In $W$ and Higgs jets, which result from the hadronisation of two quarks at Born level, a structure of two angular separated, hard energy depositions (two-prong) is expected. Top jets feature a three-prong structure from the $b$ quark and the hadronically decaying $W$ boson if all decay products are caught in one large radius jet (typically, for  $\pt\gtrsim 2m_{t}$). The dominant background are typically QCD jets which are characterised by a single hard substructure with diffuse, soft wide-angle radiation. These inputs and algorithms to calculate jet substructure observables are discussed in Sections~\ref{sec:tasinputs} and~\ref{sec:tasalgo}.

\subsubsection{Prior Mass-Cut}
Tagging variables are usually used after applying a cut on the jet mass around the interval that contains 68\% of signal events. Therefore, prior to tagging with the n-subjettiness or C2/D2 variables, a cut is applied on the calibrated mass of the large-R calorimeter jet. The mass window is calculated to cover the smallest interval around the peak mass that contains 68\% of the signal events per \pt bin. In the Higgs tagging case, there is not enough statistics to derive a conclusive result for $p_{\mathrm{T}} > 2000~\GeV$. Hence this study is restricted to the five lower $p_{\mathrm{T}}$ bins. The mass windows per \pt range are summarised in Table~\ref{table:mass_cut}.

\begin{table}[b]
\centering
\begin{tabular}{l||ll||ll||ll}
  &  \textbf{W jets}                                                    &                                 &  \textbf{Higgs jets}                                  &                                &    \textbf{Top jets}                                  &                                  \\ \hline
$p_{\mathrm{T}} \, \text{[GeV]}$   & \multicolumn{1}{l|}{Mass [GeV]} & $\frac{1}{\epsilon_{\rm bgr}}$ & \multicolumn{1}{l|}{Mass [GeV]} & $\frac{1}{\epsilon_{\rm bgr}}$ & \multicolumn{1}{l|}{Mass [GeV]}  & $\frac{1}{\epsilon_{\rm bgr}}$ \\ \hline \hline
250 - 500 & \multicolumn{1}{l|}{63 - 85}                        & 10.8                            & \multicolumn{1}{l|}{56 - 167}          & 3.8                             & \multicolumn{1}{l|}{77 - 191}          & 6.3                             \\ \cline{1-7} 
500 - 800 & \multicolumn{1}{l|}{72 - 92}                        & 13.6                            & \multicolumn{1}{l|}{92 - 150}          & 7.3                             & \multicolumn{1}{l|}{117 -205}          & 6.9                             \\ \cline{1-7} 
800 - 1200 & \multicolumn{1}{l|}{76 - 104}                       & 9.6                             & \multicolumn{1}{l|}{98 - 143}          & 9.5                             & \multicolumn{1}{l|}{122 - 218}         & 6.5                             \\ \cline{1-7} 
1200 - 1600 & \multicolumn{1}{l|}{77 - 107}                       & 7.3                             & \multicolumn{1}{l|}{103 - 149}         & 9.0                             & \multicolumn{1}{l|}{122 - 227}         & 6.3                             \\ \cline{1-7} 
1600 - 2000 & \multicolumn{1}{l|}{79 - 115}                       & 5.6                             & \multicolumn{1}{l|}{91 - 170}          & 4.4                             & \multicolumn{1}{l|}{121 - 235}         & 5.6                             \\ \cline{1-7} 
$> 2000$ & \multicolumn{1}{l|}{80 - 126}                       & 4.2                             & \multicolumn{1}{l|}{/}                 & /                               & \multicolumn{1}{l|}{123 - 251}         & 4.8                             \\ \hline
\end{tabular}
\caption{{Studied $p_{\mathrm{T}}$ regions and corresponding calculated 68\% mass intervals along with the background rejections from the mass cut for $W$, Higgs, and top jets.}} 
\label{table:mass_cut}
\end{table} 


