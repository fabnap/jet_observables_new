%-------------------------------------------------------------------------------
% This file provides a skeleton ATLAS document.
%-------------------------------------------------------------------------------
% \pdfoutput=1
% The \pdfoutput command is needed by arXiv/JHEP/JINST to ensure use of pdflatex.
% It should be included in the first 5 lines of the file.
%-------------------------------------------------------------------------------
% Specify where ATLAS LaTeX style files can be found.
\newcommand*{\ATLASLATEXPATH}{latex/}

% Use this variant if the files are in a central location, e.g. $HOME/texmf.
% \newcommand*{\ATLASLATEXPATH}{}
%-------------------------------------------------------------------------------
\documentclass[UKenglish,texlive=2013]{\ATLASLATEXPATH atlasdoc}

% The language of the document must be set: usually UKenglish or USenglish.
% british and american also work!
% Commonly used options:
%  texlive=YYYY          Specify TeX Live version (2013 is default).
%  atlasstyle=true|false Use ATLAS style for document (default).
%  coverpage             Create ATLAS draft cover page for collaboration circulation.
%                        See atlas-draft-cover.tex for a list of variables that should be defined.
%  cernpreprint          Create front page for a CERN preprint.
%                        See atlas-preprint-cover.tex for a list of variables that should be defined.
%  PAPER                 The document is an ATLAS paper (draft).
%  CONF                  The document is a CONF note (draft).
%  PUB                   The document is a PUB note (draft).
%  BOOK                  The document is of book form, like an LOI or TDR (draft)
%  txfonts=true|false    Use txfonts rather than the default newtx - needed for arXiv submission.
%  paper=a4|letter       Set paper size to A4 (default) or letter.

%-------------------------------------------------------------------------------
% Extra packages:
% \usepackage{\ATLASLATEXPATH atlaspackage}
\usepackage[subcaption]{\ATLASLATEXPATH atlaspackage}
\usepackage{overpic}	

\usepackage{verbatim}

\newcommand\crule[3][black]{\textcolor{#1}{\rule{#2}{#3}}}
% Commonly used options:
%  biblatex=true|false   Use biblatex (default) or bibtex for the bibliography.
%  backend=biber         Use the biber backend rather than bibtex.
%  subfigure|subfig|subcaption  to use one of these packages for figures in figures.
%  minimal               Minimal set of packages.
%  default               Standard set of packages.
%  full                  Full set of packages.
%-------------------------------------------------------------------------------
% Style file with biblatex options for ATLAS documents.
\usepackage{\ATLASLATEXPATH atlasbiblatex}

% Package for creating list of authors and contributors to the analysis.
\usepackage{\ATLASLATEXPATH atlascontribute}

% Useful macros
\usepackage{\ATLASLATEXPATH atlasphysics}
% See doc/atlas_physics.pdf for a list of the defined symbols.
% Default options are:
%   true:  journal, misc, particle, unit, xref
%   false: BSM, heppparticle, hepprocess, hion, jetetmiss, math, process, other, texmf
% See the package for details on the options.

%% custom stuff
% \usepackage{subcaption}
% \captionsetup{compatibility=false}

% \newcommand{\pt}{p_{T}}
\newcommand{\mta}{\ensuremath{m^\textrm{TA}}\xspace}
\newcommand{\mtas}{\ensuremath{m^\textrm{TAS}}\xspace}
\newcommand{\mcal}{\ensuremath{m^\textrm{calo}}\xspace}
\newcommand{\mcomb}{\ensuremath{m^\textrm{comb}}\xspace}
\newcommand{\mcombtas}{\ensuremath{m^\textrm{comb,\,TAS}}\xspace}
\newcommand{\cme}{\ensuremath{\sqrt{s}}\xspace}
\newcommand{\iqr}{\ensuremath{\frac{1}{2}\: \times\: 68\% \:\textrm{IQnR/median}}\xspace}
\def\spvec#1{\left(\vcenter{\halign{\hfil$##$\hfil\cr \spvecA#1;;}}\right)}
\def\spvecA#1;{\if;#1;\else #1\cr \expandafter \spvecA \fi}

% Files with references for use with biblatex.
% Note that biber gives an error if it finds empty bib files.
\addbibresource{mydocument.bib}
\addbibresource{bibtex/bib/ATLAS.bib}

% Paths for figures - do not forget the / at the end of the directory name.
\graphicspath{{logos/}{figures/}}

% Add you own definitions here (file mydocument-defs.sty).
\usepackage{mydocument-defs}
\usepackage{rotating}
\usepackage{float}
\usepackage{placeins}

\definecolor{Gray}{gray}{0.95}
\definecolor{Gray2}{gray}{0.98}
\definecolor{Red}{HTML}{E55451}

%-------------------------------------------------------------------------------
% Generic document information
%-------------------------------------------------------------------------------

% Title, abstract and document 
%-------------------------------------------------------------------------------
% This file contains the title, author and abstract.
% It also contains all relevant document numbers used by the different cover pages.
%-------------------------------------------------------------------------------

% Title
\AtlasTitle{Jet Observables using Subjet-assisted Tracks}

% Author - this does not work with revtex (add it after \begin{document})
% \author{The ATLAS Collaboration}

% Authors and list of contributors to the analysis
% \AtlasAuthorContributor also adds the name to the author list
% Include package latex/atlascontribute to use this
% Use authblk package if there are multiple authors, which is included by latex/atlascontribute
% \usepackage{authblk}
% Use the following 3 lines to have all institutes on one line
% \makeatletter
% \renewcommand\AB@affilsepx{, \protect\Affilfont}
% \makeatother
% \renewcommand\Authands{, } % avoid ``. and'' for last author
% \renewcommand\Affilfont{\itshape\small} % affiliation formatting
% \AtlasAuthorContributor{First AtlasAuthorContributor}{a}{Author's contribution.}
% \AtlasAuthorContributor{Second AtlasAuthorContributor}{b}{Author's contribution.}
% \AtlasAuthorContributor{Third AtlasAuthorContributor}{a}{Author's contribution.}
% \AtlasContributor{Fourth AtlasContributor}{Contribution to the analysis.}
\author[a]{Oleg Brandt}
\author[a]{Sascha Dreyer}
\author[a]{Fabrizio Napolitano}
\author[a]{Merve Sahinsoy}
\author[b]{Benjamin P. Nachman}
\affil[a]{Kirchhoff-Institut f\"ur Physik, Heidelberg University}
\affil[b]{Lawrence Berkeley National Laboratory}
% \affil[b]{Another Institution}

% If a special author list should be indicated via a link use the following code:
% Include the two lines below if you do not use atlasstyle:
% \usepackage[marginal,hang]{footmisc}
% \setlength{\footnotemargin}{0.5em}
% Use the following lines in all cases:
% \usepackage{authblk}
% \author{The ATLAS Collaboration%
% \thanks{The full author list can be found at:\newline
%   \url{https://atlas.web.cern.ch/Atlas/PUBNOTES/ATL-PHYS-PUB-2016-007/authorlist.pdf}}
% }

% Draft version:
% Should be 1.0 for the first circulation, and 2.0 for the second circulation.
% If given, adds draft version on front page, a 'DRAFT' box on top of each other page, 
% and line numbers.
% Comment or remove in final version.
\AtlasVersion{0.3}

% ATLAS reference code, to help ATLAS members to locate the paper
\AtlasRefCode{KIP-2016}

% ATLAS note number. Can be an COM, INT, PUB or CONF note
% \AtlasNote{ATLAS-CONF-2016-XXX}
% \AtlasNote{ATL-PHYS-PUB-2016-XXX}
% \AtlasNote{ATL-COM-PHYS-2016-XXX}

% CERN preprint number
% \PreprintIdNumber{CERN-PH-2016-XX}

% ATLAS date - arXiv submission; usually filled in by the Physics Office
% \AtlasDate{\today}

% ATLAS heading - heading at top of title page. Set for TDR etc.
% \AtlasHeading{ATLAS ABC TDR}

% arXiv identifier
% \arXivId{14XX.YYYY}

% HepData record
% \HepDataRecord{ZZZZZZZZ}

% Submission journal and final reference
% \AtlasJournal{Phys.\ Lett.\ B.}
% \AtlasJournalRef{\PLB 789 (2014) 123}
% \AtlasDOI{}

% Abstract - % directly after { is important for correct indentation
\AtlasAbstract{%
This note presents a novel approach to reconstruct jet substructure observables using subjet-assisted tracks. The momentum scale of these tracks is ``assisted'', i.e., calibrated, to the momentum scale provided by the subjets of a groomed large-radius jet.
%that the tracks are matched to. 
This approach combines the excellent angular resolution of the tracker with the superior energy measurement of the calorimeter, which takes neutral particles into account.
\\
The performance of this novel technique is studied using simulated MC events  with hadronic decays of $W$, $Z$, and Higgs bosons, as well as top quarks. A particular focus is placed on the invariant mass of the groomed large-radius jet, the single most discriminant observable against QCD multijet background. An improvement by up to 40\% in resolution is found for the jet mass reconstructed from subjet-assisted tracks $m^\textrm{TAS}$ relative to the traditional approach using the calorimeter only. 
\\
Beyond this, jet substructure observables widely used within ATLAS like the Energy Correlation Functions, $n$-subjettiness, C2, D2, $\tau_{21}$, and $\tau_{32}$ are studied. Here, an improvement in QCD multijet rejection rate by up to 100\% is found for the same identification efficiency of hadronic decays of $W$, $Z$, and Higgs bosons, as well as top quarks. An improvement by up to 50\% is found when using tracks topologically matched to subjets of the groomed large-radius jets. A strategy for the evaluation of systematic uncertainties is proposed.
}

%-------------------------------------------------------------------------------
% The following information is needed for the cover page. The commands are only defined
% if you use the coverpage option in atlasdoc or use the atlascover package
%-------------------------------------------------------------------------------

% List of supporting notes  (leave as null \AtlasCoverSupportingNote{} if you want to skip this option)
% \AtlasCoverSupportingNote{Short title note 1}{https://cds.cern.ch/record/XXXXXXX}
% \AtlasCoverSupportingNote{Short title note 2}{https://cds.cern.ch/record/YYYYYYY}
%
% OR (the 2nd option is deprecated, especially for CONF and PUB notes)
%
% Supporting material TWiki page  (leave as null \AtlasCoverTwikiURL{} if you want to skip this option)
% \AtlasCoverTwikiURL{https://twiki.cern.ch/twiki/bin/view/Atlas/WebHome}

% Comment deadline
% \AtlasCoverCommentsDeadline{DD Month 2016}

% Analysis team members - contact editors should no longer be specified
% as there is a generic email list name for the editors
% \AtlasCoverAnalysisTeam{Peter Analyser, Susan Editor1, Jenny Editor2, Alphonse Physicien}

% Editorial Board Members - indicate the Chair by a (chair) after his/her name
% Give either all members at once (then they appear on one line), or separately
% \AtlasCoverEdBoardMember{EdBoard~Chair~(chair), EB~Member~1, EB~Member~2, EB~Member~3}
% \AtlasCoverEdBoardMember{EdBoard~Chair~(chair)}
% \AtlasCoverEdBoardMember{EB~Member~1}
% \AtlasCoverEdBoardMember{EB~Member~2}
% \AtlasCoverEdBoardMember{EB~Member~3}

% A PUB note has readers and not an EdBoard -- give their names here (one line or several entries)
% \AtlasCoverReaderMember{Reader~1, Reader~2}
% \AtlasCoverReaderMember{Reader~1}
% \AtlasCoverEdBoardMember{Reader~2}

% Editors egroup
% \AtlasCoverEgroupEditors{atlas-GROUP-2016-XX-editors@cern.ch}

% EdBoard egroup
% \AtlasCoverEgroupEdBoard{atlas-GROUP-2016-XX-editorial-board@cern.ch}


% Author and title for the PDF file
\hypersetup{pdftitle={ATLAS document},pdfauthor={The ATLAS Collaboration}}

%-------------------------------------------------------------------------------
% Content
%-------------------------------------------------------------------------------
\begin{document}

\maketitle

\tableofcontents

\newpage
% List of contributors - print here or after the Bibliography.
%\PrintAtlasContribute{0.30}
%\clearpage

%-------------------------------------------------------------------------------
\section{Introduction}
\label{sec:intro}

%-------------------------------------------------------------------------------


Jets are collimated streams of particles resulting from quarks and gluons fragmentation and hadronisation.
The distribution of energy inside a jet contains information about the initiating particle. When a massive
particle such as a top quark, Higgs boson or W/Z bosons is produced with significant Lorentz boost and decays into
quarks, the entire hadronic decay may be captured inside a single jet. The mass of such jets (jet mass)
is one of the most powerful tools for distinguishing massive particle decays from the continuum multijet
background; the ratios of Energy Correlation Functions C2 and D2 (see Reference \cite{bib:ECF}) and n-Subjettiness $\tau_{21}$ and $\tau_{32}$ (see Reference \cite{bib:nsub}) provide an ad-hoc tool purposely developed for the multijet background and constitute a fundamental part of many for boson taggers.
This note documents the so-called subjet-assisted techniques with the ATLAS detector. 
The track-assisted subjet mass $m^{TAS}$ definition is presented and confronted with the standard development in ATLAS, $m^{comb}$ and $m^{TA}$. 
Energy Correlation Functions and n-Subjettiness with the modified subjet-assisted technique are presented and confronted with the standard one in ATLAS.
The note ends with conclusions for the jet observables using subjet-assisted tracks.

% %-------------------------------------------------------------------------------
% \section{Theory Introduction}
% \label{sec:theory}

% %-------------------------------------------------------------------------------








\subsection{Large-radius jet mass definitions}

% Large-radius jet, or arge-$R$ jets are jets constructed with a radius parameter of the reclustering algorithm much bigger than the standard 0.4; within ATLAS the size of large-$R$ jets is 1.0 for anti-k$_t$ and 1.2 for C/A (the area of C/A is $\sim$20\% smaller than anti-k$_t$).

% It is worth noting that, for a standard anti-k$_t$ 0.4 jet the active area \cite{antiktalgo} is $A_{anti-k_t}=\pi R^2 \simeq 0.5$, while it is $\simeq 3.14$ for 1.0 jet, i.e. around six times bigger.

% Already from this ``geometrical'' point of view, the necessity of further techniques can be understood: the effect of soft radiation contamination from Pile-Up (PU) and Underlying Event (UE) will be in this case six times bigger and spoil the efficiency of the jet mass measurements.

Large-radius jet, or large-$R$ jets are jets constructed with a radius parameter of the reclustering algorithm of 1.0 for those built using the anti-k$_t$ algorithm and 1.2 for the C/A algorithm.
SInce the active area of this jets is typically six times bigger than their counterparts of radius 0.4 which is the usual choice of jet radii within ATLAS, the necessity of further techniques is required to have control over the effect of soft radiation contamination from Pile-Up (PU) and Underlying Event (UE).

\subsubsection{Substructure: Grooming Techniques}

% This section is based on the 7 TeV article on jet Substructure \cite{substructure1}.
In order to use large-$R$ jets, it is necessary to gain additional information on the interior of these objects, i.e. using techniques that exploit its substructure allowing a jet-by-jet discrimination of the energy deposit most likely coming from the hard-scattering to other soft radiation.

A common feature in substructure is the use of \textit{sub-jet}, i.e. jets obtained from a parent jet (e.g. the large-$R$ jet), using its constituent but running the jet reclustering algorithm with a smaller radius parameter; in one large-$R$ jet, typically there are two or more sub-jets depending on the originating process and its $\pt$.

Techniques have been developed, both using sub-jets or directly constituents of a jet, which are referred to as \textit{grooming} algorithms.

Grooming algorithms are designed to retain the characteristic substructure within such a large-$R$
jet while reducing the impact of the fluctuations of the parton shower and the UE, thereby
improving the mass resolution and mitigating the influence of pile-up.

The grooming algorithms presented here are the most important ones in ATLAS: the \textit{Trimming}; other used as well, the \textit{Split-Filtering} and the \textit{Pruning} can be found in \cite{substructure1}. Details on Trimming, the most used within ATLAS and in this note, are given in the Appendix.


\subsubsection{Calorimeter Mass}

% Once the collection of constituents from the large-$R$ jet is groomed, i.e. the most likely sources of soft radiation from PU and UE are eliminated, one can start working with those and start worrying about how to get physical-related properties from it, e.g. the mass.
Once the collection of topo-clusters from the large-$R$ jet is groomed, i.e. cleaned from PU contamination through the trimming technique, it is possible to use them for the measure of physical related properties such as the jet mass, since the possible sources of soft radiation from PU and UE have been reduced.


The \textit{calorimeter mass} or $m^{calo}$ is a widely used variable which takes as input the topo-cluster information. Given that each topo-cluster $i$ has a 3D information on the energy deposit, $E_i$, $\eta$ and $\phi$, the mass can be simply calculated from 4-vector properties:
$$m^{calo}=\sqrt{\left(\sum_{i\in J}E_i\right)^2-\left(\sum_{i\in J}p_{T,i}\right)^2} $$
where $J$ labels the Large-$R$ jet and assuming the topo-clusters as massless.

\subsubsection{Track Mass}
\label{sec:tracks}
% This section briefly presents the tracks and how they are related to the properties of the large-$R$ jets.
This section briefly presents the tracks and their relation with the large-$R$ jet's properties.
% There are significant advantages of tracks and why they are interesting and possible candidate for precise mass reconstruction and a big disadvantage.
There are significant advantages and few disadvantages of their usage for precise jet mass reconstruction, which are inherited both from the detector experimental properties and from the underlying physical processes. 

First of all the performance of angular separation at low $\pt$ is intrinsically better for tracks than the calorimeter one.
The second main advantage is that tracks can be associated with the primary vertex, thus simply excluding those from PU or other beam-induced soft radiation background (this is not the case for the UE).

The requirement made on tracks to achieve optimal performance are grouped into two categories, the quality of the track, i.e. if it was fully reconstructed from the detector and separated from others with no ambiguities, and the association conditions with the primary vertex; further details are given in the appendix.

Given the set of tracks which pass this selection, the mass $m^{track}$ is calculated summing up the 4-momenta of those tracks which are ghost associated to the groomed jet.

% The main, big disadvantage is that the tracker system is completely blind to the neutral component of the jet, which, as said, amounts to c.a. a third of the total. As seen in Figure \ref{fig:trackandcalo}, the track mass (red distribution) is not only shifted towards lower values than the calorimeter mass (green distribution), but its width also degrades. 

Apart from this benefits which derive from the tracker system, there is also an important disadvantage which comes from the underlying physics: it is completely blind to the electrically neutral component (mostly $\pi^0$) of the jet. As seen in Figure \ref{fig:trackandcalo}, the track mass (red distribution) is not only shifted towards lower values than the calorimeter mass (green distribution), but its width also degrades. 

\begin{figure}[!ht]
  \centering
      \includegraphics[width=0.7\textwidth]{jet_part/trackandcalo.png}
  \caption[Mass distribution for boosted $W/Z$]{Mass distribution boosted $W/Z$: in green the $m^{calo}$ and in red the $m^{track}$. }
  \label{fig:trackandcalo}
\end{figure}

Tracks could be used either for independent mass reconstruction (and in this section is shown how this should not be done), or, most importantly, as an additional information to the calorimeter measurement.

\subsubsection{InterQuantile-Range}
The general idea of Figure of Merit (FoM) is given in the Appendix; here the InterQuantile range is described since used in this note and identical to the one used in the conference BOOST 2016.
The InterQuantile range (IQnR) is here defined as it corresponds to a sigma of a ``perfect'' Gaussian distribution: $q84\%-q16\%$ where $q84\%$ is the 84$^{th}$ percentile and $q16\%$ is the 16$^{th}$, not to be confused with the InterQua\textbf{r}tile Range (IQR) which is the $q75\%-q25\%$ and does not correspond to the sigma. The final descriptor is then divided by the Median ($\iqr$). It provides stability and high sensitivity to left-hand-side and right-hand-side tails.
% The way in which we look at the mass FoM to determine is half of the 68\% of the InterQuantile range (IQnR) (here defined such as it corresponds to a sigma of a ``perfect'' Gaussian distribution: $q84\%-q16\%$ where $q84\%$ is the 84$^{th}$ percentile and $q16\%$ is the 16$^{th}$, not to be confused with the InterQua\textbf{r}tile Range (IQR) which is the $q75\%-q25\%$ and does not correspond to the sigma) divided by the Median ($\iqr$). It provides stability and high sensitivity to left-hand-side and right-hand-side tails.

% Another important FoM, used in literature and in this note, is the response distribution: given the reconstructed mass (calorimeter, track etc.) one can compare it to its $truth$ mass ($m^{truth}$), computed from the particle at MC level before the interaction with the detector:
The IQnR is then applied to the response distribution Figure of Merit: given the reconstructed mass (calorimeter, track etc.) one can compare it to its $truth$ mass ($m^{truth}$), computed from the particle at MC level before the interaction with the detector:

$$R_m=\frac{m^{reco}}{m^{truth}}$$

Standard descriptor of the FoM e.g. in \cite{art35} and here is the IQnR of the $R_m$.
  
  
In Figure \ref{fig:iqrbin} a mass response for a single range of transverse momentum is shown, for the calorimeter mass. On the plot the contours of a standard deviation and of $q16\%$ and $q84\%$ are drawn with dashed and solid lines, respectively, showing the difference induced by the tail. This sort of plot is the key when looking quantitatively to the observable performance and can be found in the Appendix for each of the process studied in every $\pt$ range considered. 
% In this chapter will be shown, however, the quantity which describes this FOM, the IQnR, as a function of $\pt$, in order to get an understanding of the behavior in the entire spectrum and assure the exclusion of local sub-optimalities.

\begin{figure}[!ht]
  \centering
      \includegraphics[width=0.7\textwidth]{jet_part/8ResponsePTJ_h_JetRatio_mJ05CALO.pdf}
  \caption[$\mcal$ response single $\pt$ bin]{Calorimeter mass response plot for boosted $W/Z$. One the plot, right, are shown: the number of entries, the mean and the width of the fit to the Gaussian core, the integral from 0 to $\mu-\sigma$ and the one from $\mu+\sigma$ to $+\infty$, the values $\iqr$ and $\sigma/\mu$. On the distribution the dashed vertical lines represent the points $\mu-\sigma$ and $\mu+\sigma$ and the solid lines represent the $q16\%$ and $q84\%$. These lines also explicitly show the asymmetry between the left-hand-side flank, in general more pronounced, and the right-hand-side one}
  \label{fig:iqrbin}
\end{figure}

\subsection{Track-Assisted Mass ($\mta$)}
The track-assisted mass, $\mta$, was one of the first attempts to combine the information form the tracker system and from the calorimeter. It is defined as $\mta=\frac{p_T^{calo}}{p_T^{track}}\times m^{track}$, where the $p_T^{track}$ and the $m^{track}$ are calculated from the tracks which are associated to the large-radius jet, adding up their 4-momenta (hence exploiting the superior angular resolution of the tracker system); the $p_T^{calo}$ is the transverse momentum as measured from the calorimeter system. The ratio $p_T^{calo}/p_T^{track}$ restores the fraction of the missing neutral component in the $m^{track}$.
The $\mta$ has a better performance on the reconstruction of boosted objects such as $W/Z$ in the extreme kinematic regime ($\sim $ 1 TeV) and above in the transverse momentum of the decaying electroweak object. Another advantage of this observable shows up as it comes to the systematic uncertainties: in particular jet mass scale and jet mass resolution uncertainty on $\mta$ can be estimated by propagating the track reconstruction uncertainties and calorimeter-jet $\pt$ uncertainties through the definition of the variable given above. The tracking uncertainties are smaller for $\mta$ rather than $\mcal$ because a larger extent of the uncertainty cancels in the ratio $m^{track}/p_T^{track}$.
Apart all of this advantages, the track-assisted mass shows its limits when it comes to intermediate transverse momentum regimes and below ($\pt < 1 $ TeV) in $W/Z$ and for Higgs and top quarks throughout the whole kinematic space.
% The track-assisted mass, $\mta$, was one of the first attempts to combine the information form the tracker system and from the calorimeter. 
% The track mass is missing the neutral component, i.e. each measurement is missing the fraction $\frac{neutral+charged}{charged}$, but it has very good angular resolution but $\pt$ resolution degrades linearly with the transverse momentum. The calorimeter mass, on the other hand, has the limitation of the angular resolution of the topo-clusters but relative energy resolution increases at higher energies. The missing neutral fraction from the tracker system could be corrected on a jet-by-jet basis taking advantage of the two detector sub-system optimalities: this leads to the definition of the \textit{track-assisted mass} ($m^{TA}$):
% \begin{equation}
 % m^{TA}=\frac{p_T^{calo}}{p_T^{track}}\times m^{track}
% \end{equation}
% The better resolution of the $\mta$ takes place at the scale of above $\sim1$ TeV of transverse momentum for $W/Z$, while the performance is suboptimal to the calorimeter mass for all the other samples considered.
% Another advantage with respect to 
Full description of this variable is given in the ATLAS CONF Note \cite{art35}.
% The main limitation of the calorimeter mass comes from the angular resolution of the topo-clusters, which, for extreme kinematic regimes, start approaching each other at the point that they hit the granularity of the detector. The main advantage is that on the contrary the relative energy resolution increases at higher energies.

% The tracks instead have a very good angular resolution, but $\pt$ relative resolution degrades linearly with the transverse momentum. 

% One could then think about creating a variable which exploits the advantages of both and minimizes the disadvantages. As seen, the track mass is missing the neutral component, i.e. each measurement is missing the fraction $\frac{neutral+charged}{charged}$, but it could be corrected on a jet-by-jet basis: this leads to the definition of the \textit{track-assisted mass} ($m^{TA}$):
% \begin{equation}
%  m^{TA}=\frac{p_T^{calo}}{p_T^{track}}\times m^{track}
% \end{equation}

% It can be intuitively understood as follows: the term $m^{track}$ has the superior angular resolution, but misses the neutral component; the ratio $p_T^{calo}/p_T^{track}$, representing exactly the $(neutral+charged)/charged$ ratio, ``restores'' the correct value of the mass back to $charged+neutral$.
% \begin{figure}[!ht]
%   \centering
%       \includegraphics[width=0.7\textwidth]{jet_part/mta/allbinptmta.png}
%   \caption[$\mcal$ and $\mta$ mass responses]{Track-assisted mass response plot for boosted $W/Z$: in green the calorimeter mass, in red the track-assisted mass. On the right are shown properties of the fit to the Gaussian core; it can be seen than the width of the $\mta$ distribution is smaller, and the mean is slightly below the calorimeter mass.}
%   \label{fig:mta1}
% \end{figure}

% From Figure \ref{fig:mta1} the comparison of the track-assisted mass and the calorimeter mass; the width of the distribution is smaller, making this observable a good candidate for usage.


% \subsection{Advantages and Limitation of $\mta$}
% The $\mta$ has a good handle on boosted $W/Z$, looking at all the transverse momentum spectrum for these results.

% \begin{figure}[!ht]
%   \centering
%       \includegraphics[width=0.9\textwidth]{jet_part/uncert.png}
%   \caption[Comparison of the uncertainties for $\mcal$ and $\mta$]{Comparison of the uncertainties for $\mcal$, on the left, and $\mta$, on the right the rise on the high jet $\pt$ is due to statistics. From the \cite{art35}.}
%   \label{fig:uncert}
% \end{figure}

% Another big advantage which supports the use of the track-assisted mass is the relatively small uncertainties: in Figure \ref{fig:uncert} the comparison of $\mcal$ (left) and $\mta$ (right) fractional uncertainties on the JMS, shows how the tracking uncertainties are much smaller because of the ratio $m^{track}/p_T^{track}$. On the right plot the black line indicates the JMS fractional uncertainty for the $\mcal$, and is always above the $\mta$. Of course this introduces another argument in the development of new techniques, which is to look for a good balance between performance and small uncertainties: a perfect observable in terms of behavior which has very big uncertainties is not really useful.


% When looking in the extreme kinematic regime, at very high $\pt$, as in the top plot in Figure \ref{fig:mta2}, the $\mta$ shows its real strength, achieving much smaller value of the IQnR.
% However, there are some severe limitations which are worth noting, especially looking at the performance in different regions of transverse momentum: this is shown in the bottom plot of Figure \ref{fig:mta2}, where at a low $\pt$ it exhibits a much worse behavior.

% \subsubsection{Performance in $W \to q'\bar{q}$ Decays}

% \begin{figure}
%     \centering
%     \begin{subfigure}[b]{0.5\textwidth}
% 	\centering
%         \includegraphics[width=\textwidth]{jet_part/mta/highptmta.png}
   
% %         \label{fig:tiger}
%     \end{subfigure}
%     \begin{subfigure}[b]{0.5\textwidth}
% 	\centering
%         \includegraphics[width=\textwidth]{jet_part/mta/lowptmta.png}
 
% %         \label{fig:gull}
%     \end{subfigure}
%     \caption[Mass response plots for the $\mta$]{Mass response plots for selected ranges of $\pt$: on the bottom, a ``low'' range, 500 GeV $<\pt<$ 700 GeV, on the top an high $\pt$, 1900 GeV $<\pt<$ 2100 GeV. A difference in performance can be clearly seen.} 
%     \label{fig:mta2}
% \end{figure}


% The performance in all the bins of $\pt$ can be studied looking at Figure \ref{fig:mta3}; these plots have as horizontal axis the transverse momentum and as vertical one the value of the $\iqr$ calculated from the correspondingly response. For $W/Z$ jets, there is a crossing point around $\pt\sim$1 TeV, which can be understood as the point in which the two sub-jet present start merging (sub-jet multiplicity shown in Figure \ref{fig:multi} in Appendix).



% \subsubsection{Performance in $t\to q'\bar{q}b$ Decays}

% For top quarks the situation is much different: with respect to $W/Z$ jets, in fact, there are two main disparities: on one side, the mass of the top quark is much higher than the one of the electroweak bosons, hence making the separation $\Delta R=\frac{2m}{\pt}$ bigger; on the other side, the decay is not anymore two-prong (two-sub-jet-like) but rather a three-prong  (three-sub-jet-like) decay, one from the b-jet and the other two from the $W$ decay.
% $\mta$ is here never performing better than $\mcal$, as can be seen e.g. in Figure \ref{fig:mta3}, right.


% \begin{figure}[!ht]
%   \centering
%       \includegraphics[width=\textwidth]{jet_part/mtawandtop.png}
%   \caption[$m^{calo}$ and $m^{TA}$ comparison for $W/Z$ jets and top jets]{The comparison between the performance of $m^{calo}$ and $m^{TA}$ for $W/Z$ jets (left) and top jets (right); on the x-axis the transverse momentum and on the y-axes the $\iqr$ of the mass distribution, from \cite{art35}. A better observable has lower values on the y-axis. }
%   \label{fig:mta3}
% \end{figure}

% \subsubsection{Performance in $h\to b\bar{b}$ Decays}

% For boosted Higgs the $\mcal$ outperforms the $\mta$ in the spectrum of transverse momentum. Although the decay is two-pronged, the mass of the Higgs is higher than the electroweak bosons, moreover another difference lays in light quarks initiated jets and heavy quarks initiated ones, like the b-quarks from Higgs decay.
% % the b-jet poses an additional complication which comes from the branching ratio of B mesons to muons, which leave very little energy in the calorimeter system but additional tracks.

% \begin{figure}[!ht]
%   \centering
%       \includegraphics[width=0.7\textwidth]{jet_part/mta/higgsmta.png}
%   \caption[Performance of the $\mta$ with the boosted Higgs sample]{Performance of the $\mta$ with the boosted Higgs sample; the $\mta$ is the blue line, the $\mcomb$ will be described later in this chapter. From \cite{art39}. The FoM here is the resolution of the Response.}
%   \label{fig:mta4}
% \end{figure}



\subsection{The Track-Assisted Sub-jet Mass ($\mtas$)}
In this section the main outcome of the optimization of the large-radius jet mas reconstruction is presented: the \textit{track-assisted sub-jet mass} ($\mtas$).
The main idea takes inspiration from the track-assisted mass: if one can use tracks to exploit the better angular resolution and correct the missing neutral component jet-by-jet, there is an additional information that can be used. The neutral fraction, in fact, varies stochastically not only per-jet basis, but even per-sub-jet basis, since the each quark follows a different parton showering and hadronization process.
Correcting the missed neutral component per-sub-jet, it should perform better already at an intuitive level, as it accesses information from jet substructure.
There are few question in the definition of this mass observable, whose answers are in the next section:
\begin{itemize}
  \item Regarding the inputs:
  \begin{itemize}
     \item How to select the set of tracks to be used?
     \item Which kind of sub-jet should be used?
  \end{itemize}
  \item Regarding the procedure
  \begin{itemize}
  
  \item How to associate the tracks to a sub-jet?
  \item How to correct for the missed neutrals on a sub-jet basis?
  \item How to add everything back together?
 \end{itemize} 
 
\end{itemize}

Those details are given in the next subsection.


\subsection{Observable Definition: Inputs}
There are two inputs to the $\mtas$: tracks and sub-jets. The definition of the standard inputs are give here; alternative approaches are given in subsection \ref{sec:alternate}.

\subsubsection{Tracks}
Only the tracks that satisfy the quality criteria and primary vertex association, described in the appendix \ref{sec:tracks}, are used.
The tracks are additionally required to be ghost associated to the sub-jets of the groomed jet; namely only the sub-jets which survived the trimming procedure and are described in the next subsection.
Ghost association provides a clear correspondence of tracks to the sub-jets set and was therefore chosen and preferred to other kind of assignments.

\subsubsection{Sub-jets}

The choice of sub-jets must follow a simple requirement: of course we want to take those which most likely come from the hard-scattering. This means that the choice of taking them after grooming is strongly favored.

As grooming technique used, the trimming was preferred as being the standard in ATLAS and the most flexible one for optimization studies.

The standard version of the trimming uses the k$_t$ reclustering algorithm with radius of 0.2, with the transverse momentum ratio $f_{cut}$ at 5\%.

As shown later, this is also the optimal configuration for sub-jets.

\subsection{Observable Definition: Procedure}
Having tracks and sub-jets now well defined, we can describe the recipe to produce the $\mtas$. For brevity we will call the sub-jets SJ in the formulae below. 

As said, the tracks are the ones ghost-associated to the sub-jets; however, tracks which fall inside the area of the large-$R$ jet, but not inside the sub-jets area, are still much probably coming from the hard-scattering. They are then associated again to the closest sub-jets via $\Delta R$ association.

Each sub-jet will have at this point some tracks associated via ghost-association and some other via $\Delta R$ (which are maximally 5\%). We call this set of tracks, a ``custom'' Track-Jet or TJ.

At this point, the one-to-one correspondence is preserved (for each SJ there is one and only one TJ), and we can move on correcting the neutral fraction.

Getting inspired from the formula $m^{TA}=p_T^{calo}/p_T^{track}\times m^{track}$, we would like to replicate this at sub-jet level, i.e.

$$\mtas="\sum_{SJ}"\frac{p_T^{SJ}}{p_T^{TJ}}\times m^{TJ}$$

Where the summation symbol between quotation mark symbolize that the sum must be intended at 4-vector level: since now we are working inside the sub-jets, in fact, we need to change the sub-jet's 4-vector itself and not only the mass. If we call $p_\mu^{TJ}$ the Lorentz vector of the track-jet, 

$$p_\mu^{TJ} = \spvec{m^{TJ};p_T^{TJ};\eta^{TJ};\phi^{TJ}} \to p_\mu^{TA}=\spvec{m^{TJ}\times\frac{p_T^{SJ}}{p_T^{TJ}} ;p_T^{SJ};\eta^{TJ};\phi^{TJ}} $$
 
where $p_\mu^{TA}$ is the track-assisted sub-jet's 4-vector. If we label $i$ the $i$-th track-jet of the $N$ ones present in the large-$R$ jet,

$$ \mtas=\sqrt{\left(\sum_i^N p^{TA} \right)_\mu \left(\sum_i^N p^{TA} \right)^{_\mu}} $$

\subsubsection{Observable Definition: TAS Procedure}
\label{sec:tas}

As it will be shown and already stated in the introduction, the TAS procedure is being utilized with the Energy Correlation Functions and the n-Subjettiness. The four-momentum scheme which was described above and which was adopted as standard for the production of the $\mtas$ observable historically and also because of higher versatility and feasibility of implementation cannot be applied for those variable. The variable of interest to be modified which enters the computation of the ECF and n-Subjettiness is in fact the momentum.
This correction is now applied on single track rather than the whole track-jet and on the transverse momentum, not the mass.
The TAS correction reads:

$$p_\mu^{track} = \spvec{m^{track};p_T^{track};\eta^{track};\phi^{track}} \to p_\mu^{TA}=\spvec{m^{track} ;p_T^{track} \times\frac{p_T^{SJ}}{p_T^{TJ}};\eta^{track};\phi^{track}}$$

The corection factor $\frac{p_T^{SJ}}{p_T^{TJ}}$ refers to the $\pt$ of the sub-jet in which the track is associated and the $\pt$ of the track-jet associated to it.
This momentum correction was studied with previous versions of the $\mtas$ and the difference with the stardard approach was found to be negligible \cite{presentation}.

As before, these four-momenta are then summed together to give this alternative definition:

$$ \mtas=\sqrt{\left(\sum_i^M p^{TA} \right)_\mu \left(\sum_i^M p^{TA} \right)^{_\mu}} $$

where now the sum refers from the first to the M-th tracks associated to the large-$R$ jet.

\begin{figure}[!ht]
  \centering
      \includegraphics[width=0.6\textwidth]{jet_part/mtas/mtas.png}
  \caption[Pictorial event display]{Pictorial event display showing the $\eta$ $\phi$ region of a large-$R$ anti-k$_t$ trimmed jet, (in blue the catchment area of the anti-k$_t$) showing the different k$_t$ sub-jets: they are highlighted in green, fuchsia and yellow. The associated track-jets (here indicated as arrows pointing the calorimeter area) are colored with the same color of the correspondent sub-jet. Some tracks associated with $\Delta R$ procedure can be seen in the fuchsia sub-jet. The transverse momenta and mass values are also shown for the sub-jets.}
  \label{fig:mtas1}
\end{figure}

An important remark is that, in the case of a large-$R$ jet with only one sub-jet, the $\mtas$ has exactly the same definition of the $\mta$. This implies, since the angular separation of the decay product scales inversely with $\pt$, that the performance should approach the one of the $\mta$ at very high transverse momenta. However, the space for improvement is precisely in the low-intermediate $\pt$ regime.


\subsection{Jet Substructure Observables with (assisted) Tracks}\label{sec:tas_jss}


Jet substructure observables are used to distinguish between different jet topologies, corresponding to signal and background events. In $W$ and Higgs jets, originate from two quarks, a structure of two angular separated, hard energy depositions (two-prong) is expected. Top jets feature three-prong structure from the $b$ quark and the hadronically decaying $W$ boson if all decay products are caught in one large radius jet (typically $p_{\text{T}}\gtrsim 2m_{Top}$). The background are QCD jets which are characterized by a single hard substructure with diffuse, soft wide-angle radiation. These observables are calculated from the jet constituents, which are calorimeter topo-clusters in the default case or tracks and subjet-assisted tracks as studied here.

\subsubsection{Prior Mass-Cut}
Tagging variables are usually used after applying a mass-cut around the interval that contains 68\% of the signal events. Therefore, a cut is applied on the calibrated mass of the large-R calorimeter jet which is calculated to cover the smallest interval around the peak mass that contains 68\% of the signal events. In the Higgs tagging case, there is not enough statistics to derive a conclusive result for $p_{\mathrm{T}} > 2000 \, \text{GeV}$. Hence this study is restricted to the five lower $p_{\mathrm{T}}$ bins.

Prior to tagging with the n-Subjettiness or C2/D2 variables, a cut on the calibrated calorimeter jet mass is applied, given that the mass is the main discriminant in QCD jet rejection. This cut is defined to choose the smallest interval around the peak mass containing 68\% of the signal. However, the reconstructed mass depends on the $p_{\mathrm{T}}$ region, therefore a different cut was calculated for every region to meet the requirements.
\begin{table}[]
\centering
\begin{tabular}{l||ll||ll||ll}
  &  \textbf{W boson}                                                    &                                 &  \textbf{Higgs boson}                                  &                                &    \textbf{Top quark}                                  &                                  \\ \hline
$p_{\mathrm{T}} \, \text{[GeV]}$   & \multicolumn{1}{l|}{Mass [GeV]} & $\frac{1}{\epsilon_{bgr}}$ & \multicolumn{1}{l|}{Mass [GeV]} & $\frac{1}{\epsilon_{bgr}}$ & \multicolumn{1}{l|}{Mass [GeV]}  & $\frac{1}{\epsilon_{bgr}}$ \\ \hline \hline
250 - 500 & \multicolumn{1}{l|}{63 - 85}                        & 10.8                            & \multicolumn{1}{l|}{56 - 167}          & 3.8                             & \multicolumn{1}{l|}{77 - 191}          & 6.3                             \\ \cline{1-7} 
500 - 800 & \multicolumn{1}{l|}{72 - 92}                        & 13.6                            & \multicolumn{1}{l|}{92 - 150}          & 7.3                             & \multicolumn{1}{l|}{117 -205}          & 6.9                             \\ \cline{1-7} 
800 - 1200 & \multicolumn{1}{l|}{76 - 104}                       & 9.6                             & \multicolumn{1}{l|}{98 - 143}          & 9.5                             & \multicolumn{1}{l|}{122 - 218}         & 6.5                             \\ \cline{1-7} 
1200 - 1600 & \multicolumn{1}{l|}{77 - 107}                       & 7.3                             & \multicolumn{1}{l|}{103 - 149}         & 9.0                             & \multicolumn{1}{l|}{122 - 227}         & 6.3                             \\ \cline{1-7} 
1600 - 2000 & \multicolumn{1}{l|}{79 - 115}                       & 5.6                             & \multicolumn{1}{l|}{91 - 170}          & 4.4                             & \multicolumn{1}{l|}{121 - 235}         & 5.6                             \\ \cline{1-7} 
$> 2000$ & \multicolumn{1}{l|}{80 - 126}                       & 4.2                             & \multicolumn{1}{l|}{/}                 & /                               & \multicolumn{1}{l|}{123 - 251}         & 4.8                             \\ \hline
\end{tabular}
\caption{\footnotesize{Studied $p_{\mathrm{T}}$ regions and corresponding calculated 68\% mass intervals along with the background rejections from the mass cut for $W$ boson, Higgs boson and Top quark jets.}} \label{table:mass_cut}
\end{table} 




\subsubsection{Energy Correlation Functions}\label{subsec:ECF}
Information about the substructure of large-R jets can be used to discriminate between different event topologies. These are one, two and respectively three hard substructures (or prongs) inside the large-R jet. QCD jets are characterized by one hard substructure, jets originated by $W$ or $Z$ bosons feature two and Top quark jets feature three substructures (hadronic decay channels).

The \textsc{Energy Correlation Functions} ECF(N,$\beta$) or N-point correlators, described in Reference \cite{bib:ECF}, explore the substructure of a jet using a sum over the constituents. The correlation between pairs and triples of constituents is considered by the product of their $p_{\mathrm{T}}$, multiplied by the angular weighting, which is defined by the product of the pairwise angular distances of the considered constituents. This angular part can be scaled against the momentum part via an exponent $\beta$. The default value for $\beta$ is 1, corresponding to angular and momentum parts being weighted equally.
\begin{equation}
\begin{aligned}
 & \text{ECF1}  ={} \sum\limits_{constituents} p_{\mathrm{T}} \\ 
 & \text{ECF(2,$\beta$)} ={} \sum\limits_{i=1}^n \sum\limits_{j=i+1}^n p_{\mathrm{T},i}p_{\mathrm{T},j}\Delta R_{ij}^{\,\beta} \\ 
 & \text{(ECF(3,$\beta$)} ={} \sum\limits_{i=1}^n \sum\limits_{j=i+1}^n \sum\limits_{k=j+1}^n p_{\mathrm{T},i}p_{\mathrm{T},j}p_{\mathrm{T},k}(\Delta R_{ij} \Delta R_{ik} \Delta R_{jk})^{\,\beta}
\end{aligned}
\end{equation}\label{eq:ECF}
The ECF(N) variables can be expanded straightforwardly to larger values of N by considering this definition.
With this, ECF(2) uses pairwise correlation and is sensitive to two-prong structures, whereas ECF3 relies on triple-wise correlations to identify three-prong structures. ECF(1) corresponds to the $p_{\mathrm{T}}$ of the whole jet by a summation over the constituents $p_{\mathrm{T}}$, thereby serving as normalization to minimize the energy scale dependence.

The ECF(N) variable tends to very small values for collinear or soft configurations of $N$ constituents and is defined to be zero for jets with less than $N$ constituents. For ECF(2), only pairs of constituents that are angular separated but not soft result in sum terms that are non-negligible, which directly leads to the picture of two hard substructures inside the jet. A similar conclusion can be made for ECF(3) and three hard substructures. 
Resulting from this, a jet with $N$ or more hard substructures features a high ECFN value while a jet with fewer than $N$ substructures has a lower ECF(N) value. Consequently, one can define ratios of Energy Correlation Functions. Two of them, called C2 and D2 are found to be very powerful to distinguish between one- and two-prong like jets, see e.g. Reference \cite{bib:power_counting}. 
\begin{equation}
\begin{aligned}
 & \text{C2} ={} \frac{\text{ECF(3)}\cdot\text{ECF(1)}}{\text{ECF(2)}^2} \\ 
 & \text{D2} ={} \frac{\text{ECF(3)}\cdot\text{ECF(1)}^3}{\text{ECF(2)}^3}
\end{aligned}
\end{equation}\label{eq:C2D2} 
E.g. a jet originated from a $W$ boson features a small ECF(3) but a high ECF(2) value resulting in small C2/D2, corresponding to a high agreement with the two-prong hypothesis. QCD jets feature a very small ECF(3) and a small ECF(2) value. This results, considering the power of ECF(2) in the definitions, in a higher C2/D2 value as for a $W$ boson jet. 
These variables are IRC-safe for $\beta > 0$ and theoretically very well understood, see Reference \cite{bib:analytic_ECF}. D2 was found to perform slightly better for tagging $W$ boson jets as C2 in Reference \cite{bib:w_tagging}, most notably due to a more $p_{\mathrm{T}}$ robust cut value and a somewhat higher background rejection. 

% Stress default constituents are calorimeter clusters

\subsubsection{n-Subjettiness}\label{subsec:nSub}
Similar to C2/D2, ratios of \textit{n-subjettiness} variables $\tau_N$ can be used to distinguish between different jet topologies. Often used for Top tagging is the observable: 
\begin{equation}
\tau_{32} = \frac{\tau_3}{\tau_2}  
\end{equation} 
N-subjettiness $\tau_N$ quantifies the level of agreement between a given large-R jet and a certain number $N$ of subjet axes. The two mainly used subjet axes definitions are $k_\mathrm{t}$-axes and the $k_\mathrm{t}$-WTA (Winner Takes All) definition. The jet is reclustered with an exclusive $k_\mathrm{t}$-algorithm that runs the recombination just until $N$ subjets are clustered. The $k_\mathrm{t}$-axes are defined by the four-momenta of the $k_\mathrm{t}$-subjets, WTA correspond to the four-momentum of the hardest constituent in each $k_\mathrm{t}$-subjet. Used in this study is th $k_\mathrm{t}$-WTA axis definition. Calculated via a sum over the jets constituents (calorimeter clusters as default), the variable is defines as follows.
\begin{equation}
\tau_N = \frac{1}{d_0}\sum_k p_{T,k}\:min(\Delta R_{1,k},\Delta R_{2,k},...,\Delta R_{N,k})^{\beta}
\end{equation}
The constituents $p_{\mathrm{T}}$ is multiplied by the angular distance to the nearest subjet axis. The overall value is normalized with a sum over the constituents $p_{\mathrm{T}}$ times the characteristic radius parameter $R$ of the large jet.
\begin{equation}
d_0=\sum_k p_{T,k}R_0
\end{equation}
Again, the angular part can be scaled relative to the $p_{\mathrm{T}}$ factor via the exponent $\beta$. N-subjettiness is an IRC-safe variable for values of $\beta \ge 0$.

Small values of $\tau_N$ correspond to a jet with all constituents more or less aligned or near to the given $N$ subjet axes. Hence, the jet is compatible with the assumption to be composed of $N$ or fewer subjets. A higher value indicates a consistency with more than $N$ subjets as a non-negligible part is located apart of the $N$ subjet axes. Consequently, Top jets are likely to feature a small $\tau_3$ and a high $\tau_2$ value. QCD jets with their one-prong structure result in a high $\tau_{3}$ and a small $\tau_{2}$ value. While $\tau_2$ and $\tau_3$ alone provide only slightly separation, their ratio has proven to be very powerful. A Top tagging example with $\tau_{32}$ calculated with clusters in given in Figure \ref{fig:nSub_example}. The $\tau_{21}$ ratio can as well be used to tag $W$ or Higgs jets. Nevertheless, C2 and D2 were found to yield higher background rejections for $W$ tagging, see e.g. Reference \cite{bib:w_tagging}. 
\begin{figure}
\centering
\includegraphics[width=0.6\textwidth]{sascha_input/plots/Top/Beta1/h_recoJet_nSub32_bin6.pdf}
\caption{Exemplary $\tau_{32}$ distributions for Top signal jets and QCD background jets, calculated with clusters. The background tends to higher values.}\label{fig:nSub_example}
\end{figure}

\subsubsection{TAS Procedure for Jet Substructure Observables}
The concept of track assisting with the $p_{\mathrm{T}}$ ratio of the whole jet is without effect for the studied substructure variables. This can be understood from the definitions of the weighted $p_{\mathrm{T}}$ sums. If corrected with only one ratio, all tracks are scaled by the same factor $c$, which then can be put in front of the sum and cancels as soon as the ratios $\tau_{21}$ and $\tau_{32}$, respectively C2 and D2 are formed.
\begin{equation}
\begin{aligned}
 & \tau_N ={} \frac{1}{d_0}\sum_k p_{T,k} \; c \; min(\Delta R_{1,k},\Delta R_{2,k},...,\Delta R_{N,k})^{\beta} \\
 & \; \; \; \;  ={} \frac{c}{d_0}\sum_k p_{T,k}\:min(\Delta R_{1,k},\Delta R_{2,k},...,\Delta R_{N,k})^{\beta}
\end{aligned}
\end{equation}
Track assisting with ghost association to subjets (TAS), see Section \ref{sec:mtas} for $\mtas$, works with different scaling factors depending on the corresponding subjet $c_k$, which also affect ratios:
\begin{equation}
\tau_N = \frac{1}{d_0}\sum_k p_{T,k} \; c_k \; min(\Delta R_{1,k},\Delta R_{2,k},...,\Delta R_{N,k})^{\beta} 
\end{equation}\label{eq:tas_ta}
Therefore for the substructure observables the method used is the TAS, assisting single tracks only as explained in Section \ref{sec:tas}. Tracks and assisted tracks (TAS) as well as calorimeter clusters used as input for these observables are studied. The used selection for tracks is the same as for TAS, only the part of $p_{\text{T}}$ scaling is omitted.


\clearpage
\section{Figures of Merit for Performance Studies}\label{sec:FoM}



\subsection{For jet mass}
The general idea of Figure of Merit (FoM) is given in the Appendix; here the InterQuantile range is described since used in this note and identical to the one used in the conference BOOST 2016.
The InterQuantile range (IQnR) is here defined as it corresponds to a sigma of a ``perfect'' Gaussian distribution: $q84\%-q16\%$ where $q84\%$ is the 84$^{th}$ percentile and $q16\%$ is the 16$^{th}$, not to be confused with the InterQua\textbf{r}tile Range (IQR) which is the $q75\%-q25\%$ and does not correspond to the sigma. The final descriptor is then divided by the Median ($\iqr$). It provides stability and high sensitivity to left-hand-side and right-hand-side tails.
% The way in which we look at the mass FoM to determine is half of the 68\% of the InterQuantile range (IQnR) (here defined such as it corresponds to a sigma of a ``perfect'' Gaussian distribution: $q84\%-q16\%$ where $q84\%$ is the 84$^{th}$ percentile and $q16\%$ is the 16$^{th}$, not to be confused with the InterQua\textbf{r}tile Range (IQR) which is the $q75\%-q25\%$ and does not correspond to the sigma) divided by the Median ($\iqr$). It provides stability and high sensitivity to left-hand-side and right-hand-side tails.

% Another important FoM, used in literature and in this note, is the response distribution: given the reconstructed mass (calorimeter, track etc.) one can compare it to its $truth$ mass ($m^{truth}$), computed from the particle at MC level before the interaction with the detector:
The IQnR is then applied to the response distribution Figure of Merit: given the reconstructed mass (calorimeter, track etc.) one can compare it to its $truth$ mass ($m^{truth}$), computed from the particle at MC level before the interaction with the detector:

$$R_m=\frac{m^{reco}}{m^{truth}}$$

Standard descriptor of the FoM e.g. in \cite{art35} and here is the IQnR of the $R_m$.
  
  
In Figure \ref{fig:iqrbin} a mass response for a single range of transverse momentum is shown, for the calorimeter mass. On the plot the contours of a standard deviation and of $q16\%$ and $q84\%$ are drawn with dashed and solid lines, respectively, showing the difference induced by the tail. This sort of plot is the key when looking quantitatively to the observable performance and can be found in the Appendix for each of the process studied in every $\pt$ range considered. 
% In this chapter will be shown, however, the quantity which describes this FOM, the IQnR, as a function of $\pt$, in order to get an understanding of the behavior in the entire spectrum and assure the exclusion of local sub-optimalities.

\begin{figure}[!ht]
  \centering
      \includegraphics[width=0.7\textwidth]{jet_part/8ResponsePTJ_h_JetRatio_mJ05CALO.pdf}
  \caption[$\mcal$ response single $\pt$ bin]{Calorimeter mass response plot for $W/Z$ jets. One the plot, right, are shown: the number of entries, the mean and the width of the fit to the Gaussian core, the integral from 0 to $\mu-\sigma$ and the one from $\mu+\sigma$ to $+\infty$, the values $\iqr$ and $\sigma/\mu$. On the distribution the dashed vertical lines represent the points $\mu-\sigma$ and $\mu+\sigma$ and the solid lines represent the $q16\%$ and $q84\%$. These lines also explicitly show the asymmetry between the left-hand-side flank, in general more pronounced, and the right-hand-side one}
  \label{fig:iqrbin}
\end{figure}


\subsection{Receiver Operator Characteristics}\label{sec:ROC}
The separation power of discrimination variables can be studied quite intuitively by comparing the signal and background distributions of a certain variable. Another used figure of merit for the performance, especially for comparisons of different variables, is to use \textit{Receiver Operator Characteristics} (ROC) which show the achieved background rejection for different values of signal efficiency (signal fraction left after performing a cut). 
Each point is calculated from the underlying signal and background distributions by integrating the background distribution from zero \footnote[1]{If the signal distribution lies at lower values as the background.} to the point where the desired signal fraction is achieved. The fraction of background events contained in this region are kept when cutting at this signal efficiency, hence the inverse of this fraction, $\frac{1}{\epsilon_{background}}$ is an estimate for the background rejection. The lower the fraction of background events in the region, the better is the achieved exclusion. Accordingly, a good discrimination variable is represented by a ROC with preferably high values of background rejection up to high signal efficiencies.



\clearpage
%-------------------------------------------------------------------------------
\section{Performance of Track-assisted subjet mass}
\label{sec:mtas}
%-------------------------------------------------------------------------------
% The track-assisted subjet mass takes inspiration from the simpler development which is already implemented within ATLAS, the track-assisted mass which is described briefly below for completeness.
The performance of the track-assisted subjet mass is described in this section for the signal and background samples considered.

% \subsection{Track-Assisted Mass ($\mta$)}
% The track-assisted mass, $\mta$, was one of the first attempts to combine the information form the tracker system and from the calorimeter. It is defined as $\mta=\frac{p_T^{calo}}{p_T^{track}}\times m^{track}$, where the $p_T^{track}$ and the $m^{track}$ are calculated from the tracks which are associated to the large-radius jet, adding up their 4-momenta (hence exploiting the superior angular resolution of the tracker system); the $p_T^{calo}$ is the transverse momentum as measured from the calorimeter system. The ratio $p_T^{calo}/p_T^{track}$ restores the fraction of the missing neutral component in the $m^{track}$.
% The $\mta$ has a better performance on the reconstruction of boosted objects such as $W/Z$ in the extreme kinematic regime ($\sim $ 1 TeV) and above in the transverse momentum of the decaying electroweak object. Another advantage of this observable shows up as it comes to the systematic uncertainties: in particular jet mass scale and jet mass resolution uncertainty on $\mta$ can be estimated by propagating the track reconstruction uncertainties and calorimeter-jet $\pt$ uncertainties through the definition of the variable given above. The tracking uncertainties are smaller for $\mta$ rather than $\mcal$ because a larger extent of the uncertainty cancels in the ratio $m^{track}/p_T^{track}$.
% Apart all of this advantages, the track-assisted mass shows its limits when it comes to intermediate transverse momentum regimes and below ($\pt < 1 $ TeV) in $W/Z$ and for Higgs and top quarks throughout the whole kinematic space.
% % The track-assisted mass, $\mta$, was one of the first attempts to combine the information form the tracker system and from the calorimeter. 
% % The track mass is missing the neutral component, i.e. each measurement is missing the fraction $\frac{neutral+charged}{charged}$, but it has very good angular resolution but $\pt$ resolution degrades linearly with the transverse momentum. The calorimeter mass, on the other hand, has the limitation of the angular resolution of the topo-clusters but relative energy resolution increases at higher energies. The missing neutral fraction from the tracker system could be corrected on a jet-by-jet basis taking advantage of the two detector sub-system optimalities: this leads to the definition of the \textit{track-assisted mass} ($m^{TA}$):
% % \begin{equation}
%  % m^{TA}=\frac{p_T^{calo}}{p_T^{track}}\times m^{track}
% % \end{equation}
% % The better resolution of the $\mta$ takes place at the scale of above $\sim1$ TeV of transverse momentum for $W/Z$, while the performance is suboptimal to the calorimeter mass for all the other samples considered.
% % Another advantage with respect to 
% Full description of this variable is given in the ATLAS CONF Note \cite{art35}.
% % The main limitation of the calorimeter mass comes from the angular resolution of the topo-clusters, which, for extreme kinematic regimes, start approaching each other at the point that they hit the granularity of the detector. The main advantage is that on the contrary the relative energy resolution increases at higher energies.

% % The tracks instead have a very good angular resolution, but $\pt$ relative resolution degrades linearly with the transverse momentum. 

% % One could then think about creating a variable which exploits the advantages of both and minimizes the disadvantages. As seen, the track mass is missing the neutral component, i.e. each measurement is missing the fraction $\frac{neutral+charged}{charged}$, but it could be corrected on a jet-by-jet basis: this leads to the definition of the \textit{track-assisted mass} ($m^{TA}$):
% % \begin{equation}
% %  m^{TA}=\frac{p_T^{calo}}{p_T^{track}}\times m^{track}
% % \end{equation}

% % It can be intuitively understood as follows: the term $m^{track}$ has the superior angular resolution, but misses the neutral component; the ratio $p_T^{calo}/p_T^{track}$, representing exactly the $(neutral+charged)/charged$ ratio, ``restores'' the correct value of the mass back to $charged+neutral$.
% % \begin{figure}[!ht]
% %   \centering
% %       \includegraphics[width=0.7\textwidth]{jet_part/mta/allbinptmta.png}
% %   \caption[$\mcal$ and $\mta$ mass responses]{Track-assisted mass response plot for boosted $W/Z$: in green the calorimeter mass, in red the track-assisted mass. On the right are shown properties of the fit to the Gaussian core; it can be seen than the width of the $\mta$ distribution is smaller, and the mean is slightly below the calorimeter mass.}
% %   \label{fig:mta1}
% % \end{figure}

% % From Figure \ref{fig:mta1} the comparison of the track-assisted mass and the calorimeter mass; the width of the distribution is smaller, making this observable a good candidate for usage.


% % \subsection{Advantages and Limitation of $\mta$}
% % The $\mta$ has a good handle on boosted $W/Z$, looking at all the transverse momentum spectrum for these results.

% % \begin{figure}[!ht]
% %   \centering
% %       \includegraphics[width=0.9\textwidth]{jet_part/uncert.png}
% %   \caption[Comparison of the uncertainties for $\mcal$ and $\mta$]{Comparison of the uncertainties for $\mcal$, on the left, and $\mta$, on the right the rise on the high jet $\pt$ is due to statistics. From the \cite{art35}.}
% %   \label{fig:uncert}
% % \end{figure}

% % Another big advantage which supports the use of the track-assisted mass is the relatively small uncertainties: in Figure \ref{fig:uncert} the comparison of $\mcal$ (left) and $\mta$ (right) fractional uncertainties on the JMS, shows how the tracking uncertainties are much smaller because of the ratio $m^{track}/p_T^{track}$. On the right plot the black line indicates the JMS fractional uncertainty for the $\mcal$, and is always above the $\mta$. Of course this introduces another argument in the development of new techniques, which is to look for a good balance between performance and small uncertainties: a perfect observable in terms of behavior which has very big uncertainties is not really useful.


% % When looking in the extreme kinematic regime, at very high $\pt$, as in the top plot in Figure \ref{fig:mta2}, the $\mta$ shows its real strength, achieving much smaller value of the IQnR.
% % However, there are some severe limitations which are worth noting, especially looking at the performance in different regions of transverse momentum: this is shown in the bottom plot of Figure \ref{fig:mta2}, where at a low $\pt$ it exhibits a much worse behavior.

% % \subsubsection{Performance in $W \to q'\bar{q}$ Decays}

% % \begin{figure}
% %     \centering
% %     \begin{subfigure}[b]{0.5\textwidth}
% % 	\centering
% %         \includegraphics[width=\textwidth]{jet_part/mta/highptmta.png}
   
% % %         \label{fig:tiger}
% %     \end{subfigure}
% %     \begin{subfigure}[b]{0.5\textwidth}
% % 	\centering
% %         \includegraphics[width=\textwidth]{jet_part/mta/lowptmta.png}
 
% % %         \label{fig:gull}
% %     \end{subfigure}
% %     \caption[Mass response plots for the $\mta$]{Mass response plots for selected ranges of $\pt$: on the bottom, a ``low'' range, 500 GeV $<\pt<$ 700 GeV, on the top an high $\pt$, 1900 GeV $<\pt<$ 2100 GeV. A difference in performance can be clearly seen.} 
% %     \label{fig:mta2}
% % \end{figure}


% % The performance in all the bins of $\pt$ can be studied looking at Figure \ref{fig:mta3}; these plots have as horizontal axis the transverse momentum and as vertical one the value of the $\iqr$ calculated from the correspondingly response. For $W/Z$ jets, there is a crossing point around $\pt\sim$1 TeV, which can be understood as the point in which the two subjet present start merging (subjet multiplicity shown in Figure \ref{fig:multi} in Appendix).



% % \subsubsection{Performance in $t\to q'\bar{q}b$ Decays}

% % For top quarks the situation is much different: with respect to $W/Z$ jets, in fact, there are two main disparities: on one side, the mass of the top quark is much higher than the one of the electroweak bosons, hence making the separation $\Delta R=\frac{2m}{\pt}$ bigger; on the other side, the decay is not anymore two-prong (two-subjet-like) but rather a three-prong  (three-subjet-like) decay, one from the b-jet and the other two from the $W$ decay.
% % $\mta$ is here never performing better than $\mcal$, as can be seen e.g. in Figure \ref{fig:mta3}, right.


% % \begin{figure}[!ht]
% %   \centering
% %       \includegraphics[width=\textwidth]{jet_part/mtawandtop.png}
% %   \caption[$m^{calo}$ and $m^{TA}$ comparison for $W/Z$ jets and top jets]{The comparison between the performance of $m^{calo}$ and $m^{TA}$ for $W/Z$ jets (left) and top jets (right); on the x-axis the transverse momentum and on the y-axes the $\iqr$ of the mass distribution, from \cite{art35}. A better observable has lower values on the y-axis. }
% %   \label{fig:mta3}
% % \end{figure}

% % \subsubsection{Performance in $h\to b\bar{b}$ Decays}

% % For boosted Higgs the $\mcal$ outperforms the $\mta$ in the spectrum of transverse momentum. Although the decay is two-pronged, the mass of the Higgs is higher than the electroweak bosons, moreover another difference lays in light quarks initiated jets and heavy quarks initiated ones, like the b-quarks from Higgs decay.
% % % the b-jet poses an additional complication which comes from the branching ratio of B mesons to muons, which leave very little energy in the calorimeter system but additional tracks.

% % \begin{figure}[!ht]
% %   \centering
% %       \includegraphics[width=0.7\textwidth]{jet_part/mta/higgsmta.png}
% %   \caption[Performance of the $\mta$ with the boosted Higgs sample]{Performance of the $\mta$ with the boosted Higgs sample; the $\mta$ is the blue line, the $\mcomb$ will be described later in this chapter. From \cite{art39}. The FoM here is the resolution of the Response.}
% %   \label{fig:mta4}
% % \end{figure}



% \subsection{The Track-Assisted Subjet Mass ($\mtas$)}
% In this section the main outcome of the optimization of the large-radius jet mas reconstruction is presented: the \textit{track-assisted subjet mass} ($\mtas$).
% The main idea takes inspiration from the track-assisted mass: if one can use tracks to exploit the better angular resolution and correct the missing neutral component jet-by-jet, there is an additional information that can be used. The neutral fraction, in fact, varies stochastically not only per-jet basis, but even per-subjet basis, since the each quark follows a different parton showering and hadronization process.
% Correcting the missed neutral component per-subjet, it should perform better already at an intuitive level, as it accesses information from jet substructure.
% There are few question in the definition of this mass observable, whose answers are in the next section:
% \begin{itemize}
%   \item Regarding the inputs:
%   \begin{itemize}
%      \item How to select the set of tracks to be used?
%      \item Which kind of subjet should be used?
%   \end{itemize}
%   \item Regarding the procedure
%   \begin{itemize}
  
%   \item How to associate the tracks to a subjet?
%   \item How to correct for the missed neutrals on a subjet basis?
%   \item How to add everything back together?
%  \end{itemize} 
 
% \end{itemize}

% Those details are given in the next subsection.


% \subsection{Observable Definition: Inputs}
% There are two inputs to the $\mtas$: tracks and subjets. The definition of the standard inputs are give here; alternative approaches are given in subsection \ref{sec:alternate}.

% \subsubsection{Tracks}
% Only the tracks that satisfy the quality criteria and primary vertex association, described in the appendix \ref{sec:tracks}, are used.
% The tracks are additionally required to be ghost associated to the subjets of the groomed jet; namely only the subjets which survived the trimming procedure and are described in the next subsection.
% Ghost association provides a clear correspondence of tracks to the subjets set and was therefore chosen and preferred to other kind of assignments.

% \subsubsection{Subjets}

% The choice of subjets must follow a simple requirement: of course we want to take those which most likely come from the hard-scattering. This means that the choice of taking them after grooming is strongly favored.

% As grooming technique used, the trimming was preferred as being the standard in ATLAS and the most flexible one for optimization studies.

% The standard version of the trimming uses the k$_t$ reclustering algorithm with radius of 0.2, with the transverse momentum ratio $f_{cut}$ at 5\%.

% As shown later, this is also the optimal configuration for subjets.

% \subsection{Observable Definition: Procedure}
% Having tracks and subjets now well defined, we can describe the recipe to produce the $\mtas$. For brevity we will call the subjets SJ in the formulae below. 

% As said, the tracks are the ones ghost-associated to the subjets; however, tracks which fall inside the area of the large-$R$ jet, but not inside the subjets area, are still much probably coming from the hard-scattering. They are then associated again to the closest subjets via $\Delta R$ association.

% Each subjet will have at this point some tracks associated via ghost-association and some other via $\Delta R$ (which are maximally 5\%). We call this set of tracks, a ``custom'' Track-Jet or TJ.

% At this point, the one-to-one correspondence is preserved (for each SJ there is one and only one TJ), and we can move on correcting the neutral fraction.

% Getting inspired from the formula $m^{TA}=p_T^{calo}/p_T^{track}\times m^{track}$, we would like to replicate this at subjet level, i.e.

% $$\mtas="\sum_{SJ}"\frac{p_T^{SJ}}{p_T^{TJ}}\times m^{TJ}$$

% Where the summation symbol between quotation mark symbolize that the sum must be intended at 4-vector level: since now we are working inside the subjets, in fact, we need to change the subjet's 4-vector itself and not only the mass. If we call $p_\mu^{TJ}$ the Lorentz vector of the track-jet, 

% $$p_\mu^{TJ} = \spvec{m^{TJ};p_T^{TJ};\eta^{TJ};\phi^{TJ}} \to p_\mu^{TA}=\spvec{m^{TJ}\times\frac{p_T^{SJ}}{p_T^{TJ}} ;p_T^{SJ};\eta^{TJ};\phi^{TJ}} $$
 
% where $p_\mu^{TA}$ is the track-assisted subjet's 4-vector. If we label $i$ the $i$-th track-jet of the $N$ ones present in the large-$R$ jet,

% $$ \mtas=\sqrt{\left(\sum_i^N p^{TA} \right)_\mu \left(\sum_i^N p^{TA} \right)^{_\mu}} $$
 
% \begin{figure}[!ht]
%   \centering
%       \includegraphics[width=0.6\textwidth]{jet_part/mtas/mtas.png}
%   \caption[Pictorial event display]{Pictorial event display showing the $\eta$ $\phi$ region of a large-$R$ anti-k$_t$ trimmed jet, (in blue the catchment area of the anti-k$_t$) showing the different k$_t$ subjets: they are highlighted in green, fuchsia and yellow. The associated track-jets (here indicated as arrows pointing the calorimeter area) are colored with the same color of the correspondent subjet. Some tracks associated with $\Delta R$ procedure can be seen in the fuchsia subjet. The transverse momenta and mass values are also shown for the subjets.}
%   \label{fig:mtas1}
% \end{figure}

% An important remark is that, in the case of a large-$R$ jet with only one subjet, the $\mtas$ has exactly the same definition of the $\mta$. This implies, since the angular separation of the decay product scales inversely with $\pt$, that the performance should approach the one of the $\mta$ at very high transverse momenta. However, the space for improvement is precisely in the low-intermediate $\pt$ regime.

\subsection{Performance in $W \to q'\bar{q}$ Decays}
The $W/Z$ decay was the first one looked at, and with which the $\mtas$ was designed. The $\mcal$ shows a fast deterioration of the performance at high $\pt$, and, as shown in the previous section, the $\mta$ prevents this deterioration but suffers at low transverse momenta ($\pt<1$ TeV).
The $\mtas$ has a similar behavior in the extreme transverse momentum regime as the $\mta$, since the subjet multiplicity peaks at one, where there are no differences between the two observables.
In the low-$\pt$ regime, on the contrary, it exploits the difference in charged to neutral ratio for each subjet, achieving a better performance.
This is shown in Figure \ref{fig:mtas2} as a function of $\pt$: below $\sim$ 1 TeV achieves lower values of the IQnR converging from below to the $\mta$ as the number of subjets decreases to one.

% \begin{figure}[!ht]
%   \centering
%       \includegraphics[width=0.7\textwidth]{jet_part/mtas/71graphcftr_h_JetRatio_mJ12CALOIQRoMWZ.pdf}
%   \caption[$\mtas$ for boosted $W/Z$]{Performance of the $\mtas$ versus the $\mcal$ and $\mta$ for the boosted $W/Z$ sample.}
%   \label{fig:mtas2}
% \end{figure}


\subsection{Performance in $h\to b\bar{b}$ Decays}
In the Randall-Sundrum graviton to di-Higgs to four b-quark, the performance is again problematic for the $\mta$ with respect to $\mcal$, which is far beyond the latter, while the performance of the $\mtas$ is partially similar to the top-quark decay, but degrades much more in the extreme $\pt$ regime, following the $\mta$. Shown in Figure \ref{fig:mtas4}.

\begin{figure}
    \centering
    \begin{subfigure}[b]{0.45\textwidth}
  \centering
      \includegraphics[width=0.9\textwidth]{jet_part/mtas/71graphcftr_h_JetRatio_mJ12CALOIQRoMWZ.pdf}
  \caption[$\mtas$ for boosted $W/Z$]{$W/Z$ jets.}
  \label{fig:mtas2}
    \end{subfigure}%
    \begin{subfigure}[b]{0.45\textwidth}
  \centering
      \includegraphics[width=0.9\textwidth]{jet_part/mtas/71graphcftr_h_JetRatio_mJ12CALOIQRoMHiggs.pdf}
  \caption[$\mtas$ for boosted Higgs]{Higgs jets.}
  \label{fig:mtas4}
    \end{subfigure}
 \begin{subfigure}[b]{0.45\textwidth}
  \centering
      \includegraphics[width=0.9\textwidth]{jet_part/mtas/71graphcftr_h_JetRatio_mJ12CALOIQRoMTops.pdf}
  \caption[$\mtas$ for boosted tops]{Top jets.}
  \label{fig:mtas3}
    \end{subfigure}%


    \caption[Performance of the $\mtas$ versus the $\mcal$ and $\mta$]{Performance of the $\mtas$ versus the $\mcal$ and $\mta$ for $W/Z$, top left, where $\mta$ is not better than $\mcal$ in the low $\pt$ range but is outperformed by the $\mtas$;  Higgs decay, where $\mcal$ is everywhere better than $\mta$, yet comparable with $\mtas$ and top decays where the more complex topology makes critical the high $\pt$ regime} 
    % \label{fig:meanandtail}
\end{figure}


\subsection{Performance in $t\to q'\bar{q}b$ Decays}
The top decays are shown on Figure \ref{fig:mtas3}; the $\mtas$ is comparable yet slightly worse than the $\mcal$ in the low-middle $\pt$ regime, while degrades at higher $\pt$ approaching the $\mta$, which is far beyond the track-assisted subjet mass in performance.
As already noted, the worse performance can be ascribed both to the higher top-quark mass, and to its different and more complex decay topology.


% \begin{figure}[!ht]
%   \centering
%       \includegraphics[width=0.7\textwidth]{jet_part/mtas/71graphcftr_h_JetRatio_mJ12CALOIQRoMTops.pdf}
%   \caption[$\mtas$ for boosted tops]{Performance of the $\mtas$ versus the $\mcal$ and $\mta$ for the boosted top sample.}
%   \label{fig:mtas3}
% \end{figure}


% \begin{figure}[!ht]
%   \centering
%       \includegraphics[width=0.7\textwidth]{jet_part/mtas/71graphcftr_h_JetRatio_mJ12CALOIQRoMHiggs.pdf}
%   \caption[$\mtas$ for boosted Higgs]{Performance of the $\mtas$ versus the $\mcal$ and $\mta$ for the boosted Higgs sample.}
%   \label{fig:mtas4}
% \end{figure}




\subsection{Performance in QCD Multijet Events}
The behavior of the QCD multijet sample is similar to the $W/Z$ sample, where the $\mta$ exhibits a crossing point in the middle-low regime $\pt\simeq900$ GeV and proceeds with a better performance at high transverse momenta.
Again the $\mtas$ follows this similarity showing no crossing point and an optimal overall behavior, both with respect to calorimeter- and track-assisted-based mass definition. On Figure \ref{fig:mtas5}.

\begin{figure}[!ht]
  \centering
%       \includegraphics[width=\textwidth]{jet_part/mtas/qcdmtas.png}
        \includegraphics[width=0.50\textwidth]{jet_part/mtas/qcdmtastruffa.png}
   \caption[$\mtas$ for QCD jets]{Performance of the $\mtas$ versus the $\mcal$ and $\mta$ for the QCD multijet shows a much better behavior of the track-assisted subjet mass. Here shown $50\% \:\textrm{IQnR/median}$ and not the $\iqr$.}
  \label{fig:mtas5}
\end{figure}

\subsection{Performance in Massive $\tilde{W}\to q'\bar{q}$ Decays with $m_{\tilde{W}}=m_t$}
The massive $W$ sample is a special sample which was used to understand the behavior of top jets, whether its worse resolution was coming from the higher mass of the top quark or from the more complex decay topology (three-pronged instead of two-pronged decay and $b$-quark presence). 
The sample is almost identical to the $W/Z$ one ($W'\to WZ$) but in this case the SM electroweak boson have the mass of the top quark $m_{\tilde{W}}=m_t$.
In fact, from the rule $\Delta R\simeq 2m/p_T$, a bigger separation is expected between quarks from the hadronic decay.
The comparison with $\mcal$ is shown in Figure \ref{fig:mtas6}, together with the top-quark jet for completeness. As seen here, the performance of the latter is clearly worse than the former, the trend is yet very similar. This difference is interpreted in terms of different and more complex topology and hence higher subjet multiplicity: in the three subjet structure, resolving accurately the components is more challenging.

\begin{figure}[!ht]
  \centering
     \includegraphics[width=0.55\textwidth]{jet_part/mtas/71graphcftr_h_JetRatio_mJ12CALOIQRoMcalib_WmassiveVsTops.pdf}
   \caption[$\mtas$ for massive $W/Z$]{Performance of the $\mtas$ versus the $\mcal$ for the massive $W/Z$ (in red and green); shown on the same plot also the top sample (in blue and light blue).}
  \label{fig:mtas6}
\end{figure}

\subsection{Stability of Mean of Response and Left-Hand-Side Integral }
The stability of the $\mtas$ was checked, although the IQnR is already a good quantifier of stability, explicitly for the mean of the mass response distribution and for the left-hand-side tail, as a function of the transverse momentum. This was an important check to assure the overall gaussianity of the final distribution in the whole spectrum of $\pt$, and suitability in regards of the calibration step, which is not discussed in this note.

The mean of the response distribution is shown for $W/Z$ decays in Figure \ref{fig:meanandtail}, left; as seen here, despite the mean being constantly below unity, its behavior is much more flat and independent of $\pt$, especially in the low-intermediate regime. This is surprising since the $\mcal$ is already shown after the calibration step, which is not taken instead for the $\mtas$. Conversely the left-hand-side tail of the mass response which is shown in the same figure, right, shows a more enhanced behavior than the $\mcal$, but still never reaches the 10\%. Of course an enhancement of the tail causes a loss of gaussianity and a number of jets which are reconstructed with a lower mass than they should, but it is still comparable with the calorimeter mass.

Those quantifiers show analogous behavior for the other samples considered and those figures can be found in the Appendix.

\begin{figure}
    \centering
    \begin{subfigure}[b]{0.45\textwidth}
	\centering
        \includegraphics[width=\textwidth]{appendixB/mTAS_W_calibmCal_20:07:01-03-11-2016/71graph_h_JetRatio_mJ12CALO_meanResponseMvsTA.pdf}
%         \label{fig:tiger}
    \end{subfigure}
    \begin{subfigure}[b]{0.45\textwidth}
	\centering
        \includegraphics[width=\textwidth]{appendixB/mTAS_W_calibmCal_20:07:01-03-11-2016/74graph_h_JetRatio_mJ12CALO_I50ResponseMvsTAnorm.pdf}
 
%         \label{fig:gull}
    \end{subfigure}
    \caption[Mean and left-hand side integral for boosted $W/Z$]{Stability quantifiers which were checked for the $\mtas$: mean on the left and normalized left-hand side integral of the mass response distribution on the right. The mean is calculated from a Gaussian fit and the integral goes from 0 to 0.6.} 
    \label{fig:meanandtail}
\end{figure}

\subsection{Potential Improvements from Subjet Calibration}

An additional attempt of calibrating the subjet was also tried and, although the results were not substantially improved, it is presented in this subsection. This study was performed using only $W/Z$ samples.

% \subsection{Preliminary Studies on Subjet Calibration}
The \textit{perfect calibration} refers to the procedure of using $\mtas$ with truth-level information for calorimeter and tracker system, i.e. looking at the best possible scenario with an ideal detector. The performance is of course expected to be optimal, because of the use of the truth-level. This step was necessary as feasibility study, to understand whether ulterior efforts in this direction were meaningful.
% The first attempt in calibrating the subjets had as start a ``perfect calibration'', which means using the truth-level information from the MC sample \textit{before} the interaction with the calorimeter.
Truth-level tracks are the particles in the jet which have an electric charge and are stable, truth-level subjets are all the particles, charged and not, which are ghost associated to the calorimeter subjets.
There are few possibilities in doing so, here some nomenclature for this study will be introduced:
\begin{itemize}
 \item $\mtas$ using truth-level subjets and tracks; normal tracks (with all detector effects) are used to assist the truth-level subjets;
 \item $\mtas$ using truth-level tracks and truth-level subjets; the truth-level tracks are used to assist the truth-level subjets;
 \item $\mcal$ truth, calculated using only the truth subjets.
\end{itemize}


% \subsubsection{Perfect Calibration}


% \begin{figure}[!ht]
%   \centering
%       \includegraphics[width=0.7\textwidth]{jet_part/calib/perfcalib.png}
%   \caption[Perfect calibration]{Performance of the perfect calibration, using truth-level subjets and truth-level tracks. It shows room for improvement especially at low-middle $\pt$.}
%   \label{fig:perfcalib}
% \end{figure}


\subsubsection{Simple Subjet Calibration}
The perfect calibration using truth level subjets and tracks is shown in Figure \ref{fig:perfcalib4} in blue dots; since the performance exhibits room for big improvement below $\sim$ 1 TeV and moderate to small improvement above this value, the second step of a simple calibration was tried.

Following the example of calibration of jets in general, a simple approach to emulate this procedure was tried, constructing in various bins of transverse momenta the responses of the subjet's energy to derive the weights factors to be applied. The detailed procedure is as follows:
\begin{enumerate}
 \item Responses in energy $R_E=E^{reco}/E^{truth}$ were built in several bins of $\pt$, spanning to the whole transverse momentum range;
 \item The mean $\mu_R$ of this response was calculated via a fit to the Gaussian core;
 \item Those values (\textit{scale factors}) were stored and applied again to the subjets before the computation of the $\mtas$ via 4-momentum correction $E'=E/\mu_R$; the $\pt$ (the value which only enters the $\mtas$ variable) was changed then correspondingly to keep the subjet's mass constant.
\end{enumerate}

This procedure was called \textit{poor man's calibration} or PM calibration or \textit{simple calibration}.
A check on the $\pt$ response before and after calibration together with the mean of the entire Large-$R$ jet response is shown in Figure \ref{fig:calibA} and \ref{fig:calibA2} in Appendix.

The results are on Figure \ref{fig:perfcalib4}; there are only marginal improvements in few ranges of low transverse momentum where the scale factors are further away from unity, and the overall observable is not performing better than the standard $\mtas$. This is interpreted both in terms of a missing calibration as a function of the $\eta$ variables (having hence a befit from the crack region) and because the correction done on average does not provide the sufficient handle in a jet-by-jet basis, especially when all the subjets are rescaled by similar factors (which translates into a similarity of $\pt$s of the subjets, often the case for e.g. $W/Z$ decays, less for tops jets entirely contained in the large-$R$ jet).

\begin{figure}[!ht]
  \centering
      \includegraphics[width=0.55\textwidth]{jet_part/calib/perfectcalib4.png}
  \caption[Simple calibration]{Performance of the poor man's calibration. The improvement is marginal throughout the entire transverse momentum space.}
  \label{fig:perfcalib4}
\end{figure}

\subsection{Limitation of $\mtas$ from tracking}
The final effort to understand the various and competing effects, which take place in the $\mtas$ brought to a final study on the variable to understand the reason for the worsening of the resolution at high transverse momenta, using the truth MC information.

First of all the track mass resolution was studied: a response of the mass of the tracks associated to the jet ($m^{track}$) was constructed, using the truth-level tracks.

The result is shown on Figure \ref{fig:trackdegrade}: for the samples considered, it shows a linear degradation of the $m^{track}$, both for massive and SM $W/Z$.

\begin{figure}[!ht]
  \centering
      \includegraphics[width=0.55\textwidth]{jet_part/calib/71graphcftr_h_JetRatio_mJ12CALOIQRoMcalib_trkmass.pdf}
  \caption[Track mass degradation in tops and massive $W/Z$]{The performance of the track mass ($m^{track}$) in blue and red for massive $W$ sample and boosted $W/Z$ respectively; for reference in green the calorimeter mass of the large-$R$ jet.}
  \label{fig:trackdegrade}
\end{figure}
% ***change figures with the one in appendix***
The hypothesis of the degradation of the $\mtas$ driven by the tracks is also supported by the Figure \ref{fig:breakdown2} and \ref{fig:breakdown3}, where the truth-level tracks are used instead of real tracks to compute the variable; the flat behavior at high $\pt$ ascribes the worsening of the resolution to tracks at higher transverse momenta.
In particular the black dots show the $\mtas$ using truth-level subjets but real tracks for the track assistance procedure.
Even combining this truth-level information, in fact, it shows a large worsening of the performance (truth-level subjets only are shown as blue dots).
On the other side using again truth-level tracks for the track assistance procedure of the truth-level subjet, shows a recovery of the loss in performance.

% Particularly interesting is the black dots, which 
% uses truth-level subjet but real tracks, which worsen the overall performance of the truth-level subjet alone (shown in light blue dots).

% \begin{figure}[!ht]
%   \centering
%       \includegraphics[width=0.7\textwidth]{jet_part/calib/71graphcftr_h_JetRatio_mJ12CALOIQRoM4Truths.pdf}
%   \caption[Breakdown of the $\mtas$ ]{Breakdown of the $\mtas$ in its component using truth-level information for boosted $W/Z$ decays. In blue the $\mtas$ using truth-level subjets and truth level tracks, in black $\mtas$ using truth level subjets but real tracks and in light blue for reference the mass of the truth level particles associated to the subjets. As usual, in red and green the standard $\mtas$ and the $\mcal$}
%   \label{fig:breakdown2}
% \end{figure}

% Other results using truth-level information on boosted tops are shown and described in the Appendix.

\begin{figure}
    \centering
    \begin{subfigure}[b]{0.45\textwidth}
  \centering
      \includegraphics[width=0.9\textwidth]{jet_part/calib/71graphcftr_h_JetRatio_mJ12CALOIQRoM4Truths.pdf}
  \caption[Breakdown of the $\mtas$ ]{$W/Z$ jets}
  \label{fig:breakdown2}
    \end{subfigure}
    \begin{subfigure}[b]{0.45\textwidth}
  \centering
      \includegraphics[width=0.9\textwidth]{jet_part/appendixA/71graphcftr_h_JetRatio_mJ12CALOIQRoM4TruthsTops.pdf}
  \caption[Breakdown of the $\mtas$ ]{top jets}
  \label{fig:breakdown3}
 
    \end{subfigure}
    \caption[Breakdown of the $\mtas$]{Breakdown of the $\mtas$ in its component using truth-level information for $W/Z$ decays, on the left. In blue the $\mtas$ using truth-level subjets and truth level tracks, in black $\mtas$ using truth level subjets but real tracks and in light blue for reference the mass of the truth level particles associated to the subjets. As usual, in red and green the standard $\mtas$ and the $\mcal$. On the right the same for top jets.} 
    % \label{fig:meanandtail}
\end{figure}

% \chapter{Limitation of the $\mtas$}
% Additional studies on the limitation of the $\mtas$ based on MC studies without detector interactions are also presented. In particular, the truth study presented for $W/Z$ decay in were extended for top quark decays.

% As seen on Figure \ref{fig:breakdown3}, the breakdown of the $\mtas$ shows that, in particular for the high transverse momenta regimes, the tracks are subjected to fast degradation which makes their combination with the calorimeter mass not anymore an advantage. 
In particular for the top decay, the breakdown shows that the track degradation is not anymore compensated by the calorimeter for high $\pt$, making their combination not anymore an advantage.

% This limitation was expected and understood from the detector performance point of view, with the variables which are presented here $\mta$ and $\mtas$ to reach a competitive standpoint with the $\mcal$ in the extreme kinematic regime for the top quark decay.

In black, in fact, the performance of the $\mtas$ variable using tracks with detector effect and subjets without those effects, shows this intrinsic limit which takes place already at 1.5 TeV.

The crossing point is, as already pointed out for the top jets, present because of the optimal performance of the calorimeter system caused by the higher mass of the top quark, and partially also because of its more complex decay structure and difficulty to be resolved in subjets.

% \begin{figure}[!ht]
%   \centering
%       \includegraphics[width=0.7\textwidth]{jet_part/appendixA/71graphcftr_h_JetRatio_mJ12CALOIQRoM4TruthsTops.pdf}
%   \caption[Breakdown of the $\mtas$ ]{Breakdown of the $\mtas$ in its component using truth-level information for boosted top quarks decays.}
%   \label{fig:breakdown3}
% \end{figure}



% \subsection{Large-R jet: Calibration}
% The jet mass scale calibration aims to correct the reconstructed jet mass to the particle-level jet mass by applying calibration factors derived from a sample of simulated QCD multijet events, with an analogous procedure described in \ref{sec:calib} for the jet energy scale.
% 
% \subsection{Large-R jet: Uncertainties}


\subsection{Performance with Alternate Inputs to the $\mtas$}
\label{sec:alternate}

There are quite a few ways to modify the track-assisted subjet mass; however, all the alternative approaches showed worse performance, and they are mentioned here for completeness only.
The per-track four momentum correction scheme which is used for the ECF and the n-Subjettiness and also explored with the $\mtas$ with no significant difference was described in \ref{sec:tas}.

The other alternatives considered were: 
\begin{itemize}
 \item for the tracks:
 \begin{itemize}
   \item use of tracks not as input directly, but only taking those belonging to anti-k$_t$ reclustered track-jet with radius of 0.3 or 0.2;
   \item tighter or looser quality conditions were explored;
   \item tighter or looser primary vertex association requirement were explored.
 \end{itemize}
 \item for the subjets:
  \begin{itemize}
   \item the trimming procedure was modified: various radii $R_{sub}$ of the subjets were tested;
   \item the subjets were reclustered using not only the standard k$_t$, but also anti-k$_t$ and C/A.
  \end{itemize}
  \item for the procedure: different 4-momentum correction scheme was also studied in more details, see \ref{sec:tas}.
\end{itemize}

The different reclustering algorithm choice has a deep impact and was studied in details, since it changes the topo-cluster added to the subjets and the tracks associated to them. The situation is depicted in the event-display in Figure \ref{fig:evtdspl}; the display on the left shows the standard choice of k$_t$, the one on the right shows the modified approach anti-k$_t$. 

In Figure \ref{fig:allalgow} \ref{fig:allalgotop} \ref{fig:allalgohiggs} the performance for $W/Z$, tops and Higgs jets are shown, respectively. It can be seen that the k$_t$ algorithm provides the best observable definition, in all the samples considered. However, the anti-k$_t$ algorithm provides similar performances; this was an important check as the jet calibration procedure currently going on in ATLAS, the \textit{R-Scan} procedures includes the anti-k$_t$ algorithm with radius of R=0.2 and aims at providing the calibration and uncertainties that could be used directly in the computation of the $\mtas$.
% *** quantify the difference ***

\begin{figure}[!ht]
  \centering
      \includegraphics[width=0.55\textwidth]{jet_part/mtas/evtdspl.png}
  \caption[Different reclustering in event display]{An example of event-display shows the differences in the reclustering algorithm used for the subjets: on the right  k$_t$ and on the left anti-k$_t$. Highlighted some constituents trimmed away with the second choice.}
  \label{fig:evtdspl}
\end{figure}



% \begin{figure}
%     \centering
%    \includegraphics[width=\textwidth]{jet_part/mtas/71graphcftr_h_JetRatio_mJ12CALOIQRoM_Wprime_Allalgos.pdf}
   
%     \caption{Performance of $\mtas$ with different reclustering algorithm for the subjets: anti-k$_t$, k$_t$ and C/A. Boosted $W/Z$ sample.}
%     \label{fig:allalgow}
% \end{figure}

% \begin{figure}
%     \centering
%    \includegraphics[width=\textwidth]{jet_part/mtas/71graphcftr_h_JetRatio_mJ12CALOTopsCalib.pdf}
   
%     \caption{Performance of $\mtas$ with different reclustering algorithm for the subjets: anti-k$_t$, k$_t$ and C/A. Boosted top sample.}
%     \label{fig:allalgotop}
% \end{figure}

% \begin{figure}
%     \centering
%    \includegraphics[width=\textwidth]{jet_part/mtas/71graphcftr_h_JetRatio_mJ12CALOIQRoMHiggsNOCalib.pdf}
   
%     \caption{Performance of $\mtas$ with different reclustering algorithm for the subjets: anti-k$_t$, k$_t$ and C/A. Boosted higgs sample.}
%     \label{fig:allalgohiggs}
% \end{figure}

\clearpage
\newpage
\section{Performance of Combined Calorimeter and Track-Assisted SubJet Mass}
This section presents the achievement of the variable obtained combining the $\mtas$ and the $\mcal$, the $\mcombtas$ with respect to the combination of the $\mta$ and the $\mcal$, the $\mcomb$. Both these variables were defined in \ref{subsec:comb}

\begin{figure}
    \centering
    \begin{subfigure}[b]{0.45\textwidth}
        \centering
   \includegraphics[width=\textwidth]{jet_part/mtas/71graphcftr_h_JetRatio_mJ12CALOIQRoM_Wprime_Allalgos.pdf}
    \caption{$W/Z$ jets.}
    \label{fig:allalgow}
    \end{subfigure}
    \begin{subfigure}[b]{0.45\textwidth}
        \centering
   \includegraphics[width=\textwidth]{jet_part/mtas/71graphcftr_h_JetRatio_mJ12CALOIQRoMHiggsNOCalib.pdf}
    \caption{Higgs jets.}
    \label{fig:allalgohiggs}
    \end{subfigure}

    \begin{subfigure}[b]{0.45\textwidth}
        \centering
   \includegraphics[width=\textwidth]{jet_part/mtas/71graphcftr_h_JetRatio_mJ12CALOTopsCalib.pdf}
    \caption{Top jets.}
    \label{fig:allalgotop}
    \end{subfigure}

\caption{Performance of $\mtas$ with different reclustering algorithms for the subjets: anti-k$_t$, k$_t$ and C/A and for $W/Z$ jets, top left, Higgs jets, top right and top jets, bottom. In all the cases shown, the k$_t$ is producing the better results, but all the three have a very similar performance.}
\end{figure}


\subsection{Performance in $W \to q'\bar{q}$ Decays}
On the $W/Z$s decays, the $\mcombtas$ outperforms all the other definitions throughout all the transverse momentum space; on Figure \ref{fig:mcombtas3} they are shown for reference together with the $\mtas$. It can be noted here that the track-assisted subjet mass, although being sub-optimal, has comparable performance, yet presenting fewer complications due to the combination procedure.

% \begin{figure}[!ht]
%   \centering
%       \includegraphics[width=0.7\textwidth]{jet_part/mcomb/mcombtas3.pdf}
%   \caption[$\mcombtas$ on the boosted $W/Z$]{Performance of the combined mass on $W/Z$ samples; here shown the two definitions of the combined mass, $\mcomb$ and $\mcombtas$, together with the calorimeter mass and the track-assisted subjet mass.}
%   \label{fig:mcombtas3}
% \end{figure}


\begin{figure}
    \centering
    \begin{subfigure}[b]{0.45\textwidth}
  \centering
      \includegraphics[width=0.9\textwidth]{jet_part/mcomb/mcombtas3.pdf}
  \caption[$\mcombtas$ on the boosted $W/Z$]{$W/Z$ jets.}
  \label{fig:mcombtas3}
    \end{subfigure}
    \begin{subfigure}[b]{0.45\textwidth}
  \centering
      \includegraphics[width=0.9\textwidth]{jet_part/mcomb/mcombtas5.pdf}
  \caption[$\mcombtas$ on the boosted Higgs]{Higgs jets.}
  \label{fig:mcombtas5}
    \end{subfigure}

    \begin{subfigure}[b]{0.45\textwidth}
  \centering
      \includegraphics[width=0.9\textwidth]{jet_part/mcomb/mcombtas4.png}
  \caption[$\mcombtas$ on the boosted tops]{Top jets.}
  \label{fig:mcombtas4}
    \end{subfigure}
\caption{Performance of $\mcomb$ and $\mcombtas$ for different samples: the $W/Z$ jets, top left, the Higgs jets, top right and the top jets, bottom. The $\mcombtas$ outperforms the other definitions throughout the whole spectrum of transverse momentum. The $\mtas$, although being sub-optimal follows with similar performance the $\mcomb$. The Higgs and top jets presents the same properties as shown before, and the combined mass reflects these properties. }
\end{figure}

\subsection{Performance in $h\to b\bar{b}$ Decays}
Again, for the Higgs decay there are similarities as for the top sample; on Figure \ref{fig:mcombtas5} the two definitions of the combined mass, together with the simpler $\mtas$. Although this variable is slightly sub-optimal yet still comparable in the low to intermediate range in transverse momenta, where the tracks are driving a decrease in performance for the high to very-high $\pt$. The $\mcombtas$ uses this advantage to achieve optimal behavior in the entire transverse momentum spectrum, outperforming both $\mcal$ and $\mcomb$ almost everywhere.


\subsection{Performance in $t\to q'\bar{q}b$ Decays}
The top decay remains the most challenging phenomenon also with the combined mass; as seen on Figure \ref{fig:mcombtas4}, the $\mcomb$ performs quite similarly to the calorimeter based mass definition, behaving considerably better than the $\mtas$ especially at high transverse momentum. The $\mcombtas$, however, outperforms all the other definitions, and shows its optimal observable strength at intermediate $\pt$ i.e. in the range $0.8 < \pt < 1.6$ TeV.

% \begin{figure}[!ht]
%   \centering
%       \includegraphics[width=0.7\textwidth]{jet_part/mcomb/mcombtas4.png}
%   \caption[$\mcombtas$ on the boosted tops]{Performance of the combined mass on the top sample; here shown the two definitions of the combined mass, $\mcomb$ and $\mcombtas$, together with the calorimeter mass and the track-assisted subjet mass.}
%   \label{fig:mcombtas4}
% \end{figure}

% \begin{figure}[!ht]
%   \centering
%       \includegraphics[width=0.7\textwidth]{jet_part/mcomb/mcombtas5.pdf}
%   \caption[$\mcombtas$ on the boosted Higgs]{Performance of the combined mass on the Higgs decay; here shown the two definitions of the combined mass, $\mcomb$ and $\mcombtas$, together with the calorimeter mass and the track-assisted subjet mass.}
%   \label{fig:mcombtas5}
% \end{figure}







\clearpage
%-------------------------------------------------------------------------------
\section{Performance of Jet Substructure Observables with (assisted) Tracks}\label{sec:def_beta}
This section shows the performance of (assisted) tracks compared to clusters as input for the substructure observables C2/D2 and $\tau_{21}$ for $W$ and Higgs tagging as well as $\tau_{32}$ for Top tagging. For now, the angular weighting parameter $\beta$ is set to the default value of 1.
\subsection{Effect of Track Selection}
There are different collections of tracks that could be used to calculate substructure variables. Compared here are tracks that are ghost associated to the ungroomed large-R jet with the collection which is also used for the $\mtas$, see Section \ref{subsec:ObsDef_Proc}, which is ghost association to $k_T$-subjets and $\Delta R$ matching of tracks close to sub-jets.

The distributions showing the number of tracks associated to a calorimeter jet, see the left side of Figure \ref{fig:delta_R}, indicate, that on average around four tracks less are associated to the sub-jets compared to the ungroomed jet. The right side of Figure \ref{fig:delta_R} shows the angular distance $\Delta R$ between the single tracks and the axis of the large-R calorimeter jet. Both distributions are aligned in the lower $\Delta R$ region while the histogram representing the tracks associated to the ungroomed jet shows an enhancement towards larger $\Delta R$. Accordingly, these additional tracks feature an angular separation from the jet axis of more than $0.3$, and are in consequence distributed primarily around the outer regions of the large-R jet. Given the required primary vertex association, it is unlikely that these tracks originate from pile-up. Instead, the origin might be found in final- or initial state radiation. 
\begin{figure}
	\centering
	\includegraphics[width=0.45\textwidth]{sascha_input/plots/track_selection/h_customghost_number.pdf} \hspace{1mm}
	\includegraphics[width=0.45\textwidth]{sascha_input/plots/track_selection/h_customghost_dr.pdf}
\caption{\footnotesize{The number of tracks ghost associated to the large-R jet and to the sub-jets (left) and angular distance of associated tracks to the large-R calorimeter jet axis (right). Signal events were not reweighted at this step.}}\label{fig:delta_R}
\end{figure}


Figure \ref{fig:selection} shows the signal distributions of the C2/D2, and $\tau_{21}$, calculated with both selections of tracks for $W$ boson jets. The large $\Delta R$ to the jet axis of the differing tracks push the substructure variables to higher, more background like values. The broader distributions are a result of the variating nature of these tracks. C2 and D2 are more sensitive to tracks with a large $\Delta R$ to the jet axis, because the angular distance between all pairs and triples of tracks is considered, among tracks on possibly opposite ends of the large-R jet, whereas $\tau_{21}$uses distances to $k_\mathrm{T}$-WTA axes.
\begin{figure}
	\centering
	\includegraphics[width=0.3\textwidth]{sascha_input/plots/track_selection/h_ghost_sj_C2.pdf} 
	\includegraphics[width=0.3\textwidth]{sascha_input/plots/track_selection/h_ghost_sj_D2.pdf}
	\includegraphics[width=0.3\textwidth]{sascha_input/plots/track_selection/h_ghost_sj_nSub21.pdf}
\caption{\footnotesize{Substructure variables (left) C2, (right) D2 and (below) $\tau_{21}$ calculateated with calorimeter clusters as well as tracks associated to sub-jets and to the large-R jet. Signal events were not reweighted at this step.}}\label{fig:selection}
\end{figure}
For comparison, the signal and background distributions for the variables calculated with calorimeter clusters are shown as well. It is possible to anticipate that the performance of variables calculated with tracks and assisted tracks is not worse than cluster base variables.
In contrast to the previously studied jet mass variable, ratios of ECF(N) and $\tau_N$ are rather energy scale independent and are found to not be as sensitive to the missing neutral fraction with un-assisted tracks.
Starting from this observations, the performance of substructure techniques is compared with the following objects as input:
\begin{itemize}
\item Calorimeter clusters, labeled 'calo'.
\item Tracks selected as described in Section \ref{subsec:ObsDef_Proc}, labeled 'tracks'.
\item The same collection of tracks, assisted as defined in Section \ref{subsec:ta_adapt}, labeled 'TAS'.
\end{itemize}

\subsection{Performance with default $\beta$}
\input{sascha_input/results_beta1}

\clearpage
\newpage
\section{Performance of Jet Substructure Observables with (assisted) Tracks and optimised $\beta$}\label{sec:beta_opt}
The observed background rejection of variables calculated with TAS and tracks is at least as high as of calorimeter cluster based variables due to the high angular resolution of tracks. Therefore, studied are the effects of a higher weighting of the angular part of the substructure variables. For completeness, considered as well is a lower weighting. Previous studies of default calorimeter variables for $W$ boson tagging, see e.g. Reference \cite{bib:w_tagging}, found $\beta=1$ to maximize the separation power of calorimeter variables.

A scan over the values $\beta= 0.5, 1, 1.7, 2, 3$ is performed in order to identify the best variables for the specific scenarios of tagging $W$ boson, Higgs boson or top quark jets. The background rejections, achieved at the $50\,\%$ working point after mass cut and tagging are summarized in tables. The corresponding ROCs can be found in Appendix \ref{chp:best_ROC}. Pseudo-experiments were used to propagate the uncertainties on the signal and background distributions due to the finite size of the MC samples to the background rejections. 

\subsubsection{Optimisation for $W$ boson jets}
The results of the optimisation for $W$ boson jets are shown in Table \ref{table:w_scan}. As expected, tracks and TAS perform visibly worse with a low angular weighting. For higher values of $\beta$, tracks and TAS gain in separation power, verifying the significance of the angular part for track based variables. Nevertheless, the separation is observed to degrade for angular weightings too high compared to the $p_{\mathrm{T}}$ part, here $\beta=3$.

A $\beta$ of around 2 maximizes the separation power of tracks and TAS. The advantages of $\beta=2$ compared to $\beta=1$ are found at higher $p_{\mathrm{T}}$ values, minor losses are visible in the lowest energy regions. A slightly lower value of $\beta=1.7$ was able to retain the great background rejection of a large angular weighting at high $p_{\mathrm{T}}$ while still performing well at lower energies. Variables calculated with clusters are not as sensitive to a variation of the angular weighting.

C2 is found to be relatively insensitive to the track assisting, whereas separation with $\tau_{21}$ and D2 (for lower energies) is visibly poorer with tracks compared to TAS.  Starting around $1200\;$GeV, tracks perform comparably and for even higher energies equally well as TAS due to the rising amount of large-R jets with only one sub-jet. Comparing variables independently for the different inputs, $\tau_{21}$ performs worse than C2 and D2. However, e.g. $\tau_{21,\: \text{TAS}}$ can outperform $\text{C2}_{\text{calo}}$ and $\text{D2}^{\text{calo}}$.

The variables achieving the highest background rejections for $W$ boson tagging are $\text{D2}_{\text{TAS}}^{(\beta=1.7)}$ and $\text{C2}_{\text{TAS}}^{(\beta=1.7)}$, depending on the energy. D2 cut values were shown to be more $p_{\mathrm{T}}$ robust, C2 could also be used with tracks instead of TAS, which saves the assistance. For TAS variables, uncertainties on the sub-jets, whose derivation is being worked at, need to be propagated while tracks feature already well-known uncertainties.

\begin{sidewaystable}[]
\centering
\resizebox{22cm}{!}{%
\begin{tabular}{llllllllllllllll}
 \multicolumn{1}{l||}{\textbf{Calorimeter}} &  &  & C2 &  & \multicolumn{1}{l||}{} &  &  & D2 &  & \multicolumn{1}{l||}{} &  &  & $\tau_{21}$ &  & \multicolumn{1}{l|}{} \\ \hline
\multicolumn{1}{l||}{$p_{\mathrm{T}} \, \text{[GeV]}$} &  \multicolumn{1}{l|}{$\beta=0.5$} & \multicolumn{1}{l|}{1} & \multicolumn{1}{l|}{1.7} & \multicolumn{1}{l|}{2} & \multicolumn{1}{l||}{3} & \multicolumn{1}{l|}{$\beta=0.5$} & \multicolumn{1}{l|}{1} & \multicolumn{1}{l|}{1.7} & \multicolumn{1}{l|}{2} & \multicolumn{1}{l||}{3} & \multicolumn{1}{l|}{$\beta=0.5$} & \multicolumn{1}{l|}{1} & \multicolumn{1}{l|}{1.7} & \multicolumn{1}{l|}{2} & \multicolumn{1}{l|}{3} \\ \hline \hline
\multicolumn{1}{l||}{250 - 500} & 	\multicolumn{1}{l|}{29.7(1.5)} & 	 \multicolumn{1}{l|}{31.7(1.9)} & 		\multicolumn{1}{l|}{31.4(1.6)} & 	 \multicolumn{1}{l|}{30.7(1.9)} &     \multicolumn{1}{l||}{28.5(1.4)} & 	\multicolumn{1}{l|}{27.2(2.0)}& 	\multicolumn{1}{l|}{35.0(2.0)} & 		\multicolumn{1}{l|}{33.0(1.8)} & 		\multicolumn{1}{l|}{31.3(1.7)} & 	 \multicolumn{1}{l||}{25.7(1.2)} & 	   \multicolumn{1}{l|}{33.1(1.8)} & \multicolumn{1}{l|}{27.6(1.3)} & \multicolumn{1}{l|}{26.2(1.4)} & \multicolumn{1}{l|}{25.1(1.2)} & \multicolumn{1}{l|}{22.4(0.8)} \\
\multicolumn{1}{l||}{500 - 800} & 	\multicolumn{1}{l|}{44.2(1.8)} & 	 \multicolumn{1}{l|}{50.1(2.0)} & 		\multicolumn{1}{l|}{49.6(1.9)} & 	 \multicolumn{1}{l|}{48.6(1.8)} & 	  \multicolumn{1}{l||}{42.6(1.9)} &  	\multicolumn{1}{l|}{40.3(2.2)} & 	\multicolumn{1}{l|}{55.3(2.6)} & 		\multicolumn{1}{l|}{56.3(2.4)} & 		\multicolumn{1}{l|}{52.5(2.1)} & 	 \multicolumn{1}{l||}{39.3(1.3)} & 	   \multicolumn{1}{l|}{49.4(2.0)} & \multicolumn{1}{l|}{41.1(1.4)} & \multicolumn{1}{l|}{43.3(1.7)} & \multicolumn{1}{l|}{41.3(1.6)} & \multicolumn{1}{l|}{36.1(1.2)} \\
\multicolumn{1}{l||}{800 - 1200} & 	\multicolumn{1}{l|}{32.0(1.5)} & 	 \multicolumn{1}{l|}{37.5(1.7)} & 		\multicolumn{1}{l|}{35.4(1.5)} & 	 \multicolumn{1}{l|}{33.4(1.5)} & 	  \multicolumn{1}{l||}{26.8(0.9)} &  	\multicolumn{1}{l|}{34.0(2.1)} & 	\multicolumn{1}{l|}{41.1(2.0)} & 		\multicolumn{1}{l|}{38.5(1.6)} & 		\multicolumn{1}{l|}{34.9(1.3)} & 	 \multicolumn{1}{l||}{25.4(0.7)} & 	   \multicolumn{1}{l|}{30.5(1.2)} & \multicolumn{1}{l|}{30.9(1.2)} & \multicolumn{1}{l|}{33.8(1.4)} & \multicolumn{1}{l|}{32.5(1.3)} & \multicolumn{1}{l|}{28.1(0.9)} \\
\multicolumn{1}{l||}{1200 - 1600} & \multicolumn{1}{l|}{30.1(1.3)} & 	 \multicolumn{1}{l|}{34.4(1.8)} & 		\multicolumn{1}{l|}{29.4(1.3)} & 	 \multicolumn{1}{l|}{26.8(1.0)} & 	  \multicolumn{1}{l||}{20.7(0.8)} &  	\multicolumn{1}{l|}{34.1(1.8)} & 	\multicolumn{1}{l|}{38.1(1.9)} & 		\multicolumn{1}{l|}{31.4(1.4)} & 		\multicolumn{1}{l|}{27.6(1.2)} & 	 \multicolumn{1}{l||}{19.3(0.5)} & 	   \multicolumn{1}{l|}{23.1(0.9)} & \multicolumn{1}{l|}{27.3(1.)} & \multicolumn{1}{l|}{31.1(1.2)} & \multicolumn{1}{l|}{29.9(1.3)} & \multicolumn{1}{l|}{24.8(0.9)} \\
\multicolumn{1}{l||}{1600 - 2000} & \multicolumn{1}{l|}{20.9(1.3)} & 	 \multicolumn{1}{l|}{22.4(1.5)} & 		\multicolumn{1}{l|}{18.2(1.2)} & 	 \multicolumn{1}{l|}{16.5(0.9)} & 	  \multicolumn{1}{l||}{12.9(0.6)} &  	\multicolumn{1}{l|}{26.4(1.7)} & 	\multicolumn{1}{l|}{25.4(1.3)} & 		\multicolumn{1}{l|}{19.3(1.1)} & 		\multicolumn{1}{l|}{16.9(0.9)} & 	 \multicolumn{1}{l||}{11.9(0.5)} & 	   \multicolumn{1}{l|}{16.4(1.0)} & \multicolumn{1}{l|}{19.1(1.1)} & \multicolumn{1}{l|}{21.1(1.1)} & \multicolumn{1}{l|}{19.9(1.0)} & \multicolumn{1}{l|}{16.0(0.9)} \\
\multicolumn{1}{l||}{$>2000$} & 	\multicolumn{1}{l|}{16.9(1.4)} & 	 \multicolumn{1}{l|}{18.7(1.4)} & 		\multicolumn{1}{l|}{14.1(0.9)} & 	 \multicolumn{1}{l|}{12.6(0.8)} & 	  \multicolumn{1}{l||}{9.9(0.7)} & 		\multicolumn{1}{l|}{23.3(1.9)} & 	\multicolumn{1}{l|}{21.9(1.7)} & 		\multicolumn{1}{l|}{15.7(1.1)} & 		\multicolumn{1}{l|}{13.5(0.9)} & 	 \multicolumn{1}{l||}{9.2(0.4)} & 	   \multicolumn{1}{l|}{12.3(1.1)} & \multicolumn{1}{l|}{15.5(1.1)} & \multicolumn{1}{l|}{17.2(1.2)} & \multicolumn{1}{l|}{15.7(1.1)} & \multicolumn{1}{l|}{11.9(0.8)} \\ \hline
 &  &  &  &  &  &  &  &  &  &  &  &  &  &  &  \\
 \multicolumn{1}{l||}{\textbf{TAS}} &  &  & C2 &  & \multicolumn{1}{l||}{} &  &  & D2 &  & \multicolumn{1}{l||}{} &  &  & $\tau_{21}$ &  & \multicolumn{1}{l|}{} \\ \hline
\multicolumn{1}{l||}{$p_{\mathrm{T}} \, \text{[GeV]}$}   &  \multicolumn{1}{l|}{  $\beta=0.5$} & \multicolumn{1}{l|}{ 1} & \multicolumn{1}{l|}{1.7} &   \multicolumn{1}{l|}{ 2} &  \multicolumn{1}{l||}{ 3} & \multicolumn{1}{l|}{ $\beta=0.5$} &  \multicolumn{1}{l|}{ 1} & 	\multicolumn{1}{l|}{ 1.7} & 	\multicolumn{1}{l|}{ 2} & \multicolumn{1}{l||}{ 3} & \multicolumn{1}{l|}{  $\beta=0.5$} & \multicolumn{1}{l|}{ 1} & \multicolumn{1}{l|}{ 1.7} &  \multicolumn{1}{l|}{ 2} & \multicolumn{1}{l|}{ 3} \\ \hline \hline
\multicolumn{1}{l||}{250 - 500} & 	\multicolumn{1}{l|}{29.4(1.9)} & \multicolumn{1}{l|}{30.1(1.9)} & \multicolumn{1}{l|}{28.9(1.5)} & 				     	\multicolumn{1}{l|}{28.5(1.3)} & 						\multicolumn{1}{l||}{27.7(1.3)} & \multicolumn{1}{l|}{28.6(2.0)} & \multicolumn{1}{l|}{37.7(2.1)} & 		\multicolumn{1}{l|}{35.4(2.3)} & 					\multicolumn{1}{l|}{33.4(2.0)} & 	\multicolumn{1}{l||}{29.4(1.2)} & 					\multicolumn{1}{l|}{36.2(2.2)} & 	\multicolumn{1}{l|}{31.5(1.6)} & \multicolumn{1}{l|}{26.8(1.3)} & \multicolumn{1}{l|}{25.4(1.4)} & \multicolumn{1}{l|}{24.0(1.0)} \\
\multicolumn{1}{l||}{500 - 800} & 	\multicolumn{1}{l|}{48.2(2.0)} & \multicolumn{1}{l|}{55.5(2.7)} & \multicolumn{1}{l|}{58.6(2.6)} & 				     	\multicolumn{1}{l|}{59.1(2.7)} & 						\multicolumn{1}{l||}{56.8(2.0)} & \multicolumn{1}{l|}{42.8(2.3)} & \multicolumn{1}{l|}{67.2(3.1)} & 						\multicolumn{1}{l|}{67.6(3.2)} & 	\multicolumn{1}{l|}{63.7(3.0)} & 	\multicolumn{1}{l||}{52.6(2.3)} & 					\multicolumn{1}{l|}{55.7(2.6)} & 	\multicolumn{1}{l|}{51.9(2.1)} & \multicolumn{1}{l|}{45.5(2.0)} & \multicolumn{1}{l|}{44.0(1.9)} & \multicolumn{1}{l|}{41.3(1.5)} \\
\multicolumn{1}{l||}{800 - 1200} & 	\multicolumn{1}{l|}{31.0(1.2)} & \multicolumn{1}{l|}{44.6(1.9)} & \multicolumn{1}{l|}{54.6(2.8)} & 				     	\multicolumn{1}{l|}{55.2(2.8)} & 		\multicolumn{1}{l||}{53.0(3.2)} & \multicolumn{1}{l|}{26.1(1.3)} & \multicolumn{1}{l|}{47.6(2.3)} & 						\multicolumn{1}{l|}{54.9(2.4)} & 					\multicolumn{1}{l|}{52.6(2.8)} & 	\multicolumn{1}{l||}{43.1(1.5)} & 					\multicolumn{1}{l|}{36.4(1.8)} & 	\multicolumn{1}{l|}{37.3(1.7)} & \multicolumn{1}{l|}{36.2(1.8)} & \multicolumn{1}{l|}{36.2(1.6)} & \multicolumn{1}{l|}{35.5(1.6)} \\
\multicolumn{1}{l||}{1200 - 1600} & \multicolumn{1}{l|}{20.9(0.7)} & \multicolumn{1}{l|}{39.1(1.9)} & \multicolumn{1}{l|}{53.8(2.6)} & 				     	\multicolumn{1}{l|}{55.1(3.0)} & 		\multicolumn{1}{l||}{50.1(1.6)} & \multicolumn{1}{l|}{22.7(1.4)} & \multicolumn{1}{l|}{42.1(2.4)} & 						\multicolumn{1}{l|}{50.8(1.8)} & 					\multicolumn{1}{l|}{49.6(2.3)} & 	\multicolumn{1}{l||}{41.1(1.2)} & 					\multicolumn{1}{l|}{27.9(1.3)} & 	\multicolumn{1}{l|}{31.4(1.5)} & \multicolumn{1}{l|}{33.4(1.6)} & \multicolumn{1}{l|}{34.0(2.0)} & \multicolumn{1}{l|}{33.0(1.8)} \\
\multicolumn{1}{l||}{1600 - 2000} & \multicolumn{1}{l|}{16.7(0.7)} & \multicolumn{1}{l|}{36.9(2.9)} & \multicolumn{1}{l|}{50.9(4.3)} &    \multicolumn{1}{l|}{50.3(4.4)} & 						\multicolumn{1}{l||}{42.2(2.4)} & \multicolumn{1}{l|}{18.7(1.7)} & \multicolumn{1}{l|}{32.7(3.3)} & 						\multicolumn{1}{l|}{37.8(2.0)} & 					\multicolumn{1}{l|}{36.1(2.4)} & 	\multicolumn{1}{l||}{28.7(1.2)} & 					\multicolumn{1}{l|}{20.5(1.2)} & 	\multicolumn{1}{l|}{24.8(1.6)} & \multicolumn{1}{l|}{26.1(2.0)} & \multicolumn{1}{l|}{26.5(2.0)} & \multicolumn{1}{l|}{25.4(2.0)} \\
\multicolumn{1}{l||}{$>2000$} & 	\multicolumn{1}{l|}{11.6(0.6)} & \multicolumn{1}{l|}{31.2(3.2)} & \multicolumn{1}{l|}{46.1(4.7)} &    \multicolumn{1}{l|}{45.5(5.2)} & 						\multicolumn{1}{l||}{35.5(3.8)} & \multicolumn{1}{l|}{17.8(2.0)} & \multicolumn{1}{l|}{33.0(4.0)} & 						\multicolumn{1}{l|}{36.3(2.0)} & 					\multicolumn{1}{l|}{34.0(2.5)} & 	\multicolumn{1}{l||}{27.4(1.3)} & 					\multicolumn{1}{l|}{16.4(1.3)} & 	\multicolumn{1}{l|}{22.3(2.0)} & \multicolumn{1}{l|}{24.2(2.2)} & \multicolumn{1}{l|}{24.4(2.5)} & \multicolumn{1}{l|}{21.8(2.4)} \\ \hline
 &  &  &  &  &  &  &  &  &  &  &  &  &  &  &  \\
 \multicolumn{1}{l||}{\textbf{Tracks}} &  &  & C2 &  & \multicolumn{1}{l||}{} &  &  & D2 &  & \multicolumn{1}{l||}{} &  &  & $\tau_{21}$ &  & \multicolumn{1}{l|}{} \\ \hline
\multicolumn{1}{l||}{$p_{\mathrm{T}} \, \text{[GeV]}$}  & \multicolumn{1}{l|}{$\beta=0.5$} & \multicolumn{1}{l|}{1} & \multicolumn{1}{l|}{1.7} & \multicolumn{1}{l|}{2} & \multicolumn{1}{l||}{3} & \multicolumn{1}{l|}{$\beta=0.5$} & \multicolumn{1}{l|}{1} & \multicolumn{1}{l|}{1.7} & \multicolumn{1}{l|}{2} & \multicolumn{1}{l||}{3} & \multicolumn{1}{l|}{$\beta=0.5$} & \multicolumn{1}{l|}{1} & \multicolumn{1}{l|}{1.7} & \multicolumn{1}{l|}{2} & \multicolumn{1}{l|}{3} \\ \hline \hline
\multicolumn{1}{l||}{250 - 500} & 	\multicolumn{1}{l|}{27.1(1.2)} & \multicolumn{1}{l|}{28.1(1.5)} & \multicolumn{1}{l|}{28.7(1.9)} 						& \multicolumn{1}{l|}{28.7(1.9)} 					& \multicolumn{1}{l||}{28.2(1.7)} & \multicolumn{1}{l|}{21.6(1.2)} & \multicolumn{1}{l|}{28.9(2.0)} & \multicolumn{1}{l|}{29.5(1.8)} 		& \multicolumn{1}{l|}{29.1(1.6)} & \multicolumn{1}{l||}{28.1(1.3)} & \multicolumn{1}{l|}{28.7(1.8)} & \multicolumn{1}{l|}{28.0(1.7)} & \multicolumn{1}{l|}{25.6(1.3)} & \multicolumn{1}{l|}{25.1(1.3)} & \multicolumn{1}{l|}{24.2(0.9)} \\
\multicolumn{1}{l||}{500 - 800} & 	\multicolumn{1}{l|}{46.5(1.9)} & \multicolumn{1}{l|}{52.9(2.4)} & \multicolumn{1}{l|}{57.7(2.6)} 						& \multicolumn{1}{l|}{58.1(2.7)} 	& \multicolumn{1}{l||}{55.8(2.5)} & \multicolumn{1}{l|}{30.1(1.8)} & \multicolumn{1}{l|}{46.8(2.4)} & \multicolumn{1}{l|}{53.4(2.2)} 						& \multicolumn{1}{l|}{52.1(2.3)} & \multicolumn{1}{l||}{46.6(1.7)} & \multicolumn{1}{l|}{46.1(2.3)} & \multicolumn{1}{l|}{44.9(1.8)} & \multicolumn{1}{l|}{41.7(2.1)} & \multicolumn{1}{l|}{40.6(1.8)} & \multicolumn{1}{l|}{39.2(1.5)} \\
\multicolumn{1}{l||}{800 - 1200} & 	\multicolumn{1}{l|}{30.3(1.1)} & \multicolumn{1}{l|}{44.5(2.2)} & \multicolumn{1}{l|}{54.8(2.8)} 						& \multicolumn{1}{l|}{56.4(3.0)} 	& \multicolumn{1}{l||}{53.7(3.6)} & \multicolumn{1}{l|}{24.5(1.5)} & \multicolumn{1}{l|}{42.3(2.3)} & \multicolumn{1}{l|}{48.6(2.5)} 						& \multicolumn{1}{l|}{47.5(1.2)} & \multicolumn{1}{l||}{42.4(1.2)} & \multicolumn{1}{l|}{34.5(1.6)} & \multicolumn{1}{l|}{36.2(1.8)} & \multicolumn{1}{l|}{36.0(1.8)} & \multicolumn{1}{l|}{36.2(1.8)} & \multicolumn{1}{l|}{35.7(1.5)} \\
\multicolumn{1}{l||}{1200 - 1600} & \multicolumn{1}{l|}{20.7(0.6)} & \multicolumn{1}{l|}{39.0(1.9)} & \multicolumn{1}{l|}{54.2(2.7)} 						& \multicolumn{1}{l|}{55.5(3.3)} 	& \multicolumn{1}{l||}{50.9(1.7)} & \multicolumn{1}{l|}{22.7(1.3)} & \multicolumn{1}{l|}{41.0(2.2)} & \multicolumn{1}{l|}{50.0(1.6)} 						& \multicolumn{1}{l|}{47.6(2.2)} & \multicolumn{1}{l||}{41.4(1.2)} & \multicolumn{1}{l|}{27.7(1.2)} & \multicolumn{1}{l|}{31.3(1.4)} & \multicolumn{1}{l|}{33.3(1.6)} & \multicolumn{1}{l|}{33.9(1.7)} & \multicolumn{1}{l|}{33.2(1.8)} \\
\multicolumn{1}{l||}{1600 - 2000} & \multicolumn{1}{l|}{16.6(0.7)} & \multicolumn{1}{l|}{36.7(2.3)} & \multicolumn{1}{l|}{51.7(5.2)} 		& \multicolumn{1}{l|}{51.6(4.0)} 					& \multicolumn{1}{l||}{43.1(2.3)} & \multicolumn{1}{l|}{18.5(1.7)} & \multicolumn{1}{l|}{32.1(3.0)} & \multicolumn{1}{l|}{37.0(1.9)} 						& \multicolumn{1}{l|}{35.9(2.3)} & \multicolumn{1}{l||}{29.3(1.2)} & \multicolumn{1}{l|}{20.5(1.3)} & \multicolumn{1}{l|}{24.6(1.7)} & \multicolumn{1}{l|}{26.2(1.8)} & \multicolumn{1}{l|}{26.7(2.0)} & \multicolumn{1}{l|}{25.9(2.2)} \\
\multicolumn{1}{l||}{$>2000$} & 	\multicolumn{1}{l|}{11.6(0.5)} & \multicolumn{1}{l|}{31.5(3.0)} & \multicolumn{1}{l|}{46.8(5.7)} 		& \multicolumn{1}{l|}{46.0(4.2)} 					& \multicolumn{1}{l||}{36.1(4.3)} & \multicolumn{1}{l|}{17.8(2.2)} & \multicolumn{1}{l|}{33.0(3.3)} & \multicolumn{1}{l|}{35.9(2.1)} 						& \multicolumn{1}{l|}{34.2(2.6)} & \multicolumn{1}{l||}{28.1(1.0)} & \multicolumn{1}{l|}{16.4(1.4)} & \multicolumn{1}{l|}{22.5(1.8)} & \multicolumn{1}{l|}{24.5(2.4)} & \multicolumn{1}{l|}{24.7(2.6)} & \multicolumn{1}{l|}{22.2(2.6)} \\ \hline
\end{tabular}}
\caption{Listing of the QCD background rejection for $W$ boson signal achieved with C2, D2 and $\tau_{21}$ together with different angular weightings $\beta$ and for calorimeter cluster, tracks and TAS.}\label{table:w_scan}
\end{sidewaystable}

Shown in Figure \ref{fig:w_cut} are the cut values for $50\,\%$ and $25\,\%$ signal efficiency for $\text{D2}_{\text{TAS}}^{(\beta=1.7)}$ and $\text{C2}_{\text{TAS}}^{(\beta=1.7)}$. $\text{D2}_{\text{TAS}}^{(\beta=1.7)}$. As for the default cluster variables, the $\text{D2}_{\text{TAS}}^{(\beta=1.7)}$ cut is more $p_{\mathrm{T}}$ robust than the cut on $\text{C2}_{\text{TAS}}^{(\beta=1.7)}$. 
\begin{figure}
\includegraphics[width=0.5\textwidth]{sascha_input/plots/W/cut_value/c2_tas17.pdf} \hspace{1mm}
\includegraphics[width=0.5\textwidth]{sascha_input/plots/W/cut_value/d2_tas17.pdf}	
\caption{\footnotesize{Cut values for $\text{C2}_{\text{TAS}}^{(\beta=1.7)}$ (left) and $\text{D2}_{\text{TAS}}^{(\beta=1.7)}$ (right) to achieve $50\,\%$ and $25\,\%$ $W$ boson efficiency.}}\label{fig:w_cut}
\end{figure}
%The steeper $50\,\%$ working point curves can be explained with the signal distributions, see Figure \ref{fig:w_distris}. The signal distributions rise steeply for low values, which doesn't chance considerably with rising $p_{\mathrm{T}}$. The right-handed tail of the distributions in contrast becomes increasingly narrow for $\text{C2}_{\text{TAS}}^{(\beta=1.7)}$ and broader for $\text{D2}_{\text{TAS}}^{(\beta=1.7)}$. Therefore, the curves for the $50\,\%$ cut feature a larger slope as the $25\,\%$ cut, which already given by the sharp rise. 

Table \ref{table:w_improvement} lists the background rejections for $\text{D2}_{\text{TAS}}^{(\beta=1.7)}$, $\text{C2}_{\text{TAS}}^{(\beta=1.7)}$ and the currently used $\text{D2}^{\beta=1}_{calo}$ along with the corresponding improvements. For lower energies, $\text{D2}_{\text{TAS}}^{(\beta=1.7)}$ is the best choice. For very high boosts of the $W$ boson, $\text{C2}_{\text{TAS}}^{(\beta=1.7)}$ performs superior, especially for $25\,\%$ $\epsilon_{signal}$, where the background rejection with $\text{C2}_{\text{TAS}}^{(\beta=1.7)}$ is around 3.5 times as large as the QCD rejection with $\text{D2}^{(\beta=1)}_{\text{calo}}$.
%\begin{figure}[]
%\includegraphics[width=0.25\textwidth]{sascha_input/plots/W/cut_value/h_assisted_tj_C2_17_bin1.pdf} 
%\includegraphics[width=0.25\textwidth]{sascha_input/plots/W/cut_value/h_assisted_tj_C2_17_bin4.pdf} 
%\includegraphics[width=0.25\textwidth]{sascha_input/plots/W/cut_value/h_assisted_tj_D2_17_bin1.pdf} 
%\includegraphics[width=0.25\textwidth]{sascha_input/plots/W/cut_value/h_assisted_tj_D2_17_bin4.pdf} 
%\caption{\footnotesize{$W$ boson signal and QCD background distributions for C2 TAS at $\beta=1.7$ (above) and D2 TAS at $\beta=1.7$ (below) for the %first (left) and fourth (right) $p_{\mathrm{T}}$ bin.}}\label{fig:w_distris}
%\end{figure}
These enormous improvements at lower $\epsilon_{signal}$ are due to the signal distributions for TAS and tracks rising much steeper than for clusters. The tail to higher, background like values in contrast, is more comparable, leading to an alignment of the background rejection for very large $\epsilon_{signal}$. The improvements due to TAS lie around $50\,\%$ for D2 and up to a $100\,\%$ for C2 in the upper $p_{\mathrm{T}}$ regions and $50\,\%$ $W$ boson efficiency. For the lower working point, they can reach $200\,\%$ for D2 and around $250\,\%$ for C2, again for very large boosts of the $W$ boson.
\begin{table}[]
\centering
\begin{tabular}{llll}
 \multicolumn{1}{l||}{\textbf{50\% $\epsilon_{signal}$}} &                                                & \textbf{W tagging}                                         &                                          \\ \hline
\multicolumn{1}{l||}{$p_{\mathrm{T}} \, \text{[GeV]}$}           & \multicolumn{1}{l|}{$\text{D2}_{\text{calo}}^{(\beta=1)}$} & \multicolumn{1}{l|}{$\text{D2}_{\text{TAS}}^{(\beta=1.7)}$} & \multicolumn{1}{l|}{$\text{C2}_{\text{TAS}}^{(\beta=1.7)}$} \\ \hline \hline
\multicolumn{1}{l||}{250 - 500}                       & \multicolumn{1}{l|}{35.0 $\pm$ 2.0}                      & \multicolumn{1}{l|}{35.4 $\pm$ 2.3 (+1 $\pm$ 9\%)}         & \multicolumn{1}{l|}{28.9 $\pm$ 1.5 (-17 $\pm$ 6\%)}        \\
\multicolumn{1}{l||}{500 - 800}                       & \multicolumn{1}{l|}{55.3 $\pm$ 2.6}                      & \multicolumn{1}{l|}{67.6 $\pm$ 3.2 (+22 $\pm$ 8\%)}        & \multicolumn{1}{l|}{58.6 $\pm$ 2.6 (+6 $\pm$ 7\%)}         \\
\multicolumn{1}{l||}{800 - 1200}                      & \multicolumn{1}{l|}{41.1 $\pm$ 2.0}                      & \multicolumn{1}{l|}{54.9 $\pm$ 2.4 (+34 $\pm$ 9\%)}        & \multicolumn{1}{l|}{54.6 $\pm$ 2.8 (+33 $\pm$ 9\%)}        \\
\multicolumn{1}{l||}{1200 - 1600}                     & \multicolumn{1}{l|}{38.1 $\pm$ 1.9}                      & \multicolumn{1}{l|}{50.8 $\pm$ 1.8 (+33 $\pm$ 8\%)}        & \multicolumn{1}{l|}{53.8 $\pm$ 2.7 (+41 $\pm$ 10\%)}        \\
\multicolumn{1}{l||}{1600 - 2000}                     & \multicolumn{1}{l|}{25.4 $\pm$ 1.3}                      & \multicolumn{1}{l|}{37.8 $\pm$ 2.0 (+49 $\pm$ 11\%)}       & \multicolumn{1}{l|}{50.9 $\pm$ 4.3 (+100 $\pm$ 20\%)}       \\
\multicolumn{1}{l||}{$>2000$}                         & \multicolumn{1}{l|}{21.9 $\pm$ 1.7}                      & \multicolumn{1}{l|}{36.3 $\pm$ 2.0 (+66 $\pm$ 16\%)}       & \multicolumn{1}{l|}{46.1 $\pm$ 4.7 (+111 $\pm$ 27\%)}       \\ \hline
                                                     &                                                &                                          &                                          \\
 \multicolumn{1}{l||}{  \textbf{25\% $\epsilon_{signal}$}} &                                                &  \textbf{W tagging}                                         &                                          \\ \hline
\multicolumn{1}{l||}{$p_{\mathrm{T}} \, \text{[GeV]}$}           & \multicolumn{1}{l|}{$\text{D2}_{\text{calo}}^{(\beta=1)}$} & \multicolumn{1}{l|}{$\text{D2}_{\text{TAS}}^{(\beta=1.7)}$} & \multicolumn{1}{l|}{$\text{C2}_{\text{TAS}}^{(\beta=1.7)}$} \\ \hline \hline
\multicolumn{1}{l||}{250 - 500}                       & \multicolumn{1}{l|}{139.6 $\pm$ 9.8}                     	& \multicolumn{1}{l|}{146.0 $\pm$ 12.4 (+5 $\pm$ 12\%)}         & \multicolumn{1}{l|}{108.2 $\pm$ 7.5 (-22 $\pm$ 8\%)}       \\
\multicolumn{1}{l||}{500 - 800}                       & \multicolumn{1}{l|}{243.7 $\pm$ 13.2}                     	& \multicolumn{1}{l|}{360.1 $\pm$ 21.1 (+48 $\pm$ 12\%)}        & \multicolumn{1}{l|}{298.4 $\pm$ 15.9 (+22 $\pm$ 9\%)}        \\
\multicolumn{1}{l||}{800 -1200}                       & \multicolumn{1}{l|}{181.0 $\pm$ 8.8}                     	& \multicolumn{1}{l|}{308.5 $\pm$ 19.3 (+70 $\pm$ 14\%)}        & \multicolumn{1}{l|}{313.2 $\pm$ 24.4 (+78 $\pm$ 16\%)}       \\
\multicolumn{1}{l||}{1200 - 1600}                     & \multicolumn{1}{l|}{156.9 $\pm$ 8.3}                     	& \multicolumn{1}{l|}{295.4 $\pm$ 17.8 (+88 $\pm$ 15\%)}        & \multicolumn{1}{l|}{354.6 $\pm$ 25.6 (+126 $\pm$ 20\%)}      \\
\multicolumn{1}{l||}{1600 - 2000}                     & \multicolumn{1}{l|}{84.6 $\pm$ 5.7}                      	& \multicolumn{1}{l|}{219.6 $\pm$ 10.9 (+160 $\pm$ 22\%)}       & \multicolumn{1}{l|}{320.5 $\pm$ 31.4 (+279 $\pm$ 45\%)}       \\
\multicolumn{1}{l||}{$>2000$}                         & \multicolumn{1}{l|}{78.9 $\pm$ 7.6}                      	& \multicolumn{1}{l|}{233.5 $\pm$ 14.7 (+196 $\pm$ 34\%)}       & \multicolumn{1}{l|}{288.4 $\pm$ 33.3 (+266 $\pm$ 55\%)}      \\ \hline
\end{tabular}

\caption{Listing of the background rejections after the jet mass cut and tagging at 50\% and 25\% $W$ boson efficiency for the identified best variables $\text{D2}_{\text{TAS}}^{(\beta=1.7)}$ \& $\text{C2}_{\text{TAS}}^{(\beta=1.7)}$ together with the improvements over the standard choice $\text{D2}_{\text{calo}}^{(\beta=1)}$}\label{table:w_improvement}
\end{table}



\subsubsection{Optimisation for Higgs boson jets}
The results of the optimisation for Higgs boson jets are shown in Table \ref{table:higgs_scan}. The study of $\beta=1$ in the Higgs boson case, see section \ref{subsubsec:higgs_beta1}, showed no improvements in the rejection of QCD events due to tracks and TAS as input. As for the $W$ boson, the performance of tracks and TAS diminishes considerably with an angular weighting of $\beta=0.5$.

No improvement of $\tau_{21}$ is observed with tracks or TAS, clusters perform equally well for lower $p_{\mathrm{T}}$ and slightly better at high energies. Again, the QCD rejection achieved with $\tau_{21}$ is exceeded by C2 and D2. The discrimination with clusters profits from a slightly higher angular weighting, although the gain is not as significant as for tracks and TAS. This consistently shows the lower sensitivity to a variation of the angular weight. The small gain is connected to the higher separation of the Higgs decay products compared to the $W$ boson case.

For boosted Higgs tagging, D2 outperforms C2 over the whole studied energy range. Values of $\beta=1.7 \& 2$ yield the highest background rejection for track and TAS based D2. $\text{D2}_{\text{TAS}}^{(\beta=1.7,2}$ and $\text{D2}_{\text{track}}^{(\beta=1.7,2}$ perform superior to $\text{D2}_{\text{calo}}$ at high boosts, due to the low angular separation of constituents, and equally well at lower enegies.

The differences between $\beta=1.7$ and $\beta=2$ are inconclusive with minor advantages at high and slight inferiorities at low $p_{\mathrm{T}}$ for $\beta=2$. Tracks perform slightly worse than TAS for lower energies but similarly better in the two highest studied $p_{\mathrm{T}}$ regions. Chosen for further examination are $\text{D2}_{\text{TAS}^{(\beta=1.7)}}$ and $\text{D2}_{\text{track}^{(\beta=1.7)}}$.

\begin{sidewaystable}[]
\centering
\hspace{-1cm}
\resizebox{22cm}{!}{%
\begin{tabular}{llllllllllllllll}
 \multicolumn{1}{l||}{\textbf{Calorimeter}}    &                                  &                           & C2                        &                           & \multicolumn{1}{l|}{}     &                                  &                           & D2                        &                           & \multicolumn{1}{l|}{}     &                                  &                           & $\tau_{21}$               &                           & \multicolumn{1}{l|}{}     \\ \hline
\multicolumn{1}{l||}{$p_{\mathrm{T}} \, [GeV]$} & \multicolumn{1}{l|}{$\beta=0.5$} & \multicolumn{1}{l|}{1}    & \multicolumn{1}{l|}{1.7}  & \multicolumn{1}{l|}{2}    & \multicolumn{1}{l||}{3}    & \multicolumn{1}{l|}{$\beta=0.5$} & \multicolumn{1}{l|}{1}    & \multicolumn{1}{l|}{1.7}  & \multicolumn{1}{l|}{2}    & \multicolumn{1}{l||}{3}    & \multicolumn{1}{l|}{$\beta=0.5$} & \multicolumn{1}{l|}{1}    & \multicolumn{1}{l|}{1.7}  & \multicolumn{1}{l|}{2}    & \multicolumn{1}{l|}{3}    \\ \hline \hline
\multicolumn{1}{l||}{250 - 500}      & \multicolumn{1}{l|}{4.6(0.1)}         & \multicolumn{1}{l|}{5.0(0.1)} & \multicolumn{1}{l|}{5.2(0.1)}  	    & \multicolumn{1}{l|}{5.3(0.1)}  & \multicolumn{1}{l||}{5.5(0.1)}  		& \multicolumn{1}{l|}{5.7(0.1)}         & \multicolumn{1}{l|}{7.3(0.2)}  & \multicolumn{1}{l|}{8.4(0.2)}  							& \multicolumn{1}{l|}{8.4(0.2)}  	    & \multicolumn{1}{l||}{8.4(0.2)}  & \multicolumn{1}{l|}{7.6(0.2)}         & \multicolumn{1}{l|}{8.0(0.2)}  & \multicolumn{1}{l|}{7.9(0.2)}  & \multicolumn{1}{l|}{7.8(0.2)}  & \multicolumn{1}{l|}{7.5(0.2)}  \\
\multicolumn{1}{l||}{500 - 800}      & \multicolumn{1}{l|}{15.7(0.3)}        & \multicolumn{1}{l|}{16.7(0.4)} & \multicolumn{1}{l|}{17.0(0.4)} 		& \multicolumn{1}{l|}{16.9(0.4)} & \multicolumn{1}{l||}{16.2(0.4)} 		& \multicolumn{1}{l|}{13.6(0.3)}        & \multicolumn{1}{l|}{16.9(0.4)} & \multicolumn{1}{l|}{17.7(0.4)} 		& \multicolumn{1}{l|}{17.2(0.4)} 						& \multicolumn{1}{l||}{15.2(0.3)} & \multicolumn{1}{l|}{16.7(0.4)}        & \multicolumn{1}{l|}{15.4(0.3)} & \multicolumn{1}{l|}{15.2(0.3)} & \multicolumn{1}{l|}{14.8(0.3)} & \multicolumn{1}{l|}{14.0(0.3)} \\
\multicolumn{1}{l||}{800 - 1200}     & \multicolumn{1}{l|}{22.1(0.5)}        & \multicolumn{1}{l|}{23.8(0.5)} & \multicolumn{1}{l|}{25.0(0.6)} 		& \multicolumn{1}{l|}{25.0(0.6)} & \multicolumn{1}{l||}{23.4(0.5)} 		& \multicolumn{1}{l|}{18.4(0.4)}        & \multicolumn{1}{l|}{23.7(0.6)} & \multicolumn{1}{l|}{26.3(0.6)} 							& \multicolumn{1}{l|}{25.6(0.6)} 		& \multicolumn{1}{l||}{22.3(0.5)} & \multicolumn{1}{l|}{22.8(0.5)}        & \multicolumn{1}{l|}{21.9(0.5)} & \multicolumn{1}{l|}{22.6(0.5)} & \multicolumn{1}{l|}{22.1(0.5)} & \multicolumn{1}{l|}{20.9(0.5)} \\
\multicolumn{1}{l||}{1200 - 1600}    & \multicolumn{1}{l|}{24.0(0.6)}        & \multicolumn{1}{l|}{26.0(0.8)} & \multicolumn{1}{l|}{26.4(0.8)} 		& \multicolumn{1}{l|}{25.9(0.7)} & \multicolumn{1}{l||}{23.0(0.6)} 		& \multicolumn{1}{l|}{19.3(0.6)}        & \multicolumn{1}{l|}{24.9(0.7)} & \multicolumn{1}{l|}{27.0(0.8)} 		& \multicolumn{1}{l|}{26.1(0.7)} 						& \multicolumn{1}{l||}{21.9(0.5)} & \multicolumn{1}{l|}{21.3(0.5)}        & \multicolumn{1}{l|}{22.6(0.6)} & \multicolumn{1}{l|}{24.0(0.6)} & \multicolumn{1}{l|}{23.7(0.6)} & \multicolumn{1}{l|}{22.2(0.5)} \\
\multicolumn{1}{l||}{1600 - 2000}    & \multicolumn{1}{l|}{12.1(0.7)}        & \multicolumn{1}{l|}{13.9(0.8)} & \multicolumn{1}{l|}{14.3(0.7)}  	& \multicolumn{1}{l|}{14.0(0.7)} & \multicolumn{1}{l||}{12.3(0.6)} 		& \multicolumn{1}{l|}{11.1(0.7)}        & \multicolumn{1}{l|}{14.1(0.9)} & \multicolumn{1}{l|}{14.9(0.8)} 		& \multicolumn{1}{l|}{14.2(0.6)} 						& \multicolumn{1}{l||}{11.8(0.5)} & \multicolumn{1}{l|}{10.3(0.5)}        & \multicolumn{1}{l|}{11.9(0.5)} & \multicolumn{1}{l|}{13.1(0.6)} & \multicolumn{1}{l|}{13.1(0.7)} & \multicolumn{1}{l|}{12.3(0.7)} \\ \hline
                                    &                                  &                           &                           &                	           &                           &                                  &                           &                           &                           &                           &                                  &                           &                           &                           &                           \\
\multicolumn{1}{l||}{\textbf{TAS}}            &                                  &                           & C2                        &                           & \multicolumn{1}{l|}{}     &                                  &                           & D2                        &                           & \multicolumn{1}{l|}{}     &                                  &                           & $\tau_{21}$               &                           & \multicolumn{1}{l|}{}     \\ \hline
\multicolumn{1}{l||}{$p_{\mathrm{T}} \, [GeV]$} & \multicolumn{1}{l|}{$\beta=0.5$} & \multicolumn{1}{l|}{1}        & \multicolumn{1}{l|}{1.7}    & \multicolumn{1}{l|}{2}    & \multicolumn{1}{l||}{3}    & \multicolumn{1}{l|}{$\beta=0.5$} & \multicolumn{1}{l|}{1}    & \multicolumn{1}{l|}{1.7}  & \multicolumn{1}{l|}{2}    & \multicolumn{1}{l||}{3}    & \multicolumn{1}{l|}{$\beta=0.5$} & \multicolumn{1}{l|}{1}    & \multicolumn{1}{l|}{1.7}  & \multicolumn{1}{l|}{2}    & \multicolumn{1}{l|}{3}    \\ \hline \hline
\multicolumn{1}{l||}{250 - 500}      & \multicolumn{1}{l|}{4.8(0.1)}         & \multicolumn{1}{l|}{5.2(0.1)}  & \multicolumn{1}{l|}{5.5(0.1)}        & \multicolumn{1}{l|}{5.6(0.1)}  				& \multicolumn{1}{l||}{5.8(0.1)}    & \multicolumn{1}{l|}{5.9(0.1)}         	& \multicolumn{1}{l|}{7.6(0.2)}  	& \multicolumn{1}{l|}{8.5(0.2)}  	& \multicolumn{1}{l|}{8.6(0.2)}  		        & \multicolumn{1}{l||}{8.5(0.2)}  							& \multicolumn{1}{l|}{7.6(0.2)}         & \multicolumn{1}{l|}{8.0(0.2)}  &   \multicolumn{1}{l|}{7.7(0.2)}  &   \multicolumn{1}{l|}{7.6(0.2)}  & \multicolumn{1}{l|}{7.4(0.2)}  \\
\multicolumn{1}{l||}{500 - 800}      & \multicolumn{1}{l|}{16.1(0.4)}        & \multicolumn{1}{l|}{17.3(0.4)} & \multicolumn{1}{l|}{17.7(0.4)} 	     & \multicolumn{1}{l|}{17.7(0.4)} 				& \multicolumn{1}{l||}{17.6(0.4)} 	& \multicolumn{1}{l|}{14.0(0.3)}        	& \multicolumn{1}{l|}{18.2(0.4)} 	& \multicolumn{1}{l|}{18.7(0.4)} 		& \multicolumn{1}{l|}{18.3(0.4)} 						& \multicolumn{1}{l||}{16.9(0.4)} 							& \multicolumn{1}{l|}{16.2(0.4)}        & \multicolumn{1}{l|}{16.4(0.4)} & 	 \multicolumn{1}{l|}{15.4(0.4)} & 	\multicolumn{1}{l|}{15.1(0.3)} & \multicolumn{1}{l|}{14.6(0.3)} \\
\multicolumn{1}{l||}{800 - 1200}     & \multicolumn{1}{l|}{20.6(0.5)}        & \multicolumn{1}{l|}{23.5(0.5)} & \multicolumn{1}{l|}{26.2(0.6)} 		 & \multicolumn{1}{l|}{26.9(0.7)} 				& \multicolumn{1}{l||}{27.7(0.6)} 	& \multicolumn{1}{l|}{18.8(0.4)}        	& \multicolumn{1}{l|}{25.6(0.6)} 	& \multicolumn{1}{l|}{28.5(0.7)} 		& \multicolumn{1}{l|}{28.4(0.7)} 						& \multicolumn{1}{l||}{26.8(0.6)} 							& \multicolumn{1}{l|}{21.7(0.5)}        & \multicolumn{1}{l|}{22.4(0.5)} & 	 \multicolumn{1}{l|}{22.1(0.5)} & 	\multicolumn{1}{l|}{22.0(0.5)} & \multicolumn{1}{l|}{21.8(0.5)} \\
\multicolumn{1}{l||}{1200 - 1600}    & \multicolumn{1}{l|}{18.6(0.4)}        & \multicolumn{1}{l|}{22.6(0.6)} & \multicolumn{1}{l|}{27.4(0.7)} 		 & \multicolumn{1}{l|}{28.7(0.8)} 				& \multicolumn{1}{l||}{30.0(0.7)} 	& \multicolumn{1}{l|}{17.9(0.4)}        	& \multicolumn{1}{l|}{24.3(0.7)} 	& \multicolumn{1}{l|}{28.9(0.7)} 						& \multicolumn{1}{l|}{29.3(0.6)} 		& \multicolumn{1}{l||}{28.1(0.7)} 							& \multicolumn{1}{l|}{19.3(0.5)}        & \multicolumn{1}{l|}{20.0(0.5)} & 	 \multicolumn{1}{l|}{20.7(0.5)} & 	\multicolumn{1}{l|}{21.0(0.6)} & \multicolumn{1}{l|}{21.9(0.5)} \\
\multicolumn{1}{l||}{1600 - 2000}    & \multicolumn{1}{l|}{8.0(0.3)}         & \multicolumn{1}{l|}{11.3(0.5)} & \multicolumn{1}{l|}{15.4(0.9)} 		 & \multicolumn{1}{l|}{16.5(1.0)} 				& \multicolumn{1}{l||}{17.8(0.7)} 	& \multicolumn{1}{l|}{10.0(0.5)}        	& \multicolumn{1}{l|}{14.0(0.8)} 	& \multicolumn{1}{l|}{17.7(0.8)} 						& \multicolumn{1}{l|}{18.1(.9)} 		& \multicolumn{1}{l||}{17.9(0.6)} 							& \multicolumn{1}{l|}{9.8(0.4)}         & \multicolumn{1}{l|}{10.6(0.5)} & 	 \multicolumn{1}{l|}{11.4(0.6)} & 	\multicolumn{1}{l|}{11.8(0.6)} & \multicolumn{1}{l|}{12.6(0.6)} \\ \hline
                                    &                                  &                           &                           &                           &                           &                                  &                           &                           &                           &                           &                                  &                           &                           &                           &                           \\
 \multicolumn{1}{l||}{\textbf{Tracks}}         &                                  &                           & C2                        &                           & \multicolumn{1}{l|}{}     &                                  &                           & D2                        &                           & \multicolumn{1}{l|}{}     &                                  &                           & $\tau_{21}$               &                           & \multicolumn{1}{l|}{}     \\ \hline
\multicolumn{1}{l||}{$p_{\mathrm{T}} \, [GeV]$} & \multicolumn{1}{l|}{$\beta=0.5$} & \multicolumn{1}{l|}{1}    & \multicolumn{1}{l|}{1.7}  & \multicolumn{1}{l|}{2}    & \multicolumn{1}{l||}{3}    & \multicolumn{1}{l|}{$\beta=0.5$} & \multicolumn{1}{l|}{1}    & \multicolumn{1}{l|}{1.7}  & \multicolumn{1}{l|}{2}    & \multicolumn{1}{l||}{3}    & \multicolumn{1}{l|}{$\beta=0.5$} & \multicolumn{1}{l|}{1}    & \multicolumn{1}{l|}{1.7}  & \multicolumn{1}{l|}{2}    & \multicolumn{1}{l|}{3}    \\ \hline \hline
\multicolumn{1}{l||}{250 - 500}      & \multicolumn{1}{l|}{4.91(0.1)}       & \multicolumn{1}{l|}{5.2(0.1)}  		& \multicolumn{1}{l|}{5.5(0.1)}  		& \multicolumn{1}{l|}{5.6(0.1)}  								& \multicolumn{1}{l||}{5.9(0.1)}  					    & \multicolumn{1}{l|}{5.8(0.1)}         	& \multicolumn{1}{l|}{7.4(0.2)}  	& \multicolumn{1}{l|}{8.3(0.2)} 				 		& \multicolumn{1}{l|}{8.3(0.2)}  					& \multicolumn{1}{l||}{8.5(0.2)}  	& \multicolumn{1}{l|}{7.4(0.2)}         & \multicolumn{1}{l|}{7.9(0.2)}  	& \multicolumn{1}{l|}{7.8(0.2)}  	& \multicolumn{1}{l|}{7.7(0.2)}  	& \multicolumn{1}{l|}{7.6(0.2)}  \\
\multicolumn{1}{l||}{500 - 800}      & \multicolumn{1}{l|}{15.6(0.3)}        & \multicolumn{1}{l|}{17.2(0.4)} 		& \multicolumn{1}{l|}{17.8(0.4)} 		& \multicolumn{1}{l|}{17.9(0.4)} 				& \multicolumn{1}{l||}{17.7(0.4)} 						& \multicolumn{1}{l|}{13.5(0.3)}        	& \multicolumn{1}{l|}{17.1(0.4)} 	& \multicolumn{1}{l|}{17.9(0.4)}		& \multicolumn{1}{l|}{17.7(0.4)} 					& \multicolumn{1}{l||}{16.8(0.4)} 						& \multicolumn{1}{l|}{15.7(0.3)}        & \multicolumn{1}{l|}{16.1(0.4)} 	& \multicolumn{1}{l|}{15.5(0.3)} 	& \multicolumn{1}{l|}{15.3(0.3)} 	& \multicolumn{1}{l|}{14.8(0.1)} \\
\multicolumn{1}{l||}{800 - 1200}     & \multicolumn{1}{l|}{20.1(0.5)}        & \multicolumn{1}{l|}{24.0(0.5)} 		& \multicolumn{1}{l|}{26.9(0.6)} 		& \multicolumn{1}{l|}{27.7(0.7)} 								& \multicolumn{1}{l||}{28.4(0.6)} 	& \multicolumn{1}{l|}{18.8(0.4)}        	& \multicolumn{1}{l|}{25.3(0.6)} 	& \multicolumn{1}{l|}{28.0(0.7)} 						& \multicolumn{1}{l|}{28.0(0.7)} 					& \multicolumn{1}{l||}{26.9(0.6)} 						& \multicolumn{1}{l|}{22.0(0.5)}        & \multicolumn{1}{l|}{22.7(0.5)} 	& \multicolumn{1}{l|}{22.5(0.5)} 	& \multicolumn{1}{l|}{22.4(0.5)} 	& \multicolumn{1}{l|}{22.4(0.3)} \\
\multicolumn{1}{l||}{1200 - 1600}    & \multicolumn{1}{l|}{18.5(0.5)}        & \multicolumn{1}{l|}{23.8(0.6)} 		& \multicolumn{1}{l|}{28.8(0.8)} 		& \multicolumn{1}{l|}{30.0(0.8)} 								& \multicolumn{1}{l||}{31.1(0.7)} 						& \multicolumn{1}{l|}{19.4(0.5)}        	& \multicolumn{1}{l|}{26.3(0.7)} 	& \multicolumn{1}{l|}{30.0(0.8)} 						& \multicolumn{1}{l|}{30.3(0.8)} 	& \multicolumn{1}{l||}{29.2(0.7)} 						& \multicolumn{1}{l|}{20.8(0.5)}        & \multicolumn{1}{l|}{21.4(0.5)} 	& \multicolumn{1}{l|}{21.9(0.6)} 	& \multicolumn{1}{l|}{22.3(0.6)} 	& \multicolumn{1}{l|}{23.0(0.5)} \\
\multicolumn{1}{l||}{1600 - 2000}    & \multicolumn{1}{l|}{8.0(0.3)}         & \multicolumn{1}{l|}{11.7(0.5)} 		& \multicolumn{1}{l|}{16.1(0.9)} 		& \multicolumn{1}{l|}{17.1(0.9)} 								& \multicolumn{1}{l||}{18.3(0.9)} 						& \multicolumn{1}{l|}{11.0(0.7)}        	& \multicolumn{1}{l|}{15.5(0.7)} 	& \multicolumn{1}{l|}{18.5(0.8)} 						& \multicolumn{1}{l|}{18.7(0.8)} 	& \multicolumn{1}{l||}{18.4(0.6)} 						& \multicolumn{1}{l|}{10.4(0.5)}        & \multicolumn{1}{l|}{11.1(0.5)} 	& \multicolumn{1}{l|}{12.0(0.6)} 	& \multicolumn{1}{l|}{12.4(0.7)} 	& \multicolumn{1}{l|}{13.2(0.6)} \\ \hline
\end{tabular}}
\caption{Listing of the QCD background rejection for Higgs signal achieved with C2, D2 and $\tau_{21}$ together with different angular weightings $\beta$ and for calorimeter cluster, tracks and TAS.}\label{table:higgs_scan}
\end{sidewaystable}

Shown in Figure \ref{fig:higgs_cut} are the cut values for $50\,\%$ and $25\,\%$ signal efficiency for $\text{D2}_{\text{TAS}}^{(\beta=1.7)}$ and $\text{D2}_{\text{track}}^{(\beta=1.7)}$. The cut value shows a slight upward trend for rising $p_{\mathrm{T}}$. Moreover, cut values for the first bin are higher as for the second, in contrast to the overall upward trend of D2. This is the result of the low boost in the lowest $p_{\mathrm{T}}$ region resulting in a left shoulder of the mass distributions representing large-R jets containing only part of the Higgs boson decay. These jets feature one-prong structure and result in background-like D2 values. The TAS D2 cut is marginally higher than the corresponding track D2 cut since the assisted tracks have a higher $p_{\mathrm{T}}$ and the D2 cut features a rising tendency with $p_{\mathrm{T}}$. 
\begin{figure}
\includegraphics[width=0.5\textwidth]{sascha_input/plots/Higgs/cut_value/d2_tas17.pdf} \hspace{1mm}
\includegraphics[width=0.5\textwidth]{sascha_input/plots/Higgs/cut_value/d2_tracks17.pdf}
\caption{\footnotesize{Cut values for $\text{D2}_{\text{TAS}}^{(\beta=1.7)}$ (left) and $\text{D2}_{\text{track}}^{(\beta=1.7)}$ (right) to achieve 50\% and 25\% Higgs boson efficiency.}}\label{fig:higgs_cut}
\end{figure}

Listed in Table \ref{table:higgs_improvement} are the background rejections for $\text{D2}_{\text{TAS}}^{(\beta=1.7)}$, $\text{D2}_{\text{track}}^{(\beta=1.7)}$, and for the best calorimeter variable, which is $\text{D2}_{\text{calo}}^{(\beta=1)}$, with the corresponding improvements due to the use of TAS respectively tracks instead of clusters. At very high energies, the angle between the $b\bar{b}$ pair is small despite the high Higgs boson mass and the effect of the calorimeter cell size becomes significant. The improvements for D2 calculated with TAS instead of clusters are single-digit percentages for low $p_{\mathrm{T}}$ and up to $20\,\%$ for the highest studied $p_{\mathrm{T}}$ bin at $50\,\%$ Higgs boson efficiency. For the lower working point, they reach around $30\,\%$ of the QCD rejection achieved with cluster based D2.
\begin{table}[]
\centering
\begin{tabular}{llll}
 \multicolumn{1}{l||}{\textbf{50\% $\epsilon_{signal}$}} &                                                &  \textbf{Higgs tagging}                                        &                                          \\ \hline
\multicolumn{1}{l||}{$p_{\mathrm{T}} \, \text{[GeV]}$}           & \multicolumn{1}{l|}{$\text{D2}_{\text{calo}}^{(\beta=1)}$} & \multicolumn{1}{l|}{$\text{D2}_{\text{TAS}}^{(\beta=1.7)}$} & \multicolumn{1}{l|}{$\text{D2}_{\text{track}}^{(\beta=1.7)}$} \\ \hline \hline
\multicolumn{1}{l||}{250 - 500}                       & \multicolumn{1}{l|}{8.4 $\pm$ 0.2}                      & \multicolumn{1}{l|}{8.5 $\pm$ 0.2 (+1 $\pm$ 4\%)}          & \multicolumn{1}{l|}{8.3 $\pm$ 0.2 (-1 $\pm$ 3\%)}        \\
\multicolumn{1}{l||}{500 - 800}                       & \multicolumn{1}{l|}{17.7 $\pm$ 0.4}                      & \multicolumn{1}{l|}{18.7 $\pm$ 0.4 (+6 $\pm$ 3\%)}        & \multicolumn{1}{l|}{17.9 $\pm$ 0.4 (+1 $\pm$ 3\%)}         \\
\multicolumn{1}{l||}{800 - 1200}                      & \multicolumn{1}{l|}{26.3 $\pm$ 0.6}                      & \multicolumn{1}{l|}{28.5 $\pm$ 0.7 (+8 $\pm$ 4\%)}        & \multicolumn{1}{l|}{28.0 $\pm$ 0.7 (+6 $\pm$ 4\%)}        \\
\multicolumn{1}{l||}{1200 - 1600}                     & \multicolumn{1}{l|}{27.0 $\pm$ 0.8}                      & \multicolumn{1}{l|}{28.9 $\pm$ 0.7 (+7 $\pm$ 4\%)}        & \multicolumn{1}{l|}{30.0 $\pm$ 0.8 (+11 $\pm$ 4\%)}        \\
\multicolumn{1}{l||}{1600 - 2000}                     & \multicolumn{1}{l|}{14.9 $\pm$ 0.8}                      & \multicolumn{1}{l|}{17.7 $\pm$ 0.8 (+19 $\pm$ 8\%)}       & \multicolumn{1}{l|}{18.5 $\pm$ 0.8 (+24 $\pm$ 9\%)}       \\ \hline
                                                     &                                                &                                          &                                          \\
 \multicolumn{1}{l||}{\textbf{25\% $\epsilon_{signal}$}} &                                                &      \textbf{Higgs tagging}                                    &                                          \\ \hline
\multicolumn{1}{l||}{$p_{\mathrm{T}} \, \text{[GeV]}$}           & \multicolumn{1}{l|}{$\text{D2}_{\text{calo}}^{(\beta=1)}$} & \multicolumn{1}{l|}{$\text{D2}_{\text{TAS}}^{(\beta=1.7)}$} & \multicolumn{1}{l|}{$\text{D2}_{\text{track}}^{(\beta=1.7)}$} \\ \hline \hline
\multicolumn{1}{l||}{250 - 500}                       & \multicolumn{1}{l|}{25.1 $\pm$ 0.6}                     & \multicolumn{1}{l|}{28.9 $\pm$ 0.7 (+15 $\pm$ 4\%)}        & 	\multicolumn{1}{l|}{30.5 $\pm$ 0.8 (+22 $\pm$ 4\%)}       \\
\multicolumn{1}{l||}{500 - 800}                       & \multicolumn{1}{l|}{54.1 $\pm$ 1.4}                     & \multicolumn{1}{l|}{69.6 $\pm$ 1.9 (+29 $\pm$ 5\%)}       & 	\multicolumn{1}{l|}{64.9 $\pm$ 1.8 (+20 $\pm$ 5\%)}        \\
\multicolumn{1}{l||}{800 -1200}                       & \multicolumn{1}{l|}{90.8 $\pm$ 2.5}                     & \multicolumn{1}{l|}{121.3 $\pm$ 3.4 (+34 $\pm$ 5\%)}       & 	\multicolumn{1}{l|}{117.9 $\pm$ 3.2 (+30 $\pm$ 5\%)}       \\
\multicolumn{1}{l||}{1200 - 1600}                     & \multicolumn{1}{l|}{97.6 $\pm$ 3.1}                     & \multicolumn{1}{l|}{117.7 $\pm$ 3.8 (+21 $\pm$ 5\%)}       & 	\multicolumn{1}{l|}{122.4 $\pm$ 4.2 (+25 $\pm$ 6\%)}      \\
\multicolumn{1}{l||}{1600 - 2000}                     & \multicolumn{1}{l|}{54.6 $\pm$ 3.5}                     & \multicolumn{1}{l|}{74.0 $\pm$ 5.7 (+36 $\pm$ 14\%)}      & 	\multicolumn{1}{l|}{75.0 $\pm$ 5.1 (+37 $\pm$ 13\%)}       \\ \hline
\end{tabular}
\caption{Listing of the background rejections after the jet mass cut and tagging at 50\% and 25\% Higgs signal efficiency for the identified best variables $\text{D2}_{\text{TAS}}^{(\beta=1.7)}$ \& $\text{D2}_{\text{track}}^{(\beta=1.7)}$ together with the improvements over the best variable with clusters which is $\text{D2}_{\text{calo}}^{(\beta=1)}$.}\label{table:higgs_improvement}
\end{table}

\subsubsection{Optimisation for Top quark jets}
The results of the optimisation for Top quark jets are shown in Table \ref{table:top_scan}. Studied was $\tau_{32}$ with values of $\beta \ge 1$,  since the $W$ boson and Higgs boson parts affirmed the expected lower performance of track and TAS based variables with an angular weighting of $\beta \le 1$. The calorimeter $\tau_{32}$ variable profits from a higher angular weighting up to around $\beta=2$, but degrades in performance for $\beta=3$. Since the involved three prong structure of the top quark decay requires a good angular separation of the jet constituents to be resolved, tracks and TAS perform superior to clusters. A higher angular weighting does not improve the separation power of track and TAS variables, $\beta=2$ already diminishes the performance. The best discrimination is achieved with TAS and $\beta=1,1.7$. The marginal differences between both values of $\beta$ depend on the considered $p_{\mathrm{T}}$ region. Track $\tau_{32,\;\text{track}}$ achieves lower separation as $\tau_{32,\;\text{TAS}}$, except for regions with very high boosts, but as well outperforms the cluster variable.

\begin{table}[]
\centering
\begin{tabular}{lllll}
 \multicolumn{1}{l||}{\textbf{Calorimeter}}    &                                & $\tau_{32}$               &                           & \multicolumn{1}{l|}{}     \\ \hline
\multicolumn{1}{l||}{$p_{\mathrm{T}} \, [GeV]$} & \multicolumn{1}{l|}{$\beta=1$} & \multicolumn{1}{l|}{1.7}  & \multicolumn{1}{l|}{2}    & \multicolumn{1}{l|}{3}    \\ \hline \hline
\multicolumn{1}{l||}{250 - 500}      & \multicolumn{1}{l|}{9.7 $\pm$ 0.2}       	& \multicolumn{1}{l|}{9.5 $\pm$ 0.2}  					& \multicolumn{1}{l|}{9.5 $\pm$ 0.4}  					& \multicolumn{1}{l|}{9.4 $\pm$ 0.2}  \\
\multicolumn{1}{l||}{500 - 800}      & \multicolumn{1}{l|}{20.1 $\pm$ 0.5}      					& \multicolumn{1}{l|}{22.2 $\pm$ 0.6} 					& \multicolumn{1}{l|}{22.4 $\pm$ 0.6} & \multicolumn{1}{l|}{22.0 $\pm$ 0.6} \\
\multicolumn{1}{l||}{800 - 1200}     & \multicolumn{1}{l|}{17.3 $\pm$ 0.4}      					& \multicolumn{1}{l|}{20.3 $\pm$ 0.5} 					& \multicolumn{1}{l|}{20.6 $\pm$ 0.5} & \multicolumn{1}{l|}{20.3 $\pm$ 0.5} \\
\multicolumn{1}{l||}{1200 - 1600}    & \multicolumn{1}{l|}{14.3 $\pm$ 0.3}      					& \multicolumn{1}{l|}{16.4 $\pm$ 0.4} 					& \multicolumn{1}{l|}{16.6 $\pm$ 0.5} & \multicolumn{1}{l|}{16.1 $\pm$ 0.5} \\
\multicolumn{1}{l||}{1600 - 2000}    & \multicolumn{1}{l|}{11.7 $\pm$ 0.3}      					& \multicolumn{1}{l|}{13.3 $\pm$ 0.4} & \multicolumn{1}{l|}{13.3 $\pm$ 0.4} & \multicolumn{1}{l|}{12.6 $\pm$ 0.3} \\
\multicolumn{1}{l||}{$>2000$}        & \multicolumn{1}{l|}{9.6 $\pm$ 0.3}       					& \multicolumn{1}{l|}{11.0 $\pm$ 0.4} & \multicolumn{1}{l|}{10.9 $\pm$ 0.4} 					& \multicolumn{1}{l|}{10.1 $\pm$ 0.3} \\ \hline
                                    &                                &                           &                           &                           \\
 \multicolumn{1}{l||}{\textbf{TAS}}            &                                & $\tau_{32}$               &                           & \multicolumn{1}{l|}{}     \\ \hline
\multicolumn{1}{l||}{$p_{\mathrm{T}} \, [GeV]$} & \multicolumn{1}{l|}{$\beta=1$} & \multicolumn{1}{l|}{1.7}  & \multicolumn{1}{l|}{2}    & \multicolumn{1}{l|}{3}    \\ \hline \hline
\multicolumn{1}{l||}{250 - 500}      & \multicolumn{1}{l|}{10.7 $\pm$ 0.2}      	& \multicolumn{1}{l|}{10.1 $\pm$ 0.2} 					& \multicolumn{1}{l|}{9.9 $\pm$ 0.2}  & \multicolumn{1}{l|}{9.6 $\pm$ 0.2}  \\
\multicolumn{1}{l||}{500 - 800}      & \multicolumn{1}{l|}{22.8 $\pm$ 0.6}      	& \multicolumn{1}{l|}{22.8 $\pm$ 0.6} & \multicolumn{1}{l|}{22.5 $\pm$ 0.6} & \multicolumn{1}{l|}{21.6 $\pm$ 0.6} \\
\multicolumn{1}{l||}{800 - 1200}     & \multicolumn{1}{l|}{23.6 $\pm$ 0.6}      					& \multicolumn{1}{l|}{24.1 $\pm$ 0.6} & \multicolumn{1}{l|}{23.6 $\pm$ 0.6} & \multicolumn{1}{l|}{22.2 $\pm$ 0.5} \\
\multicolumn{1}{l||}{1200 - 1600}    & \multicolumn{1}{l|}{22.0 $\pm$ 0.6}      					& \multicolumn{1}{l|}{22.3 $\pm$ 0.6} & \multicolumn{1}{l|}{21.7 $\pm$ 0.6} & \multicolumn{1}{l|}{19.8 $\pm$ 0.6} \\
\multicolumn{1}{l||}{1600 - 2000}    & \multicolumn{1}{l|}{18.9 $\pm$ 0.6}      	& \multicolumn{1}{l|}{18.8 $\pm$ 0.6} 					& \multicolumn{1}{l|}{17.9 $\pm$ 0.5} & \multicolumn{1}{l|}{16.0 $\pm$ 0.5} \\
\multicolumn{1}{l||}{$>2000$}        & \multicolumn{1}{l|}{16.5 $\pm$ 0.7}      	& \multicolumn{1}{l|}{15.7 $\pm$ 0.7} 					& \multicolumn{1}{l|}{15.2 $\pm$ 0.7} & \multicolumn{1}{l|}{13.1 $\pm$ 0.6} \\ \hline
                                    &                                &                           &                           &                           \\
 \multicolumn{1}{l||}{\textbf{Tracks}}         &                                & $\tau_{32}$               &                           & \multicolumn{1}{l|}{}     \\ \hline
\multicolumn{1}{l||}{$p_{\mathrm{T}} \, [GeV]$} & \multicolumn{1}{l|}{$\beta=1$} & \multicolumn{1}{l|}{1.7}  & \multicolumn{1}{l|}{2}    & \multicolumn{1}{l|}{3}    \\ \hline \hline
\multicolumn{1}{l||}{250 - 500}      & \multicolumn{1}{l|}{10.5 $\pm$ 0.2}      	& \multicolumn{1}{l|}{9.8 $\pm$ 0.2}  					& \multicolumn{1}{l|}{9.6 $\pm$ 0.2}  & \multicolumn{1}{l|}{9.4 $\pm$ 0.2}  \\
\multicolumn{1}{l||}{500 - 800}      & \multicolumn{1}{l|}{20.6 $\pm$ 0.5}      					& \multicolumn{1}{l|}{21.3 $\pm$ 0.6} & \multicolumn{1}{l|}{21.1 $\pm$ 0.5} & \multicolumn{1}{l|}{20.3 $\pm$ 0.5} \\
\multicolumn{1}{l||}{800 - 1200}     & \multicolumn{1}{l|}{21.8 $\pm$ 0.6}      					& \multicolumn{1}{l|}{22.9 $\pm$ 0.6} & \multicolumn{1}{l|}{22.6 $\pm$ 0.6} & \multicolumn{1}{l|}{21.4 $\pm$ 0.6} \\
\multicolumn{1}{l||}{1200 - 1600}    & \multicolumn{1}{l|}{21.7 $\pm$ 0.6}      					& \multicolumn{1}{l|}{22.1 $\pm$ 0.6} & \multicolumn{1}{l|}{21.6 $\pm$ 0.6} & \multicolumn{1}{l|}{19.5 $\pm$ 0.6} \\
\multicolumn{1}{l||}{1600 - 2000}    & \multicolumn{1}{l|}{19.3 $\pm$ 0.6}      	& \multicolumn{1}{l|}{19.0 $\pm$ 0.6} 					& \multicolumn{1}{l|}{18.2 $\pm$ 0.6} & \multicolumn{1}{l|}{16.0 $\pm$ 0.5} \\
\multicolumn{1}{l||}{$>2000$}        & \multicolumn{1}{l|}{16.8 $\pm$ 0.7}      	& \multicolumn{1}{l|}{15.8 $\pm$ 0.7} 					& \multicolumn{1}{l|}{15.1 $\pm$ 0.7} & \multicolumn{1}{l|}{13.0 $\pm$ 0.5} \\ \hline
\end{tabular}
\caption{Listing of the QCD background rejection for top signal achieved with $\tau_{32}$ together with different angular weightings $\beta$ and for calorimeter cluster, tracks and TAS.}\label{table:top_scan}
\end{table}

Shown in Figure \ref{fig:top_cut} are the cut values for $50\,\%$ and $25\,\%$ signal efficiency for $\tau_{32,\;\text{TAS}}^{(\beta=1.7)}$ and $\tau_{32,\;\text{track}}^{(\beta=1.7)}$. The crack between the first and second $p_{\mathrm{T}}$ bin is more evident since the top quark with its much higher mass is here very unlikely to be reconstructed into a single large-R jet, resulting in background like signal events. Furthermore, $\tau_{32} \: (\beta=1.7)$ needs to be cut at lower values as $\tau_{32}\: (\beta=1)$ to achieve a certain signal efficiency. This is the result of the higher angular weighting that shifts the overall distributions to lower values, because the angular distance between two constituents inside a (highly) boosted large-R jet is in the majority of cases lower than one. Thus, the angular part of $\tau_{32}$ decreases with $\beta > 1$. The TAS $\tau_{32}$ cut value is observed to be robust against variations of $p_{\mathrm{T}}$, in accordance to the results of the $p_{\mathrm{T}}$ correlation plots, see \ref{fig:correlation_tau21}.
\begin{figure}[htp]
\includegraphics[width=0.5\textwidth]{sascha_input/plots/Top/cut_values/tau32_tas1.pdf} \hspace{1mm}
\includegraphics[width=0.5\textwidth]{sascha_input/plots/Top/cut_values/tau32_tas17.pdf}
\caption{\footnotesize{Cut values for $\tau_{32,\;\text{TAS}}^{(\beta=1)}$ (left) and $\tau_{32,\;\text{TAS}}^{(\beta=1.7)}$ (right) to achieve 50\% and 25\% Top quark efficiency}}\label{fig:top_cut}
\end{figure}

Listed in Table \ref{table:top_improvement} are the background rejections for $\tau_{32,\;\text{TAS}}^{(\beta=1)}$, $\tau_{32,\;\text{TAS}}^{(\beta=1.7)}$ and the best cluster based variable, $\tau_{32,\;\text{calo}}^{(\beta=2)}$. The differences between both values of $\beta$ with TAS are marginal, as well for lower signal efficiencies. Improvements due to the use of TAS instead of clusters are possible for Top quark tagging over the whole studied $p_{\mathrm{T}}$ range. These enhancements are, as expected, rising with the boost of the Top quark and can reach around $50\,\%$ for the $50\,\%$ working point and even $100\,\%$ for $25\,\%$ Top efficiency.
\begin{table}
\centering
\begin{tabular}{llll}
 \multicolumn{1}{l||}{\textbf{50\% $\epsilon_{signal}$}} &                                                & \textbf{Top Tagging}                                          &                                          \\ \hline
\multicolumn{1}{l||}{$p_{\mathrm{T}} \, \text{[GeV]}$}           & \multicolumn{1}{l|}{$\text{D2}_{\text{calo}}^{(\beta=2)}$} & \multicolumn{1}{l|}{$\tau_{32,\;\text{TAS}}^{(\beta=1)}$} & \multicolumn{1}{l|}{$\tau_{32,\;\text{TAS}}^{(\beta=1.7)}$} \\ \hline \hline
\multicolumn{1}{l||}{250 - 500}                       & \multicolumn{1}{l|}{9.5 $\pm$ 0.2}                      &  \multicolumn{1}{l|}{10.7 $\pm$ 0.2 (+13 $\pm$ 3 \%)}        & \multicolumn{1}{l|}{10.1 $\pm$ 0.2 (+6 $\pm$ 3 \%)}        \\
\multicolumn{1}{l||}{500 - 800}                       & \multicolumn{1}{l|}{22.4 $\pm$ 0.6}                      & \multicolumn{1}{l|}{22.8 $\pm$ 0.6 (+2 $\pm$ 4 \%)}         & \multicolumn{1}{l|}{22.8 $\pm$ 0.6 (+2 $\pm$ 4 \%)}         \\
\multicolumn{1}{l||}{800 - 1200}                      & \multicolumn{1}{l|}{20.6 $\pm$ 0.5}                      & \multicolumn{1}{l|}{23.6 $\pm$ 0.6 (+15 $\pm$ 4 \%)}        & \multicolumn{1}{l|}{24.1 $\pm$ 0.6 (+17 $\pm$ 4 \%)}        \\
\multicolumn{1}{l||}{1200 - 1600}                     & \multicolumn{1}{l|}{16.6 $\pm$ 0.4}                      & \multicolumn{1}{l|}{22.0 $\pm$ 0.6 (+33 $\pm$ 5 \%)}        & \multicolumn{1}{l|}{22.3 $\pm$ 0.6 (+34 $\pm$ 5 \%)}        \\
\multicolumn{1}{l||}{1600 - 2000}                     & \multicolumn{1}{l|}{13.3 $\pm$ 0.4}                      & \multicolumn{1}{l|}{18.9 $\pm$ 0.6 (+42 $\pm$ 6 \%)}        & \multicolumn{1}{l|}{18.8 $\pm$ 0.6 (+41 $\pm$ 6 \%)}       \\ 
\multicolumn{1}{l||}{$>2000$}                     	  & \multicolumn{1}{l|}{10.9 $\pm$ 0.4}                      & \multicolumn{1}{l|}{16.5 $\pm$ 0.7 (+51 $\pm$ 8 \%)}        & \multicolumn{1}{l|}{15.7 $\pm$ 0.7 (+44 $\pm$ 8 \%)}       \\ \hline
                                                     &                                                &                                          &                                          \\
 \multicolumn{1}{l||}{\textbf{25\% $\epsilon_{signal}$}} &                                                &  \textbf{Top Tagging}                                        &                                          \\ \hline
\multicolumn{1}{l||}{$p_{\mathrm{T}} \, \text{[GeV]}$}           & \multicolumn{1}{l|}{$\text{D2}_{\text{calo}}^{(\beta=2)}$} & \multicolumn{1}{l|}{$\tau_{32,\;\text{TAS}}^{(\beta=1)}$} & \multicolumn{1}{l|}{$\tau_{32,\;\text{TAS}}^{(\beta=1.7)}$} \\ \hline \hline
\multicolumn{1}{l||}{250 - 500}                       & \multicolumn{1}{l|}{33.7 $\pm$ 1.0}                     & \multicolumn{1}{l|}{37.6 $\pm$ 1.4 (+12 $\pm$ 5 \%)}       & \multicolumn{1}{l|}{36.7 $\pm$ 1.2 (+9 $\pm$ 5 \%)}       \\
\multicolumn{1}{l||}{500 - 800}                       & \multicolumn{1}{l|}{114.7 $\pm$ 3.3}                    & \multicolumn{1}{l|}{138.0 $\pm$ 4.3 (+20 $\pm$ 5 \%)}       & \multicolumn{1}{l|}{139.1 $\pm$ 4.2 (+21 $\pm$ 5 \%}        \\
\multicolumn{1}{l||}{800 -1200}                       & \multicolumn{1}{l|}{97.0 $\pm$ 2.7}                     & \multicolumn{1}{l|}{144.6 $\pm$ 4.9 (+49 $\pm$ 7 \%)}       & \multicolumn{1}{l|}{149.6 $\pm$ 5.2 (+54 $\pm$ 7 \%)}       \\
\multicolumn{1}{l||}{1200 - 1600}                     & \multicolumn{1}{l|}{68.6 $\pm$ 2.1}                     & \multicolumn{1}{l|}{133.2 $\pm$ 4.6 (+94 $\pm$ 9 \%)}       & \multicolumn{1}{l|}{134.7 $\pm$ 5.1 (+96 $\pm$ 10 \%)}      \\
\multicolumn{1}{l||}{1600 - 2000}                     & \multicolumn{1}{l|}{47.5 $\pm$ 1.6}                     & \multicolumn{1}{l|}{100.3 $\pm$ 4.2 (+111 $\pm$ 11 \%)}      & \multicolumn{1}{l|}{99.9 $\pm$ 4.4 (+110 $\pm$ 12 \%)}       \\ 
\multicolumn{1}{l||}{$>2000$}                         & \multicolumn{1}{l|}{36.3 $\pm$ 1.6}                     & \multicolumn{1}{l|}{80.2 $\pm$ 5.0 (+121 $\pm$ 17 \%)}      & \multicolumn{1}{l|}{75.5 $\pm$ 4.9 (+108 $\pm$ 16 \%)}       \\ \hline
\end{tabular}
\caption{Listing of the background rejections after the jet mass cut and tagging at 50\% and 25\% top signal efficiency for the identified best variables $\tau_{32,\;\text{TAS}}^{(\beta=1,1.7)}$ together with the improvements over the best variable with clusters which is $\text{D2}_{\text{calo}}^{(\beta=2)}$.}\label{table:top_improvement}
\end{table}





%----------------------------------

\clearpage
\section{Uncertainties on observables with sub-jet-assisted tracks}
This chapter gives a brief overview of the uncertainties on the track-assisted (sub-jet) mass variable. 
For $\mta$ the uncertainties are smaller than calorimeter-based jet mass variables because of the way it is constructed, $\mta=m^{trk}\times p_T^{calo}/p_T^{trk}$: the ratio $m^{trk}/p_T^{trk}$ causes a cancellation of the tracking uncertainties to a large extent, which are smaller than $\mcal$. The remaining term $p_T^{calo}$ is the additional one where uncertainties on this variable need to be evaluated with special care.

\begin{figure}[!ht]
  \centering
      \includegraphics[width=0.9\textwidth]{jet_part/uncert.png}
  \caption[Comparison of the uncertainties for $\mcal$ and $\mta$]{Comparison of the uncertainties for $\mcal$, on the left, and $\mta$, on the right the rise on the high jet $\pt$ is due to statistics. From the \cite{art35}.}
  \label{fig:uncert}
\end{figure}


For what concerns the $\mtas$, the tracking uncertainties are expected to be identical to the $\mta$, because of the identical use of tracks in both variables, as also discussed in the BOOST Conference Note \cite{art35}.
The only significant difference in this regard of $\mta$ with respect to $\mtas$ is the $p_T^{jet}$ instead of the $p_T^{subjet}$: the uncertainties in the first one are calculated in-situ using $p_T$ balance methods, and they are generally well-behaved; for the second one, the uncertainties are also expected to lay in the same order of magnitude. In Figure \ref{fig:uncert} the comparison of the uncertainties for the $\mcal$ on the left and $\mta$ on the right shows the smaller fractional JMS uncertainties in the use of the track-assisted method.

The path to bring $\mtas$ ready to usage needs of course to evaluate the uncertainties on the  $p_T^{subjet}$ which can be provided with the R-Scan procedure which includes the anti-k$_t$ subjet of radius of 0.2; those were already showed to have a similar performance of the k$_t$ subjet used as standard.

The $\mtas$ moreover is expected to have little to none benefit from the calibration procedure which could be provided from the R-Scan as well, as already shown and discussed in the previous chapter. 





\clearpage
%-------------------------------------------------------------------------------
\section{Conclusions \& Outlook}
\label{sec:conclusions}
%----------------------------------
% \subsection{Jet mass observables}
The \mtas variable was developed for the large-$R$ jet mass; it combines the information of the tracker- and calorimeter-system to achieve an higher precision in the jet mass reconstruction, correcting the missed neutral fraction which is absent in the tracker but not in the calorimeter.
With respect to the $\mta$, it applies this correction at sub-jet by sub-jet level and not at jet by jet level, therefore providing a more accurate reconstruction. 
It was shown in Monte Carlo simulation to be a very good observable confronting quantitatively with the other definitions which are either standard or in preparation, $\mcal$, $\mta$ and $\mcomb$.
% In fact, it behaves better in terms of $\iqr$ and all the other ways to look at the figure of merit, the mass response, for the boosted $W/Z$ and QCD sample; is always better than the $\mta$ and similar to the $\mcal$ for the boosted tops and Higgs.
% Moreover, it is a slightly worse observable than the $\mcomb$, yet being comparable, and avoiding the development of ad-hoc weights.
The optimal configuration of $\mtas$ is shown and confronted with different approaches, in particular in terms of different trimming procedure of the large-$R$ jet to be used as an input.
All the components of the observable have been studied with the use of truth Monte Carlo information without detector effect, in order to evaluate quantitatively its limits and strengths; the track $\pt$ measure degradation was found to be the cause of the variable decreasing performance at higher transverse momenta.

The $\mcombtas$ is the logical extension of the $\mtas$, which improves by construction the results beyond the $\mcal$ and the $\mtas$, combining these two variables on the same way of the $\mcomb$, but taking into account the higher correlation factor which is inherited from the sub-jet usage.
Weights for its construction can be in both cases either derived specifically for the sample considered, or constructed on average with the QCD sample, in this case getting a sub-optimal performance. 
In all the cases studied, it has a better behavior than the $\mcomb$, $\mcal$ and $\mta$.

% \subsubsection{Outlook}
% The outlook of the $\mtas$ and $\mcombtas$ variables follows two main scenarios, concerning the calibration and uncertainties determination which are necessary to get this observables ready to be used. The procedure involved are already fully understood, since the the same was applied or is being applied for the $\mta$ and $\mcomb$.

% For the simple scenario here the procedure that would take place is the direct Monte Carlo calibration of the $\mtas$, aiming at correcting the reconstructed jet mass to the particle-level jet mass by applying the calibration factors derived from QCD multijet events, an analogous procedure to the one described in Section \ref{sec:calib} for the jet energy scale.

% The more complex scenario considers an additional calibration to the subjet with R=0.2, which is already at an advanced stage within ATLAS for anti-k$_t$ reclustering algorithm (it has a slightly worse performance than k$_t$, as presented previously). 

% The uncertainties are expected to be similar to the one which were derived for the $\mta$ and which are compared to the $\mcal$ on Figure \ref{fig:uncert}; the tracking uncertainties are smaller for the track-assisted mass because of the ratio $m^{track}/p_T^{track}$ and will be smaller as well for the track-assisted  sub-jet mass since it uses the same ratio.

% In-situ uncertainties were derived for the $\mta$ with a sample of enriched top-quark; the same technology used here can be applied to the $\mtas$.

% In the more complex scenario, the uncertainties could be derived for the subjet R=0.2 reclustered with anti-k$_{t}$.
%------------------------------------------------------------------------------
% \subsection{Energy Correlation Functions and n-Subjettiness}

The performance of the Energy Correlation Functions and n-Subjettiness observables was studied with tracks and subjet-assisted tracks (TAS) compared to the default calculation with calorimeter clusters. The assisting procedure for these observables is the same single-track scaling method used for $\mtas$. It was shown, that the better angular resolution of tracks can help to reduce the QCD background in boosted $W$, Higgs and Top tagging cases, especially for very high energies. Optimal results can be achieved by tuning the angular weighting to slightly higher values. In our study, a value of $\beta=1.7$ yield the highest background rejections in most of the cases. 

Subjet-assisted tracks performed superior for lower and intermediate boosts. At very high energies, the assisting was shown to loose effect on the studied observables and tracks and TAS equally outperform the cluster based variables in terms of QCD rejection.

For the three cases of $W$, Higgs and Top tagging, the optimal variables were identified and the $p_{\text{T}}$ behavior of the cut-value for two different working points were studied. 

%For the very conclusion, both the variables constructed in the work of this thesis, $\mtas$ and $\mcombtas$, exhibit a better performance of their counterparts, $\mta$ and $\mcomb$, which are now ready to be use or in preparation within the ATLAS collaboration, and share the same advantages -and disadvantages. Further steps are necessary to get this observables to usage: calibration and uncertainties.

%%Sascha: Adding my stuff to this part if OK
For the very conclusion, the variables constructed in the work of this study, $\mtas$ and $\mcombtas$, as well as the TAS and track based C2/D2/$\tau_{21}$/$\tau_{32}$, exhibit a better performance as their counterparts, $\mta$ and $\mcomb$, respectively JSS observables calculated from topo-clusters. The introduced observables are now ready to use or in preparation within the ATLAS collaboration, and share the same advantages -and disadvantages. Further steps are necessary to get this observables to usage: calibration (mass) and uncertainties (mass and further substructure observables).

%-------------------------------------------------------------------------------
\clearpage
\appendix
\part*{Appendix}
\addcontentsline{toc}{part}{Appendix}
%-------------------------------------------------------------------------------

% In a paper, an appendix is used for technical details that would otherwise disturb the flow of the paper.
% Such an appendix should be printed before the Bibliography.

% \chapter{ATLAS Detector: Further Details}
Further details here about the ATLAS detector are given.
\section{Muon subdetectors}
\subsection{RPC}
The RPCs offer fast triggering of the muons, providing track information in 15 to 25 ns. They are utilized in the barrel region ($|\eta|<1.05$) and made out of electrically resistive parallel plates with a 2mm distance filled with a gas mixture, arranged in three layers. The plates are kept at 9800 V potential difference to assure avalanche from the gas ionization caused by charged particles, which is then read out by metallic strips. The spatial resolution of this sub-detector is rather coarse, 10 mm in $\eta,\phi$ plane, which is the price to pay for the fast response.  

\subsection{TGC}
The TGC is again offering fast track information, but they are placed in the end-caps ($1.05<|\eta|<2.4$) and with four layers. The technology adopted here is the Multi Wire Proportional Chamber (MWPC). Typical readout happens in 25 ns.

\subsection{MDT}
The MDTs consist of pressurized drift tubes and, oppositely to the RPC and TGC, provides a high precision muon momentum measurement with a slower response. It is placed in $|\eta|<2.7$ and provides average spatial resolution of 80 $\mu$m, resulting in a total resolution of 35 $\mu$m, at the cost of charge collection time of 700 ns.
Tubes are arranged in three to eight layers within each chamber, enhancing the performance of tracking pattern recognition software.

\subsection{CSC}
The CSCs, like the TGC, are made out of MWPCs, covering the innermost end-cap ($2<|\eta|<2.7$), with resolution of 40 $\mu$m and a time resolution of 7 ns per plane, making them able to accommodate the higher particle flux due to the beam vicinity up to 1000 Hz/cm$^2$, making drift tubes technology infeasible to use in this region.

\section{L1}
The L1 exploits a raw information form the calorimeters and muons system, making use of algorithms to determine the Bunch Crossing Identification (BCID) associated to those raw measurements.
It then uses a custom-made electronics to take a decision in $\sim$ 25 $\mu$s on an event-by-event basis.
The raw information are simplified e.g. geometrically grouping together the calorimeter cells in so called \textit{towers}, while the muon system makes use of dedicated sub-detectors (RPC and TGC) as described above. The trigger information for the calorimeter (L1Calo) and the muon system (L1Muon) are then merged together. After that positive decision is made, the L1 defines one or more Region of Interest (RoI) which contains the measurements from the raw information and transmits those to the HLT.
\section{HLT}
At this step, in Run 1, the L2 trigger matched the inner detector data to the RoI, and made a successive trigger decision based on ID also, having $\sim$ 40 ms for this operation.
The selected events were then passed to the EF, which performed a full-granularity event reconstruction in the RoI, using hence all the sub-detector and not anymore raw data only, including calibrations, alignment corrections etc.
The EF had here 4 s for the final decision.
The computer resources were allocated separately to L2 and EF; in Run 2 instead this two step were reduced to 1 in the HLT, which is now a unique computer farm with merged processing nodes, for simplification and dynamic resources sharing.
If the event is again positive, it is registered on disk (here the final rate is 200 Hz) and will be then used for offline analyses. 
\section{Luminosity Measurement}
The delivered luminosity can be written as a function of the accelerator parameters as:
$$\mathcal{L}=\frac{n_Bf_rn_1n_2}{2\pi\Sigma_x\Sigma_y} $$
where $n_{1,2}$ are the numbers of protons per beam (1,2), $\Sigma_{x,y}$ characterize the horizontal and vertical convoluted beam width (measured through Van der Meer scans) and $f_r$ is the revolution frequency and $n_B$ is the number of bunches traveling at frequency $f_r$. There are also alternative parametrizations, where delivered luminosity is written as a function of the visible total inelastic cross section $\sigma_{vis}$ and the average number of inelastic interactions per bunch crossing $\mu_{vis}$.
A fundamental ingredient of the ATLAS strategy to assess  and  control  the  systematic  uncertainties  affecting  the
absolute luminosity determination is to  compare the measurements  of  several  luminosity  detectors,  most  of  which
use more than one algorithm to assess the luminosity. These
multiple detectors and algorithms are characterized by a significant different acceptance, response to pile-up, and sensitivity to instrumental effects and to beam-induced backgrounds.  


\chapter{Parton Shower and Hadronization Details}
\section{Parton Shower}
To perform a quantative study of the parton shower, one can start from the simple 2 $\to$ 2 process with the further splitting in into quarks and gluons, e.g. $q\to qg$, $g\to q\bar{q}$ and $g\to gg$. Parton shower can originate from initial state radiation (ISR) or final state radiation (FSR) like in Figure \ref{fig:factorization}. For details on this section, see \cite{partonshower}. 
The resulting process can now be depicted as $$2 \to 2 \otimes \textrm{ISR} \otimes \textrm{FSR}$$   
% It is important to note here that already the 2 $\to$ 2 process must be convoluted with the flux of incoming partons $i$ and $j$ in the two incoming protons, $A$ and $B$:
For the simplest 2 $\to$ 2 process, the production cross section is given by:
$$\sigma = \sum_{i,j}\iiint dx_1 dx_2 d\hat t f_i^A(x_1,Q^2)f_j^B(x_2,Q^2)\frac{d\hat{\sigma}_{i,j}}{d\hat{t}} $$
where $i$,$j$ are the incoming partons and $f_{i,j}^{A,B}(x_{1,2},Q^2)$ are the parton distribution functions of partons in the incoming protons, $A$ and $B$ and $Q^2$ being the momentum transfer squared.
The parton distribution function (PDFs) of gluons and sea quarks are strongly peaked at small momentum fractions $x_1 \sim E_i/E_A$, $x_2 \sim E_j/E_B$.  
The first step is to study the particular case of 2 $\to$ 3 e.g. with and additional gluon radiated from a quark in the final state.
Here the cross section can be written in the form \cite{halzenmartin}:
\begin{equation}
\frac{d\sigma_{2\to3}}{\sigma_{2\to2}}=\frac{\alpha_S}{2\pi}\frac{4}{3}\frac{x_1^2+x_2^2}{(1-x_1)(1-x_2)}dx_1dx_2
\end{equation}
\label{eq:sigma}
neglecting the quark masses. Now rewriting the energy fractions $x_j$ as $1-x_2=Q^2/E_{cm}^2$, $x_1\sim z$, $x_3 \sim 1-z$, equation \ref{eq:sigma} looks as follows:
$$d\mathcal{P}=\frac{d\sigma_{2\to3}}{\sigma_{2\to2}}\simeq \frac{\alpha_S}{2\pi} \frac{dQ^2}{Q^2}\frac{4}{3}\frac{1+z^2}{1-z}dz$$
\begin{wrapfigure}{r}{0.5\textwidth}
  \centering
      \includegraphics[width=0.45\textwidth]{/afsuser/fabnap/Documents/MasterArbeit/jet_part/factorization.png}
  \caption{Example of factorization of 2$\to n$ process.}
  \label{fig:factorization}
\end{wrapfigure}

which is collinear singular ($Q^2\sim 1-\textrm{cos}\theta\to 0$ if $\theta \to 0$ ).

To generalize the probability for the process $a\to bc$, which could be then gluon radiation ($q\to qg$) gluon splitting ($g\to gg$) or quark-antiquark splitting ($g\to q\bar{q}$), the Dokshitzer-Gribov-Lipatov-Altarelli-Parisi (DGLAP) equations are used:
$$d\mathcal{P}_{a\to bc}=\frac{\alpha_S}{2\pi} \frac{dQ^2}{Q^2}P_{a\to bc}(z)dz$$
where $P_{a\to bc}$ are fragmentation functions:
$$P_{q\to qg} = \frac{4}{3}\frac{1+z^2}{1-z} $$ 
$$P_{q\to gg} = 3 \frac{(1-z(1-z))^2}{z(1-z)} $$ 
$$P_{q\to q\bar{q}} = \frac{n_f}{2}(z^2+(1-z)^2) $$ 
and with $n_f$ being the number of quark flavors.

To describe now a cascade of successive branchings, like e.g. the one depicted in Figure \ref{fig:cascade}, one has to evolve the DGLAP equation above in smaller and smaller $Q^2$ using the so-called \textit{Sudakov factor} which describes  the probability that the \textit{first} emission happens at time $T$
$$d\mathcal{P}_{first}(T) = d\mathcal{P}_{sth}(T)\exp\left(-\int_0^T\frac{d\mathcal{P}_{sth}(t)}{dt}dt\right)$$
where $\mathcal{P}_{sth}$ is the probability of branching.

Thereby, the DGLAP equations become then
$$d\mathcal{P}_{a\to bc}(T) = \frac{\alpha_S}{2\pi} \frac{dQ^2}{Q^2}P_{a\to bc}(z) dz \exp\left(-\sum_b,c\int_{Q^2}^{Q^2_{max}}\frac{dQ^{'2}}{Q^{'2}}\int \frac{\alpha_S}{2\pi}P_{a\to bc}(z') dz'\right)$$
where the exponential term is called the Sudakov form factor and intuitively represents the probability of not having already radiated a particle with higher momentum transfer.
Sudakov formulation provides by definition the orderign in $Q^2$ (from larger to smaller) or in ``times'' from smaller to larger. By introducing the $Q^2_{max}$ as $Q^2$ of the hard-process one can regulate the collinear singularities.

% The advantage of the Sudakov formulation is that provides the ordering in $Q^2$ (from bigger to smaller values implies from smaller to bigger ``times'' i.e. later emission ) of the shower and that regulates the singularity for the first emission, if confronted with the procedure of using e.g. $\mathcal{P}_{q\to qg}$ multiple times. \\


\begin{wrapfigure}{l}{0.5\textwidth}
  \centering
      \includegraphics[width=0.35\textwidth]{/afsuser/fabnap/Documents/MasterArbeit/jet_part/cascade.png}
  \caption{Example of cascade of successive branchings.}
  \label{fig:cascade}
\end{wrapfigure}

Parton showers form ISR are described in similar way, but with the additional complication of taking into account the parton distribution functions.

Due to the asymptotic freedom, quarks and gluons resulting from the fragmentation described above, behave as quasi-free particles called partons, only at short distances (order of 10$^{-2}$ fm); when these colored objects separate more than the order of 1 fm, the confinement forces become effective, which have the effect of binding the quarks and gluons in hadrons. This process is referred to as \textit{hadronization}, which is a stochastic process involving a large number of particles, also described in the next section. The hadronization proceeds in fact through the formation of jet in high energy processes.

\section{Hadronization} % (fold)

Being non-perturbative, the process of hadronization is described with phenomenological models, the most important ones being: independent jet fragmentation, Lund string model and cluster hadronization.\\

In the first one, which is first one also historically in PETRA and PEP, gluonic flux tubes appear when colored objects separate and can then split to quark-antiquark pairs balancing they energy fraction and forming then primary mesons; the process lasts with the un-hadronized quarks and until the energy decreases to a cut-off. The problem of the model was an overall unsatisfactory implementation of the energy-momentum conservation, but had the advantage of a small number of parameters and simplicity.

In the (Lund) string model, which is similar to the independent fragmentation, $q\bar{q}$ interaction is described as string interaction with $V(r)\sim kr$, with $r$ being the distance, $k=|\frac{dE}{dz}|=|\frac{dp_z}{dz}|=|\frac{dE}{dt}|=|\frac{dp_z}{dt}|$ and neglecting the Coulomb part of the interaction. When tension reaches the critical values, the string breaks, forming new $q\bar{q}$ pair. See Figure \ref{fig:string} for a schematic representation. 

In the cluster model, hadronization mechanism is based on color pre-confinement; in fact gluons split to quark-antiquark pairs and nearby partons in cascade arrange themselves in color-neutral clusters, preferably at small invariant mass, down to the QCD scale, $\mathcal{O}$(200 MeV). With respect to the string model, it has the advantage of having a simple flavor composition with fewer parameters, but a less predictive energy-momentum description with more parameters.
\chapter{Topo-Clustering and Local Calibration Weighting}
\subsubsection{Topo-Clusters}
Topological cell clustering or topo-clustering, is the process where the calorimeter cells are grouped together in order to find energy deposited from the hard-scattering process. The result of topo-clustering is the formation of topo-cluster, in a way that suppresses calorimeter contribution from noise related effects, but still maintaining the activity from the underlying physical process.
The topo-clustering works as follow: first a seed cell is defined, and then other neighboring cells are added to the seed if their energy is above a noise threshold. 
% it starts with a seed cell and iteratively adds to the cluster the neighbor of a cell already in the cluster, provided that the energy in the new cell is above a threshold defined as a function of the expected noise. 
It is efficient at suppressing noise in clusters with large numbers of cells \cite{topocluster}. 

An additional step is taken to further reduce noise contribution, enhance the performance of topo-clusters and ensure that no bias is introduced in data: a local calibration scheme (Local Calibration Weighting, LCW) \cite{lcw} is also applied, based on Monte Carlo.

The calibration weights are determined from simulations of charged and neutral pions according to the cluster topology measured in the calorimeter. The cluster properties
used are the energy density in the cells forming them, the fraction of their energy deposited in the different calorimeter layers, the cluster isolation and its depth in the calorimeter.
\begin{wrapfigure}{R}{0.5\textwidth}
  \centering
\includegraphics[width=0.45\textwidth]{/afsuser/fabnap/Documents/MasterArbeit/jet_part/images.png}
  \caption[Shower development in the accordion calorimeter]{Shower development in the accordion calorimeter, Monte Carlo simulation.}
  \label{fig:accordion}
\end{wrapfigure}
The natural requirement which has to be satisfied at this point is that the calorimeters cells should be then (three-dimensionally) ``grouped'' together, in order to reconstruct the energy of the hard-scattered particle. This is done in ATLAS in two steps: first the collection of calorimeter energy deposit represented as topo-clusters is created; then those objects are used as input for the jet reconstruction algorithm (here we are speaking of topo-clusters for the reconstruction of calorimeter jets; however other input can be tracks for track-jets and truth particles for truth-jets). 
Corrections are applied to the cluster energy to account for the energy deposited in the calorimeter, but outside of clusters and energy deposited in material before and in between the calorimeters. Jets are formed from calibrated clusters by using dedicated reclustering algorithms, described in the body of the thesis.

\chapter{Pile-Up and Underlying Event}
The \textit{pile-up} is a term used to describe the jets coming from another interaction in the same bunch-crossing (\textit{in-time pile-up}), i.e. coming form another interaction at low-$\pt$ which happens together with the hard-scattering, or in another bunch-crossing (\textit{out-of-time pile-up}), before or after the hard-scattering. When sub-detectors are sensitive to several bunch-crossing or their electronics integrate over more than 25 ns, these collisions can affect the signal in the collision of interest.

\begin{wrapfigure}{r}{0.5\textwidth}
  \centering
      \includegraphics[width=0.5\textwidth]{/afsuser/fabnap/Documents/MasterArbeit/jet_part/Jet_ungroomed_mass_pt_500.png}
  \caption[Effect of pile-up contamination]{Effect of pile-up contamination in large-$R$ jets: here shown different PU conditions parametrized by $\langle\mu\rangle$. From \cite{highlumi}.}
  \label{fig:largejetpu}
\end{wrapfigure}


In case of the in-time pile-up, this effect can be directly parametrized by the number of primary vertices $N_{PV}$ reconstructed, which is around 15 for Run 2, and by the average number of interactions per bunch-crossing $\langle \mu \rangle$ (which is a function of the instantaneous luminosity $\mathcal{L}$) by the total inelastic cross section $\sigma_{in}$ and by the average frequency of bunch-crossing at the LHC $N_{bunch}\times f_{LHC}$:

$$\langle \mu \rangle = \frac{\mathcal{L}\times\sigma_{in}}{N_{bunch}\times f_{LHC}} $$

Given the increased luminosity condition for the Run 2, the average number of interactions is around 24, making the soft radiation contamination from PU an issue of increasing seriousness. 
The Underlying Event is a term which describes, in the single proton-proton interaction, all the phenomena, besides the hard-scattering, of several softer parton-parton scatters and the fragmentation of QCD strings which connects colored objects including beam remnant and initial and final state radiation \cite{ue}. Coming from the primary vertex, those soft radiations survive the tracks requirement to come from the PV (which is not the case for pile-up).

As seen in Figure \ref{fig:largejetpu}, where the large-$R$ jet mass is shown with five different PU conditions, the contamination spoils the distribution, giving rise to a misreconstruction which gets worse as it goes to worse environment conditions.\\

Other relatively small sources of contamination can be e.g. cavern background, beam halo events and beam gas events.


\begin{figure}
    \centering
    \begin{subfigure}[b]{0.45\textwidth}
	\centering
        \includegraphics[width=0.78\textwidth]{/afsuser/fabnap/Documents/MasterArbeit/jet_part/grooming/pileup.png}
 
%         \label{fig:gull}
    \end{subfigure}
    \begin{subfigure}[b]{0.45\textwidth}
	\centering
        \includegraphics[width=\textwidth]{/afsuser/fabnap/Documents/MasterArbeit/jet_part/Jet_pileupcorrected_mass_pt_500.jpeg}
   
%         \label{fig:tiger}
    \end{subfigure}
    \caption[Effect of pile-up before and after trimming]{Left: mass reconstructed as a function of the number of primary vertices (parameterizing PU) for different samples; after trimming procedure the mass is pretty much independent of PU for all the samples. Right: mass distributions for different PU conditions: after trimming the reconstruction is not degraded as much as Figure \ref{fig:largejetpu}.} 
    \label{fig:trimmingperformance}
\end{figure}

\chapter{Additional Grooming Techniques}
The standard grooming technique used for the optimization studies of this thesis was described in the body; however there are two common choices which are worth mentioning here: the \textit{split-filtering} and the \textit{pruning}

\section{Split-Filtering}
The split-filtering was developed and optimized using C/A in jet searches of Higgs to \textit{b}-quarks. It is made out of two stages: the \textit{Mass-drop and symmetry} and the \textit{Filtering}.
For the Mass-drop and symmetry these are the steps:
\begin{itemize}
 \item the last step of C/A is undone, obtaining two sub-jets (e.g. the ones that should contain the bottom quark from Higgs decay);
 \item a significant difference between the parent jet mass and the sub-jets $j_i$ is required: $m^{j_i}/m^{jet}<\mu_{frac}$;
 \item the two $\pt$ of the sub-jets are required to be relatively similar (symmetry requirement) by the condition $\frac{min[(p_T^{j_1})^2,(p_T^{j_2})^2]}{(m^{jet})^2}\times \Delta R^2_{j_1,j_2} > y_{cut}$.
\end{itemize}
If the mass-drop and symmetry criteria are not satisfied, the jet is discarded.
The second step is the filtering:
\begin{itemize}
 \item $j_1$ and $j_2$ are reclustered with the C/A algorithm with variable radius parameter $R_{filt}=min[0.3,\Delta R^2_{j_1,j_2}/2]$, where $R_{filt}<\Delta R^2_{j_1,j_2}$;
 \item the jet is then filtered: all the constituents outside the three hardest sub-jets are discarded, in order to allow the emission of an additional radiation from the two-body decay;
 \item the split-filtered jet is composed of those three sub-jets only.
\end{itemize}
This method shows powerful sensitivity to highly collimated decays.

\section{Pruning}
The pruning algorithm is widely used in CMS; it works in a similar manner as the trimming, removing constituents (not sub-jets) with a relatively small $\pt$, but additionally applying a veto on wide angle radiation. The procedure runs as follows:
\begin{itemize}
 \item the C/A or k$_t$ reclustering algorithms is run on the constituents of the parent jet;
 \item at each reclustering step, transverse momentum \textit{or} an angular requirement has to be satisfied: being $j_1$ and $j_2$ the constituents, either $p_T^{j_1}/p_T^{j_1+j_2}>z_{cut}$ or $\Delta R^2_{j_1,j_2}< R_{cut}\times (2m^{jet}/p_T^{jet})$;
 \item $j_1$ and $j_2$ are merged only if one or both of those criteria are met, else $j_2$ is discarded and the algorithm continues.
\end{itemize}

\chapter{Limitation of the $\mtas$}

In this Appendix, additional results on the limitation of the $\mtas$ based on MC studies without detector interactions are presented. In particular, the truth study presented for boosted $W/Z$ decay in the thesis is here extended for boosted top quark decays.

As seen on Figure \ref{fig:breakdown3}, the breakdown of the $\mtas$ shows that, in particular for the high transverse momenta regimes, the tracks are subjected to fast degradation which makes their combination with the calorimeter mass not anymore an advantage. 

This is a limitation which was expected and understood from the detector performance point of view, and here shows the impossibility, with the variables which are presented here $\mta$ and $\mtas$ to reach a competitive standpoint with the $\mcal$ in the extreme kinematic regime for the top quark decay.

In black, in fact, the performance of the $\mtas$ variable using tracks with detector effect and sub-jets without those effects, shows this intrinsic limit which takes place already at 1.5 TeV.

\begin{figure}[!ht]
  \centering
      \includegraphics[width=0.7\textwidth]{/afsuser/fabnap/Documents/MasterArbeit/jet_part/appendixA/71graphcftr_h_JetRatio_mJ12CALOIQRoM4TruthsTops.pdf}
  \caption[Breakdown of the $\mtas$ ]{Breakdown of the $\mtas$ in its component using truth-level information for boosted top quarks decays.}
  \label{fig:breakdown3}
\end{figure}




\section{Trimming}\label{sec:trimming}

The trimming algorithm is the most important in ATLAS and the one mainly used in this note. It takes advantage of the fact that contamination from soft radiation has a much lower $\pt$ with respect to the hard-scattering component. Therefore uses a transverse momentum ratio to distinguish among those. The algorithm works on a two-dimensional parameter space: $R_{sub}$ and $f_{cut}$.
The steps are as follows:
\begin{itemize}
 \item k$_t$ algorithm (but of course other choices are also possible) is used to create sub-jets with a smaller radius $R_{sub}$, aiming at separating the soft radiation from the hard one in different sub-jets. Typical choices are 0.2 and 0.3 (0.2 is used as standard);
 \item for each sub-jet, the ratio $f_{cut}$ between its $\pt$ and the parent jet $p_T^{jet}$ is calculated: if then this ratio is below a certain value, the sub-jet is removed. Standard choice is $\frac{p_{T}}{p_{T}^{jet}} > f_{cut}=0.05$;
 \item the sub-jets which survived this procedure are the only one which compose the trimmed jet.
\end{itemize}

The trimming procedure is also explained in Figure \ref{fig:trimming}, an example of performance in simulation with standard parameters is shown in Appendix (Figure \ref{fig:trimmingperformance}).

\begin{figure}[!ht]
  \centering
      \includegraphics[width=0.9\textwidth]{jet_part/grooming/Figura_4.png}
  \caption{Schematic of the trimming algorithm.}
  \label{fig:trimming}
\end{figure}


\begin{figure}[!ht]
  \centering
      \includegraphics[width=0.5\textwidth]{jet_part/Jet_ungroomed_mass_pt_500.png}
  \caption[Effect of pile-up contamination]{Effect of pile-up contamination in large-$R$ jets: here shown different PU conditions parametrized by $\langle\mu\rangle$. From \cite{highlumi}.}
  \label{fig:largejetpu}
\end{figure}


\begin{figure}
    \centering
    \begin{subfigure}[b]{0.45\textwidth}
  \centering
        \includegraphics[width=0.78\textwidth]{jet_part/grooming/pileup.png}
 
%         \label{fig:gull}
    \end{subfigure}
    \begin{subfigure}[b]{0.45\textwidth}
  \centering
        \includegraphics[width=\textwidth]{jet_part/Jet_pileupcorrected_mass_pt_500.jpeg}
   
%         \label{fig:tiger}
    \end{subfigure}
    \caption[Effect of pile-up before and after trimming]{Left: mass reconstructed as a function of the number of primary vertices (parameterizing PU) for different samples; after trimming procedure the mass is pretty much independent of PU for all the samples. Right: mass distributions for different PU conditions: after trimming the reconstruction is not degraded as much as Figure \ref{fig:largejetpu}.} 
    \label{fig:trimmingperformance}
\end{figure}


\section{Tracks details}\label{sec:tracking}
The requirements applied on the track used in the work presented in this note are given here:
\begin{itemize}
 \item $p_T^{track} > 400$ MeV;
 \item $|\eta|<$ 2.5;
 \item Maximum 7 hits in the Pixel and STC sub-detectors;
 \item Maximum 1 Pixel hole;
 \item Maximum 2 silicon holes;
 \item Less than 3 shared modules;
 \item Maximum 2 mm of displacement along beam axis ($z_0$) from the primary vertex;
 \item Maximum 2.5 mm of distance in x-y plane from the primary vertex and point of closest approach ($d_0$).
\end{itemize}

\section{Alternative Performance Figure of Merit (FoM)}

A concrete, quantitative feature has to be defined in order to understand which observable is ``better'', in the sense that we would prefer one or the other according to this criterion. This is often referred to as \textit{Figure of Merit} or simply FoM.

There are few ways to look at the FoM: one can e.g. na\"ively think about the mean of the mass distribution, since closer values of the mean to the e.g. $W$ or $Z$ mass (if we are speaking about $W/Z$ decays) indicate a more correct mass reconstruction. However, this does not take into account the width of this distribution, as a large width spoils the reconstruction in terms of percentage of jets misreconstructed. Moreover, the mean is not as important since it can be rescaled to the desired value in a calibration procedure.

\subsection{Gaussian Fit}

The important feature to keep in mind, in fact, is the underlying physics which brings us to calculate the mass of a jet. In figure \ref{fig:search} this is made clear: if the width of the invariant mass distribution of the jet is smaller (highlighted), it allows a bigger background rejection, here shown as the QCD dijet, for the same signal efficiency, by means of a simple mass requirement.

\begin{figure}[!ht]
  \centering
      \includegraphics[width=0.55\textwidth]{jet_part/search.png}
  \caption[QCD and $W'$ mass distribution]{Mass distributions: in red the QCD dijet background rescaled, in green the $W/Z$ from the $W'$ sample. Highlighted the width of the 68\% of the $W/Z$ distribution.}
  \label{fig:search}
\end{figure}

The width $\sigma$ of the distribution, which can be obtained from a fit to the Gaussian core, is already a valid FoM, which has an underlying physical feature. Moreover, in order to be independent from the mean of the distribution, the width can be divided by the mean itself.
This was in fact the FoM which was used at the beginning of the work for this thesis, since it provided a simple and fast solution. However, special care must be used both in the procedure of fitting Gaussian cores of responses, since they are asymmetric, and to how the tails are treated.

\begin{figure}[!ht]
  \centering
      \includegraphics[width=0.9\textwidth]{jet_part/wrongsigma.png}
  \caption[Gaussian fit for QCD multijet]{Mass Response distributions for the QCD multijet for various $\pt$ ranges: on the right the failure of the Gaussian fit shows the limitation of this approach to serve as the Figure of Merit. On the plot the fit parameters and transverse momentum ranges.}
  \label{fig:wrongsigma32}
\end{figure}

The situation is depicted e.g. in Figure \ref{fig:wrongsigma32}, where a mass response is shown for calorimeter mass for QCD multijet: here the presence of a right-handed tail which enhances going from low to high transverse momenta makes the Gaussian fit clearly not the tool which provides the stability needed. The ideal tool should consider the presence of at least tails outside the Gaussian core and should converge to the intuition of the standard deviation for a perfect Gaussian distribution.
The closest tool to this idea was found to be the \textit{InterQuantile Range}, which is presented in the body of this note.


\section{Kinematic distributions of signal and background samples}\label{sec:kinematic}
Kinematic distributions for all the samples, $\pt$ $\eta$ and $\phi$ are shown.

\begin{figure}
    \centering
    \begin{subfigure}[b]{0.45\textwidth}
        \includegraphics[width=\textwidth]{jet_part/appendixA/tops/1cfrt_h_FatJet_pt.pdf}
        \caption{$\pt$ distribution}
        \label{fig:gull}
    \end{subfigure}
    \begin{subfigure}[b]{0.45\textwidth}
        \includegraphics[width=\textwidth]{jet_part/appendixA/tops/1cfrt_h_FatJet_eta.pdf}
        \caption{$\eta$ distribution}
        \label{fig:tiger}
    \end{subfigure}
    \begin{subfigure}[b]{0.45\textwidth}
        \includegraphics[width=\textwidth]{jet_part/appendixA/tops/1cfrt_h_FatJet_psi.pdf}
        \caption{$\phi$ distribution}
        \label{fig:mouse}
    \end{subfigure}
    \caption{Boosted tops kinematic distribution.}\label{fig:animals}
\end{figure}


\begin{figure}
    \centering
    \begin{subfigure}[b]{0.45\textwidth}
        \includegraphics[width=\textwidth]{jet_part/appendixA/higgs/1cfrt_h_FatJet_pt.pdf}
        \caption{$\pt$ distribution}
        \label{fig:gull}
    \end{subfigure}
    \begin{subfigure}[b]{0.45\textwidth}
        \includegraphics[width=\textwidth]{jet_part/appendixA/higgs/1cfrt_h_FatJet_eta.pdf}
        \caption{$\eta$ distribution}
        \label{fig:tiger}
    \end{subfigure}
    \begin{subfigure}[b]{0.45\textwidth}
        \includegraphics[width=\textwidth]{jet_part/appendixA/higgs/1cfrt_h_FatJet_psi.pdf}
        \caption{$\phi$ distribution}
        \label{fig:mouse}
    \end{subfigure}
    \caption{RS-Graviton kinematic distribution.}\label{fig:animals}
\end{figure}

\begin{figure}
    \centering
   \includegraphics[width=\textwidth]{jet_part/appendixA/qcd/qcdkinematics.png}
   
    \caption{QCD dijet transverse momentum and mass distributions.}
    \label{fig:qcdkinematics}
\end{figure}

\begin{figure}[!ht]
  \centering
      \includegraphics[width=0.5\textwidth]{jet_part/ptjcobian.png}
  \caption{The $\pt$ distribution of a 3 TeV resonance from the hadronically decaying $W$ or $Z$, in logarithmic plot. As can be seen, the jacobian peak is around $\pt\simeq m_{W'}/2\simeq 1.5$ TeV.}
  \label{fig:ptjacobian}
\end{figure}


\begin{figure}
    \centering
    \begin{subfigure}[b]{0.45\textwidth}
	\centering
        \includegraphics[width=\textwidth]{jet_part/mta/13cfrt_h_SubJet_aftersel_ptJ03TAmult.pdf}
 
%         \label{fig:gull}
    \end{subfigure}
    \begin{subfigure}[b]{0.45\textwidth}
	\centering
        \includegraphics[width=\textwidth]{jet_part/mta/13cfrt_h_SubJet_aftersel_ptJ10TAmult.pdf}
   
%         \label{fig:tiger}
    \end{subfigure}
    \caption{Sub-jet and Track-jet (jets created having tracks as input) multiplicity, for selected bins of transverse momentum.} 
    \label{fig:multi}
\end{figure}


% \begin{figure}
%     \centering
%    \includegraphics[width=\textwidth]{jet_part/mtas/71graphcftr_h_JetRatio_mJ12CALOIQRoM_Wprime_Allalgos.pdf}
   
%     \caption{Performance of $\mtas$ with different reclustering algorithm for the sub-jets: anti-k$_t$, k$_t$ and C/A. Boosted $W/Z$ sample.}
%     \label{fig:allalgow}
% \end{figure}

% \begin{figure}
%     \centering
%    \includegraphics[width=\textwidth]{jet_part/mtas/71graphcftr_h_JetRatio_mJ12CALOTopsCalib.pdf}
   
%     \caption{Performance of $\mtas$ with different reclustering algorithm for the sub-jets: anti-k$_t$, k$_t$ and C/A. Boosted top sample.}
%     \label{fig:allalgotop}
% \end{figure}

% \begin{figure}
%     \centering
%    \includegraphics[width=\textwidth]{jet_part/mtas/71graphcftr_h_JetRatio_mJ12CALOIQRoMHiggsNOCalib.pdf}
   
%     \caption{Performance of $\mtas$ with different reclustering algorithm for the sub-jets: anti-k$_t$, k$_t$ and C/A. Boosted higgs sample.}
%     \label{fig:allalgohiggs}
% \end{figure}


\begin{figure}
    \centering
    \begin{subfigure}[b]{0.5\textwidth}
	\centering
        \includegraphics[width=\textwidth]{jet_part/mta/mTA_WZ/1cfrt_h_fabsca_tascal_2.pdf}
%         \rlap{\crule[white]{1cm}{1cm}}
	\put(-34,03){\crule[white]{0.15cm}{0.19cm}}
%         \label{fig:mcomba1}
    \end{subfigure}
    \begin{subfigure}[b]{0.5\textwidth}
	\centering
        \includegraphics[width=\textwidth]{jet_part/mta/mTA_Tops/1cfrt_h_fabsca_tascal_2.pdf}
	\put(-34,03){\crule[white]{0.15cm}{0.19cm}}
%         \label{fig:mcomba2}
    \end{subfigure}
    
    \begin{subfigure}[b]{0.5\textwidth}
	\centering
        \includegraphics[width=\textwidth]{jet_part/mta/mTA_higgs/1cfrt_h_fabsca_tascal_2.pdf}
	\put(-34,03){\crule[white]{0.15cm}{0.19cm}}
%         \label{fig:mcomba3}
    \end{subfigure}
    
    \caption{Calorimeter based jet mass response vs the track-assised mass response for the three signal samples. Correlation coefficient is indicated on the top right.} 
    \label{fig:mcomba1}
\end{figure}


\begin{figure}
    \centering
   \includegraphics[width=1.2\textwidth]{jet_part/mcomb/mcomba2.png}
    \caption{Calorimeter based jet mass response vs the track-assised sub-jet mass response for the three signal samples. Correlation coefficient is indicated on the top right and highlighted.
    On the left, top, the higgs sample, bottom, the $W/Z$; on the right the top-quark sample.}
    \label{fig:mcomba2}
\end{figure}


\begin{figure}[!ht]
  \centering
      \includegraphics[width=\textwidth]{jet_part/calib/perfectcalib2.png}
  \caption{Poor's man calibration effect on mean of transverse momentum's response of the sub-jet, before, left, and after, right, the procedure.}
  \label{fig:calibA}
\end{figure}

\begin{figure}[!ht]
  \centering
      \includegraphics[width=\textwidth]{jet_part/calib/perfectcalib3.png}
  \caption{Poor's man calibration effect on the mean of the mass response of the large-R jet, before, left, and after, right, the procedure.}
  \label{fig:calibA2}
\end{figure}


\begin{figure}[!ht]
  \centering
      \includegraphics[width=\textwidth]{jet_part/calib/71graphcftr_h_JetRatio_mJ12CALOIQRoMcalib_degradW.pdf}
  \caption{Comparison of the $\mtas$ and the same variable using truth-level information for the tracks.}
  \label{fig:breakdown1}
\end{figure}

% \addcontentsline{toc}{section}{$\mtas$ distributions, boosted $W/Z$}
% \addcontentsline{toc}{section}{Additional grooming techniques}
\input{appendixB.tex}
% \addcontentsline{toc}{section}{$\mtas$ distributions, boosted tops}
\input{appendixB1.tex}
% \addcontentsline{toc}{section}{$\mtas$ distributions, boosted Higgs}
\clearpage
\onecolumn
\vspace*{\fill}
\section{$\mtas$  distributions, boosted higgs}
\vfill
\clearpage
\twocolumn

\begin{figure}
 
\includegraphics[width=0.4\textwidth]{appendixB/Higgs_mTAS_Calib_20:14:09-03-11-2016/12cfrt_h_FatJet_ptJ01m.pdf}
\caption{$\mtas$ and $\mcal$ for $p_{T}^{J}$ bin (indicated on plot) }
 
\end{figure}
 
\begin{figure}
 
\includegraphics[width=0.4\textwidth]{appendixB/Higgs_mTAS_Calib_20:14:09-03-11-2016/12cfrt_h_FatJet_ptJ02m.pdf}
\caption{$\mtas$ and $\mcal$ for $p_{T}^{J}$ bin (indicated on plot) }
 
\end{figure}
 
\begin{figure}
 
\includegraphics[width=0.4\textwidth]{appendixB/Higgs_mTAS_Calib_20:14:09-03-11-2016/12cfrt_h_FatJet_ptJ03m.pdf}
\caption{$\mtas$ and $\mcal$ for $p_{T}^{J}$ bin (indicated on plot) }
 
\end{figure}
 
\begin{figure}
 
\includegraphics[width=0.4\textwidth]{appendixB/Higgs_mTAS_Calib_20:14:09-03-11-2016/12cfrt_h_FatJet_ptJ04m.pdf}
\caption{$\mtas$ and $\mcal$ for $p_{T}^{J}$ bin (indicated on plot) }
 
\end{figure}
 
\begin{figure}
 
\includegraphics[width=0.4\textwidth]{appendixB/Higgs_mTAS_Calib_20:14:09-03-11-2016/12cfrt_h_FatJet_ptJ05m.pdf}
\caption{$\mtas$ and $\mcal$ for $p_{T}^{J}$ bin (indicated on plot) }
 
\end{figure}
 
\begin{figure}
 
\includegraphics[width=0.4\textwidth]{appendixB/Higgs_mTAS_Calib_20:14:09-03-11-2016/12cfrt_h_FatJet_ptJ06m.pdf}
\caption{$\mtas$ and $\mcal$ for $p_{T}^{J}$ bin (indicated on plot) }
 
\end{figure}
 %
\begin{figure}
 
\includegraphics[width=0.4\textwidth]{appendixB/Higgs_mTAS_Calib_20:14:09-03-11-2016/12cfrt_h_FatJet_ptJ07m.pdf}
\caption{$\mtas$ and $\mcal$ for $p_{T}^{J}$ bin (indicated on plot) }
 
\end{figure}
 
\begin{figure}
 
\includegraphics[width=0.4\textwidth]{appendixB/Higgs_mTAS_Calib_20:14:09-03-11-2016/12cfrt_h_FatJet_ptJ08m.pdf}
\caption{$\mtas$ and $\mcal$ for $p_{T}^{J}$ bin (indicated on plot) }
 
\end{figure}
 
\begin{figure}
 
\includegraphics[width=0.4\textwidth]{appendixB/Higgs_mTAS_Calib_20:14:09-03-11-2016/12cfrt_h_FatJet_ptJ09m.pdf}
\caption{$\mtas$ and $\mcal$ for $p_{T}^{J}$ bin (indicated on plot) }
 
\end{figure}
 
\begin{figure}
 
\includegraphics[width=0.4\textwidth]{appendixB/Higgs_mTAS_Calib_20:14:09-03-11-2016/12cfrt_h_FatJet_ptJ10m.pdf}
\caption{$\mtas$ and $\mcal$ for $p_{T}^{J}$ bin (indicated on plot) }
 
\end{figure}
 
\begin{figure}
 
\includegraphics[width=0.4\textwidth]{appendixB/Higgs_mTAS_Calib_20:14:09-03-11-2016/12cfrt_h_FatJet_ptJ11m.pdf}
\caption{$\mtas$ and $\mcal$ for $p_{T}^{J}$ bin (indicated on plot) }
 
\end{figure}
 
\begin{figure}
 
\includegraphics[width=0.4\textwidth]{appendixB/Higgs_mTAS_Calib_20:14:09-03-11-2016/12cfrt_h_FatJet_ptJ12m.pdf}
\caption{$\mtas$ and $\mcal$ for $p_{T}^{J}$ bin (indicated on plot) }
 
\end{figure}
\clearpage % % 
\begin{figure}
 
\includegraphics[width=0.4\textwidth]{appendixB/Higgs_mTAS_Calib_20:14:09-03-11-2016/13cfrt_h_SubJet_aftersel_ptJ01TAmult.pdf}
\caption{Track-jet R=0.2 and sub-jet multiplicity for $p_{T}^{J}$ bin (indicated on plot) }
 
\end{figure}
 
\begin{figure}
 
\includegraphics[width=0.4\textwidth]{appendixB/Higgs_mTAS_Calib_20:14:09-03-11-2016/13cfrt_h_SubJet_aftersel_ptJ02TAmult.pdf}
\caption{Track-jet R=0.2 and sub-jet multiplicity for $p_{T}^{J}$ bin (indicated on plot) }
 
\end{figure}
 
\begin{figure}
 
\includegraphics[width=0.4\textwidth]{appendixB/Higgs_mTAS_Calib_20:14:09-03-11-2016/13cfrt_h_SubJet_aftersel_ptJ03TAmult.pdf}
\caption{Track-jet R=0.2 and sub-jet multiplicity for $p_{T}^{J}$ bin (indicated on plot) }
 
\end{figure}
 
\begin{figure}
 
\includegraphics[width=0.4\textwidth]{appendixB/Higgs_mTAS_Calib_20:14:09-03-11-2016/13cfrt_h_SubJet_aftersel_ptJ04TAmult.pdf}
\caption{Track-jet R=0.2 and sub-jet multiplicity for $p_{T}^{J}$ bin (indicated on plot) }
 
\end{figure}
 
\begin{figure}
 
\includegraphics[width=0.4\textwidth]{appendixB/Higgs_mTAS_Calib_20:14:09-03-11-2016/13cfrt_h_SubJet_aftersel_ptJ05TAmult.pdf}
\caption{Track-jet R=0.2 and sub-jet multiplicity for $p_{T}^{J}$ bin (indicated on plot) }
 
\end{figure}
 
\begin{figure}
 
\includegraphics[width=0.4\textwidth]{appendixB/Higgs_mTAS_Calib_20:14:09-03-11-2016/13cfrt_h_SubJet_aftersel_ptJ06TAmult.pdf}
\caption{Track-jet R=0.2 and sub-jet multiplicity for $p_{T}^{J}$ bin (indicated on plot) }
 
\end{figure}
 %
\begin{figure}
 
\includegraphics[width=0.4\textwidth]{appendixB/Higgs_mTAS_Calib_20:14:09-03-11-2016/13cfrt_h_SubJet_aftersel_ptJ07TAmult.pdf}
\caption{Track-jet R=0.2 and sub-jet multiplicity for $p_{T}^{J}$ bin (indicated on plot) }
 
\end{figure}
 
\begin{figure}
 
\includegraphics[width=0.4\textwidth]{appendixB/Higgs_mTAS_Calib_20:14:09-03-11-2016/13cfrt_h_SubJet_aftersel_ptJ08TAmult.pdf}
\caption{Track-jet R=0.2 and sub-jet multiplicity for $p_{T}^{J}$ bin (indicated on plot) }
 
\end{figure}

\begin{figure}

\includegraphics[width=0.4\textwidth]{appendixB/Higgs_mTAS_Calib_20:14:09-03-11-2016/13cfrt_h_SubJet_aftersel_ptJ09TAmult.pdf}
\caption{Track-jet R=0.2 and sub-jet multiplicity for $p_{T}^{J}$ bin (indicated on plot) }
 
\end{figure}
 
\begin{figure}

\includegraphics[width=0.4\textwidth]{appendixB/Higgs_mTAS_Calib_20:14:09-03-11-2016/13cfrt_h_SubJet_aftersel_ptJ10TAmult.pdf}
\caption{Track-jet R=0.2 and sub-jet multiplicity for $p_{T}^{J}$ bin (indicated on plot) }

\end{figure}

\begin{figure}

\includegraphics[width=0.4\textwidth]{appendixB/Higgs_mTAS_Calib_20:14:09-03-11-2016/13cfrt_h_SubJet_aftersel_ptJ11TAmult.pdf}
\caption{Track-jet R=0.2 and sub-jet multiplicity for $p_{T}^{J}$ bin (indicated on plot) }

\end{figure}

\begin{figure}

\includegraphics[width=0.4\textwidth]{appendixB/Higgs_mTAS_Calib_20:14:09-03-11-2016/13cfrt_h_SubJet_aftersel_ptJ12TAmult.pdf}
\caption{Track-jet R=0.2 and sub-jet multiplicity for $p_{T}^{J}$ bin (indicated on plot) }

\end{figure}

\clearpage %%

\begin{figure}

\includegraphics[width=0.4\textwidth]{appendixB/Higgs_mTAS_Calib_20:14:09-03-11-2016/1cfrt_h_fabsca_tascal_2.pdf}
\caption{Scatter plot $\mtas$ versus $\mcal$ responses}

\end{figure}
 
\begin{figure}
 
\includegraphics[width=0.4\textwidth]{appendixB/Higgs_mTAS_Calib_20:14:09-03-11-2016/1cfrt_h_fabsca_tasta_2.pdf}
\caption{Scatter plot $\mtas$ versus $\mta$ responses}
 
\end{figure}
 
\begin{figure}
 
\includegraphics[width=0.4\textwidth]{appendixB/Higgs_mTAS_Calib_20:14:09-03-11-2016/1cfrt_h_FatJet_aftersel_m.pdf}
\caption{$\mtas$ distribution in all the $\pt$ bins}
 
\end{figure}
 
\begin{figure}
 
\includegraphics[width=0.4\textwidth]{appendixB/Higgs_mTAS_Calib_20:14:09-03-11-2016/1cfrt_h_FatJet_eta.pdf}
\caption{$\eta$ distribution of the large-R jet, before and after selection}
 
\end{figure}

\begin{figure}
 
\includegraphics[width=0.4\textwidth]{appendixB/Higgs_mTAS_Calib_20:14:09-03-11-2016/1cfrt_h_FatJet_mult.pdf}
\caption{large-R jet Multiplicity, before and after selection}
 
\end{figure}
 
\begin{figure}
 
\includegraphics[width=0.4\textwidth]{appendixB/Higgs_mTAS_Calib_20:14:09-03-11-2016/1cfrt_h_FatJet_psi.pdf}
\caption{$\phi$ distribution of the large-R jet, before and after selection}
 
\end{figure}
 
\begin{figure}
 
\includegraphics[width=0.4\textwidth]{appendixB/Higgs_mTAS_Calib_20:14:09-03-11-2016/1cfrt_h_FatJet_pt.pdf}
\caption{$p_{T}$ distribution of the large-R jet, before and after selection}
 
\end{figure}
 
\begin{figure}
 
\includegraphics[width=0.4\textwidth]{appendixB/Higgs_mTAS_Calib_20:14:09-03-11-2016/1cfrt_h_FatJet_ptres.pdf}
\caption{$\pt$ resolution: $\frac{p_{T,jet}^{track}-p_{T,jet}^{fat}}{p_{T,jet}^{fat}}$, before and after selection }
 
\end{figure}
 
\begin{figure}
 
\includegraphics[width=0.4\textwidth]{appendixB/Higgs_mTAS_Calib_20:14:09-03-11-2016/1cfrt_h_FatJet_TAmult.pdf}
\caption{Multiplicity of track-jets R=0.2 per large-R jet}
 
\end{figure}
 %
\begin{figure}
 
\includegraphics[width=0.4\textwidth]{appendixB/Higgs_mTAS_Calib_20:14:09-03-11-2016/1cfrt_h_JetRatio_m.pdf}
\caption{Response $m^{Reco} / m^{Truth}$ for all the $\pt$ bins}
 
\end{figure}
 
\begin{figure}
 
\includegraphics[width=0.4\textwidth]{appendixB/Higgs_mTAS_Calib_20:14:09-03-11-2016/1cfrt_h_JetRatio_pt.pdf}
\caption{Transverse momentum response $p_{T}^{Reco} / p_{T}^{Truth}$ for calorimeter and tracks}
 
\end{figure}
 
\begin{figure}
 
\includegraphics[width=0.4\textwidth]{appendixB/Higgs_mTAS_Calib_20:14:09-03-11-2016/1cfrt_h_NfminusNi.pdf}
\caption{sub-jet - track-jet Multiplicity}
 
\end{figure}

\clearpage %%

\begin{figure}
 
\includegraphics[width=0.35\textwidth]{appendixB/Higgs_mTAS_Calib_20:14:09-03-11-2016/1cfrt_h_SubJet_Delta_eta.pdf}
\caption{$| \eta_{sub-jet} - \eta_{track-jet} | $ distribution, where sub-jet and track-jet are the closest}
 
\end{figure}
 
\begin{figure}
 
\includegraphics[width=0.35\textwidth]{appendixB/Higgs_mTAS_Calib_20:14:09-03-11-2016/1cfrt_h_SubJet_Delta_m.pdf}
\caption{$| m_{sub-jet} - m_{track-jet} |$ distribution, where sub-jet and track-jet are the closest}
 
\end{figure}
 
\begin{figure}
 
\includegraphics[width=0.35\textwidth]{appendixB/Higgs_mTAS_Calib_20:14:09-03-11-2016/1cfrt_h_SubJet_Delta_psi.pdf}
\caption{$| \phi_{sub-jet} - \phi_{track-jet} | $ distribution, where sub-jet and track-jet are the closest}
 
\end{figure}
 %
\begin{figure}
 
\includegraphics[width=0.35\textwidth]{appendixB/Higgs_mTAS_Calib_20:14:09-03-11-2016/1cfrt_h_SubJet_Delta_pt.pdf}
\caption{$| p_{T,sub-jet} - p_{T,track-jet} | $ distribution, where sub-jet and track-jet are the closest}
 
\end{figure}
 
\begin{figure}
 
\includegraphics[width=0.4\textwidth]{appendixB/Higgs_mTAS_Calib_20:14:09-03-11-2016/1cfrt_h_SubJet_Delta_R.pdf}
\caption{$| R_{sub-jet} - R_{track-jet} | $ distribution, where sub-jet and track-jet are the closest}
 
\end{figure}
 
\begin{figure}
 
\includegraphics[width=0.4\textwidth]{appendixB/Higgs_mTAS_Calib_20:14:09-03-11-2016/1cfrt_h_SubJet_m.pdf}
\caption{Mass distribution of the sub-jet, calorimeter and track-assisted}
 
\end{figure}
% 
\begin{figure}
 
\includegraphics[width=0.4\textwidth]{appendixB/Higgs_mTAS_Calib_20:14:09-03-11-2016/1cfrt_h_SubJet_pt1.pdf}
\caption{$\pt$ distribution for leading, sub-leading and sub-sub-leading sub-jets}
 
\end{figure}
 %
\begin{figure}
 
\includegraphics[width=0.4\textwidth]{appendixB/Higgs_mTAS_Calib_20:14:09-03-11-2016/1cfrt_h_TrackJets_m.pdf}
\caption{Mass distribution for calorimeter and tracks associated to the large-R jet}
 
\end{figure}

%

\begin{figure}

\includegraphics[width=0.4\textwidth]{appendixB/Higgs_mTAS_Calib_20:14:09-03-11-2016/71graph_h_JetRatio_mJ12CALO_meanResponseMvsTA.pdf}
\caption{$\mu $ from fit of the mass Response vs bin of  $p_{T}^{J}$}

\end{figure}
 %
\begin{figure}

\includegraphics[width=0.4\textwidth]{appendixB/Higgs_mTAS_Calib_20:14:09-03-11-2016/72graph_h_JetRatio_mJ12CALO_sigmaResponseMvsTA.pdf}
\caption{$\sigma $ from fit of the mass Response vs bin of $p_{T}^{J}$}

\end{figure}

\begin{figure}

\includegraphics[width=0.4\textwidth]{appendixB/Higgs_mTAS_Calib_20:14:09-03-11-2016/73graph_h_JetRatio_mJ12CALO_I50ResponseMvsTA.pdf}
\caption{Left integral, $\int_{0}^{0.6} $ of the mass response, vs bin of  $p_{T}^{J}$}

\end{figure}

\begin{figure}

\includegraphics[width=0.4\textwidth]{appendixB/Higgs_mTAS_Calib_20:14:09-03-11-2016/74graph_h_JetRatio_mJ12CALO_I50ResponseMvsTAnorm.pdf}
\caption{Left integral normalized, $\int_{0}^{0.6} $ of the mass response, vs bin of  $p_{T}^{J}$}

\end{figure}
 \clearpage %%
\begin{figure}

\includegraphics[width=0.4\textwidth]{appendixB/Higgs_mTAS_Calib_20:14:09-03-11-2016/8ResponsePTJ_h_JetRatio_mJ01CALO.pdf}
\caption{Response in bin of  $p_{T}^{J}$ (indicated on plot)} 

\end{figure}

\begin{figure}

\includegraphics[width=0.4\textwidth]{appendixB/Higgs_mTAS_Calib_20:14:09-03-11-2016/8ResponsePTJ_h_JetRatio_mJ02CALO.pdf}
\caption{Response in bin of  $p_{T}^{J}$ (indicated on plot)} 

\end{figure}

\begin{figure}

\includegraphics[width=0.4\textwidth]{appendixB/Higgs_mTAS_Calib_20:14:09-03-11-2016/8ResponsePTJ_h_JetRatio_mJ03CALO.pdf}
\caption{Response in bin of  $p_{T}^{J}$ (indicated on plot)} 

\end{figure}

\begin{figure}

\includegraphics[width=0.4\textwidth]{appendixB/Higgs_mTAS_Calib_20:14:09-03-11-2016/8ResponsePTJ_h_JetRatio_mJ04CALO.pdf}
\caption{Response in bin of  $p_{T}^{J}$ (indicated on plot)} 

\end{figure}

\begin{figure}

\includegraphics[width=0.4\textwidth]{appendixB/Higgs_mTAS_Calib_20:14:09-03-11-2016/8ResponsePTJ_h_JetRatio_mJ05CALO.pdf}
\caption{Response in bin of  $p_{T}^{J}$ (indicated on plot)} 

\end{figure}

\begin{figure}

\includegraphics[width=0.4\textwidth]{appendixB/Higgs_mTAS_Calib_20:14:09-03-11-2016/8ResponsePTJ_h_JetRatio_mJ06CALO.pdf}
\caption{Response in bin of  $p_{T}^{J}$ (indicated on plot)} 

\end{figure}

%
\begin{figure}

\includegraphics[width=0.4\textwidth]{appendixB/Higgs_mTAS_Calib_20:14:09-03-11-2016/8ResponsePTJ_h_JetRatio_mJ07CALO.pdf}
\caption{Response in bin of  $p_{T}^{J}$ (indicated on plot)} 

\end{figure}


\begin{figure}

\includegraphics[width=0.4\textwidth]{appendixB/Higgs_mTAS_Calib_20:14:09-03-11-2016/8ResponsePTJ_h_JetRatio_mJ08CALO.pdf}
\caption{Response in bin of  $p_{T}^{J}$ (indicated on plot)} 

\end{figure}

\begin{figure}

\includegraphics[width=0.4\textwidth]{appendixB/Higgs_mTAS_Calib_20:14:09-03-11-2016/8ResponsePTJ_h_JetRatio_mJ09CALO.pdf}
\caption{Response in bin of  $p_{T}^{J}$ (indicated on plot)} 

\end{figure}

\begin{figure}

\includegraphics[width=0.4\textwidth]{appendixB/Higgs_mTAS_Calib_20:14:09-03-11-2016/8ResponsePTJ_h_JetRatio_mJ10CALO.pdf}
\caption{Response in bin of  $p_{T}^{J}$ (indicated on plot)} 

\end{figure}

\begin{figure}

\includegraphics[width=0.4\textwidth]{appendixB/Higgs_mTAS_Calib_20:14:09-03-11-2016/8ResponsePTJ_h_JetRatio_mJ11CALO.pdf}
\caption{Response in bin of  $p_{T}^{J}$ (indicated on plot)} 

\end{figure}

\begin{figure}

\includegraphics[width=0.4\textwidth]{appendixB/Higgs_mTAS_Calib_20:14:09-03-11-2016/8ResponsePTJ_h_JetRatio_mJ12CALO.pdf}
\caption{Response in bin of  $p_{T}^{J}$ (indicated on plot)} 

\end{figure}

% \addcontentsline{toc}{section}{$\mcombtas$ distributions, boosted $W/Z$ tops and Higgs}
\clearpage
\onecolumn
\vspace*{\fill}
\section{$\mcombtas$ response distributions, boosted $W/Z$}
\vfill
\clearpage
\twocolumn
 \clearpage %%
\begin{figure}

\includegraphics[width=0.4\textwidth]{appendixB/mTASCOMB_W_calibmCal_030ro_20:38:43-03-11-2016/8ResponsePTJ_h_JetRatio_mJ01CALO.pdf}
\caption{Response in bin of  $p_{T}^{J}$ (indicated on plot)} 

\end{figure}

\begin{figure}

\includegraphics[width=0.4\textwidth]{appendixB/mTASCOMB_W_calibmCal_030ro_20:38:43-03-11-2016/8ResponsePTJ_h_JetRatio_mJ02CALO.pdf}
\caption{Response in bin of  $p_{T}^{J}$ (indicated on plot)} 

\end{figure}

\begin{figure}

\includegraphics[width=0.4\textwidth]{appendixB/mTASCOMB_W_calibmCal_030ro_20:38:43-03-11-2016/8ResponsePTJ_h_JetRatio_mJ03CALO.pdf}
\caption{Response in bin of  $p_{T}^{J}$ (indicated on plot)} 

\end{figure}

\begin{figure}

\includegraphics[width=0.4\textwidth]{appendixB/mTASCOMB_W_calibmCal_030ro_20:38:43-03-11-2016/8ResponsePTJ_h_JetRatio_mJ04CALO.pdf}
\caption{Response in bin of  $p_{T}^{J}$ (indicated on plot)} 

\end{figure}

\begin{figure}

\includegraphics[width=0.4\textwidth]{appendixB/mTASCOMB_W_calibmCal_030ro_20:38:43-03-11-2016/8ResponsePTJ_h_JetRatio_mJ05CALO.pdf}
\caption{Response in bin of  $p_{T}^{J}$ (indicated on plot)} 

\end{figure}

\begin{figure}

\includegraphics[width=0.4\textwidth]{appendixB/mTASCOMB_W_calibmCal_030ro_20:38:43-03-11-2016/8ResponsePTJ_h_JetRatio_mJ06CALO.pdf}
\caption{Response in bin of  $p_{T}^{J}$ (indicated on plot)} 

\end{figure}

%
\begin{figure}

\includegraphics[width=0.4\textwidth]{appendixB/mTASCOMB_W_calibmCal_030ro_20:38:43-03-11-2016/8ResponsePTJ_h_JetRatio_mJ07CALO.pdf}
\caption{Response in bin of  $p_{T}^{J}$ (indicated on plot)} 

\end{figure}


\begin{figure}

\includegraphics[width=0.4\textwidth]{appendixB/mTASCOMB_W_calibmCal_030ro_20:38:43-03-11-2016/8ResponsePTJ_h_JetRatio_mJ08CALO.pdf}
\caption{Response in bin of  $p_{T}^{J}$ (indicated on plot)} 

\end{figure}

\begin{figure}

\includegraphics[width=0.4\textwidth]{appendixB/mTASCOMB_W_calibmCal_030ro_20:38:43-03-11-2016/8ResponsePTJ_h_JetRatio_mJ09CALO.pdf}
\caption{Response in bin of  $p_{T}^{J}$ (indicated on plot)} 

\end{figure}

\begin{figure}

\includegraphics[width=0.4\textwidth]{appendixB/mTASCOMB_W_calibmCal_030ro_20:38:43-03-11-2016/8ResponsePTJ_h_JetRatio_mJ10CALO.pdf}
\caption{Response in bin of  $p_{T}^{J}$ (indicated on plot)} 

\end{figure}

\begin{figure}

\includegraphics[width=0.4\textwidth]{appendixB/mTASCOMB_W_calibmCal_030ro_20:38:43-03-11-2016/8ResponsePTJ_h_JetRatio_mJ11CALO.pdf}
\caption{Response in bin of  $p_{T}^{J}$ (indicated on plot)} 

\end{figure}

\begin{figure}

\includegraphics[width=0.4\textwidth]{appendixB/mTASCOMB_W_calibmCal_030ro_20:38:43-03-11-2016/8ResponsePTJ_h_JetRatio_mJ12CALO.pdf}
\caption{Response in bin of  $p_{T}^{J}$ (indicated on plot)} 

\end{figure}

\clearpage
\onecolumn
\vspace*{\fill}
\section{$\mcombtas$ response distributions, boosted tops}
\vfill
\clearpage
\twocolumn
 \clearpage %%
\begin{figure}

\includegraphics[width=0.4\textwidth]{appendixB/mTASCOMB_Tops_calibmCal_030ro_20:20:52-03-11-2016/8ResponsePTJ_h_JetRatio_mJ01CALO.pdf}
\caption{Response in bin of  $p_{T}^{J}$ (indicated on plot)} 

\end{figure}

\begin{figure}

\includegraphics[width=0.4\textwidth]{appendixB/mTASCOMB_Tops_calibmCal_030ro_20:20:52-03-11-2016/8ResponsePTJ_h_JetRatio_mJ02CALO.pdf}
\caption{Response in bin of  $p_{T}^{J}$ (indicated on plot)} 

\end{figure}

\begin{figure}

\includegraphics[width=0.4\textwidth]{appendixB/mTASCOMB_Tops_calibmCal_030ro_20:20:52-03-11-2016/8ResponsePTJ_h_JetRatio_mJ03CALO.pdf}
\caption{Response in bin of  $p_{T}^{J}$ (indicated on plot)} 

\end{figure}

\begin{figure}

\includegraphics[width=0.4\textwidth]{appendixB/mTASCOMB_Tops_calibmCal_030ro_20:20:52-03-11-2016/8ResponsePTJ_h_JetRatio_mJ04CALO.pdf}
\caption{Response in bin of  $p_{T}^{J}$ (indicated on plot)} 

\end{figure}

\begin{figure}

\includegraphics[width=0.4\textwidth]{appendixB/mTASCOMB_Tops_calibmCal_030ro_20:20:52-03-11-2016/8ResponsePTJ_h_JetRatio_mJ05CALO.pdf}
\caption{Response in bin of  $p_{T}^{J}$ (indicated on plot)} 

\end{figure}

\begin{figure}

\includegraphics[width=0.4\textwidth]{appendixB/mTASCOMB_Tops_calibmCal_030ro_20:20:52-03-11-2016/8ResponsePTJ_h_JetRatio_mJ06CALO.pdf}
\caption{Response in bin of  $p_{T}^{J}$ (indicated on plot)} 

\end{figure}

%
\begin{figure}

\includegraphics[width=0.4\textwidth]{appendixB/mTASCOMB_Tops_calibmCal_030ro_20:20:52-03-11-2016/8ResponsePTJ_h_JetRatio_mJ07CALO.pdf}
\caption{Response in bin of  $p_{T}^{J}$ (indicated on plot)} 

\end{figure}


\begin{figure}

\includegraphics[width=0.4\textwidth]{appendixB/mTASCOMB_Tops_calibmCal_030ro_20:20:52-03-11-2016/8ResponsePTJ_h_JetRatio_mJ08CALO.pdf}
\caption{Response in bin of  $p_{T}^{J}$ (indicated on plot)} 

\end{figure}

\begin{figure}

\includegraphics[width=0.4\textwidth]{appendixB/mTASCOMB_Tops_calibmCal_030ro_20:20:52-03-11-2016/8ResponsePTJ_h_JetRatio_mJ09CALO.pdf}
\caption{Response in bin of  $p_{T}^{J}$ (indicated on plot)} 

\end{figure}

\begin{figure}

\includegraphics[width=0.4\textwidth]{appendixB/mTASCOMB_Tops_calibmCal_030ro_20:20:52-03-11-2016/8ResponsePTJ_h_JetRatio_mJ10CALO.pdf}
\caption{Response in bin of  $p_{T}^{J}$ (indicated on plot)} 

\end{figure}

\begin{figure}

\includegraphics[width=0.4\textwidth]{appendixB/mTASCOMB_Tops_calibmCal_030ro_20:20:52-03-11-2016/8ResponsePTJ_h_JetRatio_mJ11CALO.pdf}
\caption{Response in bin of  $p_{T}^{J}$ (indicated on plot)} 

\end{figure}

\begin{figure}

\includegraphics[width=0.4\textwidth]{appendixB/mTASCOMB_Tops_calibmCal_030ro_20:20:52-03-11-2016/8ResponsePTJ_h_JetRatio_mJ12CALO.pdf}
\caption{Response in bin of  $p_{T}^{J}$ (indicated on plot)} 

\end{figure}
%mTAS_Comb_Higgs_calibmCal_003ro_11:11:35-10-11-2016
\clearpage
\onecolumn
\vspace*{\fill}
\section{$\mcombtas$ response distributions, Higgs}
\vfill
\clearpage
\twocolumn
 \clearpage %%
\begin{figure}

\includegraphics[width=0.4\textwidth]{appendixB/mTAS_Comb_Higgs_calibmCal_003ro_11:11:35-10-11-2016/8ResponsePTJ_h_JetRatio_mJ01CALO.pdf}
\caption{Response in bin of  $p_{T}^{J}$ (indicated on plot)} 

\end{figure}

\begin{figure}

\includegraphics[width=0.4\textwidth]{appendixB/mTAS_Comb_Higgs_calibmCal_003ro_11:11:35-10-11-2016/8ResponsePTJ_h_JetRatio_mJ02CALO.pdf}
\caption{Response in bin of  $p_{T}^{J}$ (indicated on plot)} 

\end{figure}

\begin{figure}

\includegraphics[width=0.4\textwidth]{appendixB/mTAS_Comb_Higgs_calibmCal_003ro_11:11:35-10-11-2016/8ResponsePTJ_h_JetRatio_mJ03CALO.pdf}
\caption{Response in bin of  $p_{T}^{J}$ (indicated on plot)} 

\end{figure}

\begin{figure}

\includegraphics[width=0.4\textwidth]{appendixB/mTAS_Comb_Higgs_calibmCal_003ro_11:11:35-10-11-2016/8ResponsePTJ_h_JetRatio_mJ04CALO.pdf}
\caption{Response in bin of  $p_{T}^{J}$ (indicated on plot)} 

\end{figure}

\begin{figure}

\includegraphics[width=0.4\textwidth]{appendixB/mTAS_Comb_Higgs_calibmCal_003ro_11:11:35-10-11-2016/8ResponsePTJ_h_JetRatio_mJ05CALO.pdf}
\caption{Response in bin of  $p_{T}^{J}$ (indicated on plot)} 

\end{figure}

\begin{figure}

\includegraphics[width=0.4\textwidth]{appendixB/mTAS_Comb_Higgs_calibmCal_003ro_11:11:35-10-11-2016/8ResponsePTJ_h_JetRatio_mJ06CALO.pdf}
\caption{Response in bin of  $p_{T}^{J}$ (indicated on plot)} 

\end{figure}

%
\begin{figure}

\includegraphics[width=0.4\textwidth]{appendixB/mTAS_Comb_Higgs_calibmCal_003ro_11:11:35-10-11-2016/8ResponsePTJ_h_JetRatio_mJ07CALO.pdf}
\caption{Response in bin of  $p_{T}^{J}$ (indicated on plot)} 

\end{figure}


\begin{figure}

\includegraphics[width=0.4\textwidth]{appendixB/mTAS_Comb_Higgs_calibmCal_003ro_11:11:35-10-11-2016/8ResponsePTJ_h_JetRatio_mJ08CALO.pdf}
\caption{Response in bin of  $p_{T}^{J}$ (indicated on plot)} 

\end{figure}

\begin{figure}

\includegraphics[width=0.4\textwidth]{appendixB/mTAS_Comb_Higgs_calibmCal_003ro_11:11:35-10-11-2016/8ResponsePTJ_h_JetRatio_mJ09CALO.pdf}
\caption{Response in bin of  $p_{T}^{J}$ (indicated on plot)} 

\end{figure}

\begin{figure}

\includegraphics[width=0.4\textwidth]{appendixB/mTAS_Comb_Higgs_calibmCal_003ro_11:11:35-10-11-2016/8ResponsePTJ_h_JetRatio_mJ10CALO.pdf}
\caption{Response in bin of  $p_{T}^{J}$ (indicated on plot)} 

\end{figure}

\begin{figure}

\includegraphics[width=0.4\textwidth]{appendixB/mTAS_Comb_Higgs_calibmCal_003ro_11:11:35-10-11-2016/8ResponsePTJ_h_JetRatio_mJ11CALO.pdf}
\caption{Response in bin of  $p_{T}^{J}$ (indicated on plot)} 

\end{figure}

\begin{figure}

\includegraphics[width=0.4\textwidth]{appendixB/mTAS_Comb_Higgs_calibmCal_003ro_11:11:35-10-11-2016/8ResponsePTJ_h_JetRatio_mJ12CALO.pdf}
\caption{Response in bin of  $p_{T}^{J}$ (indicated on plot)} 

\end{figure}

\onecolumn


\section{$p_{\text{T}}$ reweighting}
\begin{figure} 
	\includegraphics[width=0.5\textwidth]{sascha_input/plots/track_selection/h_leadpt_truth.pdf} \hspace{1mm}
	\includegraphics[width=0.5\textwidth]{sascha_input/plots/track_selection/h_leadpt_truth_weight.pdf}
\caption{\footnotesize{Exemplary $p_{\mathrm{T}}$ distributions of $W$ jets (left) and QCD jets from multi-jet events with reweighted $W$ boson events (right).}}\label{fig:p_T}
\end{figure}

\section{$p_{\text{T}}$ Dependence of Substructure Observables}\label{appendix:pt_dependence}
Due to the low weights for high $p_{\mathrm{T}}$, the correlation plots are divided into the six $p_{\mathrm{T}}$ regions. For C2, see Figure \ref{fig:correlation_C2}, observed is a strong trend to lower values (signal and background) for clusters and TAS. The TAS distributions concentrate at lower values compared to calorimeter counterparts.

For D2, Figure \ref{fig:correlation_D2}, and $\tau_{21}$, Figure \ref{fig:correlation_tau21}, there is a slight upward trend of the calorimeter variables in the lower $p_{\mathrm{T}}$ regions. With rising boost this slows down and ends in a broader distribution for $\tau_{21}$. This verifies the higher $p_{\mathrm{T}}$ dependence of the C2 variable in comparison to D2 and $\tau_{21}$. The TAS counterparts feature an even more robust signal with the background moving to higher values, hence improving separation. The $p_{\mathrm{T}}$ dependence of variables calculated with tracks is very similar to the ones with TAS, therefore they are omitted.
\begin{figure}[htp]
\includegraphics[width=0.5\textwidth]{sascha_input/plots/W/beta1/scatter_plots/scatter_h_scatter_reco_C2.pdf}
\bigskip
\includegraphics[width=0.5\textwidth]{sascha_input/plots/W/beta1/scatter_plots/scatter_h_scatter_assisted_tj_C2.pdf} 
\caption{\footnotesize{Correlation between C2 at $\beta=1$ and $p_{\mathrm{T}}$ applied on $W$ boson signal (above) and QCD background (below) for calorimeter (left) and TAS (right).}}\label{fig:correlation_C2}
\end{figure}
\begin{figure}[htp]
\includegraphics[width=0.5\textwidth]{sascha_input/plots/W/beta1/scatter_plots/scatter_h_scatter_reco_D2.pdf}
\bigskip
\includegraphics[width=0.5\textwidth]{sascha_input/plots/W/beta1/scatter_plots/scatter_h_scatter_assisted_tj_D2.pdf} 
\caption{\footnotesize{Correlation between D2 at $\beta=1$ and $p_{\mathrm{T}}$ applied on $W$ boson signal (above) and QCD background (below) for calorimeter (left) and TAS (right).}}\label{fig:correlation_D2}
\end{figure}
\begin{figure}[htp]
\includegraphics[width=0.5\textwidth]{sascha_input/plots/W/beta1/scatter_plots/scatter_h_scatter_reco_nSub21.pdf}
\bigskip
\includegraphics[width=0.5\textwidth]{sascha_input/plots/W/beta1/scatter_plots/scatter_h_scatter_assisted_tj_nSub21.pdf} 
\caption{\footnotesize{Correlation between $\tau_{21}$ at $\beta=1$ and $p_{\mathrm{T}}$ applied on $W$ boson signal (above) and QCD background (below) for calorimeter (left) and TAS (right).}}\label{fig:correlation_tau21}
\end{figure}

\section{Results of $\beta$ Optimisation}
\vspace{-0.25cm}
\subsection{Performance for $W$ tagging}\label{appendix:w_optimisation}
\vspace{-5cm}
\begin{sidewaystable}[htb]
\centering
\resizebox{22cm}{!}{%
\begin{tabular}{llllllllllllllll}
\rowcolor{Gray} \multicolumn{1}{l||}{\textbf{Calorimeter}} &  &  & C2 &  & \multicolumn{1}{l||}{} &  &  & D2 &  & \multicolumn{1}{l||}{} &  &  & $\tau_{21}$ &  & \multicolumn{1}{l|}{} \\ \hline
\multicolumn{1}{l||}{$p_{\mathrm{T}} \, \text{[GeV]}$} &  \multicolumn{1}{l|}{\cellcolor{Gray2}$\beta=0.5$} & \multicolumn{1}{l|}{\cellcolor{Gray2}1} & \multicolumn{1}{l|}{\cellcolor{Gray2}1.7} & \multicolumn{1}{l|}{\cellcolor{Gray2}2} & \multicolumn{1}{l||}{\cellcolor{Gray2}3} & \multicolumn{1}{l|}{\cellcolor{Gray2}$\beta=0.5$} & \multicolumn{1}{l|}{\cellcolor{Gray2}1} & \multicolumn{1}{l|}{\cellcolor{Gray2}1.7} & \multicolumn{1}{l|}{\cellcolor{Gray2}2} & \multicolumn{1}{l||}{\cellcolor{Gray2}3} & \multicolumn{1}{l|}{\cellcolor{Gray2}$\beta=0.5$} & \multicolumn{1}{l|}{\cellcolor{Gray2}1} & \multicolumn{1}{l|}{\cellcolor{Gray2}1.7} & \multicolumn{1}{l|}{\cellcolor{Gray2}2} & \multicolumn{1}{l|}{\cellcolor{Gray2}3} \\ \hline \hline
\multicolumn{1}{l||}{250 - 500} & 	\multicolumn{1}{l|}{29.7(1.5)} & 	 \multicolumn{1}{l|}{31.7(1.9)} & 		\multicolumn{1}{l|}{31.4(1.6)} & 	 \multicolumn{1}{l|}{30.7(1.9)} &     \multicolumn{1}{l||}{28.5(1.4)} & 	\multicolumn{1}{l|}{27.2(2.0)}& 	\multicolumn{1}{l|}{\cellcolor{Red!50}35.0(2.0)} & 		\multicolumn{1}{l|}{33.0(1.8)} & 		\multicolumn{1}{l|}{31.3(1.7)} & 	 \multicolumn{1}{l||}{25.7(1.2)} & 	   \multicolumn{1}{l|}{33.1(1.8)} & \multicolumn{1}{l|}{27.6(1.3)} & \multicolumn{1}{l|}{26.2(1.4)} & \multicolumn{1}{l|}{25.1(1.2)} & \multicolumn{1}{l|}{22.4(0.8)} \\
\multicolumn{1}{l||}{500 - 800} & 	\multicolumn{1}{l|}{44.2(1.8)} & 	 \multicolumn{1}{l|}{50.1(2.0)} & 		\multicolumn{1}{l|}{49.6(1.9)} & 	 \multicolumn{1}{l|}{48.6(1.8)} & 	  \multicolumn{1}{l||}{42.6(1.9)} &  	\multicolumn{1}{l|}{40.3(2.2)} & 	\multicolumn{1}{l|}{\cellcolor{Red!50}55.3(2.6)} & 		\multicolumn{1}{l|}{56.3(2.4)} & 		\multicolumn{1}{l|}{52.5(2.1)} & 	 \multicolumn{1}{l||}{39.3(1.3)} & 	   \multicolumn{1}{l|}{49.4(2.0)} & \multicolumn{1}{l|}{41.1(1.4)} & \multicolumn{1}{l|}{43.3(1.7)} & \multicolumn{1}{l|}{41.3(1.6)} & \multicolumn{1}{l|}{36.1(1.2)} \\
\multicolumn{1}{l||}{800 - 1200} & 	\multicolumn{1}{l|}{32.0(1.5)} & 	 \multicolumn{1}{l|}{37.5(1.7)} & 		\multicolumn{1}{l|}{35.4(1.5)} & 	 \multicolumn{1}{l|}{33.4(1.5)} & 	  \multicolumn{1}{l||}{26.8(0.9)} &  	\multicolumn{1}{l|}{34.0(2.1)} & 	\multicolumn{1}{l|}{\cellcolor{Red!50}41.1(2.0)} & 		\multicolumn{1}{l|}{38.5(1.6)} & 		\multicolumn{1}{l|}{34.9(1.3)} & 	 \multicolumn{1}{l||}{25.4(0.7)} & 	   \multicolumn{1}{l|}{30.5(1.2)} & \multicolumn{1}{l|}{30.9(1.2)} & \multicolumn{1}{l|}{33.8(1.4)} & \multicolumn{1}{l|}{32.5(1.3)} & \multicolumn{1}{l|}{28.1(0.9)} \\
\multicolumn{1}{l||}{1200 - 1600} & \multicolumn{1}{l|}{30.1(1.3)} & 	 \multicolumn{1}{l|}{34.4(1.8)} & 		\multicolumn{1}{l|}{29.4(1.3)} & 	 \multicolumn{1}{l|}{26.8(1.0)} & 	  \multicolumn{1}{l||}{20.7(0.8)} &  	\multicolumn{1}{l|}{34.1(1.8)} & 	\multicolumn{1}{l|}{\cellcolor{Red!50}38.1(1.9)} & 		\multicolumn{1}{l|}{31.4(1.4)} & 		\multicolumn{1}{l|}{27.6(1.2)} & 	 \multicolumn{1}{l||}{19.3(0.5)} & 	   \multicolumn{1}{l|}{23.1(0.9)} & \multicolumn{1}{l|}{27.3(1.)} & \multicolumn{1}{l|}{31.1(1.2)} & \multicolumn{1}{l|}{29.9(1.3)} & \multicolumn{1}{l|}{24.8(0.9)} \\
\multicolumn{1}{l||}{1600 - 2000} & \multicolumn{1}{l|}{20.9(1.3)} & 	 \multicolumn{1}{l|}{22.4(1.5)} & 		\multicolumn{1}{l|}{18.2(1.2)} & 	 \multicolumn{1}{l|}{16.5(0.9)} & 	  \multicolumn{1}{l||}{12.9(0.6)} &  	\multicolumn{1}{l|}{26.4(1.7)} & 	\multicolumn{1}{l|}{\cellcolor{Red!50}25.4(1.3)} & 		\multicolumn{1}{l|}{19.3(1.1)} & 		\multicolumn{1}{l|}{16.9(0.9)} & 	 \multicolumn{1}{l||}{11.9(0.5)} & 	   \multicolumn{1}{l|}{16.4(1.0)} & \multicolumn{1}{l|}{19.1(1.1)} & \multicolumn{1}{l|}{21.1(1.1)} & \multicolumn{1}{l|}{19.9(1.0)} & \multicolumn{1}{l|}{16.0(0.9)} \\
\multicolumn{1}{l||}{$>2000$} & 	\multicolumn{1}{l|}{16.9(1.4)} & 	 \multicolumn{1}{l|}{18.7(1.4)} & 		\multicolumn{1}{l|}{14.1(0.9)} & 	 \multicolumn{1}{l|}{12.6(0.8)} & 	  \multicolumn{1}{l||}{9.9(0.7)} & 		\multicolumn{1}{l|}{23.3(1.9)} & 	\multicolumn{1}{l|}{\cellcolor{Red!50}21.9(1.7)} & 		\multicolumn{1}{l|}{15.7(1.1)} & 		\multicolumn{1}{l|}{13.5(0.9)} & 	 \multicolumn{1}{l||}{9.2(0.4)} & 	   \multicolumn{1}{l|}{12.3(1.1)} & \multicolumn{1}{l|}{15.5(1.1)} & \multicolumn{1}{l|}{17.2(1.2)} & \multicolumn{1}{l|}{15.7(1.1)} & \multicolumn{1}{l|}{11.9(0.8)} \\ \hline
 &  &  &  &  &  &  &  &  &  &  &  &  &  &  &  \\
\rowcolor{Gray} \multicolumn{1}{l||}{\textbf{TAS}} &  &  & C2 &  & \multicolumn{1}{l||}{} &  &  & D2 &  & \multicolumn{1}{l||}{} &  &  & $\tau_{21}$ &  & \multicolumn{1}{l|}{} \\ \hline
\multicolumn{1}{l||}{$p_{\mathrm{T}} \, \text{[GeV]}$}   &  \multicolumn{1}{l|}{ \cellcolor{Gray2} $\beta=0.5$} & \multicolumn{1}{l|}{\cellcolor{Gray2} 1} & \multicolumn{1}{l|}{\cellcolor{Gray2}1.7} &   \multicolumn{1}{l|}{\cellcolor{Gray2} 2} &  \multicolumn{1}{l||}{\cellcolor{Gray2} 3} & \multicolumn{1}{l|}{\cellcolor{Gray2} $\beta=0.5$} &  \multicolumn{1}{l|}{\cellcolor{Gray2} 1} & 	\multicolumn{1}{l|}{\cellcolor{Gray2} 1.7} & 	\multicolumn{1}{l|}{\cellcolor{Gray2} 2} & \multicolumn{1}{l||}{\cellcolor{Gray2} 3} & \multicolumn{1}{l|}{ \cellcolor{Gray2} $\beta=0.5$} & \multicolumn{1}{l|}{\cellcolor{Gray2} 1} & \multicolumn{1}{l|}{\cellcolor{Gray2} 1.7} &  \multicolumn{1}{l|}{\cellcolor{Gray2} 2} & \multicolumn{1}{l|}{\cellcolor{Gray2} 3} \\ \hline \hline
\multicolumn{1}{l||}{250 - 500} & 	\multicolumn{1}{l|}{29.4(1.9)} & \multicolumn{1}{l|}{30.1(1.9)} & \multicolumn{1}{l|}{28.9(1.5)} & 				     	\multicolumn{1}{l|}{28.5(1.3)} & 						\multicolumn{1}{l||}{27.7(1.3)} & \multicolumn{1}{l|}{28.6(2.0)} & \multicolumn{1}{l|}{\cellcolor{Red!50}37.7(2.1)} & 		\multicolumn{1}{l|}{35.4(2.3)} & 					\multicolumn{1}{l|}{33.4(2.0)} & 	\multicolumn{1}{l||}{29.4(1.2)} & 					\multicolumn{1}{l|}{36.2(2.2)} & 	\multicolumn{1}{l|}{31.5(1.6)} & \multicolumn{1}{l|}{26.8(1.3)} & \multicolumn{1}{l|}{25.4(1.4)} & \multicolumn{1}{l|}{24.0(1.0)} \\
\multicolumn{1}{l||}{500 - 800} & 	\multicolumn{1}{l|}{48.2(2.0)} & \multicolumn{1}{l|}{55.5(2.7)} & \multicolumn{1}{l|}{58.6(2.6)} & 				     	\multicolumn{1}{l|}{59.1(2.7)} & 						\multicolumn{1}{l||}{56.8(2.0)} & \multicolumn{1}{l|}{42.8(2.3)} & \multicolumn{1}{l|}{67.2(3.1)} & 						\multicolumn{1}{l|}{\cellcolor{Red!50}67.6(3.2)} & 	\multicolumn{1}{l|}{63.7(3.0)} & 	\multicolumn{1}{l||}{52.6(2.3)} & 					\multicolumn{1}{l|}{55.7(2.6)} & 	\multicolumn{1}{l|}{51.9(2.1)} & \multicolumn{1}{l|}{45.5(2.0)} & \multicolumn{1}{l|}{44.0(1.9)} & \multicolumn{1}{l|}{41.3(1.5)} \\
\multicolumn{1}{l||}{800 - 1200} & 	\multicolumn{1}{l|}{31.0(1.2)} & \multicolumn{1}{l|}{44.6(1.9)} & \multicolumn{1}{l|}{54.6(2.8)} & 				     	\multicolumn{1}{l|}{\cellcolor{Red!50}55.2(2.8)} & 		\multicolumn{1}{l||}{53.0(3.2)} & \multicolumn{1}{l|}{26.1(1.3)} & \multicolumn{1}{l|}{47.6(2.3)} & 						\multicolumn{1}{l|}{54.9(2.4)} & 					\multicolumn{1}{l|}{52.6(2.8)} & 	\multicolumn{1}{l||}{43.1(1.5)} & 					\multicolumn{1}{l|}{36.4(1.8)} & 	\multicolumn{1}{l|}{37.3(1.7)} & \multicolumn{1}{l|}{36.2(1.8)} & \multicolumn{1}{l|}{36.2(1.6)} & \multicolumn{1}{l|}{35.5(1.6)} \\
\multicolumn{1}{l||}{1200 - 1600} & \multicolumn{1}{l|}{20.9(0.7)} & \multicolumn{1}{l|}{39.1(1.9)} & \multicolumn{1}{l|}{53.8(2.6)} & 				     	\multicolumn{1}{l|}{\cellcolor{Red!50}55.1(3.0)} & 		\multicolumn{1}{l||}{50.1(1.6)} & \multicolumn{1}{l|}{22.7(1.4)} & \multicolumn{1}{l|}{42.1(2.4)} & 						\multicolumn{1}{l|}{50.8(1.8)} & 					\multicolumn{1}{l|}{49.6(2.3)} & 	\multicolumn{1}{l||}{41.1(1.2)} & 					\multicolumn{1}{l|}{27.9(1.3)} & 	\multicolumn{1}{l|}{31.4(1.5)} & \multicolumn{1}{l|}{33.4(1.6)} & \multicolumn{1}{l|}{34.0(2.0)} & \multicolumn{1}{l|}{33.0(1.8)} \\
\multicolumn{1}{l||}{1600 - 2000} & \multicolumn{1}{l|}{16.7(0.7)} & \multicolumn{1}{l|}{36.9(2.9)} & \multicolumn{1}{l|}{\cellcolor{Red!50}50.9(4.3)} &    \multicolumn{1}{l|}{50.3(4.4)} & 						\multicolumn{1}{l||}{42.2(2.4)} & \multicolumn{1}{l|}{18.7(1.7)} & \multicolumn{1}{l|}{32.7(3.3)} & 						\multicolumn{1}{l|}{37.8(2.0)} & 					\multicolumn{1}{l|}{36.1(2.4)} & 	\multicolumn{1}{l||}{28.7(1.2)} & 					\multicolumn{1}{l|}{20.5(1.2)} & 	\multicolumn{1}{l|}{24.8(1.6)} & \multicolumn{1}{l|}{26.1(2.0)} & \multicolumn{1}{l|}{26.5(2.0)} & \multicolumn{1}{l|}{25.4(2.0)} \\
\multicolumn{1}{l||}{$>2000$} & 	\multicolumn{1}{l|}{11.6(0.6)} & \multicolumn{1}{l|}{31.2(3.2)} & \multicolumn{1}{l|}{\cellcolor{Red!50}46.1(4.7)} &    \multicolumn{1}{l|}{45.5(5.2)} & 						\multicolumn{1}{l||}{35.5(3.8)} & \multicolumn{1}{l|}{17.8(2.0)} & \multicolumn{1}{l|}{33.0(4.0)} & 						\multicolumn{1}{l|}{36.3(2.0)} & 					\multicolumn{1}{l|}{34.0(2.5)} & 	\multicolumn{1}{l||}{27.4(1.3)} & 					\multicolumn{1}{l|}{16.4(1.3)} & 	\multicolumn{1}{l|}{22.3(2.0)} & \multicolumn{1}{l|}{24.2(2.2)} & \multicolumn{1}{l|}{24.4(2.5)} & \multicolumn{1}{l|}{21.8(2.4)} \\ \hline
 &  &  &  &  &  &  &  &  &  &  &  &  &  &  &  \\
\rowcolor{Gray} \multicolumn{1}{l||}{\textbf{Tracks}} &  &  & C2 &  & \multicolumn{1}{l||}{} &  &  & D2 &  & \multicolumn{1}{l||}{} &  &  & $\tau_{21}$ &  & \multicolumn{1}{l|}{} \\ \hline
\multicolumn{1}{l||}{$p_{\mathrm{T}} \, \text{[GeV]}$}  & \multicolumn{1}{l|}{\cellcolor{Gray2}$\beta=0.5$} & \multicolumn{1}{l|}{\cellcolor{Gray2}1} & \multicolumn{1}{l|}{\cellcolor{Gray2}1.7} & \multicolumn{1}{l|}{\cellcolor{Gray2}2} & \multicolumn{1}{l||}{\cellcolor{Gray2}3} & \multicolumn{1}{l|}{\cellcolor{Gray2}$\beta=0.5$} & \multicolumn{1}{l|}{\cellcolor{Gray2}1} & \multicolumn{1}{l|}{\cellcolor{Gray2}1.7} & \multicolumn{1}{l|}{\cellcolor{Gray2}2} & \multicolumn{1}{l||}{\cellcolor{Gray2}3} & \multicolumn{1}{l|}{\cellcolor{Gray2}$\beta=0.5$} & \multicolumn{1}{l|}{\cellcolor{Gray2}1} & \multicolumn{1}{l|}{\cellcolor{Gray2}1.7} & \multicolumn{1}{l|}{\cellcolor{Gray2}2} & \multicolumn{1}{l|}{\cellcolor{Gray2}3} \\ \hline \hline
\multicolumn{1}{l||}{250 - 500} & 	\multicolumn{1}{l|}{27.1(1.2)} & \multicolumn{1}{l|}{28.1(1.5)} & \multicolumn{1}{l|}{28.7(1.9)} 						& \multicolumn{1}{l|}{28.7(1.9)} 					& \multicolumn{1}{l||}{28.2(1.7)} & \multicolumn{1}{l|}{21.6(1.2)} & \multicolumn{1}{l|}{28.9(2.0)} & \multicolumn{1}{l|}{\cellcolor{Red!50}29.5(1.8)} 		& \multicolumn{1}{l|}{29.1(1.6)} & \multicolumn{1}{l||}{28.1(1.3)} & \multicolumn{1}{l|}{28.7(1.8)} & \multicolumn{1}{l|}{28.0(1.7)} & \multicolumn{1}{l|}{25.6(1.3)} & \multicolumn{1}{l|}{25.1(1.3)} & \multicolumn{1}{l|}{24.2(0.9)} \\
\multicolumn{1}{l||}{500 - 800} & 	\multicolumn{1}{l|}{46.5(1.9)} & \multicolumn{1}{l|}{52.9(2.4)} & \multicolumn{1}{l|}{57.7(2.6)} 						& \multicolumn{1}{l|}{\cellcolor{Red!50}58.1(2.7)} 	& \multicolumn{1}{l||}{55.8(2.5)} & \multicolumn{1}{l|}{30.1(1.8)} & \multicolumn{1}{l|}{46.8(2.4)} & \multicolumn{1}{l|}{53.4(2.2)} 						& \multicolumn{1}{l|}{52.1(2.3)} & \multicolumn{1}{l||}{46.6(1.7)} & \multicolumn{1}{l|}{46.1(2.3)} & \multicolumn{1}{l|}{44.9(1.8)} & \multicolumn{1}{l|}{41.7(2.1)} & \multicolumn{1}{l|}{40.6(1.8)} & \multicolumn{1}{l|}{39.2(1.5)} \\
\multicolumn{1}{l||}{800 - 1200} & 	\multicolumn{1}{l|}{30.3(1.1)} & \multicolumn{1}{l|}{44.5(2.2)} & \multicolumn{1}{l|}{54.8(2.8)} 						& \multicolumn{1}{l|}{\cellcolor{Red!50}56.4(3.0)} 	& \multicolumn{1}{l||}{53.7(3.6)} & \multicolumn{1}{l|}{24.5(1.5)} & \multicolumn{1}{l|}{42.3(2.3)} & \multicolumn{1}{l|}{48.6(2.5)} 						& \multicolumn{1}{l|}{47.5(1.2)} & \multicolumn{1}{l||}{42.4(1.2)} & \multicolumn{1}{l|}{34.5(1.6)} & \multicolumn{1}{l|}{36.2(1.8)} & \multicolumn{1}{l|}{36.0(1.8)} & \multicolumn{1}{l|}{36.2(1.8)} & \multicolumn{1}{l|}{35.7(1.5)} \\
\multicolumn{1}{l||}{1200 - 1600} & \multicolumn{1}{l|}{20.7(0.6)} & \multicolumn{1}{l|}{39.0(1.9)} & \multicolumn{1}{l|}{54.2(2.7)} 						& \multicolumn{1}{l|}{\cellcolor{Red!50}55.5(3.3)} 	& \multicolumn{1}{l||}{50.9(1.7)} & \multicolumn{1}{l|}{22.7(1.3)} & \multicolumn{1}{l|}{41.0(2.2)} & \multicolumn{1}{l|}{50.0(1.6)} 						& \multicolumn{1}{l|}{47.6(2.2)} & \multicolumn{1}{l||}{41.4(1.2)} & \multicolumn{1}{l|}{27.7(1.2)} & \multicolumn{1}{l|}{31.3(1.4)} & \multicolumn{1}{l|}{33.3(1.6)} & \multicolumn{1}{l|}{33.9(1.7)} & \multicolumn{1}{l|}{33.2(1.8)} \\
\multicolumn{1}{l||}{1600 - 2000} & \multicolumn{1}{l|}{16.6(0.7)} & \multicolumn{1}{l|}{36.7(2.3)} & \multicolumn{1}{l|}{\cellcolor{Red!50}51.7(5.2)} 		& \multicolumn{1}{l|}{51.6(4.0)} 					& \multicolumn{1}{l||}{43.1(2.3)} & \multicolumn{1}{l|}{18.5(1.7)} & \multicolumn{1}{l|}{32.1(3.0)} & \multicolumn{1}{l|}{37.0(1.9)} 						& \multicolumn{1}{l|}{35.9(2.3)} & \multicolumn{1}{l||}{29.3(1.2)} & \multicolumn{1}{l|}{20.5(1.3)} & \multicolumn{1}{l|}{24.6(1.7)} & \multicolumn{1}{l|}{26.2(1.8)} & \multicolumn{1}{l|}{26.7(2.0)} & \multicolumn{1}{l|}{25.9(2.2)} \\
\multicolumn{1}{l||}{$>2000$} & 	\multicolumn{1}{l|}{11.6(0.5)} & \multicolumn{1}{l|}{31.5(3.0)} & \multicolumn{1}{l|}{\cellcolor{Red!50}46.8(5.7)} 		& \multicolumn{1}{l|}{46.0(4.2)} 					& \multicolumn{1}{l||}{36.1(4.3)} & \multicolumn{1}{l|}{17.8(2.2)} & \multicolumn{1}{l|}{33.0(3.3)} & \multicolumn{1}{l|}{35.9(2.1)} 						& \multicolumn{1}{l|}{34.2(2.6)} & \multicolumn{1}{l||}{28.1(1.0)} & \multicolumn{1}{l|}{16.4(1.4)} & \multicolumn{1}{l|}{22.5(1.8)} & \multicolumn{1}{l|}{24.5(2.4)} & \multicolumn{1}{l|}{24.7(2.6)} & \multicolumn{1}{l|}{22.2(2.6)} \\ \hline
\end{tabular}}
\caption{Listing of the QCD rejection for $W$ jets achieved with C2, D2 and $\tau_{21}$ calculated with varying angular weightings $\beta$ and constituents. The highest achieved background rejection per energy range is highlighted in red.}\label{table:w_scan}
\end{sidewaystable}

\FloatBarrier
\subsection{Performance for Higgs tagging}\label{appendix:higgs_optimisation}
\begin{sidewaystable}[h]
\centering
\hspace{-1cm}
\resizebox{22cm}{!}{%
\begin{tabular}{llllllllllllllll}
\rowcolor{Gray} \multicolumn{1}{l||}{\textbf{Calorimeter}}    &                                  &                           & C2                        &                           & \multicolumn{1}{l|}{}     &                                  &                           & D2                        &                           & \multicolumn{1}{l|}{}     &                                  &                           & $\tau_{21}$               &                           & \multicolumn{1}{l|}{}     \\ \hline
\multicolumn{1}{l||}{$p_{\mathrm{T}} \, [GeV]$} & \multicolumn{1}{l|}{\cellcolor{Gray2}$\beta=0.5$} & \multicolumn{1}{l|}{\cellcolor{Gray2}1}    & \multicolumn{1}{l|}{\cellcolor{Gray2}1.7}  & \multicolumn{1}{l|}{\cellcolor{Gray2}2}    & \multicolumn{1}{l||}{\cellcolor{Gray2}3}    & \multicolumn{1}{l|}{\cellcolor{Gray2}$\beta=0.5$} & \multicolumn{1}{l|}{\cellcolor{Gray2}1}    & \multicolumn{1}{l|}{\cellcolor{Gray2}1.7}  & \multicolumn{1}{l|}{\cellcolor{Gray2}2}    & \multicolumn{1}{l||}{\cellcolor{Gray2}3}    & \multicolumn{1}{l|}{\cellcolor{Gray2}$\beta=0.5$} & \multicolumn{1}{l|}{\cellcolor{Gray2}1}    & \multicolumn{1}{l|}{\cellcolor{Gray2}1.7}  & \multicolumn{1}{l|}{\cellcolor{Gray2}2}    & \multicolumn{1}{l|}{\cellcolor{Gray2}3}    \\ \hline \hline
\multicolumn{1}{l||}{250 - 500}      & \multicolumn{1}{l|}{4.6(0.1)}         & \multicolumn{1}{l|}{5.0(0.1)} & \multicolumn{1}{l|}{5.2(0.1)}  	    & \multicolumn{1}{l|}{5.3(0.1)}  & \multicolumn{1}{l||}{5.5(0.1)}  		& \multicolumn{1}{l|}{5.7(0.1)}         & \multicolumn{1}{l|}{7.3(0.2)}  & \multicolumn{1}{l|}{\cellcolor{Red!50}8.4(0.2)}  							& \multicolumn{1}{l|}{\cellcolor{Red!50}8.4(0.2)}  	    & \multicolumn{1}{l||}{\cellcolor{Red!50}8.4(0.2)}  & \multicolumn{1}{l|}{7.6(0.2)}         & \multicolumn{1}{l|}{8.0(0.2)}  & \multicolumn{1}{l|}{7.9(0.2)}  & \multicolumn{1}{l|}{7.8(0.2)}  & \multicolumn{1}{l|}{7.5(0.2)}  \\
\multicolumn{1}{l||}{500 - 800}      & \multicolumn{1}{l|}{15.7(0.3)}        & \multicolumn{1}{l|}{16.7(0.4)} & \multicolumn{1}{l|}{17.0(0.4)} 		& \multicolumn{1}{l|}{16.9(0.4)} & \multicolumn{1}{l||}{16.2(0.4)} 		& \multicolumn{1}{l|}{13.6(0.3)}        & \multicolumn{1}{l|}{16.9(0.4)} & \multicolumn{1}{l|}{\cellcolor{Red!50}17.7(0.4)} 		& \multicolumn{1}{l|}{17.2(0.4)} 						& \multicolumn{1}{l||}{15.2(0.3)} & \multicolumn{1}{l|}{16.7(0.4)}        & \multicolumn{1}{l|}{15.4(0.3)} & \multicolumn{1}{l|}{15.2(0.3)} & \multicolumn{1}{l|}{14.8(0.3)} & \multicolumn{1}{l|}{14.0(0.3)} \\
\multicolumn{1}{l||}{800 - 1200}     & \multicolumn{1}{l|}{22.1(0.5)}        & \multicolumn{1}{l|}{23.8(0.5)} & \multicolumn{1}{l|}{25.0(0.6)} 		& \multicolumn{1}{l|}{25.0(0.6)} & \multicolumn{1}{l||}{23.4(0.5)} 		& \multicolumn{1}{l|}{18.4(0.4)}        & \multicolumn{1}{l|}{23.7(0.6)} & \multicolumn{1}{l|}{\cellcolor{Red!50}26.3(0.6)} 							& \multicolumn{1}{l|}{25.6(0.6)} 		& \multicolumn{1}{l||}{22.3(0.5)} & \multicolumn{1}{l|}{22.8(0.5)}        & \multicolumn{1}{l|}{21.9(0.5)} & \multicolumn{1}{l|}{22.6(0.5)} & \multicolumn{1}{l|}{22.1(0.5)} & \multicolumn{1}{l|}{20.9(0.5)} \\
\multicolumn{1}{l||}{1200 - 1600}    & \multicolumn{1}{l|}{24.0(0.6)}        & \multicolumn{1}{l|}{26.0(0.8)} & \multicolumn{1}{l|}{26.4(0.8)} 		& \multicolumn{1}{l|}{25.9(0.7)} & \multicolumn{1}{l||}{23.0(0.6)} 		& \multicolumn{1}{l|}{19.3(0.6)}        & \multicolumn{1}{l|}{24.9(0.7)} & \multicolumn{1}{l|}{\cellcolor{Red!50}27.0(0.8)} 		& \multicolumn{1}{l|}{26.1(0.7)} 						& \multicolumn{1}{l||}{21.9(0.5)} & \multicolumn{1}{l|}{21.3(0.5)}        & \multicolumn{1}{l|}{22.6(0.6)} & \multicolumn{1}{l|}{24.0(0.6)} & \multicolumn{1}{l|}{23.7(0.6)} & \multicolumn{1}{l|}{22.2(0.5)} \\
\multicolumn{1}{l||}{1600 - 2000}    & \multicolumn{1}{l|}{12.1(0.7)}        & \multicolumn{1}{l|}{13.9(0.8)} & \multicolumn{1}{l|}{14.3(0.7)}  	& \multicolumn{1}{l|}{14.0(0.7)} & \multicolumn{1}{l||}{12.3(0.6)} 		& \multicolumn{1}{l|}{11.1(0.7)}        & \multicolumn{1}{l|}{14.1(0.9)} & \multicolumn{1}{l|}{\cellcolor{Red!50}14.9(0.8)} 		& \multicolumn{1}{l|}{14.2(0.6)} 						& \multicolumn{1}{l||}{11.8(0.5)} & \multicolumn{1}{l|}{10.3(0.5)}        & \multicolumn{1}{l|}{11.9(0.5)} & \multicolumn{1}{l|}{13.1(0.6)} & \multicolumn{1}{l|}{13.1(0.7)} & \multicolumn{1}{l|}{12.3(0.7)} \\ \hline
                                    &                                  &                           &                           &                	           &                           &                                  &                           &                           &                           &                           &                                  &                           &                           &                           &                           \\
\rowcolor{Gray}\multicolumn{1}{l||}{\textbf{TAS}}            &                                  &                           & C2                        &                           & \multicolumn{1}{l|}{}     &                                  &                           & D2                        &                           & \multicolumn{1}{l|}{}     &                                  &                           & $\tau_{21}$               &                           & \multicolumn{1}{l|}{}     \\ \hline
\multicolumn{1}{l||}{$p_{\mathrm{T}} \, [GeV]$} & \multicolumn{1}{l|}{\cellcolor{Gray2}$\beta=0.5$} & \multicolumn{1}{l|}{\cellcolor{Gray2}1}        & \multicolumn{1}{l|}{\cellcolor{Gray2}1.7}    & \multicolumn{1}{l|}{\cellcolor{Gray2}2}    & \multicolumn{1}{l||}{\cellcolor{Gray2}3}    & \multicolumn{1}{l|}{\cellcolor{Gray2}$\beta=0.5$} & \multicolumn{1}{l|}{\cellcolor{Gray2}1}    & \multicolumn{1}{l|}{\cellcolor{Gray2}1.7}  & \multicolumn{1}{l|}{\cellcolor{Gray2}2}    & \multicolumn{1}{l||}{\cellcolor{Gray2}3}    & \multicolumn{1}{l|}{\cellcolor{Gray2}$\beta=0.5$} & \multicolumn{1}{l|}{\cellcolor{Gray2}1}    & \multicolumn{1}{l|}{\cellcolor{Gray2}1.7}  & \multicolumn{1}{l|}{\cellcolor{Gray2}2}    & \multicolumn{1}{l|}{\cellcolor{Gray2}3}    \\ \hline \hline
\multicolumn{1}{l||}{250 - 500}      & \multicolumn{1}{l|}{4.8(0.1)}         & \multicolumn{1}{l|}{5.2(0.1)}  & \multicolumn{1}{l|}{5.5(0.1)}        & \multicolumn{1}{l|}{5.6(0.1)}  				& \multicolumn{1}{l||}{5.8(0.1)}    & \multicolumn{1}{l|}{5.9(0.1)}         	& \multicolumn{1}{l|}{7.6(0.2)}  	& \multicolumn{1}{l|}{8.5(0.2)}  	& \multicolumn{1}{l|}{\cellcolor{Red!50}8.6(0.2)}  		        & \multicolumn{1}{l||}{8.5(0.2)}  							& \multicolumn{1}{l|}{7.6(0.2)}         & \multicolumn{1}{l|}{8.0(0.2)}  &   \multicolumn{1}{l|}{7.7(0.2)}  &   \multicolumn{1}{l|}{7.6(0.2)}  & \multicolumn{1}{l|}{7.4(0.2)}  \\
\multicolumn{1}{l||}{500 - 800}      & \multicolumn{1}{l|}{16.1(0.4)}        & \multicolumn{1}{l|}{17.3(0.4)} & \multicolumn{1}{l|}{17.7(0.4)} 	     & \multicolumn{1}{l|}{17.7(0.4)} 				& \multicolumn{1}{l||}{17.6(0.4)} 	& \multicolumn{1}{l|}{14.0(0.3)}        	& \multicolumn{1}{l|}{18.2(0.4)} 	& \multicolumn{1}{l|}{\cellcolor{Red!50}18.7(0.4)} 		& \multicolumn{1}{l|}{18.3(0.4)} 						& \multicolumn{1}{l||}{16.9(0.4)} 							& \multicolumn{1}{l|}{16.2(0.4)}        & \multicolumn{1}{l|}{16.4(0.4)} & 	 \multicolumn{1}{l|}{15.4(0.4)} & 	\multicolumn{1}{l|}{15.1(0.3)} & \multicolumn{1}{l|}{14.6(0.3)} \\
\multicolumn{1}{l||}{800 - 1200}     & \multicolumn{1}{l|}{20.6(0.5)}        & \multicolumn{1}{l|}{23.5(0.5)} & \multicolumn{1}{l|}{26.2(0.6)} 		 & \multicolumn{1}{l|}{26.9(0.7)} 				& \multicolumn{1}{l||}{27.7(0.6)} 	& \multicolumn{1}{l|}{18.8(0.4)}        	& \multicolumn{1}{l|}{25.6(0.6)} 	& \multicolumn{1}{l|}{\cellcolor{Red!50}28.5(0.7)} 		& \multicolumn{1}{l|}{28.4(0.7)} 						& \multicolumn{1}{l||}{26.8(0.6)} 							& \multicolumn{1}{l|}{21.7(0.5)}        & \multicolumn{1}{l|}{22.4(0.5)} & 	 \multicolumn{1}{l|}{22.1(0.5)} & 	\multicolumn{1}{l|}{22.0(0.5)} & \multicolumn{1}{l|}{21.8(0.5)} \\
\multicolumn{1}{l||}{1200 - 1600}    & \multicolumn{1}{l|}{18.6(0.4)}        & \multicolumn{1}{l|}{22.6(0.6)} & \multicolumn{1}{l|}{27.4(0.7)} 		 & \multicolumn{1}{l|}{28.7(0.8)} 				& \multicolumn{1}{l||}{30.0(0.7)} 	& \multicolumn{1}{l|}{17.9(0.4)}        	& \multicolumn{1}{l|}{24.3(0.7)} 	& \multicolumn{1}{l|}{28.9(0.7)} 						& \multicolumn{1}{l|}{\cellcolor{Red!50}29.3(0.6)} 		& \multicolumn{1}{l||}{28.1(0.7)} 							& \multicolumn{1}{l|}{19.3(0.5)}        & \multicolumn{1}{l|}{20.0(0.5)} & 	 \multicolumn{1}{l|}{20.7(0.5)} & 	\multicolumn{1}{l|}{21.0(0.6)} & \multicolumn{1}{l|}{21.9(0.5)} \\
\multicolumn{1}{l||}{1600 - 2000}    & \multicolumn{1}{l|}{8.0(0.3)}         & \multicolumn{1}{l|}{11.3(0.5)} & \multicolumn{1}{l|}{15.4(0.9)} 		 & \multicolumn{1}{l|}{16.5(1.0)} 				& \multicolumn{1}{l||}{17.8(0.7)} 	& \multicolumn{1}{l|}{10.0(0.5)}        	& \multicolumn{1}{l|}{14.0(0.8)} 	& \multicolumn{1}{l|}{17.7(0.8)} 						& \multicolumn{1}{l|}{\cellcolor{Red!50}18.1(.9)} 		& \multicolumn{1}{l||}{17.9(0.6)} 							& \multicolumn{1}{l|}{9.8(0.4)}         & \multicolumn{1}{l|}{10.6(0.5)} & 	 \multicolumn{1}{l|}{11.4(0.6)} & 	\multicolumn{1}{l|}{11.8(0.6)} & \multicolumn{1}{l|}{12.6(0.6)} \\ \hline
                                    &                                  &                           &                           &                           &                           &                                  &                           &                           &                           &                           &                                  &                           &                           &                           &                           \\
\rowcolor{Gray} \multicolumn{1}{l||}{\textbf{Tracks}}         &                                  &                           & C2                        &                           & \multicolumn{1}{l|}{}     &                                  &                           & D2                        &                           & \multicolumn{1}{l|}{}     &                                  &                           & $\tau_{21}$               &                           & \multicolumn{1}{l|}{}     \\ \hline
\multicolumn{1}{l||}{$p_{\mathrm{T}} \, [GeV]$} & \multicolumn{1}{l|}{\cellcolor{Gray2}$\beta=0.5$} & \multicolumn{1}{l|}{\cellcolor{Gray2}1}    & \multicolumn{1}{l|}{\cellcolor{Gray2}1.7}  & \multicolumn{1}{l|}{\cellcolor{Gray2}2}    & \multicolumn{1}{l||}{\cellcolor{Gray2}3}    & \multicolumn{1}{l|}{\cellcolor{Gray2}$\beta=0.5$} & \multicolumn{1}{l|}{1\cellcolor{Gray2}}    & \multicolumn{1}{l|}{\cellcolor{Gray2}1.7}  & \multicolumn{1}{l|}{\cellcolor{Gray2}2}    & \multicolumn{1}{l||}{\cellcolor{Gray2}3}    & \multicolumn{1}{l|}{\cellcolor{Gray2}$\beta=0.5$} & \multicolumn{1}{l|}{\cellcolor{Gray2}1}    & \multicolumn{1}{l|}{\cellcolor{Gray2}1.7}  & \multicolumn{1}{l|}{\cellcolor{Gray2}2}    & \multicolumn{1}{l|}{\cellcolor{Gray2}3}    \\ \hline \hline
\multicolumn{1}{l||}{250 - 500}      & \multicolumn{1}{l|}{4.91(0.1)}       & \multicolumn{1}{l|}{5.2(0.1)}  		& \multicolumn{1}{l|}{5.5(0.1)}  		& \multicolumn{1}{l|}{5.6(0.1)}  								& \multicolumn{1}{l||}{5.9(0.1)}  					    & \multicolumn{1}{l|}{5.8(0.1)}         	& \multicolumn{1}{l|}{7.4(0.2)}  	& \multicolumn{1}{l|}{8.3(0.2)} 				 		& \multicolumn{1}{l|}{8.3(0.2)}  					& \multicolumn{1}{l||}{\cellcolor{Red!50}8.5(0.2)}  	& \multicolumn{1}{l|}{7.4(0.2)}         & \multicolumn{1}{l|}{7.9(0.2)}  	& \multicolumn{1}{l|}{7.8(0.2)}  	& \multicolumn{1}{l|}{7.7(0.2)}  	& \multicolumn{1}{l|}{7.6(0.2)}  \\
\multicolumn{1}{l||}{500 - 800}      & \multicolumn{1}{l|}{15.6(0.3)}        & \multicolumn{1}{l|}{17.2(0.4)} 		& \multicolumn{1}{l|}{17.8(0.4)} 		& \multicolumn{1}{l|}{\cellcolor{Red!50}17.9(0.4)} 				& \multicolumn{1}{l||}{17.7(0.4)} 						& \multicolumn{1}{l|}{13.5(0.3)}        	& \multicolumn{1}{l|}{17.1(0.4)} 	& \multicolumn{1}{l|}{\cellcolor{Red!50}17.9(0.4)}		& \multicolumn{1}{l|}{17.7(0.4)} 					& \multicolumn{1}{l||}{16.8(0.4)} 						& \multicolumn{1}{l|}{15.7(0.3)}        & \multicolumn{1}{l|}{16.1(0.4)} 	& \multicolumn{1}{l|}{15.5(0.3)} 	& \multicolumn{1}{l|}{15.3(0.3)} 	& \multicolumn{1}{l|}{14.8(0.1)} \\
\multicolumn{1}{l||}{800 - 1200}     & \multicolumn{1}{l|}{20.1(0.5)}        & \multicolumn{1}{l|}{24.0(0.5)} 		& \multicolumn{1}{l|}{26.9(0.6)} 		& \multicolumn{1}{l|}{27.7(0.7)} 								& \multicolumn{1}{l||}{\cellcolor{Red!50}28.4(0.6)} 	& \multicolumn{1}{l|}{18.8(0.4)}        	& \multicolumn{1}{l|}{25.3(0.6)} 	& \multicolumn{1}{l|}{28.0(0.7)} 						& \multicolumn{1}{l|}{28.0(0.7)} 					& \multicolumn{1}{l||}{26.9(0.6)} 						& \multicolumn{1}{l|}{22.0(0.5)}        & \multicolumn{1}{l|}{22.7(0.5)} 	& \multicolumn{1}{l|}{22.5(0.5)} 	& \multicolumn{1}{l|}{22.4(0.5)} 	& \multicolumn{1}{l|}{22.4(0.3)} \\
\multicolumn{1}{l||}{1200 - 1600}    & \multicolumn{1}{l|}{18.5(0.5)}        & \multicolumn{1}{l|}{23.8(0.6)} 		& \multicolumn{1}{l|}{28.8(0.8)} 		& \multicolumn{1}{l|}{30.0(0.8)} 								& \multicolumn{1}{l||}{\cellcolor{Red!50}31.1(0.7)} 						& \multicolumn{1}{l|}{19.4(0.5)}        	& \multicolumn{1}{l|}{26.3(0.7)} 	& \multicolumn{1}{l|}{30.0(0.8)} 						& \multicolumn{1}{l|}{30.3(0.8)} 	& \multicolumn{1}{l||}{29.2(0.7)} 						& \multicolumn{1}{l|}{20.8(0.5)}        & \multicolumn{1}{l|}{21.4(0.5)} 	& \multicolumn{1}{l|}{21.9(0.6)} 	& \multicolumn{1}{l|}{22.3(0.6)} 	& \multicolumn{1}{l|}{23.0(0.5)} \\
\multicolumn{1}{l||}{1600 - 2000}    & \multicolumn{1}{l|}{8.0(0.3)}         & \multicolumn{1}{l|}{11.7(0.5)} 		& \multicolumn{1}{l|}{16.1(0.9)} 		& \multicolumn{1}{l|}{17.1(0.9)} 								& \multicolumn{1}{l||}{18.3(0.9)} 						& \multicolumn{1}{l|}{11.0(0.7)}        	& \multicolumn{1}{l|}{15.5(0.7)} 	& \multicolumn{1}{l|}{18.5(0.8)} 						& \multicolumn{1}{l|}{\cellcolor{Red!50}18.7(0.8)} 	& \multicolumn{1}{l||}{18.4(0.6)} 						& \multicolumn{1}{l|}{10.4(0.5)}        & \multicolumn{1}{l|}{11.1(0.5)} 	& \multicolumn{1}{l|}{12.0(0.6)} 	& \multicolumn{1}{l|}{12.4(0.7)} 	& \multicolumn{1}{l|}{13.2(0.6)} \\ \hline
\end{tabular}}
\caption{Listing of the QCD rejection for Higgs jets achieved with C2, D2 and $\tau_{21}$ calculated with varying angular weightings $\beta$ and constituents. The highest achieved background rejection per energy range is highlighted in red.}\label{table:higgs_scan}
\end{sidewaystable}
\FloatBarrier
\subsection{Performance for Top tagging}\label{appendix:top_optimisation}
\begin{table}[H]
\centering
\begin{tabular}{lllll}
\rowcolor{Gray} \multicolumn{1}{l||}{\textbf{Calorimeter}}    &                                & $\tau_{32}$               &                           & \multicolumn{1}{l|}{}     \\ \hline
\multicolumn{1}{l||}{$p_{\mathrm{T}} \, [GeV]$} & \multicolumn{1}{l|}{$\beta=1$} & \multicolumn{1}{l|}{\cellcolor{Gray2}1.7}  & \multicolumn{1}{l|}{\cellcolor{Gray2}2}    & \multicolumn{1}{l|}{\cellcolor{Gray2}3}    \\ \hline \hline
\multicolumn{1}{l||}{250 - 500}      & \multicolumn{1}{l|}{\cellcolor{Red!50}9.7 $\pm$ 0.2}       	& \multicolumn{1}{l|}{9.5 $\pm$ 0.2}  					& \multicolumn{1}{l|}{9.5 $\pm$ 0.4}  					& \multicolumn{1}{l|}{9.4 $\pm$ 0.2}  \\
\multicolumn{1}{l||}{500 - 800}      & \multicolumn{1}{l|}{20.1 $\pm$ 0.5}      					& \multicolumn{1}{l|}{22.2 $\pm$ 0.6} 					& \multicolumn{1}{l|}{\cellcolor{Red!50}22.4 $\pm$ 0.6} & \multicolumn{1}{l|}{22.0 $\pm$ 0.6} \\
\multicolumn{1}{l||}{800 - 1200}     & \multicolumn{1}{l|}{17.3 $\pm$ 0.4}      					& \multicolumn{1}{l|}{20.3 $\pm$ 0.5} 					& \multicolumn{1}{l|}{\cellcolor{Red!50}20.6 $\pm$ 0.5} & \multicolumn{1}{l|}{20.3 $\pm$ 0.5} \\
\multicolumn{1}{l||}{1200 - 1600}    & \multicolumn{1}{l|}{14.3 $\pm$ 0.3}      					& \multicolumn{1}{l|}{16.4 $\pm$ 0.4} 					& \multicolumn{1}{l|}{\cellcolor{Red!50}16.6 $\pm$ 0.5} & \multicolumn{1}{l|}{16.1 $\pm$ 0.5} \\
\multicolumn{1}{l||}{1600 - 2000}    & \multicolumn{1}{l|}{11.7 $\pm$ 0.3}      					& \multicolumn{1}{l|}{\cellcolor{Red!50}13.3 $\pm$ 0.4} & \multicolumn{1}{l|}{\cellcolor{Red!50}13.3 $\pm$ 0.4} & \multicolumn{1}{l|}{12.6 $\pm$ 0.3} \\
\multicolumn{1}{l||}{$>2000$}        & \multicolumn{1}{l|}{9.6 $\pm$ 0.3}       					& \multicolumn{1}{l|}{\cellcolor{Red!50}11.0 $\pm$ 0.4} & \multicolumn{1}{l|}{10.9 $\pm$ 0.4} 					& \multicolumn{1}{l|}{10.1 $\pm$ 0.3} \\ \hline
                                    &                                &                           &                           &                           \\
\rowcolor{Gray} \multicolumn{1}{l||}{\textbf{TAS}}            &                                & $\tau_{32}$               &                           & \multicolumn{1}{l|}{}     \\ \hline
\multicolumn{1}{l||}{$p_{\mathrm{T}} \, [GeV]$} & \multicolumn{1}{l|}{\cellcolor{Gray2}$\beta=1$} & \multicolumn{1}{l|}{\cellcolor{Gray2}1.7}  & \multicolumn{1}{l|}{\cellcolor{Gray2}2}    & \multicolumn{1}{l|}{\cellcolor{Gray2}3}    \\ \hline \hline
\multicolumn{1}{l||}{250 - 500}      & \multicolumn{1}{l|}{\cellcolor{Red!50}10.7 $\pm$ 0.2}      	& \multicolumn{1}{l|}{10.1 $\pm$ 0.2} 					& \multicolumn{1}{l|}{9.9 $\pm$ 0.2}  & \multicolumn{1}{l|}{9.6 $\pm$ 0.2}  \\
\multicolumn{1}{l||}{500 - 800}      & \multicolumn{1}{l|}{\cellcolor{Red!50}22.8 $\pm$ 0.6}      	& \multicolumn{1}{l|}{\cellcolor{Red!50}22.8 $\pm$ 0.6} & \multicolumn{1}{l|}{22.5 $\pm$ 0.6} & \multicolumn{1}{l|}{21.6 $\pm$ 0.6} \\
\multicolumn{1}{l||}{800 - 1200}     & \multicolumn{1}{l|}{23.6 $\pm$ 0.6}      					& \multicolumn{1}{l|}{\cellcolor{Red!50}24.1 $\pm$ 0.6} & \multicolumn{1}{l|}{23.6 $\pm$ 0.6} & \multicolumn{1}{l|}{22.2 $\pm$ 0.5} \\
\multicolumn{1}{l||}{1200 - 1600}    & \multicolumn{1}{l|}{22.0 $\pm$ 0.6}      					& \multicolumn{1}{l|}{\cellcolor{Red!50}22.3 $\pm$ 0.6} & \multicolumn{1}{l|}{21.7 $\pm$ 0.6} & \multicolumn{1}{l|}{19.8 $\pm$ 0.6} \\
\multicolumn{1}{l||}{1600 - 2000}    & \multicolumn{1}{l|}{\cellcolor{Red!50}18.9 $\pm$ 0.6}      	& \multicolumn{1}{l|}{18.8 $\pm$ 0.6} 					& \multicolumn{1}{l|}{17.9 $\pm$ 0.5} & \multicolumn{1}{l|}{16.0 $\pm$ 0.5} \\
\multicolumn{1}{l||}{$>2000$}        & \multicolumn{1}{l|}{\cellcolor{Red!50}16.5 $\pm$ 0.7}      	& \multicolumn{1}{l|}{15.7 $\pm$ 0.7} 					& \multicolumn{1}{l|}{15.2 $\pm$ 0.7} & \multicolumn{1}{l|}{13.1 $\pm$ 0.6} \\ \hline
                                    &                                &                           &                           &                           \\
\rowcolor{Gray} \multicolumn{1}{l||}{\textbf{Tracks}}         &                                & $\tau_{32}$               &                           & \multicolumn{1}{l|}{}     \\ \hline
\multicolumn{1}{l||}{$p_{\mathrm{T}} \, [GeV]$} & \multicolumn{1}{l|}{\cellcolor{Gray2}$\beta=1$} & \multicolumn{1}{l|}{\cellcolor{Gray2}1.7}  & \multicolumn{1}{l|}{\cellcolor{Gray2}2}    & \multicolumn{1}{l|}{\cellcolor{Gray2}3}    \\ \hline \hline
\multicolumn{1}{l||}{250 - 500}      & \multicolumn{1}{l|}{\cellcolor{Red!50}10.5 $\pm$ 0.2}      	& \multicolumn{1}{l|}{9.8 $\pm$ 0.2}  					& \multicolumn{1}{l|}{9.6 $\pm$ 0.2}  & \multicolumn{1}{l|}{9.4 $\pm$ 0.2}  \\
\multicolumn{1}{l||}{500 - 800}      & \multicolumn{1}{l|}{20.6 $\pm$ 0.5}      					& \multicolumn{1}{l|}{\cellcolor{Red!50}21.3 $\pm$ 0.6} & \multicolumn{1}{l|}{21.1 $\pm$ 0.5} & \multicolumn{1}{l|}{20.3 $\pm$ 0.5} \\
\multicolumn{1}{l||}{800 - 1200}     & \multicolumn{1}{l|}{21.8 $\pm$ 0.6}      					& \multicolumn{1}{l|}{\cellcolor{Red!50}22.9 $\pm$ 0.6} & \multicolumn{1}{l|}{22.6 $\pm$ 0.6} & \multicolumn{1}{l|}{21.4 $\pm$ 0.6} \\
\multicolumn{1}{l||}{1200 - 1600}    & \multicolumn{1}{l|}{21.7 $\pm$ 0.6}      					& \multicolumn{1}{l|}{\cellcolor{Red!50}22.1 $\pm$ 0.6} & \multicolumn{1}{l|}{21.6 $\pm$ 0.6} & \multicolumn{1}{l|}{19.5 $\pm$ 0.6} \\
\multicolumn{1}{l||}{1600 - 2000}    & \multicolumn{1}{l|}{\cellcolor{Red!50}19.3 $\pm$ 0.6}      	& \multicolumn{1}{l|}{19.0 $\pm$ 0.6} 					& \multicolumn{1}{l|}{18.2 $\pm$ 0.6} & \multicolumn{1}{l|}{16.0 $\pm$ 0.5} \\
\multicolumn{1}{l||}{$>2000$}        & \multicolumn{1}{l|}{\cellcolor{Red!50}16.8 $\pm$ 0.7}      	& \multicolumn{1}{l|}{15.8 $\pm$ 0.7} 					& \multicolumn{1}{l|}{15.1 $\pm$ 0.7} & \multicolumn{1}{l|}{13.0 $\pm$ 0.5} \\ \hline
\end{tabular}
\caption{Listing of the QCD rejection for Top jets achieved with $\tau_{32}$ calculated with varying angular weightings $\beta$ and constituents. The highest achieved background rejection per energy range is highlighted in red.}\label{table:top_scan}
\end{table}


\FloatBarrier
\section{Signal and Background Distributions}\label{app:distris}
\vspace{-0.5cm}
\subsection{$W$ Distributions}
\subsubsection*{$\beta=0.5$}
\vspace{-0.5cm}
\begin{figure}[H]
\includegraphics[width=0.3\textwidth]{sascha_input/Appendix/Distributions/w/distributions/beta05/h_assisted_tj_C2_05_bin1.pdf} \hspace{1mm}
\includegraphics[width=0.3\textwidth]{sascha_input/Appendix/Distributions/w/distributions/beta05/h_assisted_tj_C2_05_bin2.pdf} \hspace{1mm}
\includegraphics[width=0.3\textwidth]{sascha_input/Appendix/Distributions/w/distributions/beta05/h_assisted_tj_C2_05_bin3.pdf} 
\bigskip
\includegraphics[width=0.3\textwidth]{sascha_input/Appendix/Distributions/w/distributions/beta05/h_assisted_tj_C2_05_bin4.pdf} \hspace{1mm}
\includegraphics[width=0.3\textwidth]{sascha_input/Appendix/Distributions/w/distributions/beta05/h_assisted_tj_C2_05_bin5.pdf} \hspace{1mm}
\includegraphics[width=0.3\textwidth]{sascha_input/Appendix/Distributions/w/distributions/beta05/h_assisted_tj_C2_05_bin6.pdf} 
\bigskip
\includegraphics[width=0.3\textwidth]{sascha_input/Appendix/Distributions/w/distributions/beta05/h_assisted_tj_D2_05_bin1.pdf} \hspace{1mm}
\includegraphics[width=0.3\textwidth]{sascha_input/Appendix/Distributions/w/distributions/beta05/h_assisted_tj_D2_05_bin2.pdf} \hspace{1mm}
\includegraphics[width=0.3\textwidth]{sascha_input/Appendix/Distributions/w/distributions/beta05/h_assisted_tj_D2_05_bin3.pdf} 
\bigskip
\includegraphics[width=0.3\textwidth]{sascha_input/Appendix/Distributions/w/distributions/beta05/h_assisted_tj_D2_05_bin4.pdf} \hspace{1mm}
\includegraphics[width=0.3\textwidth]{sascha_input/Appendix/Distributions/w/distributions/beta05/h_assisted_tj_D2_05_bin5.pdf} \hspace{1mm}
\includegraphics[width=0.3\textwidth]{sascha_input/Appendix/Distributions/w/distributions/beta05/h_assisted_tj_D2_05_bin6.pdf}
\bigskip 
\includegraphics[width=0.3\textwidth]{sascha_input/Appendix/Distributions/w/distributions/beta05/h_assisted_tj_nSub21_05_bin1.pdf} \hspace{1mm}
\includegraphics[width=0.3\textwidth]{sascha_input/Appendix/Distributions/w/distributions/beta05/h_assisted_tj_nSub21_05_bin2.pdf} \hspace{1mm}
\includegraphics[width=0.3\textwidth]{sascha_input/Appendix/Distributions/w/distributions/beta05/h_assisted_tj_nSub21_05_bin3.pdf} 
\bigskip
\includegraphics[width=0.3\textwidth]{sascha_input/Appendix/Distributions/w/distributions/beta05/h_assisted_tj_nSub21_05_bin4.pdf} \hspace{6mm}
\includegraphics[width=0.3\textwidth]{sascha_input/Appendix/Distributions/w/distributions/beta05/h_assisted_tj_nSub21_05_bin5.pdf} \hspace{6mm}
\includegraphics[width=0.3\textwidth]{sascha_input/Appendix/Distributions/w/distributions/beta05/h_assisted_tj_nSub21_05_bin6.pdf} 
\vspace{-1.25cm}
\caption{\footnotesize{Distributions for $W$ boson tagging using TAS $\beta=0.5$. C2, D2, $\tau_{21}$ top down.}}
\end{figure}
\begin{figure}[H]
\includegraphics[width=0.3\textwidth]{sascha_input/Appendix/Distributions/w/distributions/beta05/h_normal_tj_C2_05_bin1.pdf} \hspace{1mm}
\includegraphics[width=0.3\textwidth]{sascha_input/Appendix/Distributions/w/distributions/beta05/h_normal_tj_C2_05_bin2.pdf} \hspace{1mm}
\includegraphics[width=0.3\textwidth]{sascha_input/Appendix/Distributions/w/distributions/beta05/h_normal_tj_C2_05_bin3.pdf} 
\bigskip
\includegraphics[width=0.3\textwidth]{sascha_input/Appendix/Distributions/w/distributions/beta05/h_normal_tj_C2_05_bin4.pdf} \hspace{1mm}
\includegraphics[width=0.3\textwidth]{sascha_input/Appendix/Distributions/w/distributions/beta05/h_normal_tj_C2_05_bin5.pdf} \hspace{1mm}
\includegraphics[width=0.3\textwidth]{sascha_input/Appendix/Distributions/w/distributions/beta05/h_normal_tj_C2_05_bin6.pdf} 
\bigskip
\includegraphics[width=0.3\textwidth]{sascha_input/Appendix/Distributions/w/distributions/beta05/h_normal_tj_D2_05_bin1.pdf} \hspace{1mm}
\includegraphics[width=0.3\textwidth]{sascha_input/Appendix/Distributions/w/distributions/beta05/h_normal_tj_D2_05_bin2.pdf} \hspace{1mm}
\includegraphics[width=0.3\textwidth]{sascha_input/Appendix/Distributions/w/distributions/beta05/h_normal_tj_D2_05_bin3.pdf} 
\bigskip
\includegraphics[width=0.3\textwidth]{sascha_input/Appendix/Distributions/w/distributions/beta05/h_normal_tj_D2_05_bin4.pdf} \hspace{1mm}
\includegraphics[width=0.3\textwidth]{sascha_input/Appendix/Distributions/w/distributions/beta05/h_normal_tj_D2_05_bin5.pdf} \hspace{1mm}
\includegraphics[width=0.3\textwidth]{sascha_input/Appendix/Distributions/w/distributions/beta05/h_normal_tj_D2_05_bin6.pdf}
\bigskip
\includegraphics[width=0.3\textwidth]{sascha_input/Appendix/Distributions/w/distributions/beta05/h_normal_tj_nSub21_05_bin1.pdf} \hspace{1mm}
\includegraphics[width=0.3\textwidth]{sascha_input/Appendix/Distributions/w/distributions/beta05/h_normal_tj_nSub21_05_bin2.pdf} \hspace{1mm}
\includegraphics[width=0.3\textwidth]{sascha_input/Appendix/Distributions/w/distributions/beta05/h_normal_tj_nSub21_05_bin3.pdf} 
\bigskip
\includegraphics[width=0.3\textwidth]{sascha_input/Appendix/Distributions/w/distributions/beta05/h_normal_tj_nSub21_05_bin4.pdf} \hspace{6mm}
\includegraphics[width=0.3\textwidth]{sascha_input/Appendix/Distributions/w/distributions/beta05/h_normal_tj_nSub21_05_bin5.pdf} \hspace{6mm}
\includegraphics[width=0.3\textwidth]{sascha_input/Appendix/Distributions/w/distributions/beta05/h_normal_tj_nSub21_05_bin6.pdf}
\caption{\footnotesize{Distributions for $W$ boson tagging using tracks $\beta=0.5$. C2, D2, $\tau_{21}$ top down.}}
\end{figure}
\begin{figure}[H]
\includegraphics[width=0.3\textwidth]{sascha_input/Appendix/Distributions/w/distributions/beta05/h_recoJet_C2_05_bin1.pdf} \hspace{1mm}
\includegraphics[width=0.3\textwidth]{sascha_input/Appendix/Distributions/w/distributions/beta05/h_recoJet_C2_05_bin2.pdf} \hspace{1mm}
\includegraphics[width=0.3\textwidth]{sascha_input/Appendix/Distributions/w/distributions/beta05/h_recoJet_C2_05_bin3.pdf} 
\bigskip
\includegraphics[width=0.3\textwidth]{sascha_input/Appendix/Distributions/w/distributions/beta05/h_recoJet_C2_05_bin4.pdf} \hspace{1mm}
\includegraphics[width=0.3\textwidth]{sascha_input/Appendix/Distributions/w/distributions/beta05/h_recoJet_C2_05_bin5.pdf} \hspace{1mm}
\includegraphics[width=0.3\textwidth]{sascha_input/Appendix/Distributions/w/distributions/beta05/h_recoJet_C2_05_bin6.pdf}
\bigskip
\includegraphics[width=0.3\textwidth]{sascha_input/Appendix/Distributions/w/distributions/beta05/h_recoJet_D2_05_bin1.pdf} \hspace{1mm}
\includegraphics[width=0.3\textwidth]{sascha_input/Appendix/Distributions/w/distributions/beta05/h_recoJet_D2_05_bin2.pdf} \hspace{1mm}
\includegraphics[width=0.3\textwidth]{sascha_input/Appendix/Distributions/w/distributions/beta05/h_recoJet_D2_05_bin3.pdf} 
\bigskip
\includegraphics[width=0.3\textwidth]{sascha_input/Appendix/Distributions/w/distributions/beta05/h_recoJet_D2_05_bin4.pdf} \hspace{1mm}
\includegraphics[width=0.3\textwidth]{sascha_input/Appendix/Distributions/w/distributions/beta05/h_recoJet_D2_05_bin5.pdf} \hspace{1mm}
\includegraphics[width=0.3\textwidth]{sascha_input/Appendix/Distributions/w/distributions/beta05/h_recoJet_D2_05_bin6.pdf}
\bigskip
\includegraphics[width=0.3\textwidth]{sascha_input/Appendix/Distributions/w/distributions/beta05/h_recoJet_nSub21_05_bin1.pdf} \hspace{1mm}
\includegraphics[width=0.3\textwidth]{sascha_input/Appendix/Distributions/w/distributions/beta05/h_recoJet_nSub21_05_bin2.pdf} \hspace{1mm}
\includegraphics[width=0.3\textwidth]{sascha_input/Appendix/Distributions/w/distributions/beta05/h_recoJet_nSub21_05_bin3.pdf} 
\bigskip
\includegraphics[width=0.3\textwidth]{sascha_input/Appendix/Distributions/w/distributions/beta05/h_recoJet_nSub21_05_bin4.pdf} \hspace{6mm}
\includegraphics[width=0.3\textwidth]{sascha_input/Appendix/Distributions/w/distributions/beta05/h_recoJet_nSub21_05_bin5.pdf} \hspace{6mm}
\includegraphics[width=0.3\textwidth]{sascha_input/Appendix/Distributions/w/distributions/beta05/h_recoJet_nSub21_05_bin6.pdf}
\caption{\footnotesize{Distributions for $W$ boson tagging using calorimeter clusters $\beta=0.5$. C2, D2, $\tau_{21}$ top down.}}
\end{figure}
\subsubsection*{$\beta=1$}\label{subsec:app_w_1}
\vspace{-0.5cm}
\begin{figure}[H]
\includegraphics[width=0.3\textwidth]{sascha_input/Appendix/Distributions/w/distributions/beta1/h_assisted_tj_C2_bin1.pdf} \hspace{1mm}
\includegraphics[width=0.3\textwidth]{sascha_input/Appendix/Distributions/w/distributions/beta1/h_assisted_tj_C2_bin2.pdf} \hspace{1mm}
\includegraphics[width=0.3\textwidth]{sascha_input/Appendix/Distributions/w/distributions/beta1/h_assisted_tj_C2_bin3.pdf} 
\bigskip
\includegraphics[width=0.3\textwidth]{sascha_input/Appendix/Distributions/w/distributions/beta1/h_assisted_tj_C2_bin4.pdf} \hspace{1mm}
\includegraphics[width=0.3\textwidth]{sascha_input/Appendix/Distributions/w/distributions/beta1/h_assisted_tj_C2_bin5.pdf} \hspace{1mm}
\includegraphics[width=0.3\textwidth]{sascha_input/Appendix/Distributions/w/distributions/beta1/h_assisted_tj_C2_bin6.pdf} 
\bigskip
\includegraphics[width=0.3\textwidth]{sascha_input/Appendix/Distributions/w/distributions/beta1/h_assisted_tj_D2_bin1.pdf} \hspace{1mm}
\includegraphics[width=0.3\textwidth]{sascha_input/Appendix/Distributions/w/distributions/beta1/h_assisted_tj_D2_bin2.pdf} \hspace{1mm}
\includegraphics[width=0.3\textwidth]{sascha_input/Appendix/Distributions/w/distributions/beta1/h_assisted_tj_D2_bin3.pdf} 
\bigskip
\includegraphics[width=0.3\textwidth]{sascha_input/Appendix/Distributions/w/distributions/beta1/h_assisted_tj_D2_bin4.pdf} \hspace{1mm}
\includegraphics[width=0.3\textwidth]{sascha_input/Appendix/Distributions/w/distributions/beta1/h_assisted_tj_D2_bin5.pdf} \hspace{1mm}
\includegraphics[width=0.3\textwidth]{sascha_input/Appendix/Distributions/w/distributions/beta1/h_assisted_tj_D2_bin6.pdf}
\bigskip 
\includegraphics[width=0.3\textwidth]{sascha_input/Appendix/Distributions/w/distributions/beta1/h_assisted_tj_nSub21_bin1.pdf} \hspace{1mm}
\includegraphics[width=0.3\textwidth]{sascha_input/Appendix/Distributions/w/distributions/beta1/h_assisted_tj_nSub21_bin2.pdf} \hspace{1mm}
\includegraphics[width=0.3\textwidth]{sascha_input/Appendix/Distributions/w/distributions/beta1/h_assisted_tj_nSub21_bin3.pdf} 
\bigskip
\includegraphics[width=0.3\textwidth]{sascha_input/Appendix/Distributions/w/distributions/beta1/h_assisted_tj_nSub21_bin4.pdf} \hspace{6mm}
\includegraphics[width=0.3\textwidth]{sascha_input/Appendix/Distributions/w/distributions/beta1/h_assisted_tj_nSub21_bin5.pdf} \hspace{6mm}
\includegraphics[width=0.3\textwidth]{sascha_input/Appendix/Distributions/w/distributions/beta1/h_assisted_tj_nSub21_bin6.pdf} 
\vspace{-0.5cm}
\caption{\footnotesize{Distributions for $W$ boson tagging using TAS $\beta=1$. C2, D2, $\tau_{21}$ top down.}}\label{app:W_TAS_1}
\end{figure}
\begin{figure}[H]
\includegraphics[width=0.3\textwidth]{sascha_input/Appendix/Distributions/w/distributions/beta1/h_normal_tj_C2_bin1.pdf} \hspace{1mm}
\includegraphics[width=0.3\textwidth]{sascha_input/Appendix/Distributions/w/distributions/beta1/h_normal_tj_C2_bin2.pdf} \hspace{1mm}
\includegraphics[width=0.3\textwidth]{sascha_input/Appendix/Distributions/w/distributions/beta1/h_normal_tj_C2_bin3.pdf} 
\bigskip
\includegraphics[width=0.3\textwidth]{sascha_input/Appendix/Distributions/w/distributions/beta1/h_normal_tj_C2_bin4.pdf} \hspace{1mm}
\includegraphics[width=0.3\textwidth]{sascha_input/Appendix/Distributions/w/distributions/beta1/h_normal_tj_C2_bin5.pdf} \hspace{1mm}
\includegraphics[width=0.3\textwidth]{sascha_input/Appendix/Distributions/w/distributions/beta1/h_normal_tj_C2_bin6.pdf} 
\bigskip
\includegraphics[width=0.3\textwidth]{sascha_input/Appendix/Distributions/w/distributions/beta1/h_normal_tj_D2_bin1.pdf} \hspace{1mm}
\includegraphics[width=0.3\textwidth]{sascha_input/Appendix/Distributions/w/distributions/beta1/h_normal_tj_D2_bin2.pdf} \hspace{1mm}
\includegraphics[width=0.3\textwidth]{sascha_input/Appendix/Distributions/w/distributions/beta1/h_normal_tj_D2_bin3.pdf} 
\bigskip
\includegraphics[width=0.3\textwidth]{sascha_input/Appendix/Distributions/w/distributions/beta1/h_normal_tj_D2_bin4.pdf} \hspace{1mm}
\includegraphics[width=0.3\textwidth]{sascha_input/Appendix/Distributions/w/distributions/beta1/h_normal_tj_D2_bin5.pdf} \hspace{1mm}
\includegraphics[width=0.3\textwidth]{sascha_input/Appendix/Distributions/w/distributions/beta1/h_normal_tj_D2_bin6.pdf}
\bigskip
\includegraphics[width=0.3\textwidth]{sascha_input/Appendix/Distributions/w/distributions/beta1/h_normal_tj_nSub21_bin1.pdf} \hspace{1mm}
\includegraphics[width=0.3\textwidth]{sascha_input/Appendix/Distributions/w/distributions/beta1/h_normal_tj_nSub21_bin2.pdf} \hspace{1mm}
\includegraphics[width=0.3\textwidth]{sascha_input/Appendix/Distributions/w/distributions/beta1/h_normal_tj_nSub21_bin3.pdf} 
\bigskip
\includegraphics[width=0.3\textwidth]{sascha_input/Appendix/Distributions/w/distributions/beta1/h_normal_tj_nSub21_bin4.pdf} \hspace{6mm}
\includegraphics[width=0.3\textwidth]{sascha_input/Appendix/Distributions/w/distributions/beta1/h_normal_tj_nSub21_bin5.pdf} \hspace{6mm}
\includegraphics[width=0.3\textwidth]{sascha_input/Appendix/Distributions/w/distributions/beta1/h_normal_tj_nSub21_bin6.pdf}
\caption{\footnotesize{Distributions for $W$ boson tagging using tracks $\beta=1$. C2, D2, $\tau_{21}$ top down.}}\label{app:W_track_1}
\end{figure}
\begin{figure}[H]
\includegraphics[width=0.3\textwidth]{sascha_input/Appendix/Distributions/w/distributions/beta1/h_recoJet_C2_bin1.pdf} \hspace{1mm}
\includegraphics[width=0.3\textwidth]{sascha_input/Appendix/Distributions/w/distributions/beta1/h_recoJet_C2_bin2.pdf} \hspace{1mm}
\includegraphics[width=0.3\textwidth]{sascha_input/Appendix/Distributions/w/distributions/beta1/h_recoJet_C2_bin3.pdf} 
\bigskip
\includegraphics[width=0.3\textwidth]{sascha_input/Appendix/Distributions/w/distributions/beta1/h_recoJet_C2_bin4.pdf} \hspace{1mm}
\includegraphics[width=0.3\textwidth]{sascha_input/Appendix/Distributions/w/distributions/beta1/h_recoJet_C2_bin5.pdf} \hspace{1mm}
\includegraphics[width=0.3\textwidth]{sascha_input/Appendix/Distributions/w/distributions/beta1/h_recoJet_C2_bin6.pdf}
\bigskip
\includegraphics[width=0.3\textwidth]{sascha_input/Appendix/Distributions/w/distributions/beta1/h_recoJet_D2_bin1.pdf} \hspace{1mm}
\includegraphics[width=0.3\textwidth]{sascha_input/Appendix/Distributions/w/distributions/beta1/h_recoJet_D2_bin2.pdf} \hspace{1mm}
\includegraphics[width=0.3\textwidth]{sascha_input/Appendix/Distributions/w/distributions/beta1/h_recoJet_D2_bin3.pdf} 
\bigskip
\includegraphics[width=0.3\textwidth]{sascha_input/Appendix/Distributions/w/distributions/beta1/h_recoJet_D2_bin4.pdf} \hspace{1mm}
\includegraphics[width=0.3\textwidth]{sascha_input/Appendix/Distributions/w/distributions/beta1/h_recoJet_D2_bin5.pdf} \hspace{1mm}
\includegraphics[width=0.3\textwidth]{sascha_input/Appendix/Distributions/w/distributions/beta1/h_recoJet_D2_bin6.pdf}
\bigskip
\includegraphics[width=0.3\textwidth]{sascha_input/Appendix/Distributions/w/distributions/beta1/h_recoJet_nSub21_bin1.pdf} \hspace{1mm}
\includegraphics[width=0.3\textwidth]{sascha_input/Appendix/Distributions/w/distributions/beta1/h_recoJet_nSub21_bin2.pdf} \hspace{1mm}
\includegraphics[width=0.3\textwidth]{sascha_input/Appendix/Distributions/w/distributions/beta1/h_recoJet_nSub21_bin3.pdf} 
\bigskip
\includegraphics[width=0.3\textwidth]{sascha_input/Appendix/Distributions/w/distributions/beta1/h_recoJet_nSub21_bin4.pdf} \hspace{6mm}
\includegraphics[width=0.3\textwidth]{sascha_input/Appendix/Distributions/w/distributions/beta1/h_recoJet_nSub21_bin5.pdf} \hspace{6mm}
\includegraphics[width=0.3\textwidth]{sascha_input/Appendix/Distributions/w/distributions/beta1/h_recoJet_nSub21_bin6.pdf}
\caption{\footnotesize{Distributions for $W$ boson tagging using calorimeter clusters $\beta=1$. C2, D2, $\tau_{21}$ top down.}}\label{app:W_calo_1}
\end{figure}
\subsubsection*{$\beta=1.7$}
\begin{figure}[H]
\includegraphics[width=0.3\textwidth]{sascha_input/Appendix/Distributions/w/distributions/beta17/h_assisted_tj_C2_17_bin1.pdf} \hspace{1mm}
\includegraphics[width=0.3\textwidth]{sascha_input/Appendix/Distributions/w/distributions/beta17/h_assisted_tj_C2_17_bin2.pdf} \hspace{1mm}
\includegraphics[width=0.3\textwidth]{sascha_input/Appendix/Distributions/w/distributions/beta17/h_assisted_tj_C2_17_bin3.pdf} 
\bigskip
\includegraphics[width=0.3\textwidth]{sascha_input/Appendix/Distributions/w/distributions/beta17/h_assisted_tj_C2_17_bin4.pdf} \hspace{1mm}
\includegraphics[width=0.3\textwidth]{sascha_input/Appendix/Distributions/w/distributions/beta17/h_assisted_tj_C2_17_bin5.pdf} \hspace{1mm}
\includegraphics[width=0.3\textwidth]{sascha_input/Appendix/Distributions/w/distributions/beta17/h_assisted_tj_C2_17_bin6.pdf} 
\bigskip
\includegraphics[width=0.3\textwidth]{sascha_input/Appendix/Distributions/w/distributions/beta17/h_assisted_tj_D2_17_bin1.pdf} \hspace{1mm}
\includegraphics[width=0.3\textwidth]{sascha_input/Appendix/Distributions/w/distributions/beta17/h_assisted_tj_D2_17_bin2.pdf} \hspace{1mm}
\includegraphics[width=0.3\textwidth]{sascha_input/Appendix/Distributions/w/distributions/beta17/h_assisted_tj_D2_17_bin3.pdf} 
\bigskip
\includegraphics[width=0.3\textwidth]{sascha_input/Appendix/Distributions/w/distributions/beta17/h_assisted_tj_D2_17_bin4.pdf} \hspace{1mm}
\includegraphics[width=0.3\textwidth]{sascha_input/Appendix/Distributions/w/distributions/beta17/h_assisted_tj_D2_17_bin5.pdf} \hspace{1mm}
\includegraphics[width=0.3\textwidth]{sascha_input/Appendix/Distributions/w/distributions/beta17/h_assisted_tj_D2_17_bin6.pdf}
\bigskip 
\includegraphics[width=0.3\textwidth]{sascha_input/Appendix/Distributions/w/distributions/beta17/h_assisted_tj_nSub21_17_bin1.pdf} \hspace{1mm}
\includegraphics[width=0.3\textwidth]{sascha_input/Appendix/Distributions/w/distributions/beta17/h_assisted_tj_nSub21_17_bin2.pdf} \hspace{1mm}
\includegraphics[width=0.3\textwidth]{sascha_input/Appendix/Distributions/w/distributions/beta17/h_assisted_tj_nSub21_17_bin3.pdf} 
\bigskip
\includegraphics[width=0.3\textwidth]{sascha_input/Appendix/Distributions/w/distributions/beta17/h_assisted_tj_nSub21_17_bin4.pdf} \hspace{6mm}
\includegraphics[width=0.3\textwidth]{sascha_input/Appendix/Distributions/w/distributions/beta17/h_assisted_tj_nSub21_17_bin5.pdf} \hspace{6mm}
\includegraphics[width=0.3\textwidth]{sascha_input/Appendix/Distributions/w/distributions/beta17/h_assisted_tj_nSub21_17_bin6.pdf} 
\caption{\footnotesize{Distributions for $W$ boson tagging using TAS $\beta=1.7$. C2, D2, $\tau_{21}$ top down.}} \label{app:W_TAS_17}
\end{figure}
\begin{figure}[H]
\includegraphics[width=0.3\textwidth]{sascha_input/Appendix/Distributions/w/distributions/beta17/h_normal_tj_C2_17_bin1.pdf} \hspace{1mm}
\includegraphics[width=0.3\textwidth]{sascha_input/Appendix/Distributions/w/distributions/beta17/h_normal_tj_C2_17_bin2.pdf} \hspace{1mm}
\includegraphics[width=0.3\textwidth]{sascha_input/Appendix/Distributions/w/distributions/beta17/h_normal_tj_C2_17_bin3.pdf} 
\bigskip
\includegraphics[width=0.3\textwidth]{sascha_input/Appendix/Distributions/w/distributions/beta17/h_normal_tj_C2_17_bin4.pdf} \hspace{1mm}
\includegraphics[width=0.3\textwidth]{sascha_input/Appendix/Distributions/w/distributions/beta17/h_normal_tj_C2_17_bin5.pdf} \hspace{1mm}
\includegraphics[width=0.3\textwidth]{sascha_input/Appendix/Distributions/w/distributions/beta17/h_normal_tj_C2_17_bin6.pdf} 
\bigskip
\includegraphics[width=0.3\textwidth]{sascha_input/Appendix/Distributions/w/distributions/beta17/h_normal_tj_D2_17_bin1.pdf} \hspace{1mm}
\includegraphics[width=0.3\textwidth]{sascha_input/Appendix/Distributions/w/distributions/beta17/h_normal_tj_D2_17_bin2.pdf} \hspace{1mm}
\includegraphics[width=0.3\textwidth]{sascha_input/Appendix/Distributions/w/distributions/beta17/h_normal_tj_D2_17_bin3.pdf} 
\bigskip
\includegraphics[width=0.3\textwidth]{sascha_input/Appendix/Distributions/w/distributions/beta17/h_normal_tj_D2_17_bin4.pdf} \hspace{1mm}
\includegraphics[width=0.3\textwidth]{sascha_input/Appendix/Distributions/w/distributions/beta17/h_normal_tj_D2_17_bin5.pdf} \hspace{1mm}
\includegraphics[width=0.3\textwidth]{sascha_input/Appendix/Distributions/w/distributions/beta17/h_normal_tj_D2_17_bin6.pdf}
\bigskip
\includegraphics[width=0.3\textwidth]{sascha_input/Appendix/Distributions/w/distributions/beta17/h_normal_tj_nSub21_17_bin1.pdf} \hspace{1mm}
\includegraphics[width=0.3\textwidth]{sascha_input/Appendix/Distributions/w/distributions/beta17/h_normal_tj_nSub21_17_bin2.pdf} \hspace{1mm}
\includegraphics[width=0.3\textwidth]{sascha_input/Appendix/Distributions/w/distributions/beta17/h_normal_tj_nSub21_17_bin3.pdf} 
\bigskip
\includegraphics[width=0.3\textwidth]{sascha_input/Appendix/Distributions/w/distributions/beta17/h_normal_tj_nSub21_17_bin4.pdf} \hspace{6mm}
\includegraphics[width=0.3\textwidth]{sascha_input/Appendix/Distributions/w/distributions/beta17/h_normal_tj_nSub21_17_bin5.pdf} \hspace{6mm}
\includegraphics[width=0.3\textwidth]{sascha_input/Appendix/Distributions/w/distributions/beta17/h_normal_tj_nSub21_17_bin6.pdf}
\caption{\footnotesize{Distributions for $W$ boson tagging using tracks $\beta=1.7$. C2, D2, $\tau_{21}$ top down.}}\label{app:W_track_17}
\end{figure}
\begin{figure}[H]
\includegraphics[width=0.3\textwidth]{sascha_input/Appendix/Distributions/w/distributions/beta17/h_recoJet_C2_17_bin1.pdf} \hspace{1mm}
\includegraphics[width=0.3\textwidth]{sascha_input/Appendix/Distributions/w/distributions/beta17/h_recoJet_C2_17_bin2.pdf} \hspace{1mm}
\includegraphics[width=0.3\textwidth]{sascha_input/Appendix/Distributions/w/distributions/beta17/h_recoJet_C2_17_bin3.pdf} 
\bigskip
\includegraphics[width=0.3\textwidth]{sascha_input/Appendix/Distributions/w/distributions/beta17/h_recoJet_C2_17_bin4.pdf} \hspace{1mm}
\includegraphics[width=0.3\textwidth]{sascha_input/Appendix/Distributions/w/distributions/beta17/h_recoJet_C2_17_bin5.pdf} \hspace{1mm}
\includegraphics[width=0.3\textwidth]{sascha_input/Appendix/Distributions/w/distributions/beta17/h_recoJet_C2_17_bin6.pdf}
\bigskip
\includegraphics[width=0.3\textwidth]{sascha_input/Appendix/Distributions/w/distributions/beta17/h_recoJet_D2_17_bin1.pdf} \hspace{1mm}
\includegraphics[width=0.3\textwidth]{sascha_input/Appendix/Distributions/w/distributions/beta17/h_recoJet_D2_17_bin2.pdf} \hspace{1mm}
\includegraphics[width=0.3\textwidth]{sascha_input/Appendix/Distributions/w/distributions/beta17/h_recoJet_D2_17_bin3.pdf} 
\bigskip
\includegraphics[width=0.3\textwidth]{sascha_input/Appendix/Distributions/w/distributions/beta17/h_recoJet_D2_17_bin4.pdf} \hspace{1mm}
\includegraphics[width=0.3\textwidth]{sascha_input/Appendix/Distributions/w/distributions/beta17/h_recoJet_D2_17_bin5.pdf} \hspace{1mm}
\includegraphics[width=0.3\textwidth]{sascha_input/Appendix/Distributions/w/distributions/beta17/h_recoJet_D2_17_bin6.pdf}
\bigskip
\includegraphics[width=0.3\textwidth]{sascha_input/Appendix/Distributions/w/distributions/beta17/h_recoJet_nSub21_17_bin1.pdf} \hspace{1mm}
\includegraphics[width=0.3\textwidth]{sascha_input/Appendix/Distributions/w/distributions/beta17/h_recoJet_nSub21_17_bin2.pdf} \hspace{1mm}
\includegraphics[width=0.3\textwidth]{sascha_input/Appendix/Distributions/w/distributions/beta17/h_recoJet_nSub21_17_bin3.pdf} 
\bigskip
\includegraphics[width=0.3\textwidth]{sascha_input/Appendix/Distributions/w/distributions/beta17/h_recoJet_nSub21_17_bin4.pdf} \hspace{6mm}
\includegraphics[width=0.3\textwidth]{sascha_input/Appendix/Distributions/w/distributions/beta17/h_recoJet_nSub21_17_bin5.pdf} \hspace{6mm}
\includegraphics[width=0.3\textwidth]{sascha_input/Appendix/Distributions/w/distributions/beta17/h_recoJet_nSub21_17_bin6.pdf}
\caption{\footnotesize{Distributions for $W$ boson tagging using calorimeter clusters $\beta=1.7$. C2, D2, $\tau_{21}$ top down.}}
\end{figure}
\subsubsection*{$\beta=2$}
\begin{figure}[H]
\includegraphics[width=0.3\textwidth]{sascha_input/Appendix/Distributions/w/distributions/beta2/h_assisted_tj_C2_2_bin1.pdf} \hspace{1mm}
\includegraphics[width=0.3\textwidth]{sascha_input/Appendix/Distributions/w/distributions/beta2/h_assisted_tj_C2_2_bin2.pdf} \hspace{1mm}
\includegraphics[width=0.3\textwidth]{sascha_input/Appendix/Distributions/w/distributions/beta2/h_assisted_tj_C2_2_bin3.pdf} 
\bigskip
\includegraphics[width=0.3\textwidth]{sascha_input/Appendix/Distributions/w/distributions/beta2/h_assisted_tj_C2_2_bin4.pdf} \hspace{1mm}
\includegraphics[width=0.3\textwidth]{sascha_input/Appendix/Distributions/w/distributions/beta2/h_assisted_tj_C2_2_bin5.pdf} \hspace{1mm}
\includegraphics[width=0.3\textwidth]{sascha_input/Appendix/Distributions/w/distributions/beta2/h_assisted_tj_C2_2_bin6.pdf} 
\bigskip
\includegraphics[width=0.3\textwidth]{sascha_input/Appendix/Distributions/w/distributions/beta2/h_assisted_tj_D2_2_bin1.pdf} \hspace{1mm}
\includegraphics[width=0.3\textwidth]{sascha_input/Appendix/Distributions/w/distributions/beta2/h_assisted_tj_D2_2_bin2.pdf} \hspace{1mm}
\includegraphics[width=0.3\textwidth]{sascha_input/Appendix/Distributions/w/distributions/beta2/h_assisted_tj_D2_2_bin3.pdf} 
\bigskip
\includegraphics[width=0.3\textwidth]{sascha_input/Appendix/Distributions/w/distributions/beta2/h_assisted_tj_D2_2_bin4.pdf} \hspace{1mm}
\includegraphics[width=0.3\textwidth]{sascha_input/Appendix/Distributions/w/distributions/beta2/h_assisted_tj_D2_2_bin5.pdf} \hspace{1mm}
\includegraphics[width=0.3\textwidth]{sascha_input/Appendix/Distributions/w/distributions/beta2/h_assisted_tj_D2_2_bin6.pdf}
\bigskip 
\includegraphics[width=0.3\textwidth]{sascha_input/Appendix/Distributions/w/distributions/beta2/h_assisted_tj_nSub21_2_bin1.pdf} \hspace{1mm}
\includegraphics[width=0.3\textwidth]{sascha_input/Appendix/Distributions/w/distributions/beta2/h_assisted_tj_nSub21_2_bin2.pdf} \hspace{1mm}
\includegraphics[width=0.3\textwidth]{sascha_input/Appendix/Distributions/w/distributions/beta2/h_assisted_tj_nSub21_2_bin3.pdf} 
\bigskip
\includegraphics[width=0.3\textwidth]{sascha_input/Appendix/Distributions/w/distributions/beta2/h_assisted_tj_nSub21_2_bin4.pdf} \hspace{6mm}
\includegraphics[width=0.3\textwidth]{sascha_input/Appendix/Distributions/w/distributions/beta2/h_assisted_tj_nSub21_2_bin5.pdf} \hspace{6mm}
\includegraphics[width=0.3\textwidth]{sascha_input/Appendix/Distributions/w/distributions/beta2/h_assisted_tj_nSub21_2_bin6.pdf} 
\caption{\footnotesize{Distributions for $W$ boson tagging using TAS $\beta=2$. C2, D2, $\tau_{21}$ top down.}}
\end{figure}
\begin{figure}[H]
\includegraphics[width=0.3\textwidth]{sascha_input/Appendix/Distributions/w/distributions/beta2/h_normal_tj_C2_2_bin1.pdf} \hspace{1mm}
\includegraphics[width=0.3\textwidth]{sascha_input/Appendix/Distributions/w/distributions/beta2/h_normal_tj_C2_2_bin2.pdf} \hspace{1mm}
\includegraphics[width=0.3\textwidth]{sascha_input/Appendix/Distributions/w/distributions/beta2/h_normal_tj_C2_2_bin3.pdf} 
\bigskip
\includegraphics[width=0.3\textwidth]{sascha_input/Appendix/Distributions/w/distributions/beta2/h_normal_tj_C2_2_bin4.pdf} \hspace{1mm}
\includegraphics[width=0.3\textwidth]{sascha_input/Appendix/Distributions/w/distributions/beta2/h_normal_tj_C2_2_bin5.pdf} \hspace{1mm}
\includegraphics[width=0.3\textwidth]{sascha_input/Appendix/Distributions/w/distributions/beta2/h_normal_tj_C2_2_bin6.pdf} 
\bigskip
\includegraphics[width=0.3\textwidth]{sascha_input/Appendix/Distributions/w/distributions/beta2/h_normal_tj_D2_2_bin1.pdf} \hspace{1mm}
\includegraphics[width=0.3\textwidth]{sascha_input/Appendix/Distributions/w/distributions/beta2/h_normal_tj_D2_2_bin2.pdf} \hspace{1mm}
\includegraphics[width=0.3\textwidth]{sascha_input/Appendix/Distributions/w/distributions/beta2/h_normal_tj_D2_2_bin3.pdf} 
\bigskip
\includegraphics[width=0.3\textwidth]{sascha_input/Appendix/Distributions/w/distributions/beta2/h_normal_tj_D2_2_bin4.pdf} \hspace{1mm}
\includegraphics[width=0.3\textwidth]{sascha_input/Appendix/Distributions/w/distributions/beta2/h_normal_tj_D2_2_bin5.pdf} \hspace{1mm}
\includegraphics[width=0.3\textwidth]{sascha_input/Appendix/Distributions/w/distributions/beta2/h_normal_tj_D2_2_bin6.pdf}
\bigskip
\includegraphics[width=0.3\textwidth]{sascha_input/Appendix/Distributions/w/distributions/beta2/h_normal_tj_nSub21_2_bin1.pdf} \hspace{1mm}
\includegraphics[width=0.3\textwidth]{sascha_input/Appendix/Distributions/w/distributions/beta2/h_normal_tj_nSub21_2_bin2.pdf} \hspace{1mm}
\includegraphics[width=0.3\textwidth]{sascha_input/Appendix/Distributions/w/distributions/beta2/h_normal_tj_nSub21_2_bin3.pdf} 
\bigskip
\includegraphics[width=0.3\textwidth]{sascha_input/Appendix/Distributions/w/distributions/beta2/h_normal_tj_nSub21_2_bin4.pdf} \hspace{6mm}
\includegraphics[width=0.3\textwidth]{sascha_input/Appendix/Distributions/w/distributions/beta2/h_normal_tj_nSub21_2_bin5.pdf} \hspace{6mm}
\includegraphics[width=0.3\textwidth]{sascha_input/Appendix/Distributions/w/distributions/beta2/h_normal_tj_nSub21_2_bin6.pdf}
\caption{\footnotesize{Distributions for $W$ boson tagging using tracks $\beta=2$. C2, D2, $\tau_{21}$ top down.}}
\end{figure}
\begin{figure}[H]
\includegraphics[width=0.3\textwidth]{sascha_input/Appendix/Distributions/w/distributions/beta2/h_recoJet_C2_2_bin1.pdf} \hspace{1mm}
\includegraphics[width=0.3\textwidth]{sascha_input/Appendix/Distributions/w/distributions/beta2/h_recoJet_C2_2_bin2.pdf} \hspace{1mm}
\includegraphics[width=0.3\textwidth]{sascha_input/Appendix/Distributions/w/distributions/beta2/h_recoJet_C2_2_bin3.pdf} 
\bigskip
\includegraphics[width=0.3\textwidth]{sascha_input/Appendix/Distributions/w/distributions/beta2/h_recoJet_C2_2_bin4.pdf} \hspace{1mm}
\includegraphics[width=0.3\textwidth]{sascha_input/Appendix/Distributions/w/distributions/beta2/h_recoJet_C2_2_bin5.pdf} \hspace{1mm}
\includegraphics[width=0.3\textwidth]{sascha_input/Appendix/Distributions/w/distributions/beta2/h_recoJet_C2_2_bin6.pdf}
\bigskip
\includegraphics[width=0.3\textwidth]{sascha_input/Appendix/Distributions/w/distributions/beta2/h_recoJet_D2_2_bin1.pdf} \hspace{1mm}
\includegraphics[width=0.3\textwidth]{sascha_input/Appendix/Distributions/w/distributions/beta2/h_recoJet_D2_2_bin2.pdf} \hspace{1mm}
\includegraphics[width=0.3\textwidth]{sascha_input/Appendix/Distributions/w/distributions/beta2/h_recoJet_D2_2_bin3.pdf} 
\bigskip
\includegraphics[width=0.3\textwidth]{sascha_input/Appendix/Distributions/w/distributions/beta2/h_recoJet_D2_2_bin4.pdf} \hspace{1mm}
\includegraphics[width=0.3\textwidth]{sascha_input/Appendix/Distributions/w/distributions/beta2/h_recoJet_D2_2_bin5.pdf} \hspace{1mm}
\includegraphics[width=0.3\textwidth]{sascha_input/Appendix/Distributions/w/distributions/beta2/h_recoJet_D2_2_bin6.pdf}
\bigskip
\includegraphics[width=0.3\textwidth]{sascha_input/Appendix/Distributions/w/distributions/beta2/h_recoJet_nSub21_2_bin1.pdf} \hspace{1mm}
\includegraphics[width=0.3\textwidth]{sascha_input/Appendix/Distributions/w/distributions/beta2/h_recoJet_nSub21_2_bin2.pdf} \hspace{1mm}
\includegraphics[width=0.3\textwidth]{sascha_input/Appendix/Distributions/w/distributions/beta2/h_recoJet_nSub21_2_bin3.pdf} 
\bigskip
\includegraphics[width=0.3\textwidth]{sascha_input/Appendix/Distributions/w/distributions/beta2/h_recoJet_nSub21_2_bin4.pdf} \hspace{6mm}
\includegraphics[width=0.3\textwidth]{sascha_input/Appendix/Distributions/w/distributions/beta2/h_recoJet_nSub21_2_bin5.pdf} \hspace{6mm}
\includegraphics[width=0.3\textwidth]{sascha_input/Appendix/Distributions/w/distributions/beta2/h_recoJet_nSub21_2_bin6.pdf}
\caption{\footnotesize{Distributions for $W$ boson tagging using calorimeter clusters $\beta=2$. C2, D2, $\tau_{21}$ top down.}}
\end{figure}
\subsubsection*{$\beta=3$}
\begin{figure}[H]
\includegraphics[width=0.3\textwidth]{sascha_input/Appendix/Distributions/w/distributions/beta3/h_assisted_tj_C2_3_bin1.pdf} \hspace{1mm}
\includegraphics[width=0.3\textwidth]{sascha_input/Appendix/Distributions/w/distributions/beta3/h_assisted_tj_C2_3_bin2.pdf} \hspace{1mm}
\includegraphics[width=0.3\textwidth]{sascha_input/Appendix/Distributions/w/distributions/beta3/h_assisted_tj_C2_3_bin3.pdf} 
\bigskip
\includegraphics[width=0.3\textwidth]{sascha_input/Appendix/Distributions/w/distributions/beta3/h_assisted_tj_C2_3_bin4.pdf} \hspace{1mm}
\includegraphics[width=0.3\textwidth]{sascha_input/Appendix/Distributions/w/distributions/beta3/h_assisted_tj_C2_3_bin5.pdf} \hspace{1mm}
\includegraphics[width=0.3\textwidth]{sascha_input/Appendix/Distributions/w/distributions/beta3/h_assisted_tj_C2_3_bin6.pdf} 
\bigskip
\includegraphics[width=0.3\textwidth]{sascha_input/Appendix/Distributions/w/distributions/beta3/h_assisted_tj_D2_3_bin1.pdf} \hspace{1mm}
\includegraphics[width=0.3\textwidth]{sascha_input/Appendix/Distributions/w/distributions/beta3/h_assisted_tj_D2_3_bin2.pdf} \hspace{1mm}
\includegraphics[width=0.3\textwidth]{sascha_input/Appendix/Distributions/w/distributions/beta3/h_assisted_tj_D2_3_bin3.pdf} 
\bigskip
\includegraphics[width=0.3\textwidth]{sascha_input/Appendix/Distributions/w/distributions/beta3/h_assisted_tj_D2_3_bin4.pdf} \hspace{1mm}
\includegraphics[width=0.3\textwidth]{sascha_input/Appendix/Distributions/w/distributions/beta3/h_assisted_tj_D2_3_bin5.pdf} \hspace{1mm}
\includegraphics[width=0.3\textwidth]{sascha_input/Appendix/Distributions/w/distributions/beta3/h_assisted_tj_D2_3_bin6.pdf}
\bigskip 
\includegraphics[width=0.3\textwidth]{sascha_input/Appendix/Distributions/w/distributions/beta3/h_assisted_tj_nSub21_3_bin1.pdf} \hspace{1mm}
\includegraphics[width=0.3\textwidth]{sascha_input/Appendix/Distributions/w/distributions/beta3/h_assisted_tj_nSub21_3_bin2.pdf} \hspace{1mm}
\includegraphics[width=0.3\textwidth]{sascha_input/Appendix/Distributions/w/distributions/beta3/h_assisted_tj_nSub21_3_bin3.pdf} 
\bigskip
\includegraphics[width=0.3\textwidth]{sascha_input/Appendix/Distributions/w/distributions/beta3/h_assisted_tj_nSub21_3_bin4.pdf} \hspace{6mm}
\includegraphics[width=0.3\textwidth]{sascha_input/Appendix/Distributions/w/distributions/beta3/h_assisted_tj_nSub21_3_bin5.pdf} \hspace{6mm}
\includegraphics[width=0.3\textwidth]{sascha_input/Appendix/Distributions/w/distributions/beta3/h_assisted_tj_nSub21_3_bin6.pdf} 
\caption{\footnotesize{Distributions for $W$ boson tagging using TAS $\beta=3$. C2, D2, $\tau_{21}$ top down.}}
\end{figure}
\begin{figure}[H]
\includegraphics[width=0.3\textwidth]{sascha_input/Appendix/Distributions/w/distributions/beta3/h_normal_tj_C2_3_bin1.pdf} \hspace{1mm}
\includegraphics[width=0.3\textwidth]{sascha_input/Appendix/Distributions/w/distributions/beta3/h_normal_tj_C2_3_bin2.pdf} \hspace{1mm}
\includegraphics[width=0.3\textwidth]{sascha_input/Appendix/Distributions/w/distributions/beta3/h_normal_tj_C2_3_bin3.pdf} 
\bigskip
\includegraphics[width=0.3\textwidth]{sascha_input/Appendix/Distributions/w/distributions/beta3/h_normal_tj_C2_3_bin4.pdf} \hspace{1mm}
\includegraphics[width=0.3\textwidth]{sascha_input/Appendix/Distributions/w/distributions/beta3/h_normal_tj_C2_3_bin5.pdf} \hspace{1mm}
\includegraphics[width=0.3\textwidth]{sascha_input/Appendix/Distributions/w/distributions/beta3/h_normal_tj_C2_3_bin6.pdf} 
\bigskip
\includegraphics[width=0.3\textwidth]{sascha_input/Appendix/Distributions/w/distributions/beta3/h_normal_tj_D2_3_bin1.pdf} \hspace{1mm}
\includegraphics[width=0.3\textwidth]{sascha_input/Appendix/Distributions/w/distributions/beta3/h_normal_tj_D2_3_bin2.pdf} \hspace{1mm}
\includegraphics[width=0.3\textwidth]{sascha_input/Appendix/Distributions/w/distributions/beta3/h_normal_tj_D2_3_bin3.pdf} 
\bigskip
\includegraphics[width=0.3\textwidth]{sascha_input/Appendix/Distributions/w/distributions/beta3/h_normal_tj_D2_3_bin4.pdf} \hspace{1mm}
\includegraphics[width=0.3\textwidth]{sascha_input/Appendix/Distributions/w/distributions/beta3/h_normal_tj_D2_3_bin5.pdf} \hspace{1mm}
\includegraphics[width=0.3\textwidth]{sascha_input/Appendix/Distributions/w/distributions/beta3/h_normal_tj_D2_3_bin6.pdf}
\bigskip
\includegraphics[width=0.3\textwidth]{sascha_input/Appendix/Distributions/w/distributions/beta3/h_normal_tj_nSub21_3_bin1.pdf} \hspace{1mm}
\includegraphics[width=0.3\textwidth]{sascha_input/Appendix/Distributions/w/distributions/beta3/h_normal_tj_nSub21_3_bin2.pdf} \hspace{1mm}
\includegraphics[width=0.3\textwidth]{sascha_input/Appendix/Distributions/w/distributions/beta3/h_normal_tj_nSub21_3_bin3.pdf} 
\bigskip
\includegraphics[width=0.3\textwidth]{sascha_input/Appendix/Distributions/w/distributions/beta3/h_normal_tj_nSub21_3_bin4.pdf} \hspace{6mm}
\includegraphics[width=0.3\textwidth]{sascha_input/Appendix/Distributions/w/distributions/beta3/h_normal_tj_nSub21_3_bin5.pdf} \hspace{6mm}
\includegraphics[width=0.3\textwidth]{sascha_input/Appendix/Distributions/w/distributions/beta3/h_normal_tj_nSub21_3_bin6.pdf}
\caption{\footnotesize{Distributions for $W$ boson tagging using tracks $\beta=3$. C2, D2, $\tau_{21}$ top down.}}
\end{figure}
\begin{figure}[H]
\includegraphics[width=0.3\textwidth]{sascha_input/Appendix/Distributions/w/distributions/beta3/h_recoJet_C2_3_bin1.pdf} \hspace{1mm}
\includegraphics[width=0.3\textwidth]{sascha_input/Appendix/Distributions/w/distributions/beta3/h_recoJet_C2_3_bin2.pdf} \hspace{1mm}
\includegraphics[width=0.3\textwidth]{sascha_input/Appendix/Distributions/w/distributions/beta3/h_recoJet_C2_3_bin3.pdf} 
\bigskip
\includegraphics[width=0.3\textwidth]{sascha_input/Appendix/Distributions/w/distributions/beta3/h_recoJet_C2_3_bin4.pdf} \hspace{1mm}
\includegraphics[width=0.3\textwidth]{sascha_input/Appendix/Distributions/w/distributions/beta3/h_recoJet_C2_3_bin5.pdf} \hspace{1mm}
\includegraphics[width=0.3\textwidth]{sascha_input/Appendix/Distributions/w/distributions/beta3/h_recoJet_C2_3_bin6.pdf}
\bigskip
\includegraphics[width=0.3\textwidth]{sascha_input/Appendix/Distributions/w/distributions/beta3/h_recoJet_D2_3_bin1.pdf} \hspace{1mm}
\includegraphics[width=0.3\textwidth]{sascha_input/Appendix/Distributions/w/distributions/beta3/h_recoJet_D2_3_bin2.pdf} \hspace{1mm}
\includegraphics[width=0.3\textwidth]{sascha_input/Appendix/Distributions/w/distributions/beta3/h_recoJet_D2_3_bin3.pdf} 
\bigskip
\includegraphics[width=0.3\textwidth]{sascha_input/Appendix/Distributions/w/distributions/beta3/h_recoJet_D2_3_bin4.pdf} \hspace{1mm}
\includegraphics[width=0.3\textwidth]{sascha_input/Appendix/Distributions/w/distributions/beta3/h_recoJet_D2_3_bin5.pdf} \hspace{1mm}
\includegraphics[width=0.3\textwidth]{sascha_input/Appendix/Distributions/w/distributions/beta3/h_recoJet_D2_3_bin6.pdf}
\bigskip
\includegraphics[width=0.3\textwidth]{sascha_input/Appendix/Distributions/w/distributions/beta3/h_recoJet_nSub21_3_bin1.pdf} \hspace{1mm}
\includegraphics[width=0.3\textwidth]{sascha_input/Appendix/Distributions/w/distributions/beta3/h_recoJet_nSub21_3_bin2.pdf} \hspace{1mm}
\includegraphics[width=0.3\textwidth]{sascha_input/Appendix/Distributions/w/distributions/beta3/h_recoJet_nSub21_3_bin3.pdf} 
\bigskip
\includegraphics[width=0.3\textwidth]{sascha_input/Appendix/Distributions/w/distributions/beta3/h_recoJet_nSub21_3_bin4.pdf} \hspace{6mm}
\includegraphics[width=0.3\textwidth]{sascha_input/Appendix/Distributions/w/distributions/beta3/h_recoJet_nSub21_3_bin5.pdf} \hspace{6mm}
\includegraphics[width=0.3\textwidth]{sascha_input/Appendix/Distributions/w/distributions/beta3/h_recoJet_nSub21_3_bin6.pdf}
\caption{\footnotesize{Distributions for $W$ boson tagging using calorimeter clusters $\beta=3$. C2, D2, $\tau_{21}$ top down.}}
\end{figure}


\subsection{Higgs Distributions}
\subsubsection*{$\beta=0.5$}
\begin{figure}[H]
\includegraphics[width=0.3\textwidth]{sascha_input/Appendix/Distributions/higgs/distributions/beta05/h_assisted_tj_C2_05_bin1.pdf} \hspace{1mm}
\includegraphics[width=0.3\textwidth]{sascha_input/Appendix/Distributions/higgs/distributions/beta05/h_assisted_tj_C2_05_bin2.pdf} \hspace{4mm}
\includegraphics[width=0.3\textwidth]{sascha_input/Appendix/Distributions/higgs/distributions/beta05/h_assisted_tj_C2_05_bin3.pdf} 
\bigskip
\includegraphics[width=0.3\textwidth]{sascha_input/Appendix/Distributions/higgs/distributions/beta05/h_assisted_tj_C2_05_bin4.pdf} \hspace{4mm}
\includegraphics[width=0.3\textwidth]{sascha_input/Appendix/Distributions/higgs/distributions/beta05/h_assisted_tj_C2_05_bin5.pdf} 

\bigskip
\includegraphics[width=0.3\textwidth]{sascha_input/Appendix/Distributions/higgs/distributions/beta05/h_assisted_tj_D2_05_bin1.pdf} \hspace{1mm}
\includegraphics[width=0.3\textwidth]{sascha_input/Appendix/Distributions/higgs/distributions/beta05/h_assisted_tj_D2_05_bin2.pdf} \hspace{4mm}
\includegraphics[width=0.3\textwidth]{sascha_input/Appendix/Distributions/higgs/distributions/beta05/h_assisted_tj_D2_05_bin3.pdf} 
\bigskip
\includegraphics[width=0.3\textwidth]{sascha_input/Appendix/Distributions/higgs/distributions/beta05/h_assisted_tj_D2_05_bin4.pdf} \hspace{4mm}
\includegraphics[width=0.3\textwidth]{sascha_input/Appendix/Distributions/higgs/distributions/beta05/h_assisted_tj_D2_05_bin5.pdf} 

\bigskip 
\includegraphics[width=0.3\textwidth]{sascha_input/Appendix/Distributions/higgs/distributions/beta05/h_assisted_tj_nSub21_05_bin1.pdf} \hspace{1mm}
\includegraphics[width=0.3\textwidth]{sascha_input/Appendix/Distributions/higgs/distributions/beta05/h_assisted_tj_nSub21_05_bin2.pdf} \hspace{4mm}
\includegraphics[width=0.3\textwidth]{sascha_input/Appendix/Distributions/higgs/distributions/beta05/h_assisted_tj_nSub21_05_bin3.pdf} 
\bigskip
\includegraphics[width=0.3\textwidth]{sascha_input/Appendix/Distributions/higgs/distributions/beta05/h_assisted_tj_nSub21_05_bin4.pdf} \hspace{4mm}
\includegraphics[width=0.3\textwidth]{sascha_input/Appendix/Distributions/higgs/distributions/beta05/h_assisted_tj_nSub21_05_bin5.pdf} 
\vspace{-0.75cm}
\caption{\footnotesize{Distributions for Higgs boson tagging using TAS $\beta=0.5$. C2, D2, $\tau_{21}$ top down.}}
\end{figure}
\begin{figure}[H]
\includegraphics[width=0.3\textwidth]{sascha_input/Appendix/Distributions/higgs/distributions/beta05/h_normal_tj_C2_05_bin1.pdf} 	\hspace{1mm}
\includegraphics[width=0.3\textwidth]{sascha_input/Appendix/Distributions/higgs/distributions/beta05/h_normal_tj_C2_05_bin2.pdf} 	\hspace{4mm}
\includegraphics[width=0.3\textwidth]{sascha_input/Appendix/Distributions/higgs/distributions/beta05/h_normal_tj_C2_05_bin3.pdf} 
\bigskip
\includegraphics[width=0.3\textwidth]{sascha_input/Appendix/Distributions/higgs/distributions/beta05/h_normal_tj_C2_05_bin4.pdf} 	\hspace{4mm}
\includegraphics[width=0.3\textwidth]{sascha_input/Appendix/Distributions/higgs/distributions/beta05/h_normal_tj_C2_05_bin5.pdf} 

\bigskip
\includegraphics[width=0.3\textwidth]{sascha_input/Appendix/Distributions/higgs/distributions/beta05/h_normal_tj_D2_05_bin1.pdf} 	\hspace{1mm}
\includegraphics[width=0.3\textwidth]{sascha_input/Appendix/Distributions/higgs/distributions/beta05/h_normal_tj_D2_05_bin2.pdf} 	\hspace{4mm}
\includegraphics[width=0.3\textwidth]{sascha_input/Appendix/Distributions/higgs/distributions/beta05/h_normal_tj_D2_05_bin3.pdf} 
\bigskip
\includegraphics[width=0.3\textwidth]{sascha_input/Appendix/Distributions/higgs/distributions/beta05/h_normal_tj_D2_05_bin4.pdf} 	\hspace{4mm}
\includegraphics[width=0.3\textwidth]{sascha_input/Appendix/Distributions/higgs/distributions/beta05/h_normal_tj_D2_05_bin5.pdf} 

\bigskip
\includegraphics[width=0.3\textwidth]{sascha_input/Appendix/Distributions/higgs/distributions/beta05/h_normal_tj_nSub21_05_bin1.pdf} 	\hspace{1mm}
\includegraphics[width=0.3\textwidth]{sascha_input/Appendix/Distributions/higgs/distributions/beta05/h_normal_tj_nSub21_05_bin2.pdf} 	\hspace{4mm}
\includegraphics[width=0.3\textwidth]{sascha_input/Appendix/Distributions/higgs/distributions/beta05/h_normal_tj_nSub21_05_bin3.pdf} 
\bigskip
\includegraphics[width=0.3\textwidth]{sascha_input/Appendix/Distributions/higgs/distributions/beta05/h_normal_tj_nSub21_05_bin4.pdf} \hspace{4mm}
\includegraphics[width=0.3\textwidth]{sascha_input/Appendix/Distributions/higgs/distributions/beta05/h_normal_tj_nSub21_05_bin5.pdf} 

\caption{\footnotesize{Distributions for Higgs boson tagging using tracks $\beta=0.5$. C2, D2, $\tau_{21}$ top down.}}
\end{figure}
\begin{figure}[H]
\includegraphics[width=0.3\textwidth]{sascha_input/Appendix/Distributions/higgs/distributions/beta05/h_recoJet_C2_05_bin1.pdf} \hspace{1mm}
\includegraphics[width=0.3\textwidth]{sascha_input/Appendix/Distributions/higgs/distributions/beta05/h_recoJet_C2_05_bin2.pdf} \hspace{4mm}
\includegraphics[width=0.3\textwidth]{sascha_input/Appendix/Distributions/higgs/distributions/beta05/h_recoJet_C2_05_bin3.pdf} 
\bigskip
\includegraphics[width=0.3\textwidth]{sascha_input/Appendix/Distributions/higgs/distributions/beta05/h_recoJet_C2_05_bin4.pdf} \hspace{4mm}
\includegraphics[width=0.3\textwidth]{sascha_input/Appendix/Distributions/higgs/distributions/beta05/h_recoJet_C2_05_bin5.pdf} 

\bigskip
\includegraphics[width=0.3\textwidth]{sascha_input/Appendix/Distributions/higgs/distributions/beta05/h_recoJet_D2_05_bin1.pdf} \hspace{1mm}
\includegraphics[width=0.3\textwidth]{sascha_input/Appendix/Distributions/higgs/distributions/beta05/h_recoJet_D2_05_bin2.pdf} \hspace{4mm}
\includegraphics[width=0.3\textwidth]{sascha_input/Appendix/Distributions/higgs/distributions/beta05/h_recoJet_D2_05_bin3.pdf} 
\bigskip
\includegraphics[width=0.3\textwidth]{sascha_input/Appendix/Distributions/higgs/distributions/beta05/h_recoJet_D2_05_bin4.pdf} \hspace{4mm}
\includegraphics[width=0.3\textwidth]{sascha_input/Appendix/Distributions/higgs/distributions/beta05/h_recoJet_D2_05_bin5.pdf} 

\bigskip
\includegraphics[width=0.3\textwidth]{sascha_input/Appendix/Distributions/higgs/distributions/beta05/h_recoJet_nSub21_05_bin1.pdf} \hspace{1mm}
\includegraphics[width=0.3\textwidth]{sascha_input/Appendix/Distributions/higgs/distributions/beta05/h_recoJet_nSub21_05_bin2.pdf} \hspace{4mm}
\includegraphics[width=0.3\textwidth]{sascha_input/Appendix/Distributions/higgs/distributions/beta05/h_recoJet_nSub21_05_bin3.pdf} 
\bigskip
\includegraphics[width=0.3\textwidth]{sascha_input/Appendix/Distributions/higgs/distributions/beta05/h_recoJet_nSub21_05_bin4.pdf} \hspace{4mm}
\includegraphics[width=0.3\textwidth]{sascha_input/Appendix/Distributions/higgs/distributions/beta05/h_recoJet_nSub21_05_bin5.pdf} 

\caption{\footnotesize{Distributions for Higgs boson tagging using calorimeter clusters $\beta=0.5$. C2, D2, $\tau_{21}$ top down.}}
\end{figure}
\subsubsection*{$\beta=1$}
\begin{figure}[H]
\includegraphics[width=0.3\textwidth]{sascha_input/Appendix/Distributions/higgs/distributions/beta1/h_assisted_tj_C2_bin1.pdf} \hspace{1mm}
\includegraphics[width=0.3\textwidth]{sascha_input/Appendix/Distributions/higgs/distributions/beta1/h_assisted_tj_C2_bin2.pdf} \hspace{4mm}
\includegraphics[width=0.3\textwidth]{sascha_input/Appendix/Distributions/higgs/distributions/beta1/h_assisted_tj_C2_bin3.pdf} 
\bigskip
\includegraphics[width=0.3\textwidth]{sascha_input/Appendix/Distributions/higgs/distributions/beta1/h_assisted_tj_C2_bin4.pdf} \hspace{4mm}
\includegraphics[width=0.3\textwidth]{sascha_input/Appendix/Distributions/higgs/distributions/beta1/h_assisted_tj_C2_bin5.pdf} 

\bigskip
\includegraphics[width=0.3\textwidth]{sascha_input/Appendix/Distributions/higgs/distributions/beta1/h_assisted_tj_D2_bin1.pdf} \hspace{1mm}
\includegraphics[width=0.3\textwidth]{sascha_input/Appendix/Distributions/higgs/distributions/beta1/h_assisted_tj_D2_bin2.pdf} \hspace{4mm}
\includegraphics[width=0.3\textwidth]{sascha_input/Appendix/Distributions/higgs/distributions/beta1/h_assisted_tj_D2_bin3.pdf} 
\bigskip
\includegraphics[width=0.3\textwidth]{sascha_input/Appendix/Distributions/higgs/distributions/beta1/h_assisted_tj_D2_bin4.pdf} \hspace{4mm}
\includegraphics[width=0.3\textwidth]{sascha_input/Appendix/Distributions/higgs/distributions/beta1/h_assisted_tj_D2_bin5.pdf} 

\bigskip 
\includegraphics[width=0.3\textwidth]{sascha_input/Appendix/Distributions/higgs/distributions/beta1/h_assisted_tj_nSub21_bin1.pdf} \hspace{1mm}
\includegraphics[width=0.3\textwidth]{sascha_input/Appendix/Distributions/higgs/distributions/beta1/h_assisted_tj_nSub21_bin2.pdf} \hspace{4mm}
\includegraphics[width=0.3\textwidth]{sascha_input/Appendix/Distributions/higgs/distributions/beta1/h_assisted_tj_nSub21_bin3.pdf} 
\bigskip
\includegraphics[width=0.3\textwidth]{sascha_input/Appendix/Distributions/higgs/distributions/beta1/h_assisted_tj_nSub21_bin4.pdf} \hspace{4mm}
\includegraphics[width=0.3\textwidth]{sascha_input/Appendix/Distributions/higgs/distributions/beta1/h_assisted_tj_nSub21_bin5.pdf} 

\vspace{-0.5cm}
\caption{\footnotesize{Distributions for Higgs boson tagging using TAS $\beta=1$. C2, D2, $\tau_{21}$ top down.}}
\end{figure}
\begin{figure}[H]
\includegraphics[width=0.3\textwidth]{sascha_input/Appendix/Distributions/higgs/distributions/beta1/h_normal_tj_C2_bin1.pdf} \hspace{1mm}
\includegraphics[width=0.3\textwidth]{sascha_input/Appendix/Distributions/higgs/distributions/beta1/h_normal_tj_C2_bin2.pdf} \hspace{4mm}
\includegraphics[width=0.3\textwidth]{sascha_input/Appendix/Distributions/higgs/distributions/beta1/h_normal_tj_C2_bin3.pdf} 
\bigskip
\includegraphics[width=0.3\textwidth]{sascha_input/Appendix/Distributions/higgs/distributions/beta1/h_normal_tj_C2_bin4.pdf} \hspace{4mm}
\includegraphics[width=0.3\textwidth]{sascha_input/Appendix/Distributions/higgs/distributions/beta1/h_normal_tj_C2_bin5.pdf} 

\bigskip
\includegraphics[width=0.3\textwidth]{sascha_input/Appendix/Distributions/higgs/distributions/beta1/h_normal_tj_D2_bin1.pdf} \hspace{1mm}
\includegraphics[width=0.3\textwidth]{sascha_input/Appendix/Distributions/higgs/distributions/beta1/h_normal_tj_D2_bin2.pdf} \hspace{4mm}
\includegraphics[width=0.3\textwidth]{sascha_input/Appendix/Distributions/higgs/distributions/beta1/h_normal_tj_D2_bin3.pdf} 
\bigskip
\includegraphics[width=0.3\textwidth]{sascha_input/Appendix/Distributions/higgs/distributions/beta1/h_normal_tj_D2_bin4.pdf} \hspace{4mm}
\includegraphics[width=0.3\textwidth]{sascha_input/Appendix/Distributions/higgs/distributions/beta1/h_normal_tj_D2_bin5.pdf} 

\bigskip
\includegraphics[width=0.3\textwidth]{sascha_input/Appendix/Distributions/higgs/distributions/beta1/h_normal_tj_nSub21_bin1.pdf} \hspace{1mm}
\includegraphics[width=0.3\textwidth]{sascha_input/Appendix/Distributions/higgs/distributions/beta1/h_normal_tj_nSub21_bin2.pdf} \hspace{4mm}
\includegraphics[width=0.3\textwidth]{sascha_input/Appendix/Distributions/higgs/distributions/beta1/h_normal_tj_nSub21_bin3.pdf} 
\bigskip
\includegraphics[width=0.3\textwidth]{sascha_input/Appendix/Distributions/higgs/distributions/beta1/h_normal_tj_nSub21_bin4.pdf} \hspace{4mm}
\includegraphics[width=0.3\textwidth]{sascha_input/Appendix/Distributions/higgs/distributions/beta1/h_normal_tj_nSub21_bin5.pdf} 

\caption{\footnotesize{Distributions for Higgs boson tagging using tracks $\beta=1$. C2, D2, $\tau_{21}$ top down.}}
\end{figure}
\begin{figure}[H]
\includegraphics[width=0.3\textwidth]{sascha_input/Appendix/Distributions/higgs/distributions/beta1/h_recoJet_C2_bin1.pdf} \hspace{1mm}
\includegraphics[width=0.3\textwidth]{sascha_input/Appendix/Distributions/higgs/distributions/beta1/h_recoJet_C2_bin2.pdf} \hspace{4mm}
\includegraphics[width=0.3\textwidth]{sascha_input/Appendix/Distributions/higgs/distributions/beta1/h_recoJet_C2_bin3.pdf} 
\bigskip
\includegraphics[width=0.3\textwidth]{sascha_input/Appendix/Distributions/higgs/distributions/beta1/h_recoJet_C2_bin4.pdf} \hspace{4mm}
\includegraphics[width=0.3\textwidth]{sascha_input/Appendix/Distributions/higgs/distributions/beta1/h_recoJet_C2_bin5.pdf} 

\bigskip
\includegraphics[width=0.3\textwidth]{sascha_input/Appendix/Distributions/higgs/distributions/beta1/h_recoJet_D2_bin1.pdf} \hspace{1mm}
\includegraphics[width=0.3\textwidth]{sascha_input/Appendix/Distributions/higgs/distributions/beta1/h_recoJet_D2_bin2.pdf} \hspace{4mm}
\includegraphics[width=0.3\textwidth]{sascha_input/Appendix/Distributions/higgs/distributions/beta1/h_recoJet_D2_bin3.pdf} 
\bigskip
\includegraphics[width=0.3\textwidth]{sascha_input/Appendix/Distributions/higgs/distributions/beta1/h_recoJet_D2_bin4.pdf} \hspace{4mm}
\includegraphics[width=0.3\textwidth]{sascha_input/Appendix/Distributions/higgs/distributions/beta1/h_recoJet_D2_bin5.pdf} 

\bigskip
\includegraphics[width=0.3\textwidth]{sascha_input/Appendix/Distributions/higgs/distributions/beta1/h_recoJet_nSub21_bin1.pdf} \hspace{1mm}
\includegraphics[width=0.3\textwidth]{sascha_input/Appendix/Distributions/higgs/distributions/beta1/h_recoJet_nSub21_bin2.pdf} \hspace{4mm}
\includegraphics[width=0.3\textwidth]{sascha_input/Appendix/Distributions/higgs/distributions/beta1/h_recoJet_nSub21_bin3.pdf} 
\bigskip
\includegraphics[width=0.3\textwidth]{sascha_input/Appendix/Distributions/higgs/distributions/beta1/h_recoJet_nSub21_bin4.pdf} \hspace{4mm}
\includegraphics[width=0.3\textwidth]{sascha_input/Appendix/Distributions/higgs/distributions/beta1/h_recoJet_nSub21_bin5.pdf} 

\caption{\footnotesize{Distributions for Higgs boson tagging using calorimeter clusters $\beta=1$. C2, D2, $\tau_{21}$ top down.}}
\end{figure}
\subsubsection*{$\beta=1.7$}
\begin{figure}[H]
\includegraphics[width=0.3\textwidth]{sascha_input/Appendix/Distributions/higgs/distributions/beta17/h_assisted_tj_C2_17_bin1.pdf} \hspace{1mm}
\includegraphics[width=0.3\textwidth]{sascha_input/Appendix/Distributions/higgs/distributions/beta17/h_assisted_tj_C2_17_bin2.pdf} \hspace{4mm}
\includegraphics[width=0.3\textwidth]{sascha_input/Appendix/Distributions/higgs/distributions/beta17/h_assisted_tj_C2_17_bin3.pdf} 
\bigskip
\includegraphics[width=0.3\textwidth]{sascha_input/Appendix/Distributions/higgs/distributions/beta17/h_assisted_tj_C2_17_bin4.pdf} \hspace{4mm}
\includegraphics[width=0.3\textwidth]{sascha_input/Appendix/Distributions/higgs/distributions/beta17/h_assisted_tj_C2_17_bin5.pdf} 

\bigskip
\includegraphics[width=0.3\textwidth]{sascha_input/Appendix/Distributions/higgs/distributions/beta17/h_assisted_tj_D2_17_bin1.pdf} \hspace{1mm}
\includegraphics[width=0.3\textwidth]{sascha_input/Appendix/Distributions/higgs/distributions/beta17/h_assisted_tj_D2_17_bin2.pdf} \hspace{4mm}
\includegraphics[width=0.3\textwidth]{sascha_input/Appendix/Distributions/higgs/distributions/beta17/h_assisted_tj_D2_17_bin3.pdf} 
\bigskip
\includegraphics[width=0.3\textwidth]{sascha_input/Appendix/Distributions/higgs/distributions/beta17/h_assisted_tj_D2_17_bin4.pdf} \hspace{4mm}
\includegraphics[width=0.3\textwidth]{sascha_input/Appendix/Distributions/higgs/distributions/beta17/h_assisted_tj_D2_17_bin5.pdf} 

\bigskip 
\includegraphics[width=0.3\textwidth]{sascha_input/Appendix/Distributions/higgs/distributions/beta17/h_assisted_tj_nSub21_17_bin1.pdf} \hspace{1mm}
\includegraphics[width=0.3\textwidth]{sascha_input/Appendix/Distributions/higgs/distributions/beta17/h_assisted_tj_nSub21_17_bin2.pdf} \hspace{4mm}
\includegraphics[width=0.3\textwidth]{sascha_input/Appendix/Distributions/higgs/distributions/beta17/h_assisted_tj_nSub21_17_bin3.pdf} 
\bigskip
\includegraphics[width=0.3\textwidth]{sascha_input/Appendix/Distributions/higgs/distributions/beta17/h_assisted_tj_nSub21_17_bin4.pdf} \hspace{4mm}
\includegraphics[width=0.3\textwidth]{sascha_input/Appendix/Distributions/higgs/distributions/beta17/h_assisted_tj_nSub21_17_bin5.pdf} 

\caption{\footnotesize{Distributions for Higgs boson tagging using TAS $\beta=1.7$. C2, D2, $\tau_{21}$ top down.}}
\end{figure}
\begin{figure}[H]
\includegraphics[width=0.3\textwidth]{sascha_input/Appendix/Distributions/higgs/distributions/beta17/h_normal_tj_C2_17_bin1.pdf} \hspace{1mm}
\includegraphics[width=0.3\textwidth]{sascha_input/Appendix/Distributions/higgs/distributions/beta17/h_normal_tj_C2_17_bin2.pdf} \hspace{4mm}
\includegraphics[width=0.3\textwidth]{sascha_input/Appendix/Distributions/higgs/distributions/beta17/h_normal_tj_C2_17_bin3.pdf} 
\bigskip
\includegraphics[width=0.3\textwidth]{sascha_input/Appendix/Distributions/higgs/distributions/beta17/h_normal_tj_C2_17_bin4.pdf} \hspace{4mm}
\includegraphics[width=0.3\textwidth]{sascha_input/Appendix/Distributions/higgs/distributions/beta17/h_normal_tj_C2_17_bin5.pdf} 

\bigskip
\includegraphics[width=0.3\textwidth]{sascha_input/Appendix/Distributions/higgs/distributions/beta17/h_normal_tj_D2_17_bin1.pdf} \hspace{1mm}
\includegraphics[width=0.3\textwidth]{sascha_input/Appendix/Distributions/higgs/distributions/beta17/h_normal_tj_D2_17_bin2.pdf} \hspace{4mm}
\includegraphics[width=0.3\textwidth]{sascha_input/Appendix/Distributions/higgs/distributions/beta17/h_normal_tj_D2_17_bin3.pdf} 
\bigskip
\includegraphics[width=0.3\textwidth]{sascha_input/Appendix/Distributions/higgs/distributions/beta17/h_normal_tj_D2_17_bin4.pdf} \hspace{4mm}
\includegraphics[width=0.3\textwidth]{sascha_input/Appendix/Distributions/higgs/distributions/beta17/h_normal_tj_D2_17_bin5.pdf} 

\bigskip
\includegraphics[width=0.3\textwidth]{sascha_input/Appendix/Distributions/higgs/distributions/beta17/h_normal_tj_nSub21_17_bin1.pdf} \hspace{1mm}
\includegraphics[width=0.3\textwidth]{sascha_input/Appendix/Distributions/higgs/distributions/beta17/h_normal_tj_nSub21_17_bin2.pdf} \hspace{4mm}
\includegraphics[width=0.3\textwidth]{sascha_input/Appendix/Distributions/higgs/distributions/beta17/h_normal_tj_nSub21_17_bin3.pdf} 
\bigskip
\includegraphics[width=0.3\textwidth]{sascha_input/Appendix/Distributions/higgs/distributions/beta17/h_normal_tj_nSub21_17_bin4.pdf} \hspace{4mm}
\includegraphics[width=0.3\textwidth]{sascha_input/Appendix/Distributions/higgs/distributions/beta17/h_normal_tj_nSub21_17_bin5.pdf} 

\caption{\footnotesize{Distributions for Higgs boson tagging using tracks $\beta=1.7$. C2, D2, $\tau_{21}$ top down.}}
\end{figure}
\begin{figure}[H]
\includegraphics[width=0.3\textwidth]{sascha_input/Appendix/Distributions/higgs/distributions/beta17/h_recoJet_C2_17_bin1.pdf} \hspace{1mm}
\includegraphics[width=0.3\textwidth]{sascha_input/Appendix/Distributions/higgs/distributions/beta17/h_recoJet_C2_17_bin2.pdf} \hspace{4mm}
\includegraphics[width=0.3\textwidth]{sascha_input/Appendix/Distributions/higgs/distributions/beta17/h_recoJet_C2_17_bin3.pdf} 
\bigskip
\includegraphics[width=0.3\textwidth]{sascha_input/Appendix/Distributions/higgs/distributions/beta17/h_recoJet_C2_17_bin4.pdf} \hspace{4mm}
\includegraphics[width=0.3\textwidth]{sascha_input/Appendix/Distributions/higgs/distributions/beta17/h_recoJet_C2_17_bin5.pdf} 

\bigskip
\includegraphics[width=0.3\textwidth]{sascha_input/Appendix/Distributions/higgs/distributions/beta17/h_recoJet_D2_17_bin1.pdf} \hspace{1mm}
\includegraphics[width=0.3\textwidth]{sascha_input/Appendix/Distributions/higgs/distributions/beta17/h_recoJet_D2_17_bin2.pdf} \hspace{4mm}
\includegraphics[width=0.3\textwidth]{sascha_input/Appendix/Distributions/higgs/distributions/beta17/h_recoJet_D2_17_bin3.pdf} 
\bigskip
\includegraphics[width=0.3\textwidth]{sascha_input/Appendix/Distributions/higgs/distributions/beta17/h_recoJet_D2_17_bin4.pdf} \hspace{4mm}
\includegraphics[width=0.3\textwidth]{sascha_input/Appendix/Distributions/higgs/distributions/beta17/h_recoJet_D2_17_bin5.pdf} 

\bigskip
\includegraphics[width=0.3\textwidth]{sascha_input/Appendix/Distributions/higgs/distributions/beta17/h_recoJet_nSub21_17_bin1.pdf} \hspace{1mm}
\includegraphics[width=0.3\textwidth]{sascha_input/Appendix/Distributions/higgs/distributions/beta17/h_recoJet_nSub21_17_bin2.pdf} \hspace{4mm}
\includegraphics[width=0.3\textwidth]{sascha_input/Appendix/Distributions/higgs/distributions/beta17/h_recoJet_nSub21_17_bin3.pdf} 
\bigskip
\includegraphics[width=0.3\textwidth]{sascha_input/Appendix/Distributions/higgs/distributions/beta17/h_recoJet_nSub21_17_bin4.pdf} \hspace{4mm}
\includegraphics[width=0.3\textwidth]{sascha_input/Appendix/Distributions/higgs/distributions/beta17/h_recoJet_nSub21_17_bin5.pdf}

\caption{\footnotesize{Distributions for Higgs boson tagging using calorimeter clusters $\beta=1.7$. C2, D2, $\tau_{21}$ top down.}}
\end{figure}
\subsubsection*{$\beta=2$}
\begin{figure}[H]
\includegraphics[width=0.3\textwidth]{sascha_input/Appendix/Distributions/higgs/distributions/beta2/h_assisted_tj_C2_2_bin1.pdf} \hspace{1mm}
\includegraphics[width=0.3\textwidth]{sascha_input/Appendix/Distributions/higgs/distributions/beta2/h_assisted_tj_C2_2_bin2.pdf} \hspace{4mm}
\includegraphics[width=0.3\textwidth]{sascha_input/Appendix/Distributions/higgs/distributions/beta2/h_assisted_tj_C2_2_bin3.pdf} 
\bigskip
\includegraphics[width=0.3\textwidth]{sascha_input/Appendix/Distributions/higgs/distributions/beta2/h_assisted_tj_C2_2_bin4.pdf} \hspace{4mm}
\includegraphics[width=0.3\textwidth]{sascha_input/Appendix/Distributions/higgs/distributions/beta2/h_assisted_tj_C2_2_bin5.pdf} 

\bigskip
\includegraphics[width=0.3\textwidth]{sascha_input/Appendix/Distributions/higgs/distributions/beta2/h_assisted_tj_D2_2_bin1.pdf} \hspace{1mm}
\includegraphics[width=0.3\textwidth]{sascha_input/Appendix/Distributions/higgs/distributions/beta2/h_assisted_tj_D2_2_bin2.pdf} \hspace{4mm}
\includegraphics[width=0.3\textwidth]{sascha_input/Appendix/Distributions/higgs/distributions/beta2/h_assisted_tj_D2_2_bin3.pdf} 
\bigskip
\includegraphics[width=0.3\textwidth]{sascha_input/Appendix/Distributions/higgs/distributions/beta2/h_assisted_tj_D2_2_bin4.pdf} \hspace{4mm}
\includegraphics[width=0.3\textwidth]{sascha_input/Appendix/Distributions/higgs/distributions/beta2/h_assisted_tj_D2_2_bin5.pdf} 

\bigskip 
\includegraphics[width=0.3\textwidth]{sascha_input/Appendix/Distributions/higgs/distributions/beta2/h_assisted_tj_nSub21_2_bin1.pdf} \hspace{1mm}
\includegraphics[width=0.3\textwidth]{sascha_input/Appendix/Distributions/higgs/distributions/beta2/h_assisted_tj_nSub21_2_bin2.pdf} \hspace{4mm}
\includegraphics[width=0.3\textwidth]{sascha_input/Appendix/Distributions/higgs/distributions/beta2/h_assisted_tj_nSub21_2_bin3.pdf} 
\bigskip
\includegraphics[width=0.3\textwidth]{sascha_input/Appendix/Distributions/higgs/distributions/beta2/h_assisted_tj_nSub21_2_bin4.pdf} \hspace{4mm}
\includegraphics[width=0.3\textwidth]{sascha_input/Appendix/Distributions/higgs/distributions/beta2/h_assisted_tj_nSub21_2_bin5.pdf} 

\caption{\footnotesize{Distributions for Higgs boson tagging using TAS $\beta=2$. C2, D2, $\tau_{21}$ top down.}}
\end{figure}
\begin{figure}[H]
\includegraphics[width=0.3\textwidth]{sascha_input/Appendix/Distributions/higgs/distributions/beta2/h_normal_tj_C2_2_bin1.pdf} \hspace{1mm}
\includegraphics[width=0.3\textwidth]{sascha_input/Appendix/Distributions/higgs/distributions/beta2/h_normal_tj_C2_2_bin2.pdf} \hspace{4mm}
\includegraphics[width=0.3\textwidth]{sascha_input/Appendix/Distributions/higgs/distributions/beta2/h_normal_tj_C2_2_bin3.pdf} 
\bigskip
\includegraphics[width=0.3\textwidth]{sascha_input/Appendix/Distributions/higgs/distributions/beta2/h_normal_tj_C2_2_bin4.pdf} \hspace{4mm}
\includegraphics[width=0.3\textwidth]{sascha_input/Appendix/Distributions/higgs/distributions/beta2/h_normal_tj_C2_2_bin5.pdf} 

\bigskip
\includegraphics[width=0.3\textwidth]{sascha_input/Appendix/Distributions/higgs/distributions/beta2/h_normal_tj_D2_2_bin1.pdf} \hspace{1mm}
\includegraphics[width=0.3\textwidth]{sascha_input/Appendix/Distributions/higgs/distributions/beta2/h_normal_tj_D2_2_bin2.pdf} \hspace{4mm}
\includegraphics[width=0.3\textwidth]{sascha_input/Appendix/Distributions/higgs/distributions/beta2/h_normal_tj_D2_2_bin3.pdf} 
\bigskip
\includegraphics[width=0.3\textwidth]{sascha_input/Appendix/Distributions/higgs/distributions/beta2/h_normal_tj_D2_2_bin4.pdf} \hspace{4mm}
\includegraphics[width=0.3\textwidth]{sascha_input/Appendix/Distributions/higgs/distributions/beta2/h_normal_tj_D2_2_bin5.pdf} 

\bigskip
\includegraphics[width=0.3\textwidth]{sascha_input/Appendix/Distributions/higgs/distributions/beta2/h_normal_tj_nSub21_2_bin1.pdf} \hspace{1mm}
\includegraphics[width=0.3\textwidth]{sascha_input/Appendix/Distributions/higgs/distributions/beta2/h_normal_tj_nSub21_2_bin2.pdf} \hspace{4mm}
\includegraphics[width=0.3\textwidth]{sascha_input/Appendix/Distributions/higgs/distributions/beta2/h_normal_tj_nSub21_2_bin3.pdf} 
\bigskip
\includegraphics[width=0.3\textwidth]{sascha_input/Appendix/Distributions/higgs/distributions/beta2/h_normal_tj_nSub21_2_bin4.pdf} \hspace{4mm}
\includegraphics[width=0.3\textwidth]{sascha_input/Appendix/Distributions/higgs/distributions/beta2/h_normal_tj_nSub21_2_bin5.pdf} 

\caption{\footnotesize{Distributions for Higgs boson tagging using tracks $\beta=2$. C2, D2, $\tau_{21}$ top down.}}
\end{figure}
\begin{figure}[H]
\includegraphics[width=0.3\textwidth]{sascha_input/Appendix/Distributions/higgs/distributions/beta2/h_recoJet_C2_2_bin1.pdf} \hspace{1mm}
\includegraphics[width=0.3\textwidth]{sascha_input/Appendix/Distributions/higgs/distributions/beta2/h_recoJet_C2_2_bin2.pdf} \hspace{4mm}
\includegraphics[width=0.3\textwidth]{sascha_input/Appendix/Distributions/higgs/distributions/beta2/h_recoJet_C2_2_bin3.pdf} 
\bigskip
\includegraphics[width=0.3\textwidth]{sascha_input/Appendix/Distributions/higgs/distributions/beta2/h_recoJet_C2_2_bin4.pdf} \hspace{4mm}
\includegraphics[width=0.3\textwidth]{sascha_input/Appendix/Distributions/higgs/distributions/beta2/h_recoJet_C2_2_bin5.pdf} 

\bigskip
\includegraphics[width=0.3\textwidth]{sascha_input/Appendix/Distributions/higgs/distributions/beta2/h_recoJet_D2_2_bin1.pdf} \hspace{1mm}
\includegraphics[width=0.3\textwidth]{sascha_input/Appendix/Distributions/higgs/distributions/beta2/h_recoJet_D2_2_bin2.pdf} \hspace{4mm}
\includegraphics[width=0.3\textwidth]{sascha_input/Appendix/Distributions/higgs/distributions/beta2/h_recoJet_D2_2_bin3.pdf} 
\bigskip
\includegraphics[width=0.3\textwidth]{sascha_input/Appendix/Distributions/higgs/distributions/beta2/h_recoJet_D2_2_bin4.pdf} \hspace{4mm}
\includegraphics[width=0.3\textwidth]{sascha_input/Appendix/Distributions/higgs/distributions/beta2/h_recoJet_D2_2_bin5.pdf} 

\bigskip
\includegraphics[width=0.3\textwidth]{sascha_input/Appendix/Distributions/higgs/distributions/beta2/h_recoJet_nSub21_2_bin1.pdf} \hspace{1mm}
\includegraphics[width=0.3\textwidth]{sascha_input/Appendix/Distributions/higgs/distributions/beta2/h_recoJet_nSub21_2_bin2.pdf} \hspace{4mm}
\includegraphics[width=0.3\textwidth]{sascha_input/Appendix/Distributions/higgs/distributions/beta2/h_recoJet_nSub21_2_bin3.pdf} 
\bigskip
\includegraphics[width=0.3\textwidth]{sascha_input/Appendix/Distributions/higgs/distributions/beta2/h_recoJet_nSub21_2_bin4.pdf} \hspace{4mm}
\includegraphics[width=0.3\textwidth]{sascha_input/Appendix/Distributions/higgs/distributions/beta2/h_recoJet_nSub21_2_bin5.pdf} 

\caption{\footnotesize{Distributions for Higgs boson tagging using calorimeter clusters $\beta=2$. C2, D2, $\tau_{21}$ top down.}}
\end{figure}
\subsubsection*{$\beta=3$}
\begin{figure}[H]
\includegraphics[width=0.3\textwidth]{sascha_input/Appendix/Distributions/higgs/distributions/beta3/h_assisted_tj_C2_3_bin1.pdf} \hspace{1mm}
\includegraphics[width=0.3\textwidth]{sascha_input/Appendix/Distributions/higgs/distributions/beta3/h_assisted_tj_C2_3_bin2.pdf} \hspace{4mm}
\includegraphics[width=0.3\textwidth]{sascha_input/Appendix/Distributions/higgs/distributions/beta3/h_assisted_tj_C2_3_bin3.pdf} 
\bigskip
\includegraphics[width=0.3\textwidth]{sascha_input/Appendix/Distributions/higgs/distributions/beta3/h_assisted_tj_C2_3_bin4.pdf} \hspace{4mm}
\includegraphics[width=0.3\textwidth]{sascha_input/Appendix/Distributions/higgs/distributions/beta3/h_assisted_tj_C2_3_bin5.pdf} 

\bigskip
\includegraphics[width=0.3\textwidth]{sascha_input/Appendix/Distributions/higgs/distributions/beta3/h_assisted_tj_D2_3_bin1.pdf} \hspace{1mm}
\includegraphics[width=0.3\textwidth]{sascha_input/Appendix/Distributions/higgs/distributions/beta3/h_assisted_tj_D2_3_bin2.pdf} \hspace{4mm}
\includegraphics[width=0.3\textwidth]{sascha_input/Appendix/Distributions/higgs/distributions/beta3/h_assisted_tj_D2_3_bin3.pdf} 
\bigskip
\includegraphics[width=0.3\textwidth]{sascha_input/Appendix/Distributions/higgs/distributions/beta3/h_assisted_tj_D2_3_bin4.pdf} \hspace{4mm}
\includegraphics[width=0.3\textwidth]{sascha_input/Appendix/Distributions/higgs/distributions/beta3/h_assisted_tj_D2_3_bin5.pdf} 

\bigskip 
\includegraphics[width=0.3\textwidth]{sascha_input/Appendix/Distributions/higgs/distributions/beta3/h_assisted_tj_nSub21_3_bin1.pdf} \hspace{1mm}
\includegraphics[width=0.3\textwidth]{sascha_input/Appendix/Distributions/higgs/distributions/beta3/h_assisted_tj_nSub21_3_bin2.pdf} \hspace{4mm}
\includegraphics[width=0.3\textwidth]{sascha_input/Appendix/Distributions/higgs/distributions/beta3/h_assisted_tj_nSub21_3_bin3.pdf} 
\bigskip
\includegraphics[width=0.3\textwidth]{sascha_input/Appendix/Distributions/higgs/distributions/beta3/h_assisted_tj_nSub21_3_bin4.pdf} \hspace{4mm}
\includegraphics[width=0.3\textwidth]{sascha_input/Appendix/Distributions/higgs/distributions/beta3/h_assisted_tj_nSub21_3_bin5.pdf} 

\caption{\footnotesize{Distributions for Higgs boson tagging using TAS $\beta=3$. C2, D2, $\tau_{21}$ top down.}}
\end{figure}
\begin{figure}[H]
\includegraphics[width=0.3\textwidth]{sascha_input/Appendix/Distributions/higgs/distributions/beta3/h_normal_tj_C2_3_bin1.pdf} \hspace{1mm}
\includegraphics[width=0.3\textwidth]{sascha_input/Appendix/Distributions/higgs/distributions/beta3/h_normal_tj_C2_3_bin2.pdf} \hspace{4mm}
\includegraphics[width=0.3\textwidth]{sascha_input/Appendix/Distributions/higgs/distributions/beta3/h_normal_tj_C2_3_bin3.pdf} 
\bigskip
\includegraphics[width=0.3\textwidth]{sascha_input/Appendix/Distributions/higgs/distributions/beta3/h_normal_tj_C2_3_bin4.pdf} \hspace{4mm}
\includegraphics[width=0.3\textwidth]{sascha_input/Appendix/Distributions/higgs/distributions/beta3/h_normal_tj_C2_3_bin5.pdf} 

\bigskip
\includegraphics[width=0.3\textwidth]{sascha_input/Appendix/Distributions/higgs/distributions/beta3/h_normal_tj_D2_3_bin1.pdf} \hspace{1mm}
\includegraphics[width=0.3\textwidth]{sascha_input/Appendix/Distributions/higgs/distributions/beta3/h_normal_tj_D2_3_bin2.pdf} \hspace{4mm}
\includegraphics[width=0.3\textwidth]{sascha_input/Appendix/Distributions/higgs/distributions/beta3/h_normal_tj_D2_3_bin3.pdf} 
\bigskip
\includegraphics[width=0.3\textwidth]{sascha_input/Appendix/Distributions/higgs/distributions/beta3/h_normal_tj_D2_3_bin4.pdf} \hspace{4mm}
\includegraphics[width=0.3\textwidth]{sascha_input/Appendix/Distributions/higgs/distributions/beta3/h_normal_tj_D2_3_bin5.pdf} 

\bigskip
\includegraphics[width=0.3\textwidth]{sascha_input/Appendix/Distributions/higgs/distributions/beta3/h_normal_tj_nSub21_3_bin1.pdf} \hspace{1mm}
\includegraphics[width=0.3\textwidth]{sascha_input/Appendix/Distributions/higgs/distributions/beta3/h_normal_tj_nSub21_3_bin2.pdf} \hspace{4mm}
\includegraphics[width=0.3\textwidth]{sascha_input/Appendix/Distributions/higgs/distributions/beta3/h_normal_tj_nSub21_3_bin3.pdf} 
\bigskip
\includegraphics[width=0.3\textwidth]{sascha_input/Appendix/Distributions/higgs/distributions/beta3/h_normal_tj_nSub21_3_bin4.pdf} \hspace{4mm}
\includegraphics[width=0.3\textwidth]{sascha_input/Appendix/Distributions/higgs/distributions/beta3/h_normal_tj_nSub21_3_bin5.pdf} 

\caption{\footnotesize{Distributions for Higgs boson tagging using tracks $\beta=3$. C2, D2, $\tau_{21}$ top down.}}
\end{figure}
\begin{figure}[H]
\includegraphics[width=0.3\textwidth]{sascha_input/Appendix/Distributions/higgs/distributions/beta3/h_recoJet_C2_3_bin1.pdf} \hspace{1mm}
\includegraphics[width=0.3\textwidth]{sascha_input/Appendix/Distributions/higgs/distributions/beta3/h_recoJet_C2_3_bin2.pdf} \hspace{4mm}
\includegraphics[width=0.3\textwidth]{sascha_input/Appendix/Distributions/higgs/distributions/beta3/h_recoJet_C2_3_bin3.pdf} 
\bigskip
\includegraphics[width=0.3\textwidth]{sascha_input/Appendix/Distributions/higgs/distributions/beta3/h_recoJet_C2_3_bin4.pdf} \hspace{4mm}
\includegraphics[width=0.3\textwidth]{sascha_input/Appendix/Distributions/higgs/distributions/beta3/h_recoJet_C2_3_bin5.pdf} 

\bigskip
\includegraphics[width=0.3\textwidth]{sascha_input/Appendix/Distributions/higgs/distributions/beta3/h_recoJet_D2_3_bin1.pdf} \hspace{1mm}
\includegraphics[width=0.3\textwidth]{sascha_input/Appendix/Distributions/higgs/distributions/beta3/h_recoJet_D2_3_bin2.pdf} \hspace{4mm}
\includegraphics[width=0.3\textwidth]{sascha_input/Appendix/Distributions/higgs/distributions/beta3/h_recoJet_D2_3_bin3.pdf} 
\bigskip
\includegraphics[width=0.3\textwidth]{sascha_input/Appendix/Distributions/higgs/distributions/beta3/h_recoJet_D2_3_bin4.pdf} \hspace{4mm}
\includegraphics[width=0.3\textwidth]{sascha_input/Appendix/Distributions/higgs/distributions/beta3/h_recoJet_D2_3_bin5.pdf} 

\bigskip
\includegraphics[width=0.3\textwidth]{sascha_input/Appendix/Distributions/higgs/distributions/beta3/h_recoJet_nSub21_3_bin1.pdf} \hspace{1mm}
\includegraphics[width=0.3\textwidth]{sascha_input/Appendix/Distributions/higgs/distributions/beta3/h_recoJet_nSub21_3_bin2.pdf} \hspace{4mm}
\includegraphics[width=0.3\textwidth]{sascha_input/Appendix/Distributions/higgs/distributions/beta3/h_recoJet_nSub21_3_bin3.pdf} 
\bigskip
\includegraphics[width=0.3\textwidth]{sascha_input/Appendix/Distributions/higgs/distributions/beta3/h_recoJet_nSub21_3_bin4.pdf} \hspace{4mm}
\includegraphics[width=0.3\textwidth]{sascha_input/Appendix/Distributions/higgs/distributions/beta3/h_recoJet_nSub21_3_bin5.pdf} 

\caption{\footnotesize{Distributions for Higgs boson tagging using calorimeter clusters $\beta=3$. C2, D2, $\tau_{21}$ top down.}}
\end{figure}




\subsection{Top Distributions}
\subsubsection*{$\beta=1$}
\vspace{-0.5cm}
\begin{figure}[H]
\includegraphics[width=0.3\textwidth]{sascha_input/Appendix/Distributions/top/distributions/beta1/h_assisted_tj_nSub32_bin1.pdf} \hspace{1mm}
\includegraphics[width=0.3\textwidth]{sascha_input/Appendix/Distributions/top/distributions/beta1/h_assisted_tj_nSub32_bin2.pdf} \hspace{1mm}
\includegraphics[width=0.3\textwidth]{sascha_input/Appendix/Distributions/top/distributions/beta1/h_assisted_tj_nSub32_bin3.pdf} 
\bigskip
\includegraphics[width=0.3\textwidth]{sascha_input/Appendix/Distributions/top/distributions/beta1/h_assisted_tj_nSub32_bin4.pdf} \hspace{1mm}
\includegraphics[width=0.3\textwidth]{sascha_input/Appendix/Distributions/top/distributions/beta1/h_assisted_tj_nSub32_bin5.pdf} \hspace{1mm}
\includegraphics[width=0.3\textwidth]{sascha_input/Appendix/Distributions/top/distributions/beta1/h_assisted_tj_nSub32_bin6.pdf} 
\bigskip
\includegraphics[width=0.3\textwidth]{sascha_input/Appendix/Distributions/top/distributions/beta1/h_normal_tj_nSub32_bin1.pdf} \hspace{1mm}
\includegraphics[width=0.3\textwidth]{sascha_input/Appendix/Distributions/top/distributions/beta1/h_normal_tj_nSub32_bin2.pdf} \hspace{1mm}
\includegraphics[width=0.3\textwidth]{sascha_input/Appendix/Distributions/top/distributions/beta1/h_normal_tj_nSub32_bin3.pdf} 
\bigskip
\includegraphics[width=0.3\textwidth]{sascha_input/Appendix/Distributions/top/distributions/beta1/h_normal_tj_nSub32_bin4.pdf} \hspace{1mm}
\includegraphics[width=0.3\textwidth]{sascha_input/Appendix/Distributions/top/distributions/beta1/h_normal_tj_nSub32_bin5.pdf} \hspace{1mm}
\includegraphics[width=0.3\textwidth]{sascha_input/Appendix/Distributions/top/distributions/beta1/h_normal_tj_nSub32_bin6.pdf} 
\bigskip
\includegraphics[width=0.3\textwidth]{sascha_input/Appendix/Distributions/top/distributions/beta1/h_recoJet_nSub32_bin1.pdf} \hspace{1mm}
\includegraphics[width=0.3\textwidth]{sascha_input/Appendix/Distributions/top/distributions/beta1/h_recoJet_nSub32_bin2.pdf} \hspace{1mm}
\includegraphics[width=0.3\textwidth]{sascha_input/Appendix/Distributions/top/distributions/beta1/h_recoJet_nSub32_bin3.pdf} 
\bigskip
\includegraphics[width=0.3\textwidth]{sascha_input/Appendix/Distributions/top/distributions/beta1/h_recoJet_nSub32_bin4.pdf} \hspace{6mm}
\includegraphics[width=0.3\textwidth]{sascha_input/Appendix/Distributions/top/distributions/beta1/h_recoJet_nSub32_bin5.pdf} \hspace{6mm}
\includegraphics[width=0.3\textwidth]{sascha_input/Appendix/Distributions/top/distributions/beta1/h_recoJet_nSub32_bin6.pdf}
\vspace{-0.75cm}
\caption{\footnotesize{Distributions for Top tagging using $\tau_{32}$ ($\beta=1$) using TAS, tracks and calorimeter clusters top down.}}
\end{figure}

\subsubsection*{$\beta=1.7$}
\begin{figure}[H]
\includegraphics[width=0.3\textwidth]{sascha_input/Appendix/Distributions/top/distributions/beta17/h_assisted_tj_nSub32_17_bin1.pdf} \hspace{1mm}
\includegraphics[width=0.3\textwidth]{sascha_input/Appendix/Distributions/top/distributions/beta17/h_assisted_tj_nSub32_17_bin2.pdf} \hspace{1mm}
\includegraphics[width=0.3\textwidth]{sascha_input/Appendix/Distributions/top/distributions/beta17/h_assisted_tj_nSub32_17_bin3.pdf} 
\bigskip
\includegraphics[width=0.3\textwidth]{sascha_input/Appendix/Distributions/top/distributions/beta17/h_assisted_tj_nSub32_17_bin4.pdf} \hspace{1mm}
\includegraphics[width=0.3\textwidth]{sascha_input/Appendix/Distributions/top/distributions/beta17/h_assisted_tj_nSub32_17_bin5.pdf} \hspace{1mm}
\includegraphics[width=0.3\textwidth]{sascha_input/Appendix/Distributions/top/distributions/beta17/h_assisted_tj_nSub32_17_bin6.pdf} 
\bigskip
\includegraphics[width=0.3\textwidth]{sascha_input/Appendix/Distributions/top/distributions/beta17/h_normal_tj_nSub32_17_bin1.pdf} \hspace{1mm}
\includegraphics[width=0.3\textwidth]{sascha_input/Appendix/Distributions/top/distributions/beta17/h_normal_tj_nSub32_17_bin2.pdf} \hspace{1mm}
\includegraphics[width=0.3\textwidth]{sascha_input/Appendix/Distributions/top/distributions/beta17/h_normal_tj_nSub32_17_bin3.pdf} 
\bigskip
\includegraphics[width=0.3\textwidth]{sascha_input/Appendix/Distributions/top/distributions/beta17/h_normal_tj_nSub32_17_bin4.pdf} \hspace{1mm}
\includegraphics[width=0.3\textwidth]{sascha_input/Appendix/Distributions/top/distributions/beta17/h_normal_tj_nSub32_17_bin5.pdf} \hspace{1mm}
\includegraphics[width=0.3\textwidth]{sascha_input/Appendix/Distributions/top/distributions/beta17/h_normal_tj_nSub32_17_bin6.pdf} 
\bigskip
\includegraphics[width=0.3\textwidth]{sascha_input/Appendix/Distributions/top/distributions/beta17/h_recoJet_nSub32_17_bin1.pdf} \hspace{1mm}
\includegraphics[width=0.3\textwidth]{sascha_input/Appendix/Distributions/top/distributions/beta17/h_recoJet_nSub32_17_bin2.pdf} \hspace{1mm}
\includegraphics[width=0.3\textwidth]{sascha_input/Appendix/Distributions/top/distributions/beta17/h_recoJet_nSub32_17_bin3.pdf} 
\bigskip
\includegraphics[width=0.3\textwidth]{sascha_input/Appendix/Distributions/top/distributions/beta17/h_recoJet_nSub32_17_bin4.pdf} \hspace{6mm}
\includegraphics[width=0.3\textwidth]{sascha_input/Appendix/Distributions/top/distributions/beta17/h_recoJet_nSub32_17_bin5.pdf} \hspace{6mm}
\includegraphics[width=0.3\textwidth]{sascha_input/Appendix/Distributions/top/distributions/beta17/h_recoJet_nSub32_17_bin6.pdf}
\caption{\footnotesize{Distributions for Top tagging using $\tau_{32}$ ($\beta=1.7$) using TAS, tracks and calorimeter clusters top down.}}
\end{figure}

\subsubsection*{$\beta=2$}
\begin{figure}[H]
\includegraphics[width=0.3\textwidth]{sascha_input/Appendix/Distributions/top/distributions/beta2/h_assisted_tj_nSub32_2_bin1.pdf} \hspace{1mm}
\includegraphics[width=0.3\textwidth]{sascha_input/Appendix/Distributions/top/distributions/beta2/h_assisted_tj_nSub32_2_bin2.pdf} \hspace{1mm}
\includegraphics[width=0.3\textwidth]{sascha_input/Appendix/Distributions/top/distributions/beta2/h_assisted_tj_nSub32_2_bin3.pdf} 
\bigskip
\includegraphics[width=0.3\textwidth]{sascha_input/Appendix/Distributions/top/distributions/beta2/h_assisted_tj_nSub32_2_bin4.pdf} \hspace{1mm}
\includegraphics[width=0.3\textwidth]{sascha_input/Appendix/Distributions/top/distributions/beta2/h_assisted_tj_nSub32_2_bin5.pdf} \hspace{1mm}
\includegraphics[width=0.3\textwidth]{sascha_input/Appendix/Distributions/top/distributions/beta2/h_assisted_tj_nSub32_2_bin6.pdf} 
\bigskip
\includegraphics[width=0.3\textwidth]{sascha_input/Appendix/Distributions/top/distributions/beta2/h_normal_tj_nSub32_2_bin1.pdf} \hspace{1mm}
\includegraphics[width=0.3\textwidth]{sascha_input/Appendix/Distributions/top/distributions/beta2/h_normal_tj_nSub32_2_bin2.pdf} \hspace{1mm}
\includegraphics[width=0.3\textwidth]{sascha_input/Appendix/Distributions/top/distributions/beta2/h_normal_tj_nSub32_2_bin3.pdf} 
\bigskip
\includegraphics[width=0.3\textwidth]{sascha_input/Appendix/Distributions/top/distributions/beta2/h_normal_tj_nSub32_2_bin4.pdf} \hspace{1mm}
\includegraphics[width=0.3\textwidth]{sascha_input/Appendix/Distributions/top/distributions/beta2/h_normal_tj_nSub32_2_bin5.pdf} \hspace{1mm}
\includegraphics[width=0.3\textwidth]{sascha_input/Appendix/Distributions/top/distributions/beta2/h_normal_tj_nSub32_2_bin6.pdf} 
\bigskip
\includegraphics[width=0.3\textwidth]{sascha_input/Appendix/Distributions/top/distributions/beta2/h_recoJet_nSub32_2_bin1.pdf} \hspace{1mm}
\includegraphics[width=0.3\textwidth]{sascha_input/Appendix/Distributions/top/distributions/beta2/h_recoJet_nSub32_2_bin2.pdf} \hspace{1mm}
\includegraphics[width=0.3\textwidth]{sascha_input/Appendix/Distributions/top/distributions/beta2/h_recoJet_nSub32_2_bin3.pdf} 
\bigskip
\includegraphics[width=0.3\textwidth]{sascha_input/Appendix/Distributions/top/distributions/beta2/h_recoJet_nSub32_2_bin4.pdf} \hspace{6mm}
\includegraphics[width=0.3\textwidth]{sascha_input/Appendix/Distributions/top/distributions/beta2/h_recoJet_nSub32_2_bin5.pdf} \hspace{6mm}
\includegraphics[width=0.3\textwidth]{sascha_input/Appendix/Distributions/top/distributions/beta2/h_recoJet_nSub32_2_bin6.pdf}
\caption{\footnotesize{Distributions for Top tagging using $\tau_{32}$ ($\beta=2$) using TAS, tracks and calorimeter clusters top down.}}
\end{figure}

\subsubsection*{$\beta=3$}
\begin{figure}[H]
\includegraphics[width=0.3\textwidth]{sascha_input/Appendix/Distributions/top/distributions/beta3/h_assisted_tj_nSub32_3_bin1.pdf} \hspace{1mm}
\includegraphics[width=0.3\textwidth]{sascha_input/Appendix/Distributions/top/distributions/beta3/h_assisted_tj_nSub32_3_bin2.pdf} \hspace{1mm}
\includegraphics[width=0.3\textwidth]{sascha_input/Appendix/Distributions/top/distributions/beta3/h_assisted_tj_nSub32_3_bin3.pdf} 
\bigskip
\includegraphics[width=0.3\textwidth]{sascha_input/Appendix/Distributions/top/distributions/beta3/h_assisted_tj_nSub32_3_bin4.pdf} \hspace{1mm}
\includegraphics[width=0.3\textwidth]{sascha_input/Appendix/Distributions/top/distributions/beta3/h_assisted_tj_nSub32_3_bin5.pdf} \hspace{1mm}
\includegraphics[width=0.3\textwidth]{sascha_input/Appendix/Distributions/top/distributions/beta3/h_assisted_tj_nSub32_3_bin6.pdf} 
\bigskip
\includegraphics[width=0.3\textwidth]{sascha_input/Appendix/Distributions/top/distributions/beta3/h_normal_tj_nSub32_3_bin1.pdf} \hspace{1mm}
\includegraphics[width=0.3\textwidth]{sascha_input/Appendix/Distributions/top/distributions/beta3/h_normal_tj_nSub32_3_bin2.pdf} \hspace{1mm}
\includegraphics[width=0.3\textwidth]{sascha_input/Appendix/Distributions/top/distributions/beta3/h_normal_tj_nSub32_3_bin3.pdf} 
\bigskip
\includegraphics[width=0.3\textwidth]{sascha_input/Appendix/Distributions/top/distributions/beta3/h_normal_tj_nSub32_3_bin4.pdf} \hspace{1mm}
\includegraphics[width=0.3\textwidth]{sascha_input/Appendix/Distributions/top/distributions/beta3/h_normal_tj_nSub32_3_bin5.pdf} \hspace{1mm}
\includegraphics[width=0.3\textwidth]{sascha_input/Appendix/Distributions/top/distributions/beta3/h_normal_tj_nSub32_3_bin6.pdf} 
\bigskip
\includegraphics[width=0.3\textwidth]{sascha_input/Appendix/Distributions/top/distributions/beta3/h_recoJet_nSub32_3_bin1.pdf} \hspace{1mm}
\includegraphics[width=0.3\textwidth]{sascha_input/Appendix/Distributions/top/distributions/beta3/h_recoJet_nSub32_3_bin2.pdf} \hspace{1mm}
\includegraphics[width=0.3\textwidth]{sascha_input/Appendix/Distributions/top/distributions/beta3/h_recoJet_nSub32_3_bin3.pdf} 
\bigskip
\includegraphics[width=0.3\textwidth]{sascha_input/Appendix/Distributions/top/distributions/beta3/h_recoJet_nSub32_3_bin4.pdf} \hspace{6mm}
\includegraphics[width=0.3\textwidth]{sascha_input/Appendix/Distributions/top/distributions/beta3/h_recoJet_nSub32_3_bin5.pdf} \hspace{6mm}
\includegraphics[width=0.3\textwidth]{sascha_input/Appendix/Distributions/top/distributions/beta3/h_recoJet_nSub32_3_bin6.pdf}
\caption{\footnotesize{Distributions for Top tagging using $\tau_{32}$ ($\beta=3$) using TAS, tracks and calorimeter clusters top down.}}
\end{figure}



%-------------------------------------------------------------------------------
% If you use biblatex and either biber or bibtex to process the bibliography
% just say \printbibliography here

\printbibliography
\begin{comment}
\begin{thebibliography}{9}

% \bibitem{pdg2014}
%   Particle Data Group Collaboration, K. Olive \emph{et al.},
%   \emph{ Review of Particle Physics},
% Chinese Physics C 38 (2014) 090001.

% \bibitem{zprime}
% Paul Langacker,
% \emph{The Physics of Heavy
% Z Gauge Bosons}. In: Rev. Mod. Phys.
% 81 (2009), pp. 1199–1228. doi: 10.1103/RevModPhys.81.1199. arXiv: 0801.1345
% [hep-ph].

% \bibitem{wprime}
% Martin Schmaltz and Christian Spethmann,
% \emph{Two Simple W’ Models for the Early
% LHC}. In: JHEP 07 (2011), p. 046. doi: 10.1007/JHEP07(2011)046. arXiv: 1011.
% 5918 [hep-ph].

% \bibitem{diboson}
% Georges Aad \emph{et al.}
% \emph{Combination of searches for WW, WZ, and ZZ resonances
% in pp collisions at $\sqrt{s}$ = 8 TeV with the ATLAS detector}. In: Phys. Lett. B755
% (2016), pp. 285–305. doi: 10.1016/j.physletb.2016.02.015. arXiv: 1512.05099
% [hep-ex].

% \bibitem{dibosoncms}
% The CMS Collaboration,
% \emph{Search for massive resonances decaying into pairs of boosted bosons in semi-leptonic final states at $\cme$ = 8 TeV},
% CERN-PH-EP/2013-037
% 2014/09/03
% arXiv:1405.3447v2 [hep-ex].

% \bibitem{pdgwprime}
% C. Patrignani \emph{et al.},
% Particle Data Group, Chin. Phys. C, 40, 100001 (2016).
% http://pdg.lbl.gov/2015/reviews/rpp2015-rev-wprime-searches.pdf

% \bibitem{pdgzprime}
% C. Patrignani \emph{et al.},
% Particle Data Group, Chin. Phys. C, 40, 100001 (2016).
% http://pdg.lbl.gov/2015/reviews/rpp2015-rev-zprime-searches.pdf

% \bibitem{rsgraviton}
% L. Randall and R. Sundrum,
% \emph{A Large Mass Hierarchy from a Small Extra Dimension},
% Phys. Rev. Lett. 83 (1999) 3370, arXiv: hep-ph/9905221.

% \bibitem{rsgraviton2}
% H. Davoudiasl, J.L. Hewett and T.G. Rizzo,
% \emph{Warped Phenomenology},
% Stanford Linear Accelerator Center
% Stanford CA 94309, USA,
% arXiv:hep-ph/9909255v1 6 Sep 1999.

% \bibitem{tdr}
%   ATLAS Collaboration,
%   \emph{ATLAS detector and physics performance: Technical Design Report, 1 \& 2} ,
%   Tech. Rep. ATLAS TDR 14, CERN/LHCC 99-14, Geneva, 1999.

% \bibitem{lhc}
% L. Evans and P. Bryant,
% \emph{LHC Machine}, JINST 3 (2008) S08001.

% \bibitem{qcdpdg}
% C. Patrignani \emph{et al.},
% Particle Data Group, Chin. Phys. C, 40, 100001 (2016).
% http://pdg.lbl.gov/2010/reviews/rpp2010-rev-qcd.pdf

% \bibitem{alphas}
% The CMS Collaboration,
% \emph{Measurement of the inclusive 3-jet production differential
% cross section in proton-proton collisions at 7 TeV and
% determination of the strong coupling constant in the TeV
% range},
% CERN-PH-EP/2013-037
% 2015/05/04
% arXiv:hep-ph/14121633.

% \bibitem{alpha}
% The ATLAS Collaboration,
% \emph{Improved luminosity determination in pp collisions at $\sqrt {s} = 7\ \mathrm{TeV}$ using the ATLAS detector at the LHC},
% Eur. Phys. J. C (2013) 73: 2518. doi:10.1140/epjc/s10052-013-2518-3.

% \bibitem{tasso}
% W. Bartel \emph{et al.} [JADE Collaboration],
% \emph{Observation of Planar Three Jet Events in e + e -
% Annihilation and Evidence for Gluon Bremsstrahlung},
% Phys. Lett. B 91, 142 (1980).

% \bibitem{qcdhistory}
% Ahmed Ali (DESY), Gustav Kramer (Univ. Hamburg),
% \emph{Jets and QCD: A Historical Review of the Discovery of the Quark and Gluon Jets and its Impact on QCD},
% arXiv:1012.2288v2 [hep-ph].

\bibitem{antiktalgo}
Matteo Cacciari, Gavin P. Salam, Gregory Soyez,
\emph{The anti-k$_t$ jet clustering algorithm},
arXiv:0802.1189v2 [hep-ph]. 

\bibitem{halzenmartin}
F. Halzen, A.D. Martin and D.M. Scott,  
Phys. Rev. D25 (1982) 754,
Phys. Lett. B112 (1982) 160.

\bibitem{partonshower}
Torbjorn Sjostrand,
\emph{Monte Carlo Generators},
hep-ph/0611247,
CERN-LCGAPP-2006-06.

\bibitem{topocluster}
W.~Lampl \emph{et al.},
\emph{Calorimeter Clustering Algorithms: Description and Performance},
ATL-LARG-PUB-2008-002.

\bibitem{lcw}
The ATLAS Collaboration,
\emph{Jet energy resolution in proton-proton collisions at $\sqrt{s}=$7 TeV recorded in 2010 with the ATLAS detector},
CERN-PH-EP-2012-191
arXiv:1210.6210v1  [hep-ex].

\bibitem{geant4}
GEANT4 Collaboration (S. Agostinelli \emph{et al.}), 
\emph{GEANT4: A Simulation toolkit},
Nucl.Instrum.Meth. A506 (2003) 250-303
DOI: 10.1016/S0168-9002(03)01368-8
SLAC-PUB-9350, FERMILAB-PUB-03-339.

\bibitem{calibration}
Ariel Schwartzman,
\emph{Jet energy calibration at the LHC},
SLAC National Accelerator Laboratory
% , 2575 Sand Hill Road
% Menlo Park, California 94025, USA,
arXiv:1509.05459v1 [hep-ex] 17 Sep 2015.

\bibitem{art35}
The ATLAS Collaboration,
\emph{Jet mass reconstruction with the ATLAS Detector in early Run 2 data},
ATLAS-CONF-2016-035
19 July 2016.

\bibitem{art39}
The ATLAS Collaboration,
\emph{Boosted Higgs ($\to b\bar{b}$) Boson Identification with the ATLAS
Detector at $\cme$ = 13 TeV},
ATLAS-CONF-2016-039
30th July 2016.

% \bibitem{dibosonsearch}
% The ATLAS Collaboration,
% \emph{Search for high-mass diboson resonances with boson-tagged jets in proton-proton collisions at $\cme$ = 8 TeV with the ATLAS detector},
% arXiv:1506.00962v3 [hep-ex].

% \bibitem{dibosonsearchcms}
% The CMS Collaboration,
% \emph{Combination of searches for WW, WZ, ZZ, WH, and ZH
% resonances at $\cme$= 8 and 13 TeV},
% CMS PAS B2G-16-007.

% \bibitem{zprimetotops}
% The ATLAS Collaboration,
% \emph{Search for heavy particles decaying to pairs of highly-boosted top
% quarks using lepton-plus-jets events in proton-proton collisions at
% $\cme$ = 13 TeV with the ATLAS detector},
% ATLAS-CONF-2016-014.

% \bibitem{dibosonatlas}
% The ATLAS Collaboration,
% \emph{Search for resonances with boson-tagged jets in 15.5 $fb^{-1}$ of p-p
% collisions at $\cme$ = 13 TeV collected with the ATLAS detector},
% ATLAS-CONF-2016-055.

% \bibitem{dihiggsatlas}
% The ATLAS Collaboration,
% \emph{Search for pair production of Higgs bosons in the $bb\bar{b}\bar{b}$ final state
% using proton proton collisions at $\cme$ = 13TeV with the ATLAS detector},
% ATLAS-CONF-2016-049.

\bibitem{substructure1}
The ATLAS Collaboration,
\emph{Performance of jet substructure techniques for large-R jets in proton-proton collisions at sqrt(s) = 7 TeV using the ATLAS detector},
CERN-PH-EP-2013-069
arXiv:1306.4945 [hep-ex].

\bibitem{ue}
Oleg Zenin, on behalf of the ATLAS Collaboration,
\emph{Soft QCD and underlying event measurements at ATLAS},
Nuclear and Particle Physics Proceedings
Volumes 273275, April-June 2016, Pages 2053-2058
37th International Conference on High Energy Physics (ICHEP).

% \bibitem{highlumi}
% Page title: JetSubstructureECFA2014 AtlasPublic TWiki
% Website name: Twiki.cern.ch
% URL:https://twiki.cern.ch/twiki/bin/view/AtlasPublic/JetSubstructureECFA2014
% Access date: 27th October 2016.

\bibitem{statistic}
E. Lehmann and G. Casella,
\emph{Theory of Point Estimation},
Springer Verlag, 1998,
isbn: 0387985026.

%
%\bibitem{power_counting}
%Andrew~J. Larkoski, Ian Moult, and Duff Neill.
%\emph {Power Counting to Better Jet Observables}.
% JHEP, 12:009, 2014.
%
%\bibitem{analytic_ECF}
%Andrew~J. Larkoski, Ian Moult, and Duff Neill.
%\emph {Analytic Boosted Boson Discrimination}.
% JHEP, 05:117, 2016.
%
%\bibitem{ECF}
%Andrew~J. Larkoski, Gavin~P. Salam, and Jesse Thaler.
%\emph {Energy Correlation Functions for Jet Substructure}.
% JHEP, 06:108, 2013.
%
%\bibitem{nsub}
%Jesse Thaler and Ken Van~Tilburg.
%\emph {Identifying Boosted Objects with N-subjettiness}.
% JHEP, 03:015, 2011.
%
%
%\bibitem{w_tagging}
%Georges Aad et~al.
%\emph {Identification of boosted, hadronically decaying W bosons and
%  comparisons with ATLAS data taken at $\sqrt{s} = 8$ TeV}.
% Eur. Phys. J., C76(3):154, 2016.

\bibitem{presentation}
% Link to presentation in which different 4-momentum correction scheme was applied
Jet Substructure and jet-by-jet tagging meeting:
URL: https://indico.cern.ch/event/495766/contributions/1174422/attachments/1231007/1804520/JSS\_18Feb.pdf

\end{thebibliography}
\end{comment}
%\printbibliography
% If you want to use the traditional BibTeX you need to use the syntax below.
%\bibliographystyle{bibtex/bst/atlasBibStyleWoTitle}
%\bibliography{mydocument,bibtex/bib/ATLAS}
%-------------------------------------------------------------------------------

%-------------------------------------------------------------------------------
% Print the list of contributors to the analysis
% The argument gives the fraction of the text width used for the names
%-------------------------------------------------------------------------------
% \clearpage
% \PrintAtlasContribute{0.30}

%-------------------------------------------------------------------------------
\clearpage
\appendix
% \part*{Auxiliary material}
% \addcontentsline{toc}{part}{Auxiliary material}
%-------------------------------------------------------------------------------

% In an ATLAS paper, auxiliary plots and tables that are supposed to be made public 
% should be collected in an appendix that has the title \enquote{Auxiliary material}.
% This appendix should be printed after the Bibliography.
% At the end of the paper approval procedure, this information can be split into a separate document
% -- see \texttt{atlas-auxmat.tex}.

% In an ATLAS note, use the appendices to include all the technical details of your work
% that are relevant for the ATLAS Collaboration only (e.g.\ dataset details, software release used).
% This information should be printed after the Bibliography.

\end{document}
