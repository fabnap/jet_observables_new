\subsection{Performance in $W \to q'\bar{q}$ Decays}
The $W/Z$ decay was the first one looked at, and with which the $\mtas$ was designed. The $\mcal$ shows a fast deterioration of the performance at high $\pt$, and, as shown in the previous section, the $\mta$ prevents this deterioration but suffers at low transverse momenta ($\pt<1$ TeV).
The $\mtas$ has a similar behavior in the extreme transverse momentum regime as the $\mta$, since the subjet multiplicity peaks at one, where there are no differences between the two observables.
In the low-$\pt$ regime, on the contrary, it exploits the difference in charged to neutral ratio for each subjet, achieving a better performance.
This is shown in Figure \ref{fig:mtas2} as a function of $\pt$: below $\sim$ 1 TeV achieves lower values of the IQnR converging from below to the $\mta$ as the number of subjets decreases to one.

% \begin{figure}[!ht]
%   \centering
%       \includegraphics[width=0.7\textwidth]{jet_part/mtas/71graphcftr_h_JetRatio_mJ12CALOIQRoMWZ.pdf}
%   \caption[$\mtas$ for boosted $W/Z$]{Performance of the $\mtas$ versus the $\mcal$ and $\mta$ for the boosted $W/Z$ sample.}
%   \label{fig:mtas2}
% \end{figure}

\begin{figure}[b]
    \centering
    \begin{subfigure}[b]{0.45\textwidth}
  \centering
      \includegraphics[width=0.9\textwidth]{jet_part/mtas/71graphcftr_h_JetRatio_mJ12CALOIQRoMWZ.pdf}
  \caption[$\mtas$ for boosted $W/Z$]{$W/Z$ jets.}
  \label{fig:mtas2}
    \end{subfigure}%
    \begin{subfigure}[b]{0.45\textwidth}
  \centering
      \includegraphics[width=0.9\textwidth]{jet_part/mtas/71graphcftr_h_JetRatio_mJ12CALOIQRoMHiggs.pdf}
  \caption[$\mtas$ for boosted Higgs]{Higgs jets.}
  \label{fig:mtas4}
    \end{subfigure}
 \begin{subfigure}[b]{0.45\textwidth}
  \centering
      \includegraphics[width=0.9\textwidth]{jet_part/mtas/71graphcftr_h_JetRatio_mJ12CALOIQRoMTops.pdf}
  \caption[$\mtas$ for boosted tops]{Top jets.}
  \label{fig:mtas3}
    \end{subfigure}%
    \caption[Performance of the $\mtas$ versus the $\mcal$ and $\mta$]{Performance of the $\mtas$ versus the $\mcal$ and $\mta$ for $W/Z$, top left, where $\mta$ is not better than $\mcal$ in the low $\pt$ range but is outperformed by the $\mtas$;  Higgs decay, where $\mcal$ is everywhere better than $\mta$, yet comparable with $\mtas$ and top decays where the more complex topology makes critical the high $\pt$ regime} 
    % \label{fig:meanandtail}
\end{figure}


\subsection{Performance in $h\to b\bar{b}$ Decays}
In the Randall-Sundrum graviton to di-Higgs to four b-quark, the performance is again problematic for the $\mta$ with respect to $\mcal$, which is far beyond the latter, while the performance of the $\mtas$ is partially similar to the top-quark decay, but degrades much more in the extreme $\pt$ regime, following the $\mta$. Shown in Figure \ref{fig:mtas4}.


\subsection{Performance in $t\to q'\bar{q}b$ Decays}
The top decays are shown on Figure \ref{fig:mtas3}; the $\mtas$ is comparable yet slightly worse than the $\mcal$ in the low-middle $\pt$ regime, while degrades at higher $\pt$ approaching the $\mta$, which is far beyond the track-assisted subjet mass in performance.
As already noted, the worse performance can be ascribed both to the higher top-quark mass, and to its different and more complex decay topology.


% \begin{figure}[!ht]
%   \centering
%       \includegraphics[width=0.7\textwidth]{jet_part/mtas/71graphcftr_h_JetRatio_mJ12CALOIQRoMTops.pdf}
%   \caption[$\mtas$ for boosted tops]{Performance of the $\mtas$ versus the $\mcal$ and $\mta$ for the boosted top sample.}
%   \label{fig:mtas3}
% \end{figure}


% \begin{figure}[!ht]
%   \centering
%       \includegraphics[width=0.7\textwidth]{jet_part/mtas/71graphcftr_h_JetRatio_mJ12CALOIQRoMHiggs.pdf}
%   \caption[$\mtas$ for boosted Higgs]{Performance of the $\mtas$ versus the $\mcal$ and $\mta$ for the boosted Higgs sample.}
%   \label{fig:mtas4}
% \end{figure}




\subsection{Performance in QCD Multijet Events}
The behavior of the QCD multijet sample is similar to the $W/Z$ sample, where the $\mta$ exhibits a crossing point in the middle-low regime $\pt\simeq900$ GeV and proceeds with a better performance at high transverse momenta.
Again the $\mtas$ follows this similarity showing no crossing point and an optimal overall behavior, both with respect to calorimeter- and track-assisted-based mass definition. On Figure \ref{fig:mtas5}.

\begin{figure}[!ht]
  \centering
%       \includegraphics[width=\textwidth]{jet_part/mtas/qcdmtas.png}
        \includegraphics[width=0.50\textwidth]{jet_part/mtas/qcdmtastruffa.png}
   \caption[$\mtas$ for QCD jets]{Performance of the $\mtas$ versus the $\mcal$ and $\mta$ for the QCD multijet shows a much better behavior of the track-assisted subjet mass. Here shown $50\% \:\textrm{IQnR/median}$ and not the $\iqr$.}
  \label{fig:mtas5}
\end{figure}

\subsection{Performance in Massive $\tilde{W}\to q'\bar{q}$ Decays with $m_{\tilde{W}}=m_t$}
The massive $W$ sample is a special sample which was used to understand the behavior of top jets, whether its worse resolution was coming from the higher mass of the top quark or from the more complex decay topology (three-pronged instead of two-pronged decay and $b$-quark presence). 
The sample is almost identical to the $W/Z$ one ($W'\to WZ$) but in this case the SM electroweak boson have the mass of the top quark $m_{\tilde{W}}=m_t$.
In fact, from the rule $\Delta R\simeq 2m/p_T$, a bigger separation is expected between quarks from the hadronic decay.
The comparison with $\mcal$ is shown in Figure \ref{fig:mtas6}, together with the top-quark jet for completeness. As seen here, the performance of the latter is clearly worse than the former, the trend is yet very similar. This difference is interpreted in terms of different and more complex topology and hence higher subjet multiplicity: in the three subjet structure, resolving accurately the components is more challenging.

\begin{figure}[!ht]
  \centering
     \includegraphics[width=0.55\textwidth]{jet_part/mtas/71graphcftr_h_JetRatio_mJ12CALOIQRoMcalib_WmassiveVsTops.pdf}
   \caption[$\mtas$ for massive $W/Z$]{Performance of the $\mtas$ versus the $\mcal$ for the massive $W/Z$ (in red and green); shown on the same plot also the top sample (in blue and light blue).}
  \label{fig:mtas6}
\end{figure}

\subsection{Stability of Mean of Response and Left-Hand-Side Integral }
The stability of the $\mtas$ was checked, although the IQnR is already a good quantifier of stability, explicitly for the mean of the mass response distribution and for the left-hand-side tail, as a function of the transverse momentum. This was an important check to assure the overall gaussianity of the final distribution in the whole spectrum of $\pt$, and suitability in regards of the calibration step, which is not discussed in this note.

The mean of the response distribution is shown for $W/Z$ decays in Figure \ref{fig:meanandtail}, left; as seen here, despite the mean being constantly below unity, its behavior is much more flat and independent of $\pt$, especially in the low-intermediate regime. This is surprising since the $\mcal$ is already shown after the calibration step, which is not taken instead for the $\mtas$. Conversely the left-hand-side tail of the mass response which is shown in the same figure, right, shows a more enhanced behavior than the $\mcal$, but still never reaches the 10\%. Of course an enhancement of the tail causes a loss of gaussianity and a number of jets which are reconstructed with a lower mass than they should, but it is still comparable with the calorimeter mass.

Those quantifiers show analogous behavior for the other samples considered and those figures can be found in the Appendix.

\begin{figure}
    \centering
    \begin{subfigure}[b]{0.45\textwidth}
	\centering
        \includegraphics[width=\textwidth]{appendixB/mTAS_W_calibmCal_20:07:01-03-11-2016/71graph_h_JetRatio_mJ12CALO_meanResponseMvsTA.pdf}
%         \label{fig:tiger}
    \end{subfigure}
    \begin{subfigure}[b]{0.45\textwidth}
	\centering
        \includegraphics[width=\textwidth]{appendixB/mTAS_W_calibmCal_20:07:01-03-11-2016/74graph_h_JetRatio_mJ12CALO_I50ResponseMvsTAnorm.pdf}
 
%         \label{fig:gull}
    \end{subfigure}
    \caption[Mean and left-hand side integral for boosted $W/Z$]{Stability quantifiers which were checked for the $\mtas$: mean on the left and normalized left-hand side integral of the mass response distribution on the right. The mean is calculated from a Gaussian fit and the integral goes from 0 to 0.6.} 
    \label{fig:meanandtail}
\end{figure}

\subsection{Potential Improvements from Subjet Calibration}

An additional attempt of calibrating the subjet was also tried and, although the results were not substantially improved, it is presented in this subsection. This study was performed using only $W/Z$ samples.

% \subsection{Preliminary Studies on Subjet Calibration}
The \textit{perfect calibration} refers to the procedure of using $\mtas$ with truth-level information for calorimeter and tracker system, i.e. looking at the best possible scenario with an ideal detector. The performance is of course expected to be optimal, because of the use of the truth-level. This step was necessary as feasibility study, to understand whether ulterior efforts in this direction were meaningful.
% The first attempt in calibrating the subjets had as start a ``perfect calibration'', which means using the truth-level information from the MC sample \textit{before} the interaction with the calorimeter.
Truth-level tracks are the particles in the jet which have an electric charge and are stable, truth-level subjets are all the particles, charged and not, which are ghost associated to the calorimeter subjets.
There are few possibilities in doing so, here some nomenclature for this study will be introduced:
\begin{itemize}
 \item $\mtas$ using truth-level subjets and tracks; normal tracks (with all detector effects) are used to assist the truth-level subjets;
 \item $\mtas$ using truth-level tracks and truth-level subjets; the truth-level tracks are used to assist the truth-level subjets;
 \item $\mcal$ truth, calculated using only the truth subjets.
\end{itemize}


% \subsubsection{Perfect Calibration}


% \begin{figure}[!ht]
%   \centering
%       \includegraphics[width=0.7\textwidth]{jet_part/calib/perfcalib.png}
%   \caption[Perfect calibration]{Performance of the perfect calibration, using truth-level subjets and truth-level tracks. It shows room for improvement especially at low-middle $\pt$.}
%   \label{fig:perfcalib}
% \end{figure}


\subsubsection{Simple Subjet Calibration}
The perfect calibration using truth level subjets and tracks is shown in Figure \ref{fig:perfcalib4} in blue dots; since the performance exhibits room for big improvement below $\sim$ 1 TeV and moderate to small improvement above this value, the second step of a simple calibration was tried.

Following the example of calibration of jets in general, a simple approach to emulate this procedure was tried, constructing in various bins of transverse momenta the responses of the subjet's energy to derive the weights factors to be applied. The detailed procedure is as follows:
\begin{enumerate}
 \item Responses in energy $R_E=E^{reco}/E^{truth}$ were built in several bins of $\pt$, spanning to the whole transverse momentum range;
 \item The mean $\mu_R$ of this response was calculated via a fit to the Gaussian core;
 \item Those values (\textit{scale factors}) were stored and applied again to the subjets before the computation of the $\mtas$ via 4-momentum correction $E'=E/\mu_R$; the $\pt$ (the value which only enters the $\mtas$ variable) was changed then correspondingly to keep the subjet's mass constant.
\end{enumerate}

This procedure was called \textit{poor man's calibration} or PM calibration or \textit{simple calibration}.
A check on the $\pt$ response before and after calibration together with the mean of the entire Large-$R$ jet response is shown in Figure \ref{fig:calibA} and \ref{fig:calibA2} in Appendix.

The results are on Figure \ref{fig:perfcalib4}; there are only marginal improvements in few ranges of low transverse momentum where the scale factors are further away from unity, and the overall observable is not performing better than the standard $\mtas$. This is interpreted both in terms of a missing calibration as a function of the $\eta$ variables (having hence a befit from the crack region) and because the correction done on average does not provide the sufficient handle in a jet-by-jet basis, especially when all the subjets are rescaled by similar factors (which translates into a similarity of $\pt$s of the subjets, often the case for e.g. $W/Z$ decays, less for tops jets entirely contained in the large-$R$ jet).

\begin{figure}[!ht]
  \centering
      \includegraphics[width=0.55\textwidth]{jet_part/calib/perfectcalib4.png}
  \caption[Simple calibration]{Performance of the poor man's calibration. The improvement is marginal throughout the entire transverse momentum space.}
  \label{fig:perfcalib4}
\end{figure}

\subsection{Limitation of $\mtas$ from tracking}
The final effort to understand the various and competing effects, which take place in the $\mtas$ brought to a final study on the variable to understand the reason for the worsening of the resolution at high transverse momenta, using the truth MC information.

First of all the track mass resolution was studied: a response of the mass of the tracks associated to the jet ($m^\textrm{trk}$) was constructed, using the truth-level tracks.

The result is shown on Figure \ref{fig:trackdegrade}: for the samples considered, it shows a linear degradation of the $m^\textrm{trk}$, both for massive and SM $W/Z$.

\begin{figure}[!ht]
  \centering
      \includegraphics[width=0.55\textwidth]{jet_part/calib/71graphcftr_h_JetRatio_mJ12CALOIQRoMcalib_trkmass.pdf}
  \caption[Track mass degradation in tops and massive $W/Z$]{The performance of the track mass ($m^\textrm{trk}$) in blue and red for massive $W$ sample and boosted $W/Z$ respectively; for reference in green the calorimeter mass of the large-$R$ jet.}
  \label{fig:trackdegrade}
\end{figure}
% ***change figures with the one in appendix***
The hypothesis of the degradation of the $\mtas$ driven by the tracks is also supported by the Figure \ref{fig:breakdown2} and \ref{fig:breakdown3}, where the truth-level tracks are used instead of real tracks to compute the variable; the flat behavior at high $\pt$ ascribes the worsening of the resolution to tracks at higher transverse momenta.
In particular the black dots show the $\mtas$ using truth-level subjets but real tracks for the track assistance procedure.
Even combining this truth-level information, in fact, it shows a large worsening of the performance (truth-level subjets only are shown as blue dots).
On the other side using again truth-level tracks for the track assistance procedure of the truth-level subjet, shows a recovery of the loss in performance.

% Particularly interesting is the black dots, which 
% uses truth-level subjet but real tracks, which worsen the overall performance of the truth-level subjet alone (shown in light blue dots).

% \begin{figure}[!ht]
%   \centering
%       \includegraphics[width=0.7\textwidth]{jet_part/calib/71graphcftr_h_JetRatio_mJ12CALOIQRoM4Truths.pdf}
%   \caption[Breakdown of the $\mtas$ ]{Breakdown of the $\mtas$ in its component using truth-level information for boosted $W/Z$ decays. In blue the $\mtas$ using truth-level subjets and truth level tracks, in black $\mtas$ using truth level subjets but real tracks and in light blue for reference the mass of the truth level particles associated to the subjets. As usual, in red and green the standard $\mtas$ and the $\mcal$}
%   \label{fig:breakdown2}
% \end{figure}

% Other results using truth-level information on boosted tops are shown and described in the Appendix.

\begin{figure}
    \centering
    \begin{subfigure}[b]{0.45\textwidth}
  \centering
      \includegraphics[width=0.9\textwidth]{jet_part/calib/71graphcftr_h_JetRatio_mJ12CALOIQRoM4Truths.pdf}
  \caption[Breakdown of the $\mtas$ ]{$W/Z$ jets}
  \label{fig:breakdown2}
    \end{subfigure}
    \begin{subfigure}[b]{0.45\textwidth}
  \centering
      \includegraphics[width=0.9\textwidth]{jet_part/appendixA/71graphcftr_h_JetRatio_mJ12CALOIQRoM4TruthsTops.pdf}
  \caption[Breakdown of the $\mtas$ ]{top jets}
  \label{fig:breakdown3}
 
    \end{subfigure}
    \caption[Breakdown of the $\mtas$]{Breakdown of the $\mtas$ in its component using truth-level information for $W/Z$ decays, on the left. In blue the $\mtas$ using truth-level subjets and truth level tracks, in black $\mtas$ using truth level subjets but real tracks and in light blue for reference the mass of the truth level particles associated to the subjets. As usual, in red and green the standard $\mtas$ and the $\mcal$. On the right the same for top jets.} 
    % \label{fig:meanandtail}
\end{figure}

% \Section{Limitation of the $\mtas$}
% Additional studies on the limitation of the $\mtas$ based on MC studies without detector interactions are also presented. In particular, the truth study presented for $W/Z$ decay in were extended for top quark decays.

% As seen on Figure \ref{fig:breakdown3}, the breakdown of the $\mtas$ shows that, in particular for the high transverse momenta regimes, the tracks are subjected to fast degradation which makes their combination with the calorimeter mass not anymore an advantage. 
In particular for the top decay, the breakdown shows that the track degradation is not anymore compensated by the calorimeter for high $\pt$, making their combination not anymore an advantage.

% This limitation was expected and understood from the detector performance point of view, with the variables which are presented here $\mta$ and $\mtas$ to reach a competitive standpoint with the $\mcal$ in the extreme kinematic regime for the top quark decay.

In black, in fact, the performance of the $\mtas$ variable using tracks with detector effect and subjets without those effects, shows this intrinsic limit which takes place already at 1.5 TeV.

The crossing point is, as already pointed out for the top jets, present because of the optimal performance of the calorimeter system caused by the higher mass of the top quark, and partially also because of its more complex decay structure and difficulty to be resolved in subjets.

% \begin{figure}[!ht]
%   \centering
%       \includegraphics[width=0.7\textwidth]{jet_part/appendixA/71graphcftr_h_JetRatio_mJ12CALOIQRoM4TruthsTops.pdf}
%   \caption[Breakdown of the $\mtas$ ]{Breakdown of the $\mtas$ in its component using truth-level information for boosted top quarks decays.}
%   \label{fig:breakdown3}
% \end{figure}



% \subsection{Large-R jet: Calibration}
% The jet mass scale calibration aims to correct the reconstructed jet mass to the particle-level jet mass by applying calibration factors derived from a sample of simulated QCD multijet events, with an analogous procedure described in \ref{sec:calib} for the jet energy scale.
% 
% \subsection{Large-R jet: Uncertainties}


\subsection{Performance with Alternate Inputs to the $\mtas$}
\label{sec:alternate}

There are quite a few ways to modify the track-assisted subjet mass; however, all the alternative approaches showed worse performance, and they are mentioned here for completeness only.
The per-track four momentum correction scheme which is used for the ECF and the n-subjettiness and also explored with the $\mtas$ with no significant difference was described in \ref{sec:tas}.

The other alternatives considered were: 
\begin{itemize}
 \item for the tracks:
 \begin{itemize}
   \item use of tracks not as input directly, but only taking those belonging to anti-k$_t$ reclustered track-jet with radius of 0.3 or 0.2;
   \item tighter or looser quality conditions were explored;
   \item tighter or looser primary vertex association requirement were explored.
 \end{itemize}
 \item for the subjets:
  \begin{itemize}
   \item the trimming procedure was modified: various radii $R_{sub}$ of the subjets were tested;
   \item the subjets were reclustered using not only the standard k$_t$, but also anti-k$_t$ and C/A.
  \end{itemize}
  \item for the procedure: different 4-momentum correction scheme was also studied in more details, see \ref{sec:tas}.
\end{itemize}

The different reclustering algorithm choice has a deep impact and was studied in details, since it changes the topo-cluster added to the subjets and the tracks associated to them. The situation is depicted in the event-display in Figure \ref{fig:evtdspl}; the display on the left shows the standard choice of k$_t$, the one on the right shows the modified approach anti-k$_t$. 

In Figure \ref{fig:allalgow} \ref{fig:allalgotop} \ref{fig:allalgohiggs} the performance for $W/Z$, tops and Higgs jets are shown, respectively. It can be seen that the k$_t$ algorithm provides the best observable definition, in all the samples considered. However, the anti-k$_t$ algorithm provides similar performances; this was an important check as the jet calibration procedure currently going on in ATLAS, the \textit{R-Scan} procedures includes the anti-k$_t$ algorithm with radius of R=0.2 and aims at providing the calibration and uncertainties that could be used directly in the computation of the $\mtas$.
% *** quantify the difference ***

\begin{figure}[!ht]
  \centering
      \includegraphics[width=0.85\textwidth]{jet_part/mtas/evtdspl.png}
  \caption[Different reclustering in event display]{An example of event-display shows the differences in the reclustering algorithm used for the subjets: on the right  k$_t$ and on the left anti-k$_t$. Highlighted some constituents trimmed away with the second choice.}
  \label{fig:evtdspl}
\end{figure}



% \begin{figure}
%     \centering
%    \includegraphics[width=\textwidth]{jet_part/mtas/71graphcftr_h_JetRatio_mJ12CALOIQRoM_Wprime_Allalgos.pdf}
   
%     \caption{Performance of $\mtas$ with different reclustering algorithm for the subjets: anti-k$_t$, k$_t$ and C/A. Boosted $W/Z$ sample.}
%     \label{fig:allalgow}
% \end{figure}

% \begin{figure}
%     \centering
%    \includegraphics[width=\textwidth]{jet_part/mtas/71graphcftr_h_JetRatio_mJ12CALOTopsCalib.pdf}
   
%     \caption{Performance of $\mtas$ with different reclustering algorithm for the subjets: anti-k$_t$, k$_t$ and C/A. Boosted top sample.}
%     \label{fig:allalgotop}
% \end{figure}

% \begin{figure}
%     \centering
%    \includegraphics[width=\textwidth]{jet_part/mtas/71graphcftr_h_JetRatio_mJ12CALOIQRoMHiggsNOCalib.pdf}
   
%     \caption{Performance of $\mtas$ with different reclustering algorithm for the subjets: anti-k$_t$, k$_t$ and C/A. Boosted higgs sample.}
%     \label{fig:allalgohiggs}
% \end{figure}

%\clearpage


\begin{figure}
    \centering
    \begin{subfigure}[b]{0.45\textwidth}
        \centering
   \includegraphics[width=\textwidth]{jet_part/mtas/71graphcftr_h_JetRatio_mJ12CALOIQRoM_Wprime_Allalgos.pdf}
    \caption{$W/Z$ jets.}
    \label{fig:allalgow}
    \end{subfigure}
    \begin{subfigure}[b]{0.45\textwidth}
        \centering
   \includegraphics[width=\textwidth]{jet_part/mtas/71graphcftr_h_JetRatio_mJ12CALOIQRoMHiggsNOCalib.pdf}
    \caption{Higgs jets.}
    \label{fig:allalgohiggs}
    \end{subfigure}

    \begin{subfigure}[b]{0.45\textwidth}
        \centering
   \includegraphics[width=\textwidth]{jet_part/mtas/71graphcftr_h_JetRatio_mJ12CALOTopsCalib.pdf}
    \caption{Top jets.}
    \label{fig:allalgotop}
    \end{subfigure}

\caption{Performance of $\mtas$ with different reclustering algorithms for the subjets: anti-k$_t$, k$_t$ and C/A and for $W/Z$ jets, top left, Higgs jets, top right and top jets, bottom. In all the cases shown, the k$_t$ is producing the better results, but all the three have a very similar performance.}
\end{figure}



\clearpage

\section{Performance of Combined Calorimeter and Track-Assisted SubJet Mass}
\label{sec:mcombtas}
This section presents the achievement of the variable obtained combining the $\mtas$ and the $\mcal$, the $\mcombtas$ with respect to the combination of the $\mta$ and the $\mcal$, the $\mcomb$. Both these variables were defined in \ref{subsec:comb}


\subsection{Performance in $W \to q'\bar{q}$ Decays}
On the $W/Z$s decays, the $\mcombtas$ outperforms all the other definitions throughout all the transverse momentum space; on Figure \ref{fig:mcombtas3} they are shown for reference together with the $\mtas$. It can be noted here that the track-assisted subjet mass, although being sub-optimal, has comparable performance, yet presenting fewer complications due to the combination procedure.

% \begin{figure}[!ht]
%   \centering
%       \includegraphics[width=0.7\textwidth]{jet_part/mcomb/mcombtas3.pdf}
%   \caption[$\mcombtas$ on the boosted $W/Z$]{Performance of the combined mass on $W/Z$ samples; here shown the two definitions of the combined mass, $\mcomb$ and $\mcombtas$, together with the calorimeter mass and the track-assisted subjet mass.}
%   \label{fig:mcombtas3}
% \end{figure}


\begin{figure}[!ht]
    \centering
    \begin{subfigure}[b]{0.45\textwidth}
  \centering
      \includegraphics[width=0.9\textwidth]{jet_part/mcomb/mcombtas3.pdf}
  \caption[$\mcombtas$ on the boosted $W/Z$]{$W/Z$ jets.}
  \label{fig:mcombtas3}
    \end{subfigure}
    \begin{subfigure}[b]{0.45\textwidth}
  \centering
      \includegraphics[width=0.9\textwidth]{jet_part/mcomb/mcombtas5.pdf}
  \caption[$\mcombtas$ on the boosted Higgs]{Higgs jets.}
  \label{fig:mcombtas5}
    \end{subfigure}

    \begin{subfigure}[b]{0.45\textwidth}
  \centering
      \includegraphics[width=0.9\textwidth]{jet_part/mcomb/mcombtas4.png}
  \caption[$\mcombtas$ on the boosted tops]{Top jets.}
  \label{fig:mcombtas4}
    \end{subfigure}
\caption{Performance of $\mcomb$ and $\mcombtas$ for different samples: the $W/Z$ jets, top left, the Higgs jets, top right and the top jets, bottom. The $\mcombtas$ outperforms the other definitions throughout the whole spectrum of transverse momentum. The $\mtas$, although being sub-optimal follows with similar performance the $\mcomb$. The Higgs and top jets presents the same properties as shown before, and the combined mass reflects these properties. }
\end{figure}

\subsection{Performance in $h\to b\bar{b}$ Decays}
Again, for the Higgs decay there are similarities as for the top sample; on Figure \ref{fig:mcombtas5} the two definitions of the combined mass, together with the simpler $\mtas$. Although this variable is slightly sub-optimal yet still comparable in the low to intermediate range in transverse momenta, where the tracks are driving a decrease in performance for the high to very-high $\pt$. The $\mcombtas$ uses this advantage to achieve optimal behavior in the entire transverse momentum spectrum, outperforming both $\mcal$ and $\mcomb$ almost everywhere.


\subsection{Performance in $t\to q'\bar{q}b$ Decays}
The top decay remains the most challenging phenomenon also with the combined mass; as seen on Figure \ref{fig:mcombtas4}, the $\mcomb$ performs quite similarly to the calorimeter based mass definition, behaving considerably better than the $\mtas$ especially at high transverse momentum. The $\mcombtas$, however, outperforms all the other definitions, and shows its optimal observable strength at intermediate $\pt$ i.e. in the range $0.8 < \pt < 1.6$ TeV.

% \begin{figure}[!ht]
%   \centering
%       \includegraphics[width=0.7\textwidth]{jet_part/mcomb/mcombtas4.png}
%   \caption[$\mcombtas$ on the boosted tops]{Performance of the combined mass on the top sample; here shown the two definitions of the combined mass, $\mcomb$ and $\mcombtas$, together with the calorimeter mass and the track-assisted subjet mass.}
%   \label{fig:mcombtas4}
% \end{figure}

% \begin{figure}[!ht]
%   \centering
%       \includegraphics[width=0.7\textwidth]{jet_part/mcomb/mcombtas5.pdf}
%   \caption[$\mcombtas$ on the boosted Higgs]{Performance of the combined mass on the Higgs decay; here shown the two definitions of the combined mass, $\mcomb$ and $\mcombtas$, together with the calorimeter mass and the track-assisted subjet mass.}
%   \label{fig:mcombtas5}
% \end{figure}






